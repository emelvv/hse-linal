\documentclass[a4paper]{article}
\usepackage{setspace}
\usepackage[T2A]{fontenc} %
\usepackage[utf8]{inputenc} % подключение русского языка
\usepackage[russian]{babel} %
\usepackage[14pt]{extsizes}
\usepackage{mathtools}
\usepackage{graphicx}
\usepackage{fancyhdr}
\usepackage{amssymb}
\usepackage{amsmath, amsfonts, amssymb, amsthm, mathtools}
\usepackage{tikz}

\usetikzlibrary{positioning}
\setstretch{1.3}

\newcommand{\mat}[1]{\begin{pmatrix} #1 \end{pmatrix}}
\renewcommand{\f}[2]{\frac{#1}{#2}}
\newcommand{\dspace}{\space\space}
\newcommand{\s}[2]{\sum\limits_{#1}^{#2}}
\newcommand{\sq}[1]{\left[ {#1} \right]}
\newcommand{\gath}[1]{\left[ \begin{array}{@{}l@{}} #1 \end{array} \right.}
\newcommand{\case}[1]{\begin{cases} #1 \end{cases}}
\newcommand{\ts}{\text{\space}}

\newcommand{\lr}{\Leftrightarrow}
\renewcommand{\r}{\Rightarrow}
\newcommand{\rr}{\rightarrow}
\renewcommand{\geq}{\geqslant}
\renewcommand{\leq}{\leqslant}
\newcommand{\RR}{\mathbb{R}}
\newcommand{\CC}{\mathbb{C}}
\newcommand{\QQ}{\mathbb{Q}}
\newcommand{\ZZ}{\mathbb{Z}}
\newcommand{\VV}{\mathbb{V}}
\newcommand{\NN}{\mathbb{N}}

\DeclarePairedDelimiter\abs{\lvert}{\rvert} %
\makeatletter                               % \abs{}
\let\oldabs\abs                             %
\def\abs{\@ifstar{\oldabs}{\oldabs*}}       %

\begin{document}

\section*{ИДЗ (Линейная алгебра)}
{\large Емельянов Владимир, ПМИ гр №247}\\\\
\begin{enumerate}
    \item[\textbf{1}]
    $$A = \mat{-1 & -4 & -1 \\ -1 & 0 & 6}, \ts B = \mat{7 & -1 & 6 \\ 3 & 0 & 1}, \ts C = \mat{6 & 1 \\ 0 & 6}, \ts D = \mat{13 & 13 \\ 13 & 13}$$
    $$tr(B^TB)DAA^T+tr((3BA^T+4AB^T)D+D(2BA^T-3AB^T))(B+A)(B^T-A^T)+$$
    $$+9C^2-6CD + D^2$$
    1)$$tr(B^TB) = \mat{7 & 3  \\ -1 & 0  \\ 6 & 1 }\cdot \mat{7 & -1 & 6 \\ 3 & 0 & 1} = tr(\mat{58 & -7 & 45 \\ -7 & 1 & -6 \\ 45 & -6 & 37}) = $$
    $$= 58+1+37 = 96$$
    $$DA = \mat{13 & 13 \\ 13 & 13} \cdot \mat{-1 & -4 & -1 \\ -1 & 0 & 6} = \mat{-26 & -52 & 65 \\ -26 & -52 & 65} \r $$
    $$\r DAA^T = \mat{-26 & -52 & 65 \\ -26 & -52 & 65} \cdot \mat{-1 & -1  \\ -4 & 0 \\ -1 & 6  } = \mat{169 & 416 \\ 169 & 416}$$
    $$tr(B^TB)DAA^T = 96\mat{169 & 416 \\ 169 & 416} = \mat{16224 & 39936 \\ 16224 & 39936}$$
    
    2)$$D^T = D$$
    $$tr((3BA^T+4AB^T)D+D(2BA^T-3AB^T)) = $$
    $$=tr((3BA^T+4AB^T)D) + tr(D(2BA^T-3AB^T)) = $$
    $$=tr((3BA^T+4AB^T)D) + tr((2AB^T-3BA^T)D^T) = $$
    $$=tr((3BA^T+4AB^T)D) + tr((2AB^T-3BA^T)D) = $$
    $$=tr((3BA^T+4AB^T)D+(2AB^T-3BA^T)D) = $$
    $$=tr(6AB^TD)$$
    $$AB^T = \mat{-1 & -4 & -1 \\ -1 & 0 & 6} \cdot \mat{7 & 3  \\ -1 & 0  \\ 6 & 1 } = \mat{-9 & -4 \\ 29 & 3}$$
    $$AB^TD = \mat{-9 & -4 \\ 29 & 3} \cdot \mat{13 & 13 \\ 13 & 13} = \mat{-169 & -169 \\ 416 & 416}$$
    $$6AB^TD = 6\mat{-169 & -169 \\ 416 & 416} = \mat{-1014 & -1014 \\ 2496 & 2496}$$
    $$tr(6AB^TD) = -1014+2496=1482$$

    3) $$(B+A)(B^T-A^T) = (B+A)(B-A)^T$$
    $$B+A = \mat{7 & -1 & 6 \\ 3 & 0 & 1} + \mat{-1 & -4 & -1 \\ -1 & 0 & 6} = \mat{6 & -5 & 5 \\ 2 & 0 & 7}$$
    $$B-A = \mat{7 & -1 & 6 \\ 3 & 0 & 1} - \mat{-1 & -4 & -1 \\ -1 & 0 & 6} = \mat{8 & 3 & 7 \\ 4 & 0 & -5}$$
    $$(B+A)(B-A)^T = \mat{6 & -5 & 5 \\ 2 & 0 & 7} \cdot \mat{8 & 4 \\ 3 & 0 \\ 7 & -5} = \mat{68 & -1 \\ 65 & -27}$$
    $$tr((3BA^T+4AB^T)D+D(2BA^T-3AB^T))(B+A)(B^T-A^T) = $$
    $$=1482\mat{68 & -1 \\ 65 & -27} = \mat{100776 & -1482 \\ 96330 & -40014}$$

    4) $$C^2 = \mat{6 & 1 \\ 0 & 6}^2 = \mat{36 & 12 \\ 0 & 36}$$
    $$9C^2 = 9\mat{36 & 12 \\ 0 & 36} = \mat{324 & 108 \\ 0 & 324}$$
    $$CD = \mat{6 & 1 \\ 0 & 6} \cdot \mat{13 & 13 \\ 13 & 13} = \mat{91 & 91 \\ 78 & 78}$$
    $$-6CD = -6\mat{91 & 91 \\ 78 & 78} = \mat{-546 & -546 \\ -468 & -468}$$
    $$D^2 = \mat{13 & 13 \\ 13 & 13}^2 = \mat{338 & 338 \\ 338 & 338}$$
    $$9C^2-6CD + D^2 = \mat{324 & 108 \\ 0 & 324} + \mat{-546 & -546 \\ -468 & -468} + \mat{338 & 338 \\ 338 & 338} = $$
    $$= \mat{116 & -100 \\ -130 & 194}$$
    
    5)$$tr(B^TB)DAA^T+tr((3BA^T+4AB^T)D+D(2BA^T-3AB^T))(B+A)(B^T-A^T)+$$
    $$+9C^2-6CD + D^2 = \mat{16224 & 39936 \\ 16224 & 39936} + \mat{100776 & -1482 \\ 96330 & -40014} + \mat{116 & -100 \\ -130 & 194} =$$
    $$= \mat{117116 & 38354 \\ 112424 & 116}$$

    \textbf{Ответ: } $\mat{117116 & 38354 \\ 112424 & 116}$

    \item[\textbf{2.}]
    $$A = \mat{a_{11} & a_{12} & a_{13} & a_{14} \\ a_{12} & a_{22} & a_{23} & a_{24} \\ a_{13} & a_{23} & a_{33} & a_{34} \\ a_{14} & a_{24} & a_{34} & a_{44}}, \ts 
    B = \mat{0 & b_{12} & b_{13} & b_{14} \\ -b_{12} & 0 & b_{23} & b_{24} \\ -b_{13} & -b_{23} & 0 & b_{34} \\ -b_{14} & -b_{24} & -b_{34} & 0}$$
    $$A+B = \mat{a_{11} & a_{12} & a_{13} & a_{14} \\ a_{12} & a_{22} & a_{23} & a_{24} \\ a_{13} & a_{23} & a_{33} & a_{34} \\ a_{14} & a_{24} & a_{34} & a_{44}} + 
    \mat{0 & b_{12} & b_{13} & b_{14} \\ -b_{12} & 0 & b_{23} & b_{24} \\ -b_{13} & -b_{23} & 0 & b_{34} \\ -b_{14} & -b_{24} & -b_{34} & 0} = $$
    $$=\mat{-18 & 42 & 28 & 8 \\ 20 & 4 & 22 &38 \\ -12 & -18 & -22 &6 \\ 0 & 0 & 52 & 38}$$
    $$\case{
        a_{11} = -18 \\
        a_{12} = 42 - b_{12} = b_{12}+20\\
        a_{13} = 28 - b_{13} = b_{13}-12\\
        a_{14} = 8 - b_{14} = b_{14}\\
        a_{22} = 4 \\
        a_{23} = 22 - b_{23} = b_{23}-18\\
        a_{24} = 38 - b_{24} = b_{24}\\
        a_{33} = -22 \\
        a_{34} = 6 - b_{34} = b_{34}+52\\
        a_{44} = 38
    }$$

    $$\case{
        a_{11} = -18\\
        a_{12} = 31, b_{12} = 11\\
        a_{13} = 8, b_{13} = 20\\
        a_{14} = 4, b_{14} = 4\\
        a_{22} = 4\\
        a_{23} = 2, b_{23} = 20 \\
        a_{24} = 19, b_{24} = 19 \\
        a_{33} = -22\\
        a_{34} = 29, b_{34} = -23 \\
        a_{44} = 38
    }$$
    $$A = \mat{-18 & 31 & 8 & 4 \\ 31 & 4 & 2 & 19 \\ 8 & 2 & -22 & 29 \\ 4 & 19 & 29 & 38}, \ts
    B = \mat{0 & 11 & 20 & 4 \\ -11 & 0 & 20 & 19 \\ -20 & -20 & 0 & -23 \\ -4 & -19 & 23 & 0}$$
    $$AB = \mat{-18 & 31 & 8 & 4 \\ 31 & 4 & 2 & 19 \\ 8 & 2 & -22 & 29 \\ 4 & 19 & 29 & 38} \cdot \mat{0 & 11 & 20 & 4 \\ -11 & 0 & 20 & 19 \\ -20 & -20 & 0 & -23 \\ -4 & -19 & 23 & 0} =$$
    $$= \mat{-517 & -434 & 352 & 333 \\ -160 & -60 & 1137 & 154 \\ 302 & -23 & 867 & 576 \\ -941 & -1258 & 1334 & -290}$$
    \textbf{Ответ: } $\mat{-517 & -434 & 352 & 333 \\ -160 & -60 & 1137 & 154 \\ 302 & -23 & 867 & 576 \\ -941 & -1258 & 1334 & -290}$

    \item[\textbf{3.}]
    $$C = \mat{1 & -3 & -5 \\ 0 & 1 &2 \\ 0 & 0 & 1}\ts 
    J = \mat{-1 & 1 & 0 \\ 0 & -1 & 1 \\ 0 & 0 & -1} \ts 
    D = \mat{1 & 3 & -1 \\ 0 & 1 & -2 \\ 0 & 0 & 1}$$
    $$DC = \mat{1 & 3 & -1 \\ 0 & 1 & -2 \\ 0 & 0 & 1} \cdot \mat{1 & -3 & -5 \\ 0 & 1 &2 \\ 0 & 0 & 1} = E$$
    $$A^{2021} = CJDCJDCJD\dots CJD = CJEJEJE \dots EJD = CJ^{2021}D$$
    $$S = E + C(J+\dots+J^{2021})D$$
    $$a_1 = J^1 = \mat{-1 & 1 & 0 \\ 0 & -1 & 1 \\ 0 & 0 & -1}$$
    $$a_2 = J^1 + J^2 = \mat{0 & -1 & 1 \\ 0 & 0 & -1 \\ 0 & 0 & 0}$$
    $$a_3 = J^1 + J^2 + J^3 = \mat{-1 & 2 & -2 \\ 0 & -1 & 2 \\ 0 & 0 & -1}$$
    $$\dots$$
    Доказать через мат. индукцию:
    $$a_k = \mat{-1 & \f{k+1}{2} & -\f{k-1}{2}(\f{k-1}{2}+1) \\ 0 & -1 & \f{k+1}{2} \\ 0 & 0 & -1} \text{ для всех нечётных } k \r $$
    $$\r a_{2021} = J^1 + \dots + J^{2021} = \mat{-1 & 1011 & -1021110 \\ 0 & -1 & 1011 \\ 0 & 0 & -1} \r$$
    $$ C\cdot a_{2021}\cdot D =$$
    $$= \mat{1 & -3 & -5 \\ 0 & 1 &2 \\ 0 & 0 & 1} \cdot \mat{-1 & 1011 & -1021110 \\ 0 & -1 & 1011 \\ 0 & 0 & -1} \cdot \mat{1 & 3 & -1 \\ 0 & 1 & -2 \\ 0 & 0 & 1} =$$
    $$=\mat{-1 & 1014 & -1024138 \\ 0 & -1 & 1009 \\ 0 & 0 & -1}\cdot \mat{1 & 3 & -1 \\ 0 & 1 & -2 \\ 0 & 0 & 1} =$$
    $$=\mat{-1 & 1011 & -1026165 \\ 0 & -1 & 1011 \\ 0 & 0 & -1} \r $$
    $$\r S = E + \mat{-1 & 1011 & -1026165 \\ 0 & -1 & 1011 \\ 0 & 0 & -1} = \mat{0 & 1011 & -1026165 \\ 0 & 0 & 1011 \\ 0 & 0 & 0}$$
    \textbf{Ответ: } $\mat{0 & 1011 & -1026165 \\ 0 & 0 & 1011 \\ 0 & 0 & 0}$

    \item[\textbf{4.}]
    $$S = \mat{-15 & -30 & 30 \\ -9 & -18 & 18 \\ -18 & -36 & 36}$$
    $$u = \mat{x_1, y_1, z_1}, \ts v^T = \mat{x_2 \\ y_2 \\ z_2}$$
    $$uv^T = \mat{x_1 \\ y_1 \\ z_1} \cdot \mat{x_2 & y_2 & z_2} = \mat{x_1x_2 & x_1y_2 & x_1z_2\\ y_1x_2 & y_1y_2 & y_1z_2 \\ z_1x_2 & z_1y_2 & z_1z_2}$$
    $$\case{x_1x_2 = -15 \\ 
        x_1y_2 = -30 \\ 
        x_1z_2 = 30 \\ 
        y_1x_2 = -9 \\ 
        y_1y_2 = -18 \\ 
        y_1z_2 = 18 \\ 
        z_1x_2 = -18 \\ 
        z_1y_2 = -36 \\ 
        z_1z_2 = 36
    }\r u = \mat{15 \\ 9 \\ 18}, \ts v = \mat{-1 \\ -2 \\ 2} \text{ (частное решение)}$$

    $$S = \mat{15 \\ 9 \\ 18} \cdot \mat{-1 & -2 & 2} = uv^T$$
    $$v^Tu = \mat{-1 & -2 & 2} \cdot \mat{15 \\ 9 \\ 18} = \mat{3}$$
    $$S^{10} = u(3)(3)(3)(3)(3)(3)(3)(3)(3)v^T = u(19683)v^T =$$
    $$= 19683\mat{15 \\ 9 \\ 18}\cdot \mat{-1 & -2 & 2} = \mat{295245 \\ 177147 \\ 354294} \cdot \mat{-1 & -2 & 2} = $$
    $$=\mat{-295245 & -590490 & 590490 \\ -177147 & -354294 & 354294 \\ -354294 & -708588 & 708588}$$
    \textbf{Ответ: } $\mat{-295245 & -590490 & 590490 \\ -177147 & -354294 & 354294 \\ -354294 & -708588 & 708588}$

    \item[\textbf{5.}]
    \begin{enumerate}
        \item[(а)]$$
        \case{
            4x_1 - 7x_2 - 10x_3 + x_4 = -8, \\
            8x_1 - 4x_2 + 12x_4 = -6, \\
            6x_1 - 5x_2 - 4x_3 + 7x_4 = 1, \\
            -5x_1 + x_2 - 3x_3 - 9x_4 = 9.
        }$$
        Запишем в виде расширенной матрицы СЛУ:
        $$A = \mat{
            4  & -7  & -10  & 1  & | & -8  \\
            8  & -4  & 0    & 12 & | & -6  \\
            6  & -5  & -4   & 7  & | & 1   \\
            -5 & 1   & -3   & -9 & | & 9
        }$$
        $$A[0] = A[0]/4$$
        $$\mat{
            1      & -\frac{7}{4}  & -\frac{5}{2}  & \frac{1}{4}  & | & -2  \\
            8      & -4            & 0             & 12          & | & -6  \\
            6      & -5            & -4            & 7           & | & 1   \\
            -5     & 1             & -3            & -9          & | & 9
        }$$
        $$A[1] = A[1]-8*A[0]$$
        $$\mat{
            1      & -\frac{7}{4}  & -\frac{5}{2}  & \frac{1}{4}  & | & -2  \\
            0      & 10            & 20            & 10          & | & 10  \\
            6      & -5            & -4            & 7           & | & 1   \\
            -5     & 1             & -3            & -9          & | &9
        }$$
        $$A[1] = A[1]/10$$
        $$\mat{
            1      & -\frac{7}{4}  & -\frac{5}{2}  & \frac{1}{4}  & | & -2  \\
            0      & 1             & 2             & 1           & |&  1   \\
            6      & -5            & -4            & 7           & | &1   \\
            -5     & 1             & -3            & -9          & | &9
        }$$
        $$A[2] = A[2] - 6*A[0]$$
        $$\mat{
            1      & -\frac{7}{4}  & -\frac{5}{2}  & \frac{1}{4}  & | &-2  \\
            0      & 1             & 2             & 1           & | &1   \\
            0      & \frac{11}{2}  & 11            & \frac{11}{2} & | &13  \\
            -5     & 1             & -3            & -9          & | &9
        }$$
        $$A[2] = A[2]*2/11$$
        $$\mat{
            1      & -\frac{7}{4}  & -\frac{5}{2}  & \frac{1}{4}  & | &-2  \\
            0      & 1             & 2             & 1           & |& 1   \\
            0      & 1             & 2             & 1           & | &\frac{26}{11} \\
            -5     & 1             & -3            & -9          & | &9
        }$$
        $$A[2] = A[2]-A[1]$$
        $$ \mat{1 & -\frac{7}{4} & -\frac{5}{2} & \frac{1}{4} & | & -2 \\ 0 & 1 & 2 & 1 & | & 1 \\ 0 & 0 & 0 & 0 & | & \frac{15}{11} \\ -5 & 1 & -3 & -9 & | & 9}$$
        $$0 = \f{15}{11} \text{ - } \varnothing \r \text{ решений нет}$$
        \textbf{Ответ:} $\varnothing$

        \item[(б)]
        $$\begin{cases}
            4x_1 - 7x_2 - 10x_3 + x_4 &= -51, \\
            8x_1 - 4x_2 + 12x_4 &= -52, \\
            6x_1 - 5x_2 - 4x_3 + 7x_4 &= -49, \\
            -5x_1 + x_2 - 3x_3 - 9x_4 &= 25.
          \end{cases}$$
        Запишем в виде расширенной матрицы СЛУ:
        $$\mat{
            4 & -7 & -10 &  1 & | & -51 \\
            8 &  -4 &  0  & 12 & | & -52 \\
            6 & -5 & -4  &  7 & | & -49 \\
           -5 &  1 & -3  & -9 & | &  25
        }$$
        $$A[0] = A[0]/4$$
        $$ \mat{1 & -\frac{7}{4} & -\frac{5}{2} & \frac{1}{4} & | & -\frac{51}{4} \\ 8 & -4 & 0 & 12 & | & -52 \\ 6 & -5 & -4 & 7 & | & -49 \\ -5 & 1 & -3 & -9 & | & 25} $$
        $$A[1] = A[1] - 8*A[0]$$
        $$ \mat{1 & -\frac{7}{4} & -\frac{5}{2} & \frac{1}{4} & | & -\frac{51}{4} \\ 0 & 10 & 20 & 10 & | & 50 \\ 6 & -5 & -4 & 7 & | & -49 \\ -5 & 1 & -3 & -9 & | & 25} $$
        $$A[1] = A[1]/10$$
        $$ \mat{1 & -\frac{7}{4} & -\frac{5}{2} & \frac{1}{4} & | & -\frac{51}{4} \\ 0 & 1 & 2 & 1 & | & 5 \\ 6 & -5 & -4 & 7 & | & -49 \\ -5 & 1 & -3 & -9 & | & 25} $$
        $$A[2] = A[2]-6*A[0]$$
        $$ \mat{1 & -\frac{7}{4} & -\frac{5}{2} & \frac{1}{4} & | & -\frac{51}{4} \\ 0 & 1 & 2 & 1 & | & 5 \\ 0 & \frac{11}{2} & 11 & \frac{11}{2} & | & \frac{55}{2} \\ -5 & 1 & -3 & -9 & | & 25} $$
        $$A[2] = A[2]-A[1]*11/2$$
        $$ \mat{1 & -\frac{7}{4} & -\frac{5}{2} & \frac{1}{4} & | & -\frac{51}{4} \\ 0 & 1 & 2 & 1 & | & 5 \\ 0 & 0 & 0 & 0 & | & 0 \\ -5 & 1 & -3 & -9 & | & 25} $$
        $$A[2], A[3] = A[3], A[2]$$
        $$A[2] = A[2]+A[0]*5$$
        $$ \mat{1 & -\frac{7}{4} & -\frac{5}{2} & \frac{1}{4} & | & -\frac{51}{4} \\ 0 & 1 & 2 & 1 & | & 5 \\ 0 & -\frac{31}{4} & -\frac{31}{2} & -\frac{31}{4} & | & -\frac{155}{4} \\ 0 & 0 & 0 & 0 & | & 0} $$
        $$A[2] = A[2]+A[1]*31/4$$
        $$ \mat{1 & -\frac{7}{4} & -\frac{5}{2} & \frac{1}{4} & | & -\frac{51}{4} \\ 0 & 1 & 2 & 1 & | & 5 \\ 0 & 0 & 0 & 0 & | & 0 \\ 0 & 0 & 0 & 0 & | & 0} $$
        $$A[0] = A[0]+A[1]*7/4$$
        $$ \mat{1 & 0 & 1 & 2 & | & -4 \\ 0 & 1 & 2 & 1 & | & 5 \\ 0 & 0 & 0 & 0 & | & 0 \\ 0 & 0 & 0 & 0 & | & 0} $$
        $$\case{
            x_{1} = -4 -x_{3}-2x_{4}\\
            x_{2} = 5 - 2x_{3}-x_4
        } \forall x_3, x_4 \in \RR$$
        $$x_3 = 1, x_4 = 1 \r x_1 = -7, x_2 = 2$$
        \textbf{Ответ: }$\case{
            x_{1} = -4 -x_{3}-2x_{4}\\
            x_{2} = 5 - 2x_{3}-x_4
        } \forall x_3, x_4 \in \RR$ - общее решение, $x_1 = -7, x_2 = 2, x_3 = 1, x_4 = 1$ - частное

    \end{enumerate}

\end{enumerate}

\end{document}