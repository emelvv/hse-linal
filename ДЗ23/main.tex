\documentclass[a4paper]{article}
\usepackage{setspace}
\usepackage[T2A]{fontenc} %
\usepackage[utf8]{inputenc} % подключение русского языка
\usepackage[russian]{babel} %
\usepackage[12pt]{extsizes}
\usepackage{mathtools}
\usepackage{graphicx}
\usepackage{fancyhdr}
\usepackage{amssymb}
\usepackage{amsmath, amsfonts, amssymb, amsthm, mathtools}
\usepackage{tikz}

\usetikzlibrary{positioning}
\setstretch{1.3}

\newcommand{\mat}[1]{\begin{pmatrix} #1 \end{pmatrix}}
\newcommand{\vmat}[1]{\begin{vmatrix} #1 \end{vmatrix}}
\renewcommand{\f}[2]{\frac{#1}{#2}}
\newcommand{\dspace}{\space\space}
\newcommand{\s}[2]{\sum\limits_{#1}^{#2}}
\newcommand{\mul}[2]{\prod_{#1}^{#2}}
\newcommand{\sq}[1]{\left[ {#1} \right]}
\newcommand{\gath}[1]{\left[ \begin{array}{@{}l@{}} #1 \end{array} \right.}
\newcommand{\case}[1]{\begin{cases} #1 \end{cases}}
\newcommand{\ts}{\text{\space}}
\newcommand{\lm}[1]{\underset{#1}{\lim}}
\newcommand{\suplm}[1]{\underset{#1}{\overline{\lim}}}
\newcommand{\inflm}[1]{\underset{#1}{\underline{\lim}}}
\newcommand{\Ker}[1]{\operatorname{Ker}}

\renewcommand{\phi}{\varphi}
\newcommand{\lr}{\Leftrightarrow}
\renewcommand{\l}{\left(}
\renewcommand{\r}{\right)}
\newcommand{\rr}{\rightarrow}
\renewcommand{\geq}{\geqslant}
\renewcommand{\leq}{\leqslant}
\newcommand{\RR}{\mathbb{R}}
\newcommand{\CC}{\mathbb{C}}
\newcommand{\QQ}{\mathbb{Q}}
\newcommand{\ZZ}{\mathbb{Z}}
\newcommand{\VV}{\mathbb{V}}
\newcommand{\NN}{\mathbb{N}}
\newcommand{\OO}{\underline{O}}
\newcommand{\oo}{\overline{o}}
\renewcommand{\Ker}{\operatorname{Ker}}
\renewcommand{\Im}{\operatorname{Im}}
\newcommand{\vol}{\text{vol}}
\newcommand{\Vol}{\text{Vol}}

\DeclarePairedDelimiter\abs{\lvert}{\rvert} %
\makeatletter                               % \abs{}
\let\oldabs\abs                             %
\def\abs{\@ifstar{\oldabs}{\oldabs*}}       %

\begin{document}

\section*{Домашнее задание на 05.04 (Линейная алгебра)}
 {\large Емельянов Владимир, ПМИ гр №247}\\\\
\begin{enumerate}
    \item[\textbf{№1}]Пусть $v_1 = (1, 2, 0, 3)^T$ и $v_2 = (-1, 3, 0, 4)^T$, тогда:
    \[\vol(P) = \sqrt{\det G(v_1, v_2)} = \sqrt{ \vmat{14 & 17 \\ 17 & 26}} 
    = \sqrt{75} = 5\sqrt{3}\]
    \textbf{Ответ:} $5\sqrt{3}$

    \item[\textbf{№2}]$a = (8, 4, 1)^T, b = (2, -2, 1)^T$
    \begin{enumerate}
        \item[2.1)]Найдём угол $\alpha$ между векторами $a$ и $b$
        $$\cos(\alpha) = \f{(a, b)}{|a||b|} = \f{9}{9\cdot 3} = \f{1}{3}$$
        \textbf{Ответ: } $\arccos{\f{1}{3}}$

        \item[2.2)]Найдём $c$:
        $$c' = [a, b] = \vmat{e_1 & e_2 & e_3 \\ a_1 & a_2 & a_3 \\ b_1 & b_2 & b_3} = 
        6e_1 -6e_2 -24e_3 \implies$$
        $$ \implies c' = (6, -6, -24)^T \implies c = \f{c'}{|c'|} = 
        \l \f{1}{3\sqrt{2}}, -\f{1}{3\sqrt{2}}, -\f{4}{3\sqrt{2}} \r^T$$
        Проверим, правая ли тройка:
        $$\Vol (P(a, b, c)) = (a, b, c) = ([a, b], c) = (c', c) > 0 \implies \text{ да }$$
        \textbf{Ответ: } $\l \f{1}{3\sqrt{2}}, -\f{1}{3\sqrt{2}}, -\f{4}{3\sqrt{2}} \r$

        \item[2.3)]Пусть $e = -c$:
        $$e = -c = \l -\f{1}{3\sqrt{2}}, \f{1}{3\sqrt{2}}, \f{4}{3\sqrt{2}} \r$$
        Тогда:
        $$\Vol (P(a, b, e)) = (a, b, e) = ([a, b], e) = (c', e) < 0$$
        При этом:
        $$\Vol (P(a, b, d)) = (a, b, d) = ([a, b], d) = (c', d) = 24-72 < 0$$
        Следовательно, тройки векторов $a, b, e$ и $a, b, d$ имеют одинаковую ориентацию

        \textbf{Ответ: } $\l -\f{1}{3\sqrt{2}}, \f{1}{3\sqrt{2}}, \f{4}{3\sqrt{2}} \r$\\

        \item[2.4)]Вектор $f$ должен удовлетворять системе:
        $$\case{
            (a, b, f) = 0\\
            \cos(\angle(f, a)) = 0\\
            |f| = |a| \\
            cos(\angle(f, b)) < 0
        }$$
        Это равносильно системе:
        $$\begin{cases}
            6f_1 - 6f_2 - 24f_3 = 0 \\
            8f_1 + 4f_2 + f_3 = 0 \\
            f_1^2 + f_2^2 + f_3^2 = 81 \\
            2f_1 - 2f_2 + f_3 < 0
        \end{cases} \implies f = \left( -\dfrac{5\sqrt{2}}{2}, \dfrac{11\sqrt{2}}{2}, -2\sqrt{2} \right)$$
        \textbf{Ответ: } $\left( -\dfrac{5\sqrt{2}}{2}, \dfrac{11\sqrt{2}}{2}, -2\sqrt{2} \right)$  \\  
    \end{enumerate}

    \item[\textbf{№3}]$A = (-1, 0, 1), B = (0, 1, 2), C = (-2, 2, 5), D = (-4, 0, 3)$. 
    Посчитаем смешанное произведение векторов $\vec{AB}, \vec{BC}, \vec{CD}$:
    $$\vec{AB} = (1, 1, 1)^T, \quad \vec{BC} = (-2, 1, 3)^T, \quad \vec{CD} = (-2, -2, -2)^T$$
    $$(\vec{AB}, \vec{BC}, \vec{CD}) = \vmat{1 & 1 & 1 \\ -2 & 1 & 3 \\ -2 & -2 & -2} = 
    -2-6+4-(-2-6+4) = 0$$
    Следовательно, векторы $\vec{AB}, \vec{BC}, \vec{CD}$ компланарны, а значит точки $A, B, C, D$ лежат в одной плоскости. 
    Теперь найдём площать по формуле:
    $$S_{ABCD} = \f{1}{2}\left| \left[\vec{AC}, \vec{BD}\right] \right|$$
    Найдём векторное произведение:
    $$\vec{AC} = (-1, 2, 4), \quad \vec{BD} = (-4, -1, 1)$$
    $$\left[\vec{AC}, \vec{BD}\right] = (6, 15, 9)^T$$
    Найдём площать:
    $$S_{ABCD} = \f{1}{2}\left|(6, 15, 9)^T\right| = \f{1}{2} \cdot 3\sqrt{38} = \f{3\sqrt{38}}{2}$$
    \textbf{Ответ: } $\f{3\sqrt{38}}{2}$

    \item[\textbf{№4}]Объём тетраэдра $ABCD$:
    $$V_{ABCD} = \f{1}{6}\vol(P(\vec{AB}, \vec{AC}, \vec{AD}))
    = \f{1}{6}|\det(\vec{AB}| \vec{AC}| \vec{AD})|$$
    Найдём определитель:
    $$\vec{AB} = (1, 2, 1), \quad \vec{AC} = (3, 1, 2), \quad \vec{AD} = (-1, 5, 0)$$
    $$|\det(\vec{AB}| \vec{AC}| \vec{AD})| = -2$$
    Следовательно
    $$V_{ABCD} = \f{1}{6}|\det(\vec{AB}| \vec{AC}| \vec{AD})| =
    \f{1}{6}\cdot 2  = \f{1}{3}$$
    \textbf{Ответ:} $\f{1}{3}$\\
\end{enumerate}
\end{document}