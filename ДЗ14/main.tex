\documentclass[a4paper]{article}
\usepackage{setspace}
\usepackage[T2A]{fontenc} %
\usepackage[utf8]{inputenc} % подключение русского языка
\usepackage[russian]{babel} %
\usepackage[12pt]{extsizes}
\usepackage{mathtools}
\usepackage{graphicx}
\usepackage{fancyhdr}
\usepackage{amssymb}
\usepackage{amsmath, amsfonts, amssymb, amsthm, mathtools}
\usepackage{tikz}

\usetikzlibrary{positioning}
\setstretch{1.3}

\newcommand{\mat}[1]{\begin{pmatrix} #1 \end{pmatrix}}
\renewcommand{\det}[1]{\begin{vmatrix} #1 \end{vmatrix}}
\renewcommand{\f}[2]{\frac{#1}{#2}}
\newcommand{\dspace}{\space\space}
\newcommand{\s}[2]{\sum\limits_{#1}^{#2}}
\newcommand{\mul}[2]{\prod_{#1}^{#2}}
\newcommand{\sq}[1]{\left[ {#1} \right]}
\newcommand{\gath}[1]{\left[ \begin{array}{@{}l@{}} #1 \end{array} \right.}
\newcommand{\case}[1]{\begin{cases} #1 \end{cases}}
\newcommand{\ts}{\text{\space}}
\newcommand{\lm}[1]{\underset{#1}{\lim}}
\newcommand{\suplm}[1]{\underset{#1}{\overline{\lim}}}
\newcommand{\inflm}[1]{\underset{#1}{\underline{\lim}}}

\renewcommand{\phi}{\varphi}
\newcommand{\lr}{\Leftrightarrow}
\renewcommand{\r}{\Rightarrow}
\newcommand{\rr}{\rightarrow}
\renewcommand{\geq}{\geqslant}
\renewcommand{\leq}{\leqslant}
\newcommand{\RR}{\mathbb{R}}
\newcommand{\CC}{\mathbb{C}}
\newcommand{\QQ}{\mathbb{Q}}
\newcommand{\ZZ}{\mathbb{Z}}
\newcommand{\VV}{\mathbb{V}}
\newcommand{\NN}{\mathbb{N}}
\newcommand{\OO}{\underline{O}}
\newcommand{\oo}{\overline{o}}


\DeclarePairedDelimiter\abs{\lvert}{\rvert} %
\makeatletter                               % \abs{}
\let\oldabs\abs                             %
\def\abs{\@ifstar{\oldabs}{\oldabs*}}       %

\begin{document}

\section*{Домашнее задание на 22.01 (Линейная алгебра)}
 {\large Емельянов Владимир, ПМИ гр №247}\\\\
\begin{enumerate}
    \item[\textbf{№1}]Векторы $U$:
    $$
    u_1 = (0,0,0,1,1)^{T}, \quad u_2 = (0,0,3,2,-3)^{T}, \quad u_3 = (0,0,3,5,0)^{T}, \quad u_4 = (0,2,1,3,0)^{T}
    $$
    Векторы $W$:
    $$
    w_1 = (1,-1,2,0,0)^{T}, \quad w_2 = (1,1,1,0,0)^{T}
    $$
    Докажем, что:
    $$\case{
        \RR^5 = U + W\\
        \dim(U + W) = \dim(U) + \dim(W)
    } \lr \case{
        \dim \RR^5 = \dim(U+W)\\
        \dim(U + W) = \dim(U) + \dim(W)
    } \lr$$
    $$\lr \dim(U) + \dim(W) = \dim(U + W) = 5$$
    Для начала найдём какие из векторов $u_1, u_2, u_3, u_4, w_1, w_2$ - ЛНЗ:
    $$\mat{0 & 0 &0 & 0&1& 1\\
    0 & 0 & 0 & 2 & -1&1\\
    0 & 3 &3  & 1 & 2&1\\
    1 & 2 & 5 & 3 & 0&0\\
    1& -3 & 0 & 0 & 0&0} \implies \text{УСВ: } \mat{1 & 0 & 3 & 0 & 0 & 0 \\
    0 & 1 & 1 & 0 & 0 & 0 \\
    0 & 0 & 0 & 1 & 0 & 0 \\
    0 & 0 & 0 & 0 & 1 & 0 \\
    0 & 0 & 0 & 0 & 0 & 1}$$
    Следовательно, $u_1, u_2, u_4, w_1, w_2$ - ЛНЗ, а значит:
    $$\dim U = 3, \quad \dim W = 2, \quad \dim(U + W) = 5$$
    Значит, выполняется:
    $$\dim(U) + \dim(W) = \dim(U + W) = 5$$
    Поэтому:
    $$\RR^5 = U \oplus V$$
    Найдём проекцию вектора $v$ на $U$ вдоль $W$ и на $W$ вдоль $U$:
    $$v = \mat{6&4&6&4&6}^T$$
    Вектор $v$ выражается как:
    $$v = u + w, \text{ где $u \in U$, $w \in W$}$$
    Здесь $u$ и $w$ будут искомыми проекциями. Их можно представить как:
    $$u = \lambda_1 u_1 + \lambda_2 u_2 + \lambda_3 u_4, \quad w = \mu_1 w_1 + \mu_2w_2$$
    Найдём $\lambda_1, \lambda_2, \lambda_3, \mu_1, \mu_2$ решив $u+w = v$:
    $$\mat{0 & 0  & 0&1& 1 & | & 6\\
    0 & 0 & 2 & -1&1 & | & 4\\
    0 & 3 & 1 & 2&1 & | & 6\\
    1 & 2 & 3 & 0&0 & | & 4\\
    1& -3 & 0 & 0&0 & | & 6} \implies \text{УСВ: } \mat{1 & 0 & 0 & 0 & 0 & | & 3 \\
    0 & 1 & 0 & 0 & 0 & | & -1 \\
    0 & 0 & 1 & 0 & 0 & | & 1 \\
    0 & 0 & 0 & 1 & 0 & | & 2 \\
    0 & 0 & 0 & 0 & 1 & | & 4}\implies \mat{\lambda_1\\\lambda_2\\ \lambda_3\\ \mu_1\\ \mu_2} = \mat{3 \\ -1\\ 1\\ 2\\4}$$
    Найдём $u$:
    $$
    u = 3u_1 - 1u_2 + 1u_4 = 3 \begin{pmatrix} 0 \\ 0 \\ 0 \\ 1 \\ 1 \end{pmatrix} - 1 \begin{pmatrix} 0 \\ 0 \\ 3 \\ 2 \\ -3 \end{pmatrix} + 1 \begin{pmatrix} 0 \\ 2 \\ 1 \\ 3 \\ 0 \end{pmatrix} = \begin{pmatrix} 0 \\ 2 \\ -2 \\ 4 \\ 6 \end{pmatrix}
    $$
    Найдём $w$:
    $$w = 2w_1 + 4w_2 = 2 \begin{pmatrix} 1 \\ -1 \\ 2 \\ 0 \\ 0 \end{pmatrix} + 4 \begin{pmatrix} 1 \\ 1 \\ 1 \\ 0 \\ 0 \end{pmatrix} = \begin{pmatrix} 6 \\ 2 \\ 8 \\ 0 \\ 0 \end{pmatrix}$$

    \textbf{Ответ: } На $U$: $(0, 2, -2, 4, 6)^T$, на $W$: $(6, 2, 8, 0, 0)^T$\\

    \item[\textbf{№2}] $U$ задано уравнением:
    $$x_1 + x_2 + \dots + x_n = 0$$
    $W$ задано уравнением:
    $$x_1 = \dots = x_n $$
    Докажем, что 
    $$\RR^n = U \oplus W$$
    Это эквивалентно:
    $$\case{
        \RR^n = U+W\\
        \dim U + \dim W = \dim(U + W)
    } \lr \case{
        n = \dim(U+W)\\
        \dim U + \dim W = n
    } \lr$$
    $$ \lr \case{
        n = \dim U + \dim W - \dim(U \cap W)\\
        \dim U + \dim W = n
    } \lr \case{
        \dim(U \cap W) = 0 \\
        \dim U + \dim W = n
    }$$
    Найдём $\dim U$:
    $$\mat{1 & 1 & \dots & 1 & | & 0} \implies x_1 = -x_2-x_3-\dots-x_n$$
    $$\mat{x_1 \\ x_2 \\ \dots \\ x_n} = \mat{-x_2-x_3-\dots-x_n\\x_2 \\ \dots \\ x_n} = \mat{-1\\1 \\ \dots \\ 0}x_2 + \mat{-1\\0 \\ \dots \\ 0}x_3 + \dots \implies$$
    $$\implies \dim U = n-1$$
    Найдём $\dim W$, линейная оболочка будет иметь вид:
    $$\mat{1\\1\\\dots\\1}x_1 \quad \forall x_1 \in \RR$$
    Поэтому:
    $$\dim W = 1$$
    Найдём $\dim( U \cap W)$:
    $$\case{
        x_1 + x_2 + \dots + x_n = 0\\
        x_1 = \dots = x_n 
    } \implies x_1 = \dots = x_n = 0 \implies \dim (U \cap W )= 0$$
    Получается, что выполняется условие:
    $$\case{
        \dim(U \cap W) = 0 \\
        \dim U + \dim W = n
    }$$
    А значит:
    $$\RR^n = U \oplus W$$
    Нам известно, что:
    $$U = \langle \mat{-1\\1 \\ \dots \\ 0},\mat{-1\\0 \\ \dots \\ 0}, \dots\rangle, \quad W = \langle \mat{1\\1\\\dots\\1} \rangle$$
    Найдём, проекции $v = (a_1, a_2, \dots, a_n)^T$:
    $$v = u + w = \alpha_1u_1 + \alpha_2u_2 + \dots + \alpha_{n-1}u_{n-1} + \mu_1w_1$$
    $$\mat{
        -1 & -1 & -1 & \dots & -1 & 1 & | & a_1\\
         1 & 0 & 0 & \dots & 0 & 1 & | & a_2\\
         0 & 1 & 0 & \dots & 0 & 1 & | & a_3\\
         0 & 0 & 1 & \dots & 0 & 1 & | & a_4 \\
         \dots & \dots &\dots &\dots &\dots & \dots & \dots & \dots &
    } \implies$$$$\implies \mat{
        1 & 0 & 0 & \dots & 0 & 1 & | & a_2\\
        0 & 1 & 0 & \dots & 0 & 1 & | & a_3\\
        0 & 0 & 1 & \dots & 0 & 1 & | & a_4 \\
        \dots & \dots &\dots &\dots &\dots & \dots & \dots & \dots \\
        0 & 0 & 0 & \dots & 1 & 1 & | & a_{n} \\
        0 & 0 & 0 & \dots & 0 & n & | & a_1 + a_2+\dots + a_n\\
    } \implies $$
    $$\implies \mat{
        1 & 0 & 0 & \dots & 0 & 1 & | & a_2\\
        0 & 1 & 0 & \dots & 0 & 1 & | & a_3\\
        0 & 0 & 1 & \dots & 0 & 1 & | & a_4 \\
        \dots & \dots &\dots &\dots &\dots & \dots & \dots & \dots \\
        0 & 0 & 0 & \dots & 1 & 1 & | & a_{n} \\
        0 & 0 & 0 & \dots & 0 & 1 & | & \f{a_1 + a_2+\dots + a_n}{n}\\
    } \r $$
    $$\r \mat{\alpha_1 \\ \alpha_2 \\ \dots \\ \alpha_{n-1}\\ \mu_1} = \mat{\f{-a_1 + a_2(n-1) - a_3 -\dots - a_n}{n} \\ \f{-a_1 - a_2 + a_3(n-1) -\dots - a_n}{n} \\ \dots \\ \f{-a_1 - a_2 - a_3 -\dots + a_n(n-1)}{n} \\ \f{a_1 + a_2+\dots + a_n}{n}}$$
    Получается проекция $v$ на $U$ равна:
    \small{$$\f{-a_1 + a_2(n-1) - a_3 -\dots - a_n}{n} \cdot \mat{-1\\1 \\ \dots \\ 0} + \dots + \f{-a_1 - a_2 - a_3 -\dots + a_n(n-1)}{n} \cdot \mat{-1\\0 \\ \dots \\ 1} $$}
    А проекция $v$ на $W$:
    $$\f{a_1 + a_2+\dots + a_n}{n} \cdot \mat{1\\1\\\dots\\1}$$

    \item[\textbf{№3}] Мы знаем размерность пространства матриц $n \times n$:
    $$\dim{M_n(\RR)} = n^2$$
    Размерность пространства симметрических матриц:
    $$\dim \text{Sym} = \f{n(n+1)}{2}$$
    Размерность пространства строго верхнетреугольных матриц:
    $$\dim N = (n-1) + (n-2) + \dots + 1 = \f{n(n-1)}{2}$$
    Размерность пространства $\text{Sym} \cap N$:
    $$\dim (\text{Sym} \cap N) = 0$$
    Следовательно, выполняется условие:
    $$\case{
        \dim (\text{Sym} \cap N) = 0\\
        \dim(M_n(\RR)) = \dim(\text{Sym}) + \dim(N) \lr n^2 = \f{n(n+1)}{2}+\f{n(n-1)}{2} \text{ - верно}
    }$$
    А значит:
    $$M_n(\RR) = \text{Sym} \oplus N$$
    Найдём проекции матрицы $A$:
    $$A = \mat{0 & 1 & 2 & \ldots & n-1 \\
     -1 & 0 & 1 & \ddots & \vdots \\ 
     -2 & -1 & 0 & \ddots & 2 \\
      \vdots & \ddots & \ddots & \ddots & 1 \\ 
      -n+1 & \ldots & -2 & -1 & 0}$$
    Мы знаем, что:
    $$A = B + C$$ где $B$ - симметрическая матрица, а $C$ - строго верхнетругольная/

    Получается, что:
    $$B = \mat{0 & -1 & -2 & \ldots & -n+1 \\
    -1 & 0 & -1 & \ddots & \vdots \\ 
    -2 & -1 & 0 & \ddots & -2 \\
     \vdots & \ddots & \ddots & \ddots & -1 \\ 
     -n+1 & \ldots & -2 & -1 & 0}, \quad C = \mat{
        0 & 2 & 4 & \ldots & 2n-2 \\
        0 & 0 & 2 & \ddots & \vdots \\ 
        0 & 0 & 0 & \ddots & 4 \\
        \vdots & \ddots & \ddots & \ddots & 2 \\ 
        0 & \ldots & 0 & 0 & 0}$$
    Следовательно, $B$ - проекция на $Sym$ вдоль $N$, а $C$ - проекция на $N$ вдоль $Sym$.

    \item[\textbf{№4}] $U$ - пространство верхнетреугольных матриц
    $$\dim U = \f{n(n+1)}{2}$$
    $N$ - подпространство строго верхнетреугольных матриц
    $$\dim N = \f{n(n-1)}{2}$$
    $S$ - подпространство скалярных матрица
    $$\dim S = 1$$
    $Z$ - подпространство диагональных матриц со следом ноль
    $$\dim Z = n-1$$
    Докажем, что $U = N \oplus S \oplus Z$, это эквивалентно:
    $$\case{
        U = N + S + Z\\
        \dim U = \dim N + \dim S + \dim Z
    }\lr \case{
        \dim U = \dim(N + S + Z)\\
        \dim U = \dim N + \dim S + \dim Z
    } \lr$$
    $$\lr \case{
        \dim U = \dim(N + S + Z)\\
        \dim U = \dim N + \dim S + \dim Z
    } \lr$$$$\lr \case{
        \dim U= \dim (N+S) + \dim Z - \dim ((N+S) \cap Z)\\
        \dim U = \dim N + \dim S + \dim Z
    }$$
    Найдём размерность $N + S$, по сути мы дополняем строго верхнетреугольную матрицу диагональю, задающуюся одним числом, поэтому:
    $$\dim (N + S) =  \f{n(n-1)}{2} + 1$$
    Пересечение таких матриц и матриц $Z$, даёт нулевую матрицу:
    $$\dim ((N+S) \cap Z) = 0$$
    Получается, что:
    $$\dim (N+S) + \dim Z - \dim ((N+S) \cap Z) = $$
    $$=\f{n(n-1)}{2} + 1 + n-1 - 0  = \f{n(n-1)}{2} + n = \f{n(n+1)}{2} = \dim U$$
    А также:
    $$\dim N + \dim S + \dim Z = \f{n(n-1)}{2} + 1 + n-1 = \f{n(n+1)}{2}= \dim U$$
    Значит, выполняется условие:
    $$\case{
        \dim U= \dim (N+S) + \dim Z - \dim ((N+S) \cap Z)\\
        \dim U = \dim N + \dim S + \dim Z
    }$$
    Следовательно:
    $$U = N \oplus S \oplus Z$$
    
\end{enumerate}
\end{document}