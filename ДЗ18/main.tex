\documentclass[a4paper]{article}
\usepackage{setspace}
\usepackage[T2A]{fontenc} %
\usepackage[utf8]{inputenc} % подключение русского языка
\usepackage[russian]{babel} %
\usepackage[12pt]{extsizes}
\usepackage{mathtools}
\usepackage{graphicx}
\usepackage{fancyhdr}
\usepackage{amssymb}
\usepackage{amsmath, amsfonts, amssymb, amsthm, mathtools}
\usepackage{tikz}

\usetikzlibrary{positioning}
\setstretch{1.3}

\newcommand{\mat}[1]{\begin{pmatrix} #1 \end{pmatrix}}
\newcommand{\vmat}[1]{\begin{vmatrix} #1 \end{vmatrix}}
\renewcommand{\f}[2]{\frac{#1}{#2}}
\newcommand{\dspace}{\space\space}
\newcommand{\s}[2]{\sum\limits_{#1}^{#2}}
\newcommand{\mul}[2]{\prod_{#1}^{#2}}
\newcommand{\sq}[1]{\left[ {#1} \right]}
\newcommand{\gath}[1]{\left[ \begin{array}{@{}l@{}} #1 \end{array} \right.}
\newcommand{\case}[1]{\begin{cases} #1 \end{cases}}
\newcommand{\ts}{\text{\space}}
\newcommand{\lm}[1]{\underset{#1}{\lim}}
\newcommand{\suplm}[1]{\underset{#1}{\overline{\lim}}}
\newcommand{\inflm}[1]{\underset{#1}{\underline{\lim}}}
\newcommand{\Ker}[1]{\operatorname{Ker}}

\renewcommand{\phi}{\varphi}
\newcommand{\lr}{\Leftrightarrow}
\renewcommand{\r}{\Rightarrow}
\newcommand{\rr}{\rightarrow}
\renewcommand{\geq}{\geqslant}
\renewcommand{\leq}{\leqslant}
\newcommand{\RR}{\mathbb{R}}
\newcommand{\CC}{\mathbb{C}}
\newcommand{\QQ}{\mathbb{Q}}
\newcommand{\ZZ}{\mathbb{Z}}
\newcommand{\VV}{\mathbb{V}}
\newcommand{\NN}{\mathbb{N}}
\newcommand{\OO}{\underline{O}}
\newcommand{\oo}{\overline{o}}
\renewcommand{\Ker}{\operatorname{Ker}}
\renewcommand{\Im}{\operatorname{Im}}


\DeclarePairedDelimiter\abs{\lvert}{\rvert} %
\makeatletter                               % \abs{}
\let\oldabs\abs                             %
\def\abs{\@ifstar{\oldabs}{\oldabs*}}       %

\begin{document}

\section*{Домашнее задание на 12.02 (Линейная алгебра)}
 {\large Емельянов Владимир, ПМИ гр №247}\\\\
\begin{enumerate}
    \item[\textbf{№1}]
    \begin{enumerate}
        \item[1.1]
        $
        \beta(x, y) = x_{1} y_{1} + 6 x_{2} y_{2} + 12 x_{3} y_{3} 
        - 2 x_{1} y_{2} - 2 x_{2} y_{1} - 3 x_{1} y_{3} - 3 x_{3} y_{1}
        + 5 x_{2} y_{3} + 5 x_{3} y_{2}.
        $

        Матрица билинейной формы:
        $$
        B = 
        \begin{pmatrix}
        1 & -2 & -3 \\
        -2 & 6 & 5 \\
        -3 & 5 & 12
        \end{pmatrix}
        $$
        Найдём диагональный вид:
        $$\begin{pmatrix}
            1 & -2 & -3 & | & 1 &0 & 0\\
            -2 & 6 & 5 &| & 0 & 1& 0\\
            -3 & 5 & 12 & | & 0 & 0 & 1
            \end{pmatrix} \implies \begin{pmatrix}
                1 & -2 & -3 & | & 1 &0 & 0\\
                0 & 2 & -1 &| & 2 & 1& 0\\
                -3 & 5 & 12 & | & 0 & 0 & 1
                \end{pmatrix} $$
        $$
        \implies \begin{pmatrix}
            1 & 0 & -3 & | & 1 &0 & 0\\
            0 & 2 & -1 &| & 2 & 1& 0\\
            -3 & -1 & 12 & | & 0 & 0 & 1
            \end{pmatrix}\implies \begin{pmatrix}
                1 & 0 & -3 & | & 1 &0 & 0\\
                0 & 2 & -1 &| & 2 & 1& 0\\
                0 & -1 & 3 & | & 3 & 0 & 1
                \end{pmatrix}$$
        $$\implies \begin{pmatrix}
            1 & 0 & 0 & | & 1 &0 & 0\\
            0 & \f{1}{2} & -\f{1}{2} &| & 1 & \f{1}{2} & 0\\
            0 & -\f{1}{2} & 3 & | & 3 & 0 & 1
            \end{pmatrix}\implies \begin{pmatrix}
                1 & 0 & 0 & | & 1 &0 & 0\\
                0 & \f{1}{2} & -\f{1}{2} &| & 1 & \f{1}{2} & 0\\
                0 & 0 & \f{5}{2} & | & 4 & \f{1}{2} & 1
                \end{pmatrix}$$
        $$\implies \begin{pmatrix}
            1 & 0 & 0 & | & 1 &0 & 0\\
            0 & \f{1}{2} & 0 &| & 1 & \f{1}{2} & 0\\
            0 & 0 & \f{5}{2} & | & 4 & \f{1}{2} & 1
            \end{pmatrix}$$
        
        Матрица перехода от $e$ к $e'$:
        $$C = \mat{
            1&1&4\\
            0&\f{1}{2}&\f{1}{2}\\
            0&0&1
        }$$
        Следовательно $e'$:
        $$e_1' = e_1, \quad e_2' = e_1+\f{1}{2}e_2, \quad e_3' = 4e_1+ \f{1}{2}e_2 + e_3$$
        
        \item[1.2]$ \beta(x, y) = x_{1} y_{2} + x_{2} y_{1} + x_{1} y_{3} +
         x_{3} y_{1} + x_{2} y_{3} + x_{3} y_{2} $
         
         Матрица билинейной формы:
         $$B = \mat{0 & 1 & 1 \\
         1&0&1\\
         1&1&0}$$
         Найдём диагональный вид:
        $$\mat{0 & 1 & 1 & | & 1 & 0 &0\\
        1&0&1&| & 0 & 1 & 0\\
        1&1&0 & | & 0 & 0 & 1}\implies \mat{\f{1}{2} & 0 & 0 & | & \f{1}{2} & \f{1}{2} &0\\
        0&-\f{1}{2}&0&| & -\f{1}{2} & \f{1}{2} & 0\\
        0&0&-2 & | & -1 & -1 & 1}$$
        Матрица перехода:
        $$C^T = \mat{\f{1}{2} & \f{1}{2} &0\\
       -\f{1}{2} & \f{1}{2} & 0\\
         -1 & -1 & 1}$$
        Новый базис $e'$:
        $$e_1' = \f{1}{2}e_1+\f{1}{2}e_2, \quad e_2' = -\f{1}{2}e_1+\f{1}{2}e_2, 
        \quad e_3' = -e_1-e_2+e_3$$

        \item[1.3]$ \beta(x, y) = x_{1} y_{1} + 9 x_{2} y_{2} +
        4 x_{3} y_{3} + 3 x_{1} y_{2} + 3 x_{2} y_{1} - 2 x_{1} y_{3} -
        2 x_{3} y_{1} + 6 x_{2} y_{3} + 6 x_{3} y_{2} $

        Матрица билинейной формы:
        $$B = \mat{1&3&-2\\
        3&9&6\\
        -2&6&4}$$
        Приведём к диагональному виду:
        $$\mat{
            1&3&-2&|&1&0&0\\
            3&9&6&|&0&1&0\\
            -2&6&4&|&0&0&1
        }\implies\mat{
            1&0&0&|&1&0&0\\
            0&6&0&|&-\f{1}{2}&\f{1}{2}&\f{1}{2}\\
            0&0&-6&|&\f{5}{2}&-\f{1}{2}&\f{1}{2}
        }$$
        Матрица перехода:
        $$C^T =\mat{
            1&0&0\\
            -\f{1}{2}&\f{1}{2}&\f{1}{2}\\
            \f{5}{2}&-\f{1}{2}&\f{1}{2}
        }$$
        Следовательно, новый базис $e'$:
        $$e_1' = e_1, \quad e_2' = -\f{1}{2}e_1 + \f{1}{2}e_2 + \f{1}{2}e_3, \quad
        e_3 = \f{5}{2}e_1 - \f{1}{2}e_2 + \f{1}{2}e_3$$

    \end{enumerate}

    \item[\textbf{№2}]\begin{enumerate}
        \item[2.1]$Q(x) = x_1^2+6x_2^2+12x_3^2-4x_1x_2-6x_1x_3+10x_2x_3$
        
        Матрица квадратичной формы:
        $$\begin{pmatrix}
            1 & -2 & -3 \\
            -2 & 6 & 5 \\
            -3 & 5 & 12
            \end{pmatrix}$$

        Матрица несимметричной билинейной формы:
        $$\begin{pmatrix}
            1 & -1 & -3 \\
            -3 & 6 & 5 \\
            -3 & 5 & 12
            \end{pmatrix}$$

        \item[2.2]$ Q(x) = 2x_1x_2 + 2x_1x_3 + 2x_2x_3 $
        
        Матрица квадратичной формы:
        \[
        \begin{pmatrix}
        0 & 1 & 1 \\
        1 & 0 & 1 \\
        1 & 1 & 0
        \end{pmatrix}
        \]  

        Матрица несимметричной билинейной формы:
        \[
        \begin{pmatrix}
        0 & 2 & 2 \\
        0 & 0 & 2 \\
        0 & 0 & 0
        \end{pmatrix}
        \]

        \item[2.3]\( Q(x) = x_1^2 + 9x_2^2 + 4x_3^2 + 6x_1x_2 - 4x_1x_3 + 12x_2x_3 \)
        
        Матрица квадратичной формы:
        \[
        \begin{pmatrix}
        1 & 3 & -2 \\
        3 & 9 & 6 \\
        -2 & 6 & 4
        \end{pmatrix}
        \] 

        Матрица несимметричной билинейной формы:
        \[
        \begin{pmatrix}
        1 & 4 & -3 \\
        2 & 9 & 8 \\
        -1 & 4 & 4
        \end{pmatrix}
        \]
          
    \end{enumerate}

    \item[\textbf{№3}]$Q(x) = x_1^2-12x_3^2-4x_1x_2+4x_1x_3-24x_2x_3$
    
    Матрица квадратичной формы:
    $$\mat{
        1 & -2 & 2\\
        -2 & 0 & -12\\
        2 & -12 & -12
    }$$
    Приведём её к каноническому виду, симметричным методом гауса:
    $$\mat{
        1 & -2 & 2&| & 1 & 0 & 0\\
        -2 & 0 & -12 & | & 0 & 1 & 0\\
        2 & -12 & -12 & | & 0 & 0 & 1
    } \implies \mat{
        1 & 0 & 0&| & 1 & 0 & 0\\
        0 & -4 & 0 & | & 2 & 1 & 0\\
        0 & 0 & 0 & | & -6 & -2 & 1
    }$$
    Получаем матрицу перехода:
    $$C = \mat{
        1 & 2 & -6\\
        0  & 1 & -2\\
        0 & 0 & 1 
    }$$

    Следовательно, новый базис:
    $$e' = e_1, \quad e_2' = 2e_1+e_2, \quad e_3' = -6e_1-2e_2+e_3$$

    \item[\textbf{№4}]\begin{enumerate}
        \item[1)]$Q(x) = x_1^2+6x_2^2+12x_3^2-4x_1x_2-6x_1x_3+10x_2x_3$
        
        Матрица квадратичной формы:
        $$\begin{pmatrix}
            1 & -2 & -3 \\
            -2 & 6 & 5 \\
            -3 & 5 & 12
            \end{pmatrix}$$
        Приведём к каноническому виду:
        $$\begin{pmatrix}
            1 & -2 & -3 & | & 1 &0 & 0\\
            -2 & 6 & 5 &| & 0 & 1& 0\\
            -3 & 5 & 12 & | & 0 & 0 & 1
            \end{pmatrix} \implies \begin{pmatrix}
                1 & -2 & -3 & | & 1 &0 & 0\\
                0 & 2 & -1 &| & 2 & 1& 0\\
                -3 & 5 & 12 & | & 0 & 0 & 1
                \end{pmatrix} \implies$$
        $$\implies \begin{pmatrix}
            1 & 0 & 0  & |& 1 & 0 & 0 \\
            0 & 1 & 0  & |& \sqrt{2} & \frac{\sqrt{2}}{2} & 0 \\
            0 & 0 & 1  & |& \frac{4\sqrt{10}}{5} & \frac{\sqrt{10}}{10} & 
            \frac{\sqrt{10}}{5}
            \end{pmatrix}$$
        Слева матрица нормального вида, справа траспонированная матрица перехода, новый базис:
        $$e_1' = e_1, \quad
         e_2' = \sqrt{2}e_1+\frac{\sqrt{2}}{2}e_2, \quad
        e_3' =  \frac{4\sqrt{10}}{5}e_1+ \frac{\sqrt{10}}{10}e_2
        +\frac{\sqrt{10}}{5}e_3$$
        Индексы инерции:
        $$\text{положительный:} 3, \quad \text{отрицательный:} 0$$\\

        \item[2)]
        $ Q(x) = 2x_1x_2 + 2x_1x_3 + 2x_2x_3 $
        
        Матрица квадратичной формы:
        \[
        \begin{pmatrix}
        0 & 1 & 1 \\
        1 & 0 & 1 \\
        1 & 1 & 0
        \end{pmatrix}
        \]  
        Приведём её к каноническому виду:
        $$\mat{0 & 1 & 1 & | & 1 & 0 &0\\
        1&0&1&| & 0 & 1 & 0\\
        1&1&0 & | & 0 & 0 & 1}\implies \mat{\f{1}{2} & 0 & 0 & | & \f{1}{2} & \f{1}{2} &0\\
        0&-\f{1}{2}&0&| & -\f{1}{2} & \f{1}{2} & 0\\
        0&0&-2 & | & -1 & -1 & 1}\implies$$
        $$\implies 
        \begin{pmatrix}
            1 & 0 & 0 & \big| & \frac{\sqrt{2}}{2} & \frac{\sqrt{2}}{2} & 0 \\
            0 & -1 & 0 & \big| & -\frac{\sqrt{2}}{2} & \frac{\sqrt{2}}{2} & 0 \\
            0 & 0 & -1 & \big| & -\frac{\sqrt{2}}{2} & -\frac{\sqrt{2}}{2} & 
            \frac{\sqrt{2}}{2}
            \end{pmatrix}$$
        Слева матрица нормального вида, справа траспонированная матрица перехода, новый базис:
        $$e_1' =\frac{\sqrt{2}}{2}e_1+ \frac{\sqrt{2}}{2}e_2, \quad
        e_2' = -\frac{\sqrt{2}}{2}e_1 + \frac{\sqrt{2}}{2}e_2, \quad
        e_3' =  -\frac{\sqrt{2}}{2}e_1 -\frac{\sqrt{2}}{2}e_2 + \frac{\sqrt{2}}{2}e_3$$
        Индексы инерции:
        $$\text{положительный:} 1, \quad \text{отрицательный:} 2$$\\


        \item[3)]\( Q(x) = x_1^2 + 9x_2^2 + 4x_3^2 + 6x_1x_2 - 4x_1x_3 + 12x_2x_3 \)
        
        Матрица квадратичной формы:
        \[
        \begin{pmatrix}
        1 & 3 & -2 \\
        3 & 9 & 6 \\
        -2 & 6 & 4
        \end{pmatrix}
        \] 
        Приведём к каноническому виду:
        $$\mat{
            1&3&-2&|&1&0&0\\
            3&9&6&|&0&1&0\\
            -2&6&4&|&0&0&1
        }\implies\mat{
            1&0&0&|&1&0&0\\
            0&6&0&|&-\f{1}{2}&\f{1}{2}&\f{1}{2}\\
            0&0&-6&|&\f{5}{2}&-\f{1}{2}&\f{1}{2}
        }$$
        $$\implies \begin{pmatrix}
            1 & 0 & 0 & \big| & 1 & 0 & 0 \\
            0 & 1 & 0 & \big| & -\frac{\sqrt{6}}{12} & \frac{\sqrt{6}}{12} & 
            \frac{\sqrt{6}}{12} \\
            0 & 0 & -1 & \big| & \frac{5\sqrt{6}}{12} & -\frac{\sqrt{6}}{12} &
             \frac{\sqrt{6}}{12}
            \end{pmatrix}$$
        Слева матрица нормального вида, справа траспонированная матрица перехода, новый базис:
        $$e_1' = e_1, \quad
        e_2' =-\frac{\sqrt{6}}{12} e_1+\frac{\sqrt{6}}{12}e_2 + \frac{\sqrt{6}}{12}e_3, \quad
        e_3' = \frac{5\sqrt{6}}{12}e_1-\frac{\sqrt{6}}{12}e_2+\frac{\sqrt{6}}{12}e_3$$
        Индексы инерции:
        $$\text{положительный:} 2, \quad \text{отрицательный:} 1$$
        
        \item[4)]  $Q(x) = x_1^2-12x_3^2-4x_1x_2+4x_1x_3-24x_2x_3$
        $$\mat{
            1 & -2 & 2&| & 1 & 0 & 0\\
            -2 & 0 & -12 & | & 0 & 1 & 0\\
            2 & -12 & -12 & | & 0 & 0 & 1
        } \implies \mat{
            1 & 0 & 0&| & 1 & 0 & 0\\
            0 & -4 & 0 & | & 2 & 1 & 0\\
            0 & 0 & 0 & | & -6 & -2 & 1
        }\implies $$
        $$\implies\mat{
            1 & 0 & 0&| & 1 & 0 & 0\\
            0 & -1 & 0 & | & 1 & \f{1}{2} & 0\\
            0 & 0 & 0 & | & -6 & -2 & 1
        }$$
        Слева матрица нормального вида, справа траспонированная матрица перехода, новый базис:
        $$e_1' = e_1, \quad
        e_2' = e_1+\f{1}{2}e_2, \quad
         e_3' = -6e_1-2e_2+e_3$$
        Индексы инерции:
        $$\text{положительный:} 1, \quad \text{отрицательный:} 1$$
    \end{enumerate}
    
\end{enumerate}
\end{document}