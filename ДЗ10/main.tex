\documentclass[a4paper]{article}
\usepackage{setspace}
\usepackage[T2A]{fontenc} %
\usepackage[utf8]{inputenc} % подключение русского языка
\usepackage[russian]{babel} %
\usepackage[12pt]{extsizes}
\usepackage{mathtools}
\usepackage{graphicx}
\usepackage{fancyhdr}
\usepackage{amssymb}
\usepackage{amsmath, amsfonts, amssymb, amsthm, mathtools}
\usepackage{tikz}

\usetikzlibrary{positioning}
\setstretch{1.3}

\newcommand{\mat}[1]{\begin{pmatrix} #1 \end{pmatrix}}
\renewcommand{\det}[1]{\begin{vmatrix} #1 \end{vmatrix}}
\renewcommand{\f}[2]{\frac{#1}{#2}}
\newcommand{\dspace}{\space\space}
\newcommand{\s}[2]{\sum\limits_{#1}^{#2}}
\newcommand{\mul}[2]{\prod_{#1}^{#2}}
\newcommand{\sq}[1]{\left[ {#1} \right]}
\newcommand{\gath}[1]{\left[ \begin{array}{@{}l@{}} #1 \end{array} \right.}
\newcommand{\case}[1]{\begin{cases} #1 \end{cases}}
\newcommand{\ts}{\text{\space}}
\newcommand{\lm}[1]{\underset{#1}{\lim}}
\newcommand{\suplm}[1]{\underset{#1}{\overline{\lim}}}
\newcommand{\inflm}[1]{\underset{#1}{\underline{\lim}}}

\renewcommand{\phi}{\varphi}
\newcommand{\lr}{\Leftrightarrow}
\renewcommand{\r}{\Rightarrow}
\newcommand{\rr}{\rightarrow}
\renewcommand{\geq}{\geqslant}
\renewcommand{\leq}{\leqslant}
\newcommand{\RR}{\mathbb{R}}
\newcommand{\CC}{\mathbb{C}}
\newcommand{\QQ}{\mathbb{Q}}
\newcommand{\ZZ}{\mathbb{Z}}
\newcommand{\VV}{\mathbb{V}}
\newcommand{\NN}{\mathbb{N}}
\newcommand{\OO}{\underline{O}}
\newcommand{\oo}{\overline{o}}


\DeclarePairedDelimiter\abs{\lvert}{\rvert} %
\makeatletter                               % \abs{}
\let\oldabs\abs                             %
\def\abs{\@ifstar{\oldabs}{\oldabs*}}       %

\begin{document}

\section*{Домашнее задание на 27.11 (Линейная алгебра)}
 {\large Емельянов Владимир, ПМИ гр №247}\\\\
\begin{enumerate}
    \item[\textbf{1.}]
    \begin{enumerate}
        \item[1.1.] Обозначим:
        $$
        f(x) = x^{10}
        $$
        и рассмотрим линейную комбинацию:
        $$
        c_0 \cdot 1 + c_1 (x-1) + c_2 (x-1)^{2} + \ldots + c_{10} (x-1)^{10} = f(x)
        $$
        $x^{10}$ представим в виде этой линейной комбинации, так как 
        $$x^{10} = \sum_{k=0}^{10} \binom{10}{k} (x-1)^k$$
        То есть при $c_i = \binom{10}{i}$ \\
        Таким образом, функция $ x^{10} $ представима в виде линейной комбинации функций $ 1, x-1, (x-1)^{2}, \ldots, (x-1)^{10} $.
        
        \item[1.2.]
        Линейная оболочка этих функций состоит из всех возможных линейных комбинаций:
        $$
        c_0 \cdot 1 + c_1 \sin x + c_2 \cos x + c_3 \sin 2x + c_4 \cos 2x
        $$
        где $ c_0, c_1, c_2, c_3, c_4 $ - некоторые коэффициенты.\\
        Функция $ x $ является многочленом первой степени, а функции $ \sin x $ и $ \cos x $ являются тригонометрическими функциями, которые не могут быть представлены в виде многочлена. Линейная комбинация указанных функций не может дать результат в виде многочлена, так как они не содержат членов с целыми степенями.
        
        Таким образом, функция $ x $ не может быть представлена в виде линейной комбинации указанных функций.
        
    \end{enumerate}

    \item[\textbf{2.}]
    Мы ищем такие коэффициенты $\lambda_1, \lambda_2, \lambda_3$, что:
    $$
    \lambda_1 (5, 4, 3, 0)^T + \lambda_2 (3, 3, 2, 1)^T + \lambda_3 (8, 1, 3, -9)^T = (0, 0, 0, 0)^T
    $$
    
    Это приводит к следующей системе уравнений:
    $$
    \case{
    5\lambda_1 + 3\lambda_2 + 8\lambda_3 &= 0 \quad \text{(1)} \\
    4\lambda_1 + 3\lambda_2 + 1\lambda_3 &= 0 \quad \text{(2)} \\
    3\lambda_1 + 2\lambda_2 + 3\lambda_3 &= 0 \quad \text{(3)} \\
    0\lambda_1 + 1\lambda_2 - 9\lambda_3 &= 0 \quad \text{(4)}
    }
    $$
    
    Решим систему уравнений, начиная с уравнения (4):
    $$
    \lambda_2 - 9\lambda_3 = 0 \implies \lambda_2 = 9\lambda_3
    $$

    Теперь подставим $\lambda_2$ в уравнения (1), (2) и (3).

    Подстановка в уравнение (1):
    $$
    5\lambda_1 + 3(9\lambda_3) + 8\lambda_3 = 0 \implies 5\lambda_1 + 27\lambda_3 + 8\lambda_3 = 0 \implies 5\lambda_1 + 35\lambda_3 = 0
    $$
    $$
    \implies \lambda_1 = -7\lambda_3 \quad \text{(5)}
    $$

    Подстановка в уравнение (2):
    $$
    4\lambda_1 + 3(9\lambda_3) + 1\lambda_3 = 0 \implies 4\lambda_1 + 27\lambda_3 + \lambda_3 = 0 \implies 4\lambda_1 + 28\lambda_3 = 0
    $$
    Подставим $\lambda_1$ из (5):
    $$
    4(-7\lambda_3) + 28\lambda_3 = 0 \implies -28\lambda_3 + 28\lambda_3 = 0
    $$
    Это уравнение всегда верно.

    Подстановка в уравнение (3):
    $$
    3\lambda_1 + 2(9\lambda_3) + 3\lambda_3 = 0 \implies 3\lambda_1 + 18\lambda_3 + 3\lambda_3 = 0 \implies 3\lambda_1 + 21\lambda_3 = 0
    $$
    Подставим $\lambda_1$ из (5):
    $$
    3(-7\lambda_3) + 21\lambda_3 = 0 \implies -21\lambda_3 + 21\lambda_3 = 0
    $$
    Это уравнение также всегда верно.

    Мы получили, что:
    $$
    \lambda_1 = -7\lambda_3, \quad \lambda_2 = 9\lambda_3
    $$
    где $\lambda_3$ может быть любым ненулевым числом. Это означает, что существует нетривиальная линейная комбинация векторов, равная нулевому вектору.

    Следовательно, векторы $(5,4,3,0)^{T}, (3,3,2,1)^{T}, (8,1,3,-9)^{T}$ являются \textbf{линейно зависимыми}.
    
    \item[\textbf{3.}]Чтобы доказать, что столбец $ b $ представляется в виде линейной комбинации столбцов $ a_{1}, \ldots, a_{k} $, воспользуемся определениями линейной независимости и линейной зависимости.
    \begin{enumerate}
        \item[1)]
        Столбцы $ a_{1}, \ldots, a_{k} $ линейно независимы, что означает, что единственное решение уравнения
        $$
        \lambda_1 a_1 + \lambda_2 a_2 + \ldots + \lambda_k a_k = 0
        $$
        — это тривиальное решение, то есть $ \lambda_1 = \lambda_2 = \ldots = \lambda_k = 0 $.\\
        \item[2)]
        Столбцы $ a_{1}, \ldots, a_{k}, b $ линейно зависимы, что означает, что существует нетривиальная линейная комбинация, равная нулевому вектору:
        $$
        \mu_1 a_1 + \mu_2 a_2 + \ldots + \mu_k a_k + \mu_{k+1} b = 0
        $$
        где не все коэффициенты $ \mu_1, \mu_2, \ldots, \mu_{k+1} $ равны нулю.\\
    \end{enumerate}


    Поскольку $ a_{1}, \ldots, a_{k} $ линейно независимы, если $ \mu_{k+1} \neq 0 $, мы можем выразить $ b $ через $ a_{1}, \ldots, a_{k} $:
    $$
    b = -\frac{1}{\mu_{k+1}} (\mu_1 a_1 + \mu_2 a_2 + \ldots + \mu_k a_k)
    $$
    Это означает, что $ b $ может быть представлен как линейная комбинация столбцов $ a_{1}, \ldots, a_{k} $.

    Если $ \mu_{k+1} = 0 $, то уравнение
    $$
    \mu_1 a_1 + \mu_2 a_2 + \ldots + \mu_k a_k = 0
    $$
    должно иметь нетривиальное решение, что противоречит предположению о линейной независимости $ a_{1}, \ldots, a_{k} $.

    
    Следовательно, в любом случае, столбец $ b $ \underbar{может быть представлен} \underbar{в виде линейной комбинации столбцов} $ a_{1}, \ldots, a_{k} $. \\

    \item[\textbf{4.}]Чтобы выяснить, являются ли векторы $ b_{1}, b_{2}, b_{3} $ линейно независимыми, мы можем выразить их через векторы $ a_{1}, a_{2}, a_{3}, a_{4} $ и проверить, существует ли нетривиальная линейная комбинация, равная нулевому вектору.

    Векторы $ b_{1}, b_{2}, b_{3} $ записаны как:
    $$
    \case{
    b_{1} &= 3 a_{1} + 2 a_{2} + a_{3} + a_{4}, \\
    b_{2} &= 2 a_{1} + 5 a_{2} + 3 a_{3} + 2 a_{4}, \\
    b_{3} &= 3 a_{1} + 4 a_{2} + 2 a_{3} + 3 a_{4}.
    }
    $$

    Линейная комбинация векторов $ b_{1}, b_{2}, b_{3} $

    Рассмотрим линейную комбинацию векторов $ b_{1}, b_{2}, b_{3} $:
    $$
    \lambda_1 b_{1} + \lambda_2 b_{2} + \lambda_3 b_{3} = 0.
    $$
    Подставим выражения для $ b_{1}, b_{2}, b_{3} $:
    $$
    \lambda_1 (3 a_{1} + 2 a_{2} + a_{3} + a_{4}) + \lambda_2 (2 a_{1} + 5 a_{2} + 3 a_{3} + 2 a_{4}) + \lambda_3 (3 a_{1} + 4 a_{2} + 2 a_{3} + 3 a_{4}) = 0.
    $$
    Соберем все коэффициенты при $ a_{1}, a_{2}, a_{3}, a_{4} $:
    $$
    (3\lambda_1 + 2\lambda_2 + 3\lambda_3) a_{1} + (2\lambda_1 + 5\lambda_2 + 4\lambda_3) a_{2} + (\lambda_1 + 3\lambda_2 + 2\lambda_3) a_{3} + (\lambda_1 + 2\lambda_2 + 3\lambda_3) a_{4} = 0.
    $$
    Это приводит к следующей системе уравнений:
    $$
    \case{
    3\lambda_1 + 2\lambda_2 + 3\lambda_3 &= 0 \quad \text{(1)} \\
    2\lambda_1 + 5\lambda_2 + 4\lambda_3 &= 0 \quad \text{(2)} \\
    \lambda_1 + 3\lambda_2 + 2\lambda_3 &= 0 \quad \text{(3)} \\
    \lambda_1 + 2\lambda_2 + 3\lambda_3 &= 0 \quad \text{(4)}
    }
    $$
    Запишем в виде матрицы СЛУ:
    $$
    R = \begin{pmatrix}
    3 & 2 & 3 & | & 0 \\
    2 & 5 & 4 & | & 0 \\
    1 & 3 & 2 & | & 0 \\
    1 & 2 & 3 & | & 0
    \end{pmatrix}
    $$
    $$
    R_1 \rightarrow \frac{1}{3} R_1 \implies \begin{pmatrix}
    1 & \frac{2}{3} & 1 & | & 0 \\
    2 & 5 & 4 & | & 0 \\
    1 & 3 & 2 & | & 0 \\
    1 & 2 & 3 & | & 0
    \end{pmatrix}\; \; \;
    R_2 \rightarrow R_2 - 2R_1 \implies \begin{pmatrix}
    1 & \frac{2}{3} & 1 & | & 0 \\
    0 & \frac{11}{3} & 2 & | & 0 \\
    1 & 3 & 2 & | & 0 \\
    1 & 2 & 3 & | & 0
    \end{pmatrix}
    $$
    $$
    R_3 \rightarrow R_3 - R_1 \implies \begin{pmatrix}
    1 & \frac{2}{3} & 1 & | & 0 \\
    0 & \frac{11}{3} & 2 & | & 0 \\
    0 & \frac{7}{3} & 1 & | & 0 \\
    1 & 2 & 3 & | & 0
    \end{pmatrix} \; \; \;
    R_4 \rightarrow R_4 - R_1 \implies \begin{pmatrix}
    1 & \frac{2}{3} & 1 & | & 0 \\
    0 & \frac{11}{3} & 2 & | & 0 \\
    0 & \frac{7}{3} & 1 & | & 0 \\
    0 & \frac{4}{3} & 2 & | & 0
    \end{pmatrix}
    $$
    $$
    R_2 \rightarrow \frac{3}{11} R_2 \implies \begin{pmatrix}
    1 & \frac{2}{3} & 1 & | & 0 \\
    0 & 1 & \frac{6}{11} & | & 0 \\
    0 & \frac{7}{3} & 1 & | & 0 \\
    0 & \frac{4}{3} & 2 & | & 0
    \end{pmatrix}
    \; \; \;
    R_3 \rightarrow R_3 - \frac{7}{3}R_2 \implies \begin{pmatrix}
    1 & \frac{2}{3} & 1 & | & 0 \\
    0 & 1 & \frac{6}{11} & | & 0 \\
    0 & 0 & -\frac{3}{11} & | & 0 \\
    0 & \frac{4}{3} & 2 & | & 0
    \end{pmatrix}
    $$
    $$
    R_4 \rightarrow R_4 - \frac{4}{3}R_2 \implies \begin{pmatrix}
    1 & \frac{2}{3} & 1 & | & 0 \\
    0 & 1 & \frac{6}{11} & | & 0 \\
    0 & 0 & -\frac{3}{11} & | & 0 \\
    0 & 0 & \frac{14}{11} & | & 0
    \end{pmatrix}\; \; \;
    R_3 \rightarrow -\frac{11}{3} R_3 \implies \begin{pmatrix}
    1 & \frac{2}{3} & 1 & | & 0 \\
    0 & 1 & \frac{6}{11} & | & 0 \\
    0 & 0 & 1 & | & 0 \\
    0 & 0 & \frac{14}{11} & | & 0
    \end{pmatrix}
    $$
    $$
    R_4 \rightarrow R_4 - \frac{14}{11}R_3 \implies \begin{pmatrix}
    1 & \frac{2}{3} & 1 & | & 0 \\
    0 & 1 & \frac{6}{11} & | & 0 \\
    0 & 0 & 1 & | & 0 \\
    0 & 0 & 0 & | & 0
    \end{pmatrix}
    $$
        
    Теперь у нас есть система уравнений в верхнем треугольном виде:
    $$
    \case{
    \lambda_1 + \frac{2}{3}\lambda_2 + \lambda_3 &= 0 \quad \text{(1)} \\
    \lambda_2 + \frac{6}{11}\lambda_3 &= 0 \quad \text{(2)} \\
    \lambda_3 &= 0 \quad \text{(3)}
    }
    $$
    
    Подставим $\lambda_3 = 0$ в (2):
    $$
    \lambda_2 + \frac{6}{11} \cdot 0 = 0 \implies \lambda_2 = 0.
    $$

    Теперь подставим $\lambda_2 = 0$ в (1):
    $$
    \lambda_1 + \frac{2}{3} \cdot 0 + 0 = 0 \implies \lambda_1 = 0.
    $$
        
    Таким образом, единственное решение для системы уравнений:
    $$
    \lambda_1 = 0, \quad \lambda_2 = 0, \quad \lambda_3 = 0.
    $$
    Это означает, что векторы $ b_{1}, b_{2}, b_{3} $ являются \underbar{линейно независимыми}.\\

    \item[\textbf{5}] Функции $ f_1, f_2, \ldots, f_n $ линейно независимы, если уравнение 

    $$
    c_1 f_1 + c_2 f_2 + \ldots + c_n f_n = 0
    $$
    
    выполняется только при $ c_1 = c_2 = \ldots = c_n = 0 $.

    \begin{enumerate}
        \item[5.1.]Система функций: $ \sin 2x, \cos x, \sin x $

        Рассмотрим линейную комбинацию:
        $$
        c_1 \sin 2x + c_2 \cos x + c_3 \sin x = 0
        $$      
        \begin{enumerate}
            \item[1)]
            Подставим $ x = 0 $:
            $$
            c_1 \sin(0) + c_2 \cos(0) + c_3 \sin(0) = 0
            $$
            Это упрощается до:
            $$
            0 + c_2 \cdot 1 + 0 = 0 \implies c_2 = 0
            $$

            \item[2)]
            Теперь подставим $ x = \frac{\pi}{2} $:
            $$
            c_1 \sin\left(2 \cdot \frac{\pi}{2}\right) + c_2 \cos\left(\frac{\pi}{2}\right) + c_3 \sin\left(\frac{\pi}{2}\right) = 0
            $$
            Это упрощается до:
            $$
            c_1 \sin(\pi) + 0 + c_3 \cdot 1 = 0 \implies 0 + c_3 = 0 \implies c_3 = 0
            $$

            \item[3)]
            Теперь подставим $ x = \frac{\pi}{4} $:
            $$
            c_1 \sin\left(2 \cdot \frac{\pi}{4}\right) + c_2 \cos\left(\frac{\pi}{4}\right) + c_3 \sin\left(\frac{\pi}{4}\right) = 0
            $$
            Это упрощается до:
            $$
            c_1 \sin\left(\frac{\pi}{2}\right) + c_2 \cdot \frac{1}{\sqrt{2}} + c_3 \cdot \frac{1}{\sqrt{2}} = 0
            $$
            Подставляя $ c_2 = 0 $ и $ c_3 = 0 $:
            $$
            c_1 \cdot 1 + 0 + 0 = 0 \implies c_1 = 0
            $$
        \end{enumerate}
        
        Мы получили, что $ c_1 = c_2 = c_3 = 0 $. Следовательно, функции $ \sin 2x, \cos x, \sin x $ линейно независимы.\\

        \item[5.2]Линейная независимость функций $ 1, \cos x, \cos^2 x, \cos^3 x, \ldots, \cos^n x $

        Рассмотрим линейную комбинацию:
        $$
        c_0 \cdot 1 + c_1 \cos x + c_2 \cos^2 x + \ldots + c_n \cos^n x = 0
        $$
        для всех $ x $. Мы хотим показать, что все коэффициенты $ c_0, c_1, \ldots, c_n $ равны нулю.

        Возьмем производную от обеих сторон уравнения:
        $$
        \frac{d}{dx} \left( c_0 \cdot 1 + c_1 \cos x + c_2 \cos^2 x + \ldots + c_n \cos^n x \right) = 0
        $$
        Получаем:
        $$
        0 + c_1 (-\sin x) + c_2 \cdot 2 \cos x (-\sin x) + \ldots + c_n \cdot n \cos^{n-1} x (-\sin x) = 0
        $$
        $$
        -\sin x \left( c_1 + 2c_2 \cos x + 3c_3 \cos^2 x + \ldots + n c_n \cos^{n-1} x \right) = 0
        $$
        Поскольку это уравнение должно выполняться для всех $ x $, включая те, для которых $ \sin x \neq 0 $ (например, $ x = \frac{\pi}{4} $), мы можем заключить, что:
        $$
        c_1 + 2c_2 \cos x + 3c_3 \cos^2 x + \ldots + n c_n \cos^{n-1} x = 0
        $$
        Теперь мы можем снова взять производную от этого уравнения:
        $$
        0 + 2c_2 (-\sin x) + 3c_3 \cdot 2 \cos x (-\sin x) + \ldots + n c_n (n-1) \cos^{n-2} x (-\sin x) = 0
        $$
        Это уравнение также должно выполняться для всех $ x $. Мы можем продолжать этот процесс, беря производные, пока не получим уравнение, в котором останется только один коэффициент $ c_n $.
        Из этого уравнения получим, что $c_n = 0$. То же самое проделаем и для $c_{n-1}, \dots, c_{0}$ 

        Таким образом, единственное решение для всех коэффициентов:
        $$
        c_0 = c_1 = c_2 = \ldots = c_n = 0
        $$

        Следовательно, функции $ 1, \cos x, \cos^2 x, \ldots, \cos^n x $ являются линейно независимыми.\\
    \end{enumerate}

    \item[\textbf{6.}]Чтобы определить, являются ли указанные наборы векторов базисом векторного пространства $ V = \mathbb{R}[x]_{\leqslant 2} $, необходимо проверить два условия:

    1. Линейная независимость векторов.\\
    2. Линейная оболочка векторов должна совпадать с $ V $.
    
    Пространство $ V $ состоит из многочленов степени не выше 2, то есть имеет размерность 3. Базисом этого пространства могут быть три линейно независимых вектора.

    \begin{enumerate}
        \item[6.1.]
        \textbf{Набор 1:} $ 1, x, x^2 $:
        
        1. Линейная независимость:\\
        Рассмотрим линейную комбинацию:
        $$
        a_0 \cdot 1 + a_1 \cdot x + a_2 \cdot x^2 = 0
        $$
        Это уравнение выполняется только если $ a_0 = a_1 = a_2 = 0 $. Следовательно, векторы линейно независимы.

        2. Линейная оболочка: \\
        Любой многочлен степени не выше 2 можно представить в виде $ a_0 + a_1 x + a_2 x^2 $, где $ a_0, a_1, a_2 \in \mathbb{R} $. Таким образом, линейная оболочка этого набора равна всему пространству $ V $.

        Следовательно, набор $ 1, x, x^2 $ является базисом.\\

        
        \textbf{Набор 2:} $ x^2 + x, 3x^2 - 2x $

        1. Линейная независимость:\\
        Рассмотрим линейную комбинацию:
        $$
        a_1 (x^2 + x) + a_2 (3x^2 - 2x) = 0
        $$
        Это можно переписать как:
        $$
        (a_1 + 3a_2)x^2 + (a_1 - 2a_2)x = 0
        $$
        Для равенства нулю необходимо, чтобы коэффициенты при $ x^2 $ и $ x $ были равны нулю:
        $$
        a_1 + 3a_2 = 0 \quad \text{и} \quad a_1 - 2a_2 = 0
        $$
        Решая эту систему, получаем $ a_1 = 0 $ и $ a_2 = 0 $. Следовательно, векторы линейно независимы.

        2. Линейная оболочка: \\
        Однако, оба вектора имеют степень 2, и их линейная комбинация не может дать все многочлены степени 0 и 1. Таким образом, линейная оболочка этого набора не равна всему пространству $ V $.

        Следовательно, набор $ x^2 + x, 3x^2 - 2x $ не является базисом.\\

                
        \textbf{Набор 3:} $ 1, x, x^2, 1 + x + x^2 $

        1. Линейная независимость: \\
        Рассмотрим линейную комбинацию:
        $$
        a_0 \cdot 1 + a_1 \cdot x + a_2 \cdot x^2 + a_3 (1 + x + x^2) = 0
        $$
        Это можно переписать как:
        $$
        (a_0 + a_3) + (a_1 + a_3)x + (a_2 + a_3)x^2 = 0
        $$
        Для равенства нулю необходимо, чтобы все коэффициенты были равны нулю:
        $$
        a_0 + a_3 = 0, \quad a_1 + a_3 = 0, \quad a_2 + a_3 = 0
        $$
        Это приводит к $ a_0 = a_1 = a_2 = a_3 = 0 $. Следовательно, векторы линейно независимы.

        2. Линейная оболочка: \\
        Набор состоит из 4 векторов, но размерность пространства $ V $ равна 3. Поэтому, несмотря на линейную независимость, этот набор не может быть базисом.

        Следовательно, набор $ 1, x, x^2, 1 + x + x^2 $ не является базисом.\\

        \item[6.2.]
                
        \textbf{Набор 1:} $ 1, x + 1, x^2 + 1 $

        1. Линейная независимость:\\ 
        Рассмотрим линейную комбинацию:
        $$
        a_0 \cdot 1 + a_1 (x + 1) + a_2 (x^2 + 1) = 0
        $$
        Это можно переписать как:
        $$
        (a_0 + a_1 + a_2) + a_1 x + a_2 x^2 = 0
        $$
        Для равенства нулю необходимо, чтобы все коэффициенты были равны нулю:
        $$
        a_0 + a_1 + a_2 = 0, \quad a_1 = 0, \quad a_2 = 0
        $$
        Это приводит к $ a_0 = 0 $. Следовательно, векторы линейно независимы.

        2. Линейная оболочка: \\
        Любой многочлен степени не выше 2 можно выразить через этот набор. Таким образом, линейная оболочка этого набора равна всему пространству $ V $.

        Следовательно, набор $ 1, x + 1, x^2 + 1 $ является базисом.\\

        
        \textbf{Набор 2:} $ 1 + x, 2 + 3x, 4 + 5x $

        1. Линейная независимость: \\
        Рассмотрим линейную комбинацию:
        $$
        a_1 (1 + x) + a_2 (2 + 3x) + a_3 (4 + 5x) = 0
        $$
        Это можно переписать как:
        $$
        (a_1 + 2a_2 + 4a_3) + (a_1 + 3a_2 + 5a_3)x = 0
        $$
        Для равенства нулю необходимо, чтобы все коэффициенты были равны нулю:
        $$
        a_1 + 2a_2 + 4a_3 = 0, \quad a_1 + 3a_2 + 5a_3 = 0
        $$
        Решая эту систему, мы можем выразить $ a_1 $ через $ a_2 $ и $ a_3 $, что указывает на линейную зависимость.

        Следовательно, набор $ 1 + x, 2 + 3x, 4 + 5x $ не является базисом.

    \end{enumerate}

\end{enumerate}
\end{document}