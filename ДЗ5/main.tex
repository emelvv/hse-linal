\documentclass[a4paper]{article}
\usepackage{setspace}
\usepackage[T2A]{fontenc} %
\usepackage[utf8]{inputenc} % подключение русского языка
\usepackage[russian]{babel} %
\usepackage[14pt]{extsizes}
\usepackage{mathtools}
\usepackage{graphicx}
\usepackage{fancyhdr}
\usepackage{amssymb}
\usepackage{amsmath, amsfonts, amssymb, amsthm, mathtools}
\usepackage{tikz}

\usetikzlibrary{positioning}
\setstretch{1.3}

\newcommand{\mat}[1]{\begin{pmatrix} #1 \end{pmatrix}}
\renewcommand{\f}[2]{\frac{#1}{#2}}
\newcommand{\dspace}{\space\space}
\newcommand{\s}[2]{\sum\limits_{#1}^{#2}}
\newcommand{\sq}[1]{\left[ {#1} \right]}
\newcommand{\gath}[1]{\left[ \begin{array}{@{}l@{}} #1 \end{array} \right.}
\newcommand{\case}[1]{\begin{cases} #1 \end{cases}}
\newcommand{\ts}{\text{\space}}

\newcommand{\lr}{\Leftrightarrow}
\renewcommand{\r}{\Rightarrow}
\newcommand{\rr}{\rightarrow}
\renewcommand{\geq}{\geqslant}
\renewcommand{\leq}{\leqslant}
\newcommand{\RR}{\mathbb{R}}
\newcommand{\CC}{\mathbb{C}}
\newcommand{\QQ}{\mathbb{Q}}
\newcommand{\ZZ}{\mathbb{Z}}
\newcommand{\VV}{\mathbb{V}}
\newcommand{\NN}{\mathbb{N}}

\DeclarePairedDelimiter\abs{\lvert}{\rvert} %
\makeatletter                               % \abs{}
\let\oldabs\abs                             %
\def\abs{\@ifstar{\oldabs}{\oldabs*}}       %

\begin{document}

\section*{Домашняя работа №5 (Линейная алгебра)}
 {\large Емельянов Владимир, ПМИ гр №247}\\\\
\begin{enumerate}
      \item[\textbf{1.}] \indent \\
            \begin{enumerate}
                  \item[1.1.] $$\mat{1 & 2 & 3 & 4 & 5 & 6 \\ 4 & 2 & 6 & 3 & 5 & 1} $$
                        $$\text{Инверсии: } 3 + 1 + 3 + 1 + 1 = 9$$
                        $$sgn = (-1)^9 = -1$$

                  \item[1.2.] {\small $\begin{pmatrix}1 & 2 & 3 \; \ldots \; n & n+1 & n+2 & n+3 \; \ldots \;\; 2n \quad & 2n+1 & 2n+2 & 2n+3 \; \ldots \;\; 3n \;\; \\ 3 & 6 & 9 \; \ldots \; 3n & 2 & 5 & \; 8 \quad \ldots \quad 3n-1 &  1 & 4 & \; 7 \quad \ldots \quad 3n-2\end{pmatrix}$}
                        $$\text{Инверсии: } 2 + 4 + 6 +\dots+2n + 1+2+3 + \dots + n + 0 + 0 + \dots = $$
                        $$ = n\f{2+2n}{2}+n\f{1+n}{2} = \f{3n^2+3n}{2}$$
                        $$sgn = (-1)^{\f{3n^2+3n}{2}}$$
            \end{enumerate}

      \item[\textbf{2.}]\indent \\
            \begin{itemize}
                  \item $\mat{1 & 2 & 3 & 4 & 5 & 6 & 7 & 8 & 9 \\ 5 & 8 & 9 & 2 & 1 & 4 & 3 & 6 & 7}
                              = (1 \; 5)(2 \; 8 \; 6 \; 4)(3 \; 9 \; 7)$\\
                        $(1 \; 5 \; 3)(2 \; 4) = \mat{1 & 2 & 3 & 4 & 5 \\ 5 & 4 & 1 & 2 & 3}$

                  \item $\sigma = (1 \; 4)(3 \; 6 \; 5) \in S_6 = \mat{1 & 2 & 3 & 4 & 5 & 6\\ 4 & 2 & 6 & 1 & 3 & 5}$\\
                        $\r \sigma^{-1} = (4 \; 1)(5 \; 6 \; 3) = \mat{1 & 2 & 3 & 4 & 5 & 6 \\ 4 & 2 & 5 & 1 & 6 & 3}$
            \end{itemize}
      \item[\textbf{3.}]\indent \\
            $\sigma = \begin{pmatrix}1 & 2 & 3 & 4 & 5\\2 & 3 & 1 & 5 & 4\end{pmatrix} $ и $\tau = \begin{pmatrix}1 & 2 & 3 & 4 & 5\\2 & 1 & 4 & 3 & 5\end{pmatrix}$\\
            $\sigma \tau = \mat{1 & 2 & 3 & 4 & 5 \\ 3 & 2 & 5 & 1 & 4} = (1 \; 3 \; 5\; 4)$\\
            $\tau \sigma = \mat{1 & 2 & 3 & 4 & 5 \\ 1 & 4 & 2 & 5 & 3} = (2 \; 4 \; 5 \; 3)$

      \item[\textbf{4.}]\indent \\
            $\sigma = \begin{pmatrix}1 & 2 & 3 & 4 & 5 \\ 3 & 5 & 1 & 2 & 4\end{pmatrix} = (1 \; 3)(2 \; 5 \; 4)$\\
            $\sigma ^ N = id \r min(N) = \text{НОК}(2, 3) = 6$\\
            $\sigma^{36} = id$\\
            $\sigma^{37} = \sigma = (1 \; 3)(2 \; 5 \; 4)$\\
            $\sigma^{38} = \sigma^{2} = (2 \; 4 \; 5)$\\
            $\sigma^{-1} = (3 \; 1)(4 \; 5 \; 2)$\\
            $\sigma^{35} = \sigma^{-1} = (3 \; 1)(4 \; 5 \; 2)$

      \item[\textbf{5.}]\indent \\
            $\sigma = \sigma = \begin{pmatrix}1 & 2 & 3 & 4 & 5 & 6 & 7 & 8 & 9 & 10 \;\;\; 11\\ 8 & 10 & 6 & 7 & 5 & 9 & 3 & 2 & 11 & 1 \;\;\;\; 4\end{pmatrix}=(1 \; 8 \; 2 \; 10)(3 \; 6 \; 9 \; 11 \; 4 \; 7)$\\
            $\sigma^{36} = id$\\
            $\sigma^{37} = \sigma = (1 \; 8 \; 2 \; 10)(3 \; 6 \; 9 \; 11 \; 4 \; 7)$\\
            $\sigma^{5} = (1 \; 8 \; 2 \; 10)(3\; 7 \; 4 \; 11 \; 9 \; 6)$\\

      \item[\textbf{6.}]\indent \\
            $(a_1 \; a_2)(a_2 \; a_3)\dots(a_{k-2} \; a_{k-1})(a_{k-1} \; a_{k})$\\
            $(a_{k-1} \; a_{k}) = \mat{a_1 & a_2 & a_3 & a_4 & \dots & a_{k-2} & a_{k-1} & a_{k} \\ a_1 & a_2 & a_3 & a_4 & \dots & a_{k-2} & a_{k} & a_{k-1}}$\\
            $(a_{k-2} \; a_{k-1}) = \mat{a_1 & a_2 & a_3 & a_4 & \dots & a_{k-2} & a_{k-1} & a_{k} \\ a_1 & a_2 & a_3 & a_4 & \dots & a_{k-1} & a_{k-2} & a_{k}}$\\
            $(a_{k-3} \; a_{k-2}) = \mat{a_1 & \dots & a_{k-3} & a_{k-2} & a_{k-1} & a_{k} \\ a_1 & \dots &  a_{k-2} & a_{k-3} & a_{k-1} & a_{k}}$\\
            $(a_{k-2} \; a_{k-1})(a_{k-1} \; a_{k}) = \mat{a_1 & \dots & a_{k-3} & a_{k-2} & a_{k-1} & a_{k} \\ a_1 & \dots & a_{k-3}  & a_{k-1} & a_{k} & a_{k-2}}$\\
            $(a_{k-3} \; a_{k-2})(a_{k-2} \; a_{k-1})(a_{k-1} \; a_{k}) = \mat{a_1 & \dots & a_{k-3} & a_{k-2} & a_{k-1} & a_{k} \\ a_1 & \dots & a_{k-2}  & a_{k-1} & a_{k} & a_{k-3}}$\\
            $\dots$\\
            $(a_1 \; a_2)\dots (a_{k-3} \; a_{k-2})(a_{k-2} \; a_{k-1})(a_{k-1} \; a_{k}) = \mat{a_1 & a_2 & \dots & a_{k-3} & a_{k-2} & a_{k-1} & a_{k} \\ a_2 & a_3 & \dots & a_{k-2}  & a_{k-1} & a_{k} & a_{1}}$\\\\
            $sgn((a_1 \; a_2)(a_2 \; a_3)\dots(a_{k-2} \; a_{k-1})(a_{k-1} \; a_{k})) = $\\
            $= sgn((a_1 \; a_2))sgn((a_2 \; a_3))\dots sgn((a_{k-2} \; a_{k-1}))sgn((a_{k-1} \; a_{k}))\r $ Так как знак транспозиции всегда равен -1, то \\
            $sgn((a_1 \; a_2)(a_2 \; a_3)\dots(a_{k-2} \; a_{k-1})(a_{k-1} \; a_{k})) = (-1)^{k-1}$\\
            \textbf{Ответ: } $\mat{a_1 & a_2 & \dots & a_{k-3} & a_{k-2} & a_{k-1} & a_{k} \\ a_2 & a_3 & \dots & a_{k-2}  & a_{k-1} & a_{k} & a_{1}}$, $sgn = (-1)^{k-1}$

      \item[\textbf{7.}] \indent \\
            \begin{enumerate}
                  \item[7.1]$$\sigma^2X\tau\rho= \sigma^2\tau$$
                        $$\sigma^2X= \sigma^2\tau \rho^{-1} \tau^{-1}$$
                        $$X= \tau \rho^{-1} \tau^{-1}$$
                        \textbf{Ответ: } $\tau \rho^{-1} \tau^{-1}$

                  \item[7.2]
                        $\sigma = \begin{pmatrix}1 & 2 & 3 & 4 & 5\\3 & 5 & 1 & 2 & 4\end{pmatrix}$,
                        $\tau = \begin{pmatrix}1 & 2 & 3 & 4 & 5\\3 & 4 & 1 & 2 & 5\end{pmatrix}$ и
                        $\rho = \begin{pmatrix}1 & 2 & 3 & 4 & 5\\2 & 4 & 5 & 1 & 3\end{pmatrix}$ решить уравнение $\sigma X \tau = \rho$.
                        $$\sigma X \tau = \rho$$
                        $$X \tau =\sigma^{-1} \rho$$
                        $$X =\sigma^{-1} \rho \tau^{-1}$$
                        $\tau^{-1} = \begin{pmatrix}1 & 2 & 3 & 4 & 5\\3 & 4 & 1 & 2 & 5\end{pmatrix}$,
                        $\rho = \begin{pmatrix}1 & 2 & 3 & 4 & 5\\2 & 4 & 5 & 1 & 3\end{pmatrix}$,
                        $\sigma^{-1} = \begin{pmatrix}1 & 2 & 3 & 4 & 5\\3 & 4 & 1 & 5 & 2\end{pmatrix}$
                        $$X =\sigma^{-1} \rho \tau^{-1} = \mat{1 & 2 & 3 & 4 & 5 \\ 2 & 3 & 4 & 5 & 1}$$
                        \textbf{Ответ: } $\mat{1 & 2 & 3 & 4 & 5 \\ 2 & 3 & 4 & 5 & 1}$

            \end{enumerate}

      \item[\textbf{8.}]
            $$\sigma = (3 \; 5 \; 7 \; \ldots 99)(2 \; 4 \; \ldots \; 98) \in S_{99}$$
            Декремент равен: $99 - 3 = 97$ - чётное $\r sgn(\sigma) = 1$
            $$\sigma = \mat{1 & 2 & 3 & \dots & 96 & 97 & 98 & 99 \\ 1 & 4 & 5 & \dots & 98 & 99 & 2 & 3}$$
            Инверсий: $0 + 2 + 2 + \dots + 2 + 0 + 0 = 96*2 = 192 \r$
            $$\r sgn(\sigma) = (-1)^{192} = 1$$
            \textbf{Ответ: } $sgn(\sigma) = 1$

      \item[\textbf{9.}]
            $X \in S_5$, что $X^2 = (1 \; 2 \; 3 \; 4 \; 5)$\\
            $(1 \; 2 \; 3 \; 4 \; 5)$ может получаться только из $(1 \; 4 \; 2 \; 5 \; 3)$\\
            \textbf{Ответ: } $(1 \; 4 \; 2 \; 5 \; 3)$


      \item[\textbf{10.}] Возьмём произвольную перестановку $\sigma \in S_n$.\\
      Пусть $m$ - количество циклов в разложении $\sigma$, включая циклы длины 1. \\
      Каждый $i$-ый цикл длины $k_i$ можно представить как произведение $(k_i-1)$ транспозиций \\
      Сумма длин всех циклов равна: $k_1+k_2+\dots+k_m = n$. \\
      Разложим каждый цикл на транспозиции, тогда всего транспозиций будет: $(k_1 -1 )+ (k_2 - 1) + \dots + (k_m - 1) = k_1 + k_2+\dots+k_m - m = n-m$.\\
      Следовательно, количество транспозиций равно декременту, а так как число инверсий определяется чётностью числа транспозиций (из задания 6), то чётность перестановки $\sigma$ равна чётности декремента.
      

      
\end{enumerate}

\end{document}