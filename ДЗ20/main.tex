\documentclass[a4paper]{article}
\usepackage{setspace}
\usepackage[T2A]{fontenc} %
\usepackage[utf8]{inputenc} % подключение русского языка
\usepackage[russian]{babel} %
\usepackage[12pt]{extsizes}
\usepackage{mathtools}
\usepackage{graphicx}
\usepackage{fancyhdr}
\usepackage{amssymb}
\usepackage{amsmath, amsfonts, amssymb, amsthm, mathtools}
\usepackage{tikz}

\usetikzlibrary{positioning}
\setstretch{1.3}

\newcommand{\mat}[1]{\begin{pmatrix} #1 \end{pmatrix}}
\newcommand{\vmat}[1]{\begin{vmatrix} #1 \end{vmatrix}}
\renewcommand{\f}[2]{\frac{#1}{#2}}
\newcommand{\dspace}{\space\space}
\newcommand{\s}[2]{\sum\limits_{#1}^{#2}}
\newcommand{\mul}[2]{\prod_{#1}^{#2}}
\newcommand{\sq}[1]{\left[ {#1} \right]}
\newcommand{\gath}[1]{\left[ \begin{array}{@{}l@{}} #1 \end{array} \right.}
\newcommand{\case}[1]{\begin{cases} #1 \end{cases}}
\newcommand{\ts}{\text{\space}}
\newcommand{\lm}[1]{\underset{#1}{\lim}}
\newcommand{\suplm}[1]{\underset{#1}{\overline{\lim}}}
\newcommand{\inflm}[1]{\underset{#1}{\underline{\lim}}}
\newcommand{\Ker}[1]{\operatorname{Ker}}

\renewcommand{\phi}{\varphi}
\newcommand{\lr}{\Leftrightarrow}
\renewcommand{\r}{\Rightarrow}
\newcommand{\rr}{\rightarrow}
\renewcommand{\geq}{\geqslant}
\renewcommand{\leq}{\leqslant}
\newcommand{\RR}{\mathbb{R}}
\newcommand{\CC}{\mathbb{C}}
\newcommand{\QQ}{\mathbb{Q}}
\newcommand{\ZZ}{\mathbb{Z}}
\newcommand{\VV}{\mathbb{V}}
\newcommand{\NN}{\mathbb{N}}
\newcommand{\OO}{\underline{O}}
\newcommand{\oo}{\overline{o}}
\renewcommand{\Ker}{\operatorname{Ker}}
\renewcommand{\Im}{\operatorname{Im}}


\DeclarePairedDelimiter\abs{\lvert}{\rvert} %
\makeatletter                               % \abs{}
\let\oldabs\abs                             %
\def\abs{\@ifstar{\oldabs}{\oldabs*}}       %

\begin{document}

\section*{Домашнее задание на 07.03 (Линейная алгебра)}
 {\large Емельянов Владимир, ПМИ гр №247}\\\\
\begin{enumerate}
    \item[\textbf{№1}]$\beta(x, y) = x_1y_1 + x_1y_2 + x_2y_1 + ax_2y_2 + bx_2y_3 - x_3y_2 + 2x_3y_3$
    Чтобы данная билинейная форма была скалярным произведением, она должна удовлетворять следующим свойствам:
    \begin{enumerate}
        \item[1)] Симметрия:
        
        То есть \(\langle x,y\rangle = \langle y,x\rangle\) для всех \(x,y\). Если записать билинейную форму в виде \(x^T M y\), то матрица \(M\) имеет вид
        \[
        M=\begin{pmatrix}
        1 & 1 & 0\\[1mm]
        1 & a & b\\[1mm]
        0 & -1 & 2
        \end{pmatrix}
        \]
        Симметрия требует \(M_{ij} = M_{ji}\) для всех \(i,j\). Видно, что
        при \(i=2,j=3\): \(M_{23}=b\) и \(M_{32}=-1\). 
        
        Поэтому условие симметрии дает
        \[
        b=-1.
        \]

        \item[2)] Положительная определённость
        
        Из первого пункта у нас получилась матрица:
        \[
        \begin{pmatrix}
        1 & 1 & 0\\[1mm]
        1 & a & -1\\[1mm]
        0 & -1 & 2
        \end{pmatrix}
        \]
        Найдём её угловые миноры:
        $$\delta_1 = 1, \quad \delta_2 = a-1, \quad \delta_3 = 2a-3$$
        Чтобы матрица была положительно определена, нужно, чтобы:
        $$\case{
            a-1 >0\\
            2a-3>0
        } \implies \case{
            a > 1\\
            a > \f{3}{2}
        } \implies a > \f{3}{2}$$

    \end{enumerate}

    \textbf{Ответ: } $b= -1, \quad a > \f{3}{2}$\\

    \item[\textbf{№2}]
    Рассмотрим векторное пространство
    \[
    E = \mathbb{R}[x]_{\leq n},
    \]
    сформированное многочленами степени не выше \( n \), и определим на нём форму
    \[
    (f,g) = f(0)g(0) + f(1)g(1) + \dots + f(n)g(n).
    \]
    Проверим условия евклидова пространства:
    \begin{enumerate}
        \item[1)]Симметричность
        
        Очевидно, что форма симметрична, то есть \( (f,g) = (g,f) \) для всех \( f, g \in E \), а также линейна по каждой из аргументов, поскольку операция умножения и сложения чисел удовлетворяет свойствам линейности.
    
        \item[2)]Положительная определённость
        
        Для любого \( f \in E \) имеем:
        \[
        (f,f) = f(0)^2 + f(1)^2 + \dots + f(n)^2.
        \]
        Каждая слагаемая \( f(k)^2 \ge 0 \). Рассмотрим когда $(f, f) = 0$:
        \[
        f(0) = f(1) = \dots = f(n) = 0.
        \]
        Следовательно,  $(f, f) = 0$ только при $f \equiv 0$. Получаем:
        $$(f,f) > 0, \text{ при } f \neq 0$$
        
    \end{enumerate}

    \textbf{Вывод:} Форма \( (f, g) \) является скалярным произведением, то есть векторное пространство \( E \) является евклидовым.\\

    \item[\textbf{№3}]
    Пусть \( \mathbf{u} = (2, 5, 4) \) и \( \mathbf{v} = (6, 0, -3) \). Стандартное скалярное произведение в \(\mathbb{R}^3\) определяется как
    \[
    \langle \mathbf{u}, \mathbf{v} \rangle = 2\cdot 6 + 5\cdot 0 + 4\cdot (-3).
    \]
    Вычисляем:
    \[
    \langle \mathbf{u}, \mathbf{v} \rangle = 12 + 0 - 12 = 0.
    \]
    
    Так как \(\langle \mathbf{u}, \mathbf{v} \rangle = 0\), векторы \(\mathbf{u}\) и \(\mathbf{v}\) ортогональны. Следовательно, угол \(\theta\) между ними равен \(90^\circ\).
    
    \textbf{Ответ: } $90$\\

    \item[\textbf{№4}]
    Пусть \( f(x) = x^3 \) и \( g(x) = x^2 + x + 1 \). Тогда скалярное произведение определяется как
    \[
    \begin{aligned}
        (f, g) = \int_0^1 f(x) g(x) \, dx = \int_0^1 x^3 (x^2+x+1) \, dx =\\
        = \int_0^1 \left( x^5 + x^4 + x^3 \right) dx = \f{37}{60}
    \end{aligned}
    \]
    Найдем нормы векторов:
    \[
    \|f\| = \sqrt{(f, f)} = \sqrt{\int_0^1 x^6 \, dx} = \sqrt{\frac{1}{7}} = \frac{1}{\sqrt{7}},
    \]
    а для \(g\):
    \[
    \|g\| = \sqrt{(g, g)} = \sqrt{\int_0^1 (x^2+x+1)^2 \, dx} = \sqrt{\f{37}{10}}
    \]
    Определим теперь угол \( \theta \) между \( f \) и \( g \) по формуле
    \[
    \cos \theta = \frac{(f, g)}{\|f\|\, \|g\|}.
    \]
    Подставляем найденные значения:
    \[
    \cos \theta = \frac{\frac{37}{60}}{\frac{1}{\sqrt{7}} \cdot \frac{\sqrt{37}}{\sqrt{10}}} = \frac{37}{60} \cdot \frac{\sqrt{7}\sqrt{10}}{\sqrt{37}} = \frac{37\sqrt{70}}{60\sqrt{37}}= \sqrt{\f{259}{360}}
    \]
    Таким образом, угол \( \theta \) равен
    \[
    \theta = \arccos\left(\sqrt{\f{259}{360}}\right)
    \]
    \textbf{Ответ:} $\arccos\sqrt{\f{259}{360}}$\\

    \item[\textbf{№5}]
    Пусть \( s \) и \( t \) --- единичные векторы, т.е.
    \[
    \|s\| = \|t\| = 1,
    \]
    и угол между ними равен \(\theta\), тогда
    \[
    (s, t) = \cos \theta.
    \]
    
    Заданы векторы
    \[
    p = s + 2t \quad \text{и} \quad q = 5s - 4t.
    \]
    Они взаимно перпендикулярны, то есть
    \[
    (p, q) = 0.
    \]
    Воспользуемся билинейностью скалярного произведения:
    \[
    (p, q) = (s + 2t,\, 5s - 4t) = (s, 5s - 4t) + (2t, 5s - 4t).
    \]
    \[
    (s, 5s - 4t) = 5(s,s) - 4(s,t) = 5 - 4\cos \theta,
    \]
    \[
    (2t, 5s - 4t) = 2\left[5(t,s) - 4(t,t)\right] = 10\cos \theta - 8.
    \]
    Таким образом,
    \[
    (p, q) = \bigl(5 - 4\cos \theta\bigr) + \bigl(10\cos \theta - 8\bigr) = (5 - 8) + ( -4\cos \theta + 10\cos \theta ) = -3 + 6\cos \theta.
    \]
    Так как \((p,q) = 0\), то
    \[
    -3 + 6\cos \theta = 0 \quad \Longrightarrow \quad \cos \theta = \frac{1}{2}.
    \]
    Отсюда
    \[
    \theta = \arccos\left(\frac{1}{2}\right) = 60^\circ.
    \]
    \textbf{Ответ: } $\f{\pi}{3}$\\
    
    \item[\textbf{№6}]
    Пусть \( p = (b,c)a - (a,c)b \) и \( c \) --- произвольный вектор пространства \( E \). Докажем, что \( p \) и \( c \) ортогональны, то есть \( (p, c) = 0 \).
    
    Вычислим скалярное произведение:
    \[
    (p, c) = \Bigl( (b,c)a - (a,c)b,\; c \Bigr) = (b,c)(a,c) - (a,c)(b,c).
    \]
    Заметим, что \( (b,c)(a,c) - (a,c)(b,c) = 0 \).
    
    Таким образом, \( (p, c) = 0 \), что и требовалось доказать.
    
    \item[\textbf{№7}]
    Мы имеем следующие матрицы:
    \[
    A_1=\begin{pmatrix} 2 & 3 \\ 4 & 7 \end{pmatrix}, \quad
    A_2=\begin{pmatrix} 2 & 4 \\ 4 & 5 \end{pmatrix}, \quad
    A_3=\begin{pmatrix} 2 & 4 \\ 4 & 11 \end{pmatrix}, \quad
    A_4=\begin{pmatrix} 2 & 4 \\ 4 & 8 \end{pmatrix}.
    \]
    Матрица Грама для набора векторов должна быть \emph{симметричной} и \emph{неотрицательно определённой}.
    
    \textbf{1. Матрица \(A_1\)}\\
    Заметим, что
    \[
    (A_1)_{12}=3 \quad \text{и} \quad (A_1)_{21}=4.
    \]
    Так как \(3\neq 4\), матрица \(A_1\) не симметрична, следовательно, она не может быть матрицей Грама.

    
    \textbf{2. Матрица \(A_2\)}\\
    Матрица \(A_2\) симметрична, но проверим её угловые миноры:
    $$\delta_1 = 2, \quad \delta_2 = 10-16 < 0$$
    Следовательно, матрица не определена, значит, она не является матрицей Грама.

        
    \textbf{3. Матрица \(A_3\)}\\
    Матрица \(A_3\) симметрична. Проверим её определитель:
    \[
    \det A_3 = 2\cdot 11 - 4^2 = 22-16 = 6 > 0.
    \]
    Также, \(a_{11}=2>0\). Следовательно, \(A_3\) \emph{положительно определена} и может быть матрицей Грама некоторого набора из двух векторов.

    Найдём, например, в \(\mathbb{R}^2\) два вектора \(v_1\) и \(v_2\) такие, что
    \[
    (v_1,v_1)=2,\quad (v_1,v_2)=4,\quad (v_2,v_2)=11.
    \]
    Выберем
    \[
    v_1=(\sqrt{2},\, 0).
    \]
    Тогда условие \((v_1,v_2)=4\) даёт
    \[
    \sqrt{2}\cdot (v_2)_1 = 4 \quad \Longrightarrow \quad (v_2)_1=\frac{4}{\sqrt{2}}=2\sqrt{2}.
    \]
    Чтобы \((v_2,v_2)=11\) выполнялось, запишем:
    \[
    (v_2)_1^2+(v_2)_2^2=11 \quad \Longrightarrow \quad (2\sqrt{2})^2+(v_2)_2^2=11,
    \]
    \[
    8+(v_2)_2^2=11 \quad \Longrightarrow \quad (v_2)_2^2=3.
    \]
    Возьмём, например, \((v_2)_2=\sqrt{3}\). Тогда
    \[
    v_2=(2\sqrt{2},\, \sqrt{3}).
    \]
    Проверим:
    \[
    (v_1,v_1)=2,\quad (v_1,v_2)=\sqrt{2}\cdot 2\sqrt{2}=4,\quad (v_2,v_2)=8+3=11.
    \]
    Таким образом, \(A_3\) является матрицей Грама набора \(\{v_1,v_2\}\).

    \textbf{4. Матрица \(A_4\)}\\
    Матрица \(A_4\) симметрична. Вычислим определитель:
    \[
    \det A_4 = 2\cdot 8 - 4^2 = 16-16 = 0.
    \]
    Определитель равен нулю, что означает, что матрица \emph{неотрицательно определена}

    Найдём векторы \(w_1\) и \(w_2\) в \(\mathbb{R}^2\) такие, что
    \[
    (w_1,w_1)=2,\quad (w_1,w_2)=4,\quad (w_2,w_2)=8.
    \]
    Можно взять \(w_1=(\sqrt{2},\,0)\). Тогда требование \((w_1,w_2)=4\) даёт:
    \[
    \sqrt{2}\cdot (w_2)_1 = 4 \quad \Longrightarrow \quad (w_2)_1=2\sqrt{2}.
    \]
    Теперь условие \((w_2,w_2)=8\) требует:
    \[
    (2\sqrt{2})^2+(w_2)_2^2=8 \quad \Longrightarrow \quad 8+(w_2)_2^2=8,
    \]
    что даёт \((w_2)_2^2=0\) и, следовательно, \((w_2)_2=0\). Таким образом,
    \[
    w_2=(2\sqrt{2},\,0).
    \]
    Заметим, что \(w_2=2w_1\), то есть векторы линейно зависимы, а их матрица Грама:
    \[
    \begin{pmatrix} (w_1,w_1) & (w_1,w_2) \\ (w_2,w_1) & (w_2,w_2) \end{pmatrix} =
    \begin{pmatrix} 2 & 4 \\ 4 & 8 \end{pmatrix},
    \]
    совпадает с \(A_4\).

    \textbf{Вывод:}\\
    Единственными матрицами из предложенных, являющимися матрицами Грама, являются \(A_3\) и \(A_4\). Примеры наборов векторов:
    \[
    A_3: \quad v_1=(\sqrt{2},\,0), \quad v_2=(2\sqrt{2},\,\sqrt{3});
    \]
    \[
    A_4: \quad w_1=(\sqrt{2},\,0), \quad w_2=(2\sqrt{2},\,0).
    \]

\end{enumerate}
\end{document}