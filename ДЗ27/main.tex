\documentclass[a4paper]{article}
\usepackage{setspace}
\usepackage[utf8]{inputenc}
\usepackage[russian]{babel}
\usepackage[12pt]{extsizes}
\usepackage{mathtools}
\usepackage{graphicx}
\usepackage{fancyhdr}
\usepackage{amssymb}
\usepackage{amsmath, amsfonts, amssymb, amsthm, mathtools}
\usepackage{tikz}

\usetikzlibrary{positioning}
\setstretch{1.3}

\newcommand{\mat}[1]{\begin{pmatrix} #1 \end{pmatrix}}
\newcommand{\vmat}[1]{\begin{vmatrix} #1 \end{vmatrix}}
\renewcommand{\f}[2]{\frac{#1}{#2}}
\newcommand{\dspace}{\space\space}
\newcommand{\s}[2]{\sum\limits_{#1}^{#2}}
\newcommand{\mul}[2]{\prod_{#1}^{#2}}
\newcommand{\sq}[1]{\left[ {#1} \right]}
\newcommand{\gath}[1]{\left[ \begin{array}{@{}l@{}} #1 \end{array} \right.}
\newcommand{\case}[1]{\begin{cases} #1 \end{cases}}
\newcommand{\ts}{\text{\space}}
\newcommand{\lm}[1]{\underset{#1}{\lim}}
\newcommand{\suplm}[1]{\underset{#1}{\overline{\lim}}}
\newcommand{\inflm}[1]{\underset{#1}{\underline{\lim}}}
\newcommand{\Ker}[1]{\operatorname{Ker}}

\renewcommand{\phi}{\varphi}
\newcommand{\lr}{\Leftrightarrow}
\renewcommand{\l}{\left(}
\renewcommand{\r}{\right)}
\newcommand{\rr}{\rightarrow}
\renewcommand{\geq}{\geqslant}
\renewcommand{\leq}{\leqslant}
\newcommand{\RR}{\mathbb{R}}
\newcommand{\CC}{\mathbb{C}}
\newcommand{\QQ}{\mathbb{Q}}
\newcommand{\ZZ}{\mathbb{Z}}
\newcommand{\VV}{\mathbb{V}}
\newcommand{\NN}{\mathbb{N}}
\newcommand{\OO}{\underline{O}}
\newcommand{\oo}{\overline{o}}
\renewcommand{\Ker}{\operatorname{Ker}}
\renewcommand{\Im}{\operatorname{Im}}
\newcommand{\vol}{\text{vol}}
\newcommand{\Vol}{\text{Vol}}

\DeclarePairedDelimiter\abs{\lvert}{\rvert} %
\makeatletter                               % \abs{}
\let\oldabs\abs                             %
\def\abs{\@ifstar{\oldabs}{\oldabs*}}       %

\begin{document}

\section*{Домашнее задание на 24.04 (Линейная алгебра)}
{\large Емельянов Владимир, ПМИ гр №247}\\\\
\begin{enumerate}
  \item[\textbf{№1}]Мы знаем с.в. и с.з.:
  $$\lambda_1=1+i,\quad v_1=\begin{pmatrix}1\\i\end{pmatrix}
  \qquad
  \lambda_2=1-i,\quad v_2=\begin{pmatrix}1\\-i\end{pmatrix}$$
  Диагонализуем \(A\):
  \[
  A = P\,D\,P^{-1},\quad
  P=\bigl(v_1,\;v_2\bigr)
  =\begin{pmatrix}1&1\\i&-i\end{pmatrix},\quad
  D=\operatorname{diag}(1+i,\;1-i)
  \]
  Тогда
  $$
  A^{21}
  = P\,D^{21}\,P^{-1}
  = P\,\begin{pmatrix}(1+i)^{21}&0\\0&(1-i)^{21}\end{pmatrix}P^{-1}
  $$
  Получаем
  $$
  A^{21}
  = P\,\begin{pmatrix}-2^{10}(1+i)&0\\0&-2^{10}(1-i)\end{pmatrix}P^{-1}
  =-2^{10}\,P\,D\,P^{-1}
  =-2^{10}\,A
  =-1024\begin{pmatrix}1&1\\-1&1\end{pmatrix}
  $$
  \textbf{Ответ: } $-1024\begin{pmatrix}1&1\\-1&1\end{pmatrix}$\\

  \item[\textbf{№2}]Рассмотрим оператор
  $$
  \phi:\begin{pmatrix}x\\y\end{pmatrix}\;\longmapsto\;
  \begin{pmatrix}
  0.85x+0.10y\\[6pt]
  0.15x+0.90y
  \end{pmatrix},
  $$
  действующий на векторе
  $\displaystyle v_0=\begin{pmatrix}x_0\\y_0\end{pmatrix}$,
  где $x_0$m– число здоровых, $y_0$ – число больных на острове с населением 10000 человек (из них вначале выявлено 100 больных).

  Ищем с.з. из
  $$
  \det\!\bigl(M-\lambda I\bigr)
  =\det
  \begin{pmatrix}
  0.85-\lambda & 0.10\\[3pt]
  0.15 & 0.90-\lambda
  \end{pmatrix}
  =0.
  $$
  Вычислим:
  $$
  (0.85-\lambda)(0.90-\lambda)-0.10\cdot0.15
  =\lambda^2-1.75\lambda+0.75=0
  $$
  Для $\lambda_1=1$ находим собственный вектор $v\ne0$:
  $$
  M v = v
  \;\Longrightarrow\;
  \begin{cases}
  0.85x+0.10y = x,\\
  0.15x+0.90y = y,
  \end{cases}
  \quad
  \Longrightarrow
  \quad
  0.10y=0.15x
  \;\Longrightarrow\;
  y=1.5\,x.
  $$
  Можно взять $v_1=(1,1.5)^\top$.

  Для $\lambda_2=0.75$ получился бы другой вектор, но при возведении в степень вклад этого собственного компонента $0.75^n\to0$ при $n\to\infty$.
  $$
  \phi^n(v_0)
  = c_1\,(1)^n\,v_1 \;+\;c_2\,(0.75)^n\,v_2
  $$
  поскольку $(0.75)^n\to0$. Константу $c_1$ выбираем так, чтобы сумма компонент равнялась общей численности 10000:
  $$
  v_1=(1,1.5),\quad
  1+1.5=2.5,
  \quad
  c_1=\frac{10000}{2.5}=4000
  $$
  Итак
  $$
  \lim_{n\to\infty}\phi^n(v_0)
  =c_1\,v_1
  =4000\begin{pmatrix}1\\1.5\end{pmatrix}
  =\begin{pmatrix}4000\\6000\end{pmatrix}
  $$


\end{enumerate}
\end{document}