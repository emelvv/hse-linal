\documentclass[a4paper]{article}
\usepackage{setspace}
\usepackage[utf8]{inputenc}
\usepackage[russian]{babel}
\usepackage[12pt]{extsizes}
\usepackage{mathtools}
\usepackage{graphicx}
\usepackage{fancyhdr}
\usepackage{amssymb}
\usepackage{amsmath, amsfonts, amssymb, amsthm, mathtools}
\usepackage{tikz}

\usetikzlibrary{positioning}
\setstretch{1.3}

\newcommand{\mat}[1]{\begin{pmatrix} #1 \end{pmatrix}}
\newcommand{\vmat}[1]{\begin{vmatrix} #1 \end{vmatrix}}
\renewcommand{\f}[2]{\frac{#1}{#2}}
\newcommand{\dspace}{\space\space}
\newcommand{\s}[2]{\sum\limits_{#1}^{#2}}
\newcommand{\mul}[2]{\prod_{#1}^{#2}}
\newcommand{\sq}[1]{\left[ {#1} \right]}
\newcommand{\gath}[1]{\left[ \begin{array}{@{}l@{}} #1 \end{array} \right.}
\newcommand{\case}[1]{\begin{cases} #1 \end{cases}}
\newcommand{\ts}{\text{\space}}
\newcommand{\lm}[1]{\underset{#1}{\lim}}
\newcommand{\suplm}[1]{\underset{#1}{\overline{\lim}}}
\newcommand{\inflm}[1]{\underset{#1}{\underline{\lim}}}
\newcommand{\Ker}[1]{\operatorname{Ker}}

\renewcommand{\phi}{\varphi}
\newcommand{\lr}{\Leftrightarrow}
\renewcommand{\l}{\left(}
\renewcommand{\r}{\right)}
\newcommand{\rr}{\rightarrow}
\renewcommand{\geq}{\geqslant}
\renewcommand{\leq}{\leqslant}
\newcommand{\RR}{\mathbb{R}}
\newcommand{\CC}{\mathbb{C}}
\newcommand{\QQ}{\mathbb{Q}}
\newcommand{\ZZ}{\mathbb{Z}}
\newcommand{\VV}{\mathbb{V}}
\newcommand{\NN}{\mathbb{N}}
\newcommand{\OO}{\underline{O}}
\newcommand{\oo}{\overline{o}}
\renewcommand{\Ker}{\operatorname{Ker}}
\renewcommand{\Im}{\operatorname{Im}}
\newcommand{\vol}{\text{vol}}
\newcommand{\Vol}{\text{Vol}}

\DeclarePairedDelimiter\abs{\lvert}{\rvert} %
\makeatletter                               % \abs{}
\let\oldabs\abs                             %
\def\abs{\@ifstar{\oldabs}{\oldabs*}}       %

\begin{document}

\section*{Домашнее задание на 24.04 (Линейная алгебра)}
{\large Емельянов Владимир, ПМИ гр №247}\\\\
\begin{enumerate}
  \item[\textbf{№1}]Мы знаем с.в. и с.з.:
  $$\lambda_1=1+i,\quad v_1=\begin{pmatrix}1\\i\end{pmatrix}
  \qquad
  \lambda_2=1-i,\quad v_2=\begin{pmatrix}1\\-i\end{pmatrix}$$
  Диагонализуем \(A\):
  \[
  A = P\,D\,P^{-1},\quad
  P=\bigl(v_1,\;v_2\bigr)
  =\begin{pmatrix}1&1\\i&-i\end{pmatrix},\quad
  D=\operatorname{diag}(1+i,\;1-i)
  \]
  Тогда
  $$
  A^{21}
  = P\,D^{21}\,P^{-1}
  = P\,\begin{pmatrix}(1+i)^{21}&0\\0&(1-i)^{21}\end{pmatrix}P^{-1}
  $$
  Получаем
  $$
  A^{21}
  = P\,\begin{pmatrix}-2^{10}(1+i)&0\\0&-2^{10}(1-i)\end{pmatrix}P^{-1}
  =-2^{10}\,P\,D\,P^{-1}
  =-2^{10}\,A
  =-1024\begin{pmatrix}1&1\\-1&1\end{pmatrix}
  $$
  \textbf{Ответ: } $-1024\begin{pmatrix}1&1\\-1&1\end{pmatrix}$\\

  \item[\textbf{№2}]Рассмотрим оператор
  $$
  \phi:\begin{pmatrix}x\\y\end{pmatrix}\;\longmapsto\;
  \begin{pmatrix}
  0.85x+0.10y\\[6pt]
  0.15x+0.90y
  \end{pmatrix},
  $$
  действующий на векторе
  $\displaystyle v_0=\begin{pmatrix}x_0\\y_0\end{pmatrix}$,
  где $x_0$ – число здоровых, $y_0$ – число больных на острове с населением 10000 человек (из них вначале выявлено 100 больных).

  Ищем с.з. из
  $$
  \det\!\bigl(M-\lambda I\bigr)
  =\det
  \begin{pmatrix}
  0.85-\lambda & 0.10\\[3pt]
  0.15 & 0.90-\lambda
  \end{pmatrix}
  =0.
  $$
  Вычислим:
  $$
  (0.85-\lambda)(0.90-\lambda)-0.10\cdot0.15
  =\lambda^2-1.75\lambda+0.75=0
  $$
  Для $\lambda_1=1$ находим собственный вектор $v\ne0$:
  $$
  M v = v
  \;\Longrightarrow\;
  \begin{cases}
  0.85x+0.10y = x,\\
  0.15x+0.90y = y,
  \end{cases}
  \quad
  \Longrightarrow
  \quad
  0.10y=0.15x
  \;\Longrightarrow\;
  y=1.5\,x.
  $$
  Можно взять $v_1=(1,1.5)^\top$.

  Для $\lambda_2=0.75$ получился бы другой вектор, но при возведении в степень вклад этого собственного компонента $0.75^n\to0$ при $n\to\infty$.
  $$
  \phi^n(v_0)
  = c_1\,(1)^n\,v_1 \;+\;c_2\,(0.75)^n\,v_2 \to c_1v_1
  $$
  поскольку $(0.75)^n\to0$. Константу $c_1$ выбираем так, чтобы сумма компонент равнялась общей численности 10000:
  $$
  v_1=(1,1.5),\quad
  1+1.5=2.5,
  \quad
  c_1=\frac{10000}{2.5}=4000
  $$
  Итак
  $$
  \lim_{n\to\infty}\phi^n(v_0)
  =c_1\,v_1
  =4000\begin{pmatrix}1\\1.5\end{pmatrix}
  =\begin{pmatrix}4000\\6000\end{pmatrix}
  $$
  \textbf{Ответ:} число здоровых стремится к $4000$, число больных к $6000$\\

  \item[\textbf{№3}]Найдём с.з. и с.в.:
  $$A=\begin{pmatrix}
    5 & 2 & -4\\
    2 & 8 & 2\\
    -4 & 2 & 5
    \end{pmatrix} $$
    
  $$
  -\chi(\lambda)=\det(A-\lambda E)
  =\det\begin{pmatrix}
  5-\lambda & 2 & -4\\
  2 & 8-\lambda & 2\\
  -4 & 2 & 5-\lambda
  \end{pmatrix}
  $$
  Разложим по первой строке:
  $$
  \begin{aligned}
  -\chi(\lambda)&=(5-\lambda)\bigl((8-\lambda)(5-\lambda)-4\bigr)
  -2\bigl(2(5-\lambda)-(-8)\bigr)
  -4\bigl(2\cdot2-(8-\lambda)(-4)\bigr)\\
  &=-\lambda^3+18\lambda^2-81\lambda\\
  &=-\lambda(\lambda-9)^2
  \end{aligned}
  $$
  Найдём собственные векторы:
  \begin{enumerate}
    \item[1)]для $\lambda=0$:\\
    $$
    \begin{cases}
    5x+2y-4z=0,\\
    2x+8y+2z=0,\\
    -4x+2y+5z=0.
    \end{cases}
    \quad \lr \quad v_{0}=\langle (2,\,-1,\,2)^T \rangle$$
    Нормируем:
    $$f_1 = \f{1}{3} (2,\,-1,\,2)^T $$

    \item[2)]для $\lambda=9$:\\
    $$
    \begin{cases}
    -4x+2y-4z=0,\\
    2x-y+2z=0,\\
    -4x+2y-4z=0.
    \end{cases}
    \quad \lr \quad 
    v_{9}=\langle (1,\,2,\,0)^T, (0,\,2,\,1)^T \rangle$$
    Ортогонализуем векторы:
    $$f_2' = (1, 2, 0)^T, \quad f_3' = (-4,2,5)^T$$
    Нормируем:
    $$f_2 = \f{1}{\sqrt{5}}(1, 2, 0)^T, \quad f_3 = \f{1}{3\sqrt{5}}(-4,2,5)^T$$
  \end{enumerate}
  Получаем ОНБ:
  $$f_1 = \f{1}{3} (2,\,-1,\,2)^T,\quad f_2 = \f{1}{\sqrt{5}}(1, 2, 0)^T, \quad f_3 = \f{1}{3\sqrt{5}}(-4,2,5)^T$$
  Диагональный вид:
  $$\text{diag}(0, 9, 9)$$\\

  \item[\textbf{№4}]Для квадратичной формы
  $$
  Q(x)=x_1^2+x_2^2-2x_3^2+8x_1x_2+4x_1x_3+4x_2x_3
  $$
  соответствует симметричная матрица $A$
  $$
  A=\begin{pmatrix}
  1 & 4 & 2\\
  4 & 1 & 2\\
  2 & 2 & -2
  \end{pmatrix}
  $$
  Рассмотрим $\phi: \RR^3 \to \RR^3$ с матрицей $A$:
  $$-\chi(\lambda) = (\lambda-6)(\lambda+3)^2$$
  \begin{enumerate}
    \item[1)] для $\lambda = 6$:
    $$v_6 = \langle (2,2,1)^T \rangle$$
    Нормируем:
    $$f_1 = \f{1}{3}(2,2,1)^T$$
    \item[2)] для $\lambda = -3$:
    $$v_{-3} = \langle (1,0,-2)^T,\;(0,1,-2)^T\rangle$$
    Ортогонализуем:
    $$f_2' = (1,0,-2)^T, \quad f_3' = (-4,5,-2)^T$$
    Нормируем:
    $$f_2 = \f{1}{\sqrt{5}}(1,0,-2)^T, \quad f_3 = \f{1}{3\sqrt{5}}(-4,5,-2)^T$$
  \end{enumerate}
  Получаем ОНБ:
  $$f_1 = \f{1}{3}(2,2,1)^T, \quad f_2 = \f{1}{\sqrt{5}}(1,0,-2)^T, \quad f_3 = \f{1}{3\sqrt{5}}(-4,5,-2)^T$$
  Канонический вид:
  $$6x_1'^2 -3x_2'^2 - 3x_3'^2$$\\

  \item[\textbf{№5}]Для оператора
  $$
  \phi: \RR^3 \to \RR^3
  $$
  с матрицей в стандартном базисе
  $$
  A=\begin{pmatrix}
  -\tfrac23 & -\tfrac23 & \tfrac13\\[3pt]
  -\tfrac23 & \tfrac13 & -\tfrac23\\[3pt]
  -\tfrac13 & \tfrac23 & \tfrac23
  \end{pmatrix}
  $$
  Характеристический многочлен даёт корни
  $$
  \lambda_1=-1,\quad 
  \lambda_{2,3}=\frac23\pm\frac{\sqrt5}{3}\,i,
  $$
  по собственному вектору при $\lambda=-1$ происходит отражение, в ортогональной ему плоскости действует чистое вращение на угол $\theta$, где
  $$
    \cos\theta=\tfrac23,\quad \sin\theta=\tfrac{\sqrt5}{3}, 
    \quad\theta=\arccos\tfrac23\,.
  $$
  собственный вектор при $\lambda=-1$:
  $$
     v_1=(2,1,0)^T,\quad \|v_1\|=\sqrt{5},  
     \quad e_1=\frac1{\sqrt5}(2,1,0)^T.
   $$
\end{enumerate}
\end{document}