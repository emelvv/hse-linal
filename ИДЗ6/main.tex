\documentclass[a4paper]{article}
\usepackage{setspace}
\usepackage[T2A]{fontenc} %
\usepackage[utf8]{inputenc} % подключение русского языка
\usepackage[russian]{babel} %
\usepackage[12pt]{extsizes}
\usepackage{mathtools}
\usepackage{graphicx}
\usepackage{fancyhdr}
\usepackage{amssymb}
\usepackage{amsmath, amsfonts, amssymb, amsthm, mathtools}
\usepackage{tikz}

\usetikzlibrary{positioning}
\setstretch{1.3}

\newcommand{\mat}[1]{\begin{pmatrix} #1 \end{pmatrix}}
\renewcommand{\det}[1]{\begin{vmatrix} #1 \end{vmatrix}}
\renewcommand{\f}[2]{\frac{#1}{#2}}
\newcommand{\dspace}{\space\space}
\newcommand{\s}[2]{\sum\limits_{#1}^{#2}}
\newcommand{\mul}[2]{\prod_{#1}^{#2}}
\newcommand{\sq}[1]{\left[ {#1} \right]}
\newcommand{\gath}[1]{\left[ \begin{array}{@{}l@{}} #1 \end{array} \right.}
\newcommand{\case}[1]{\begin{cases} #1 \end{cases}}
\newcommand{\ts}{\text{\space}}
\newcommand{\lm}[1]{\underset{#1}{\lim}}
\newcommand{\suplm}[1]{\underset{#1}{\overline{\lim}}}
\newcommand{\inflm}[1]{\underset{#1}{\underline{\lim}}}
\newcommand{\Ker}[1]{\operatorname{Ker}}

\renewcommand{\phi}{\varphi}
\newcommand{\lr}{\Leftrightarrow}
\renewcommand{\r}{\Rightarrow}
\newcommand{\rr}{\rightarrow}
\renewcommand{\geq}{\geqslant}
\renewcommand{\leq}{\leqslant}
\newcommand{\RR}{\mathbb{R}}
\newcommand{\CC}{\mathbb{C}}
\newcommand{\QQ}{\mathbb{Q}}
\newcommand{\ZZ}{\mathbb{Z}}
\newcommand{\VV}{\mathbb{V}}
\newcommand{\NN}{\mathbb{N}}
\newcommand{\OO}{\underline{O}}
\newcommand{\oo}{\overline{o}}
\renewcommand{\Ker}{\operatorname{Ker}}
\renewcommand{\Im}{\operatorname{Im}}


\DeclarePairedDelimiter\abs{\lvert}{\rvert} %
\makeatletter                               % \abs{}
\let\oldabs\abs                             %
\def\abs{\@ifstar{\oldabs}{\oldabs*}}       %

\begin{document}

\section*{ИДЗ №6 (вариант 12) (Линейная алгебра)}
 {\large Емельянов Владимир, ПМИ гр №247}\\\\
\begin{enumerate}
    \item[\textbf{№1}]Дано пространство многочленов $ V = \mathbb{R}[x]_{\leqslant 2} $. Линейное отображение $ \varphi: V \rightarrow \mathbb{R}^{2} $ задано в базисе 
    $$ e= \left(-3-x+2x^2, 1-x-x^2, 1+x-x^2\right) $$ 
    пространства $ V $ и базисе 
    $$f=  ((-1,-2),(3,1)) $$ 
    пространства $ \mathbb{R}^{2} $. 

    Матрица отображения $ \varphi $ имеет вид:
    $$
    \begin{pmatrix}
    -1 & 5 & -7 \\
    7 & 4 & 3
    \end{pmatrix}
    $$
    
    Чтобы найти $ \varphi\left(7-x-5x^2\right) $ для начала найдём координаты $7-x-5x^2$ в базисе $e$:
    $$(-3-x+2x^2)a+(1-x-x^2)b+(1+x-x^2)c = 7-x-5x^2$$
    Получаем систему:
    $$\case{
        -3a+b+c = 7\\
        -a-b+c=-1\\
        2a-b-c=-5
    } \implies \mat{
        -3 & 1 & 1 & | & 7\\
        -1 & -1 & 1 & | & -1\\
        2 & -1 & -1 & | & -5
    }\implies \text{УСВ: } \mat{1 & 0 & 0 & | & -2 \\
    0 & 1 & 0 & | & 2 \\
    0 & 0 & 1 & | & -1}$$
    Следовательно координаты $7-x-5x^2$ в базисе $e$:
    $$\mat{-2\\2\\-1}$$
    Теперь найдём координаты $\phi\left(7-x-5x^2\right)$ в базисе $f$:
    $$A\mat{a\\b\\c} = \begin{pmatrix}
        -1 & 5 & -7 \\
        7 & 4 & 3
        \end{pmatrix} \mat{-2\\2\\-1} = \mat{19 \\
        -9}$$
    Получается, что:
    $$\varphi\left(7-x-5x^2\right) = 19\cdot \mat{-1\\-2}-9\cdot\mat{3\\1} = \mat{-46\\-47}$$
    \textbf{Ответ: } $\mat{-46\\-47}$\\

    \item[\textbf{№2}]\begin{enumerate}
        \item[(a)]Докажем существование единственного линейного отображения 
        $\varphi: \mathbb{R}^{5} \rightarrow \mathbb{R}^{3}$, которое переводит 
        векторы $$a = a_{1}, a_{2}, a_{3}, a_{4}, a_{5}$$ в векторы 
        $$b = b_{1}, b_{2}, b_{3}, b_{4}, b_{5}$$

        Рассмотрим векторы из $a$:
        % $$\begin{pmatrix}
        %     -16 & -9 & -7 & 11 & 4 \\
        %     67 & -112 & -101 & -42 & -73 \\
        %     83 & -103 & -94 & -53 & -77
        %     \end{pmatrix} \implies \text{УСВ: } \mat{1 & 0 & -\frac{25}{479} & -\frac{322}{479} & -\frac{221}{479} \\
        %     0 & 1 & \frac{417}{479} & -\frac{13}{479} & \frac{180}{479} \\
        %     0 & 0 & 0 & 0 & 0}$$

        $$\mat{3 & 2 & 3 & -4 & 1 \\
        -3 & -3 & -4 & -5 & -3 \\
        -4 & 4 & 4 & 4 & 4 \\
        3 & 4 & 2 & -4 & -2 \\
        -3 & -3 & -1 & 0 & 3} \implies \text{УСВ: } 
        \mat{1 & 0 & 0 & 0 & 0 \\
        0 & 1 & 0 & 0 & 0 \\
        0 & 0 & 1 & 0 & 0 \\
        0 & 0 & 0 & 1 & 0 \\
        0 & 0 & 0 & 0 & 1}$$
        Следовательно, векторы $a$ ЛНЗ, и их $\dim(\langle a\rangle) = 5$ значит, 
        $a$ - базис $\RR^5$, пусть $e$ - стандартный базис в $\RR^3$

        По условию, мы знаем, куда переходят векторы из $a$. 
        Получается, что мы задали отображение базисных векторов, а
        значит, мы единнственным образом определили линейное отображение
        
        \item[(b)]Найдём матрицу линейного отображения $A$,
         из базиса $a$ пространства  $\RR^5$  и
        стандартного базиса $\RR^3$
        
        Получается матрица:
        $$A = \begin{pmatrix}
            -16 & -9 & -7 & 11 & 4 \\
            67 & -112 & -101 & -42 & -73  \\
            83 & -103 & -94 & -53 & -77 
            \end{pmatrix}$$
        Найдём матрицу перехода $C$ от базиса $a$ к стандартному базису $\RR^5$
        $$C = \mat{3 & 2 & 3 & -4 & 1 \\
        -3 & -3 & -4 & -5 & -3 \\
        -4 & 4 & 4 & 4 & 4 \\
        3 & 4 & 2 & -4 & -2 \\
        -3 & -3 & -1 & 0 & 3}$$
        Получается, что матрица $A'$ линейного отображения $\phi$ из стандартного базиса $\RR^5$
        в стандартный базис $\RR^3$ будет выглядить так:
        $$A' = AC = \begin{pmatrix}
            -16 & -9 & -7 & 11 & 4 \\
            67 & -112 & -101 & -42 & -73  \\
            83 & -103 & -94 & -53 & -77 
            \end{pmatrix} \cdot \mat{3 & 2 & 3 & -4 & 1 \\
            -3 & -3 & -4 & -5 & -3 \\
            -4 & 4 & 4 & 4 & 4 \\
            3 & 4 & 2 & -4 & -2 \\
            -3 & -3 & -1 & 0 & 3}$$
            $$A' = \mat{28 & -1 & -22 & 37 & -27 \\
            1034 & 117 & 234 & 56 & -136 \\
            1006 & 118 & 256 & 19 & -109}$$

        Найдём базис ядра:
        $$
        \mat{28 & -1 & -22 & 37 & -27& | & 0 \\
        1034 & 117 & 234 & 56 & -136 & | & 0\\
        1006 & 118 & 256 & 19 & -109& | & 0} \r \text{УСВ:}
        \mat{1 & 0 & -\frac{234}{431} & \frac{877}{862} & -\frac{659}{862}& | & 0 \\
        0 & 1 & \frac{2930}{431} & -\frac{3669}{431} & \frac{2411}{431}& | & 0 \\
        0 & 0 & 0 & 0 & 0& | & 0}$$
        $$\case{
            x_1 = \frac{234}{431}x_3-\frac{877}{862}x_4+\frac{659}{862}x_5\\
            x_2 = -\frac{2930}{431}x_3 + \frac{3669}{431}x_4-\frac{2411}{431}x_5
            }$$
        Следовательно, базис ядра в стандартном базисе:
        $$\mat{\frac{234}{431}\\-\frac{2930}{431}\\1\\0\\0},
        \mat{-\frac{877}{862}\\\frac{3669}{431}\\0\\1\\0},
        \mat{\frac{659}{862}\\-\frac{2411}{431}\\0\\0\\1} $$
        Дополним до $\RR^5$:
        $$\mat{1\\0\\0\\0\\0}, \mat{0\\1\\0\\0\\0}$$
        Базис образа:
        $$\mat{28\\1034\\1006}, \mat{-1\\117\\118}$$
        
        \textbf{Ответ: } $\mat{\frac{234}{431}\\-\frac{2930}{431}\\1\\0\\0},
        \mat{-\frac{877}{862}\\\frac{3669}{431}\\0\\1\\0},
        \mat{\frac{659}{862}\\-\frac{2411}{431}\\0\\0\\1} $ и 
        $\mat{28\\1034\\1006}, \mat{-1\\117\\118}$\\
    \end{enumerate}

    \item[\textbf{№3}]Дано линейное отображение $\varphi: \mathbb{R}^{5} \rightarrow \mathbb{R}^{5}$ с матрицей
    $$\mat{22 & 25 & 8 & -10 & 9 \\
    6 & 4 & -1 & 1 & 5 \\
    -6 & -11 & -7 & 19 & 2 \\
    -1 & 1 & 2 & 3 & -2 \\
    3 & 6 & 4 & -3 & -1}$$
        
    Требуется построить линейное отображение $\psi: \mathbb{R}^{5} \rightarrow \mathbb{R}^{5}$, для которого выполняются условия:
    $$
    \operatorname{Ker} \psi = \operatorname{Im} \varphi
    $$
    $$
    \operatorname{Im} \psi = \operatorname{Ker} \varphi
    $$
    и записать его матрицу в паре стандартных базисов.

    Для начала найдём базис ядра $\phi$:
    $$\mat{22 & 25 & 8 & -10 & 9 &|&0\\
        6 & 4 & -1 & 1 & 5 &|&0\\
        -6 & -11 & -7 & 19 & 2 &|&0\\
        -1 & 1 & 2 & 3 & -2&|&0 \\
        3 & 6 & 4 & -3 & -1&|&0} \implies \text{УСВ: } 
    \mat{1 & 0 & 0 & -101 & -5 & | & 0 \\
        0 & 1 & 0 & 124 & 7 & | & 0 \\
        0 & 0 & 1 & -111 & -7 & | & 0 \\
        0 & 0 & 0 & 0 & 0 & | & 0 \\
        0 & 0 & 0 & 0 & 0 & | & 0}$$
    $$\mat{x_1\\x_2\\x_3\\x_4\\x_5} = \mat{101\\-124\\111\\1\\0}x_4 + \mat{5\\-7\\7\\0\\1}x_5$$  
    Базис ядра:
    $$v_1 = \mat{101\\-124\\111\\1\\0}, \quad v_2 = \mat{5\\-7\\7\\0\\1}$$
    Теперь дополним базис ядра до $\RR^5$, для этого нужно взять векторы:
    $$\mat{1\\0\\0\\0\\0}, \quad \mat{0\\1\\0\\0\\0}, \quad \mat{0\\0\\1\\0\\0}$$
    Следовательно, базис образа:
    $$\mat{22\\6\\-6\\-1\\3}, \mat{25\\4\\-11\\1\\6}, \mat{8\\-1\\-7\\2\\4}$$
    Пусть:
    $$\psi(x) = f_1(x)v_1 + f_2(x)v_2$$
    где $f_1, f_2$ - какие-то линейные функционалы.

    При таком определении отображения $\psi$ у нас выполняется :
    $$\Ker \phi = \Im \phi = \langle v_1, v_2\rangle$$
    (равенство т.к. $\dim \Ker \phi = \dim \Im \phi$)

    Чтобы добиться условия $\Ker \psi = \Im \phi$ нужно узнать при каких $f_1, f_2$
    выполняется:
    $$f_1(x)v_1 + f_2(x)v_2 = 0 \quad \forall x \in \Im \phi$$
    Т.к. $v_1, v_2$ - ЛНЗ, то это равносильно:
    $$\case{f_1(x) = 0\\ f_2(x) =0} \forall x \in \Im \phi$$
    Найдём базис всех таких функционалов $f(x) = 0$. 
    Для каждого функционала должно выполняться:
    $$\case{
        (a, b, c, d, e)\mat{22\\6\\-6\\-1\\3} = 0\\
        (a, b, c, d, e)\mat{25\\4\\-11\\1\\6} = 0\\
        (a, b, c, d, e)\mat{8\\-1\\-7\\2\\4}= 0
    }\lr \case{
        22a+6b-6c-d+3e = 0\\ 
        25a+4b-11c+d+6e = 0\\
        8a-b-7c+2d+4e=0
    }\lr$$
    $$\lr \mat{
        22 &6 & -6 & -1 & 3 & | & 0\\
        25 & 4 & -11 & 1 & 6 & | & 0\\
        8 & -1 & -7 & 2 & 4 & | & 0 
    } \implies \text{УСВ: } \mat{1 & 0 & 0 & \frac{1}{2} & \frac{1}{2} & 0 \\
    0 & 1 & 0 & -\frac{3}{2} & -\frac{7}{6} & 0 \\
    0 & 0 & 1 & \frac{1}{2} & \frac{1}{6} & 0}$$
    $$\mat{a\\b\\c\\d\\e} = 
    \mat{-\f{1}{2}d-\f{1}{2}e\\\f{3}{2}d+\f{7}{6}e\\-\f{1}{2}d-\f{1}{6}e\\d\\e} = 
    \mat{-\f{1}{2}\\\f{3}{2}\\-\f{1}{2}\\1\\0}d + 
    \mat{-\f{1}{2}\\\f{7}{6}\\-\f{1}{6}\\0\\1}e$$
    Пусть:
    $$f_1 = \mat{-\f{1}{2}\\\f{3}{2}\\-\f{1}{2}\\1\\0}*2 = \mat{-1\\3\\-1\\2\\0}$$
    $$f_2 = \mat{-\f{1}{2}\\\f{7}{6}\\-\f{1}{6}\\0\\1}*6 = \mat{-3\\7\\-1\\0\\6}$$
    Такие функционалы будут удовлетворять условию:
    $$\case{f_1(x) = 0\\ f_2(x) =0} \forall x \in \Im \phi$$
    Чтобы найти искомую матрицу $A$ отображения $\psi$, нужно решить уравнение на $A$:
    $$Ax = (f_1)^T x v_1 + (f_2)^T x v_2 \quad \text{ - верно $\forall x \in \RR^5$}$$
    Подставим $x = (1, 0, 0, 0, 0)^T$:
    $$A^{(1)} = (f_1^T)^{(1)} v_1 + (f_2^T)^{(1)} v_2 
    = -1\cdot \mat{101\\-124\\111\\1\\0}-3\cdot \mat{5\\-7\\7\\0\\1}=
    \mat{ -116 \\ 145 \\ -132 \\ -1 \\ -3}$$
    Аналогично посчитаем 2, 3, 4, 5 столбцы:
    $$A^{(2)} = (f_1^T)^{(2)} v_1 + (f_2^T)^{(2)} v_2 
    = 3\cdot \mat{101\\-124\\111\\1\\0}+7\cdot \mat{5\\-7\\7\\0\\1}=
    \mat{ 338 \\ -421 \\ 382 \\ 3 \\ 7}$$
    $$A^{(3)} = (f_1^T)^{(3)} v_1 + (f_2^T)^{(3)} v_2 
    = -1\cdot \mat{101\\-124\\111\\1\\0}-1\cdot \mat{5\\-7\\7\\0\\1}=
    \mat{ -106 \\ 131 \\ -118 \\ -1 \\ -1 }$$
    $$A^{(4)} = (f_1^T)^{(4)} v_1 + (f_2^T)^{(4)} v_2 
    = 2\cdot \mat{101\\-124\\111\\1\\0}+0\cdot \mat{5\\-7\\7\\0\\1}=
    \mat{202 \\ -248 \\ 222 \\ 2 \\ 0 }$$
    $$A^{(5)} = (f_1^T)^{(5)} v_1 + (f_2^T)^{(5)} v_2 
    = 0\cdot \mat{101\\-124\\111\\1\\0}+6\cdot \mat{5\\-7\\7\\0\\1}=
    \mat{30 \\ -42 \\ 42 \\ 0 \\ 6}$$
    В итоге получается матрица:
    $$A = \mat{-116 & 338 & -106 & 202 & 30 \\
    145 & -421 & 131 & -248 & -42 \\
    -132 & 382 & -118 & 222 & 42 \\
    -1 & 3 & -1 & 2 & 0 \\
    -3 & 7 & -1 & 0 & 6}$$
    \textbf{Ответ:} $\mat{-116 & 338 & -106 & 202 & 30 \\
    145 & -421 & 131 & -248 & -42 \\
    -132 & 382 & -118 & 222 & 42 \\
    -1 & 3 & -1 & 2 & 0 \\
    -3 & 7 & -1 & 0 & 6}$\\

    \item[\textbf{№4}]Линейное отображение
     $\varphi: \mathbb{R}^{5} \rightarrow \mathbb{R}^{4}$
      имеет в базисах 
      $e=\left(e_{1}, e_{2}, e_{3}, e_{4}, e_{5}\right)$ и 
      $f=\left(f_{1}, f_{2}, f_{3}, f_{4}\right)$ матрицу
    $$
    A=\left(\begin{array}{ccccc}
    -17 & -15 & 14 & 15 & 18 \\
    10 & 6 & -3 & -3 & -7 \\
    34 & 9 & 1 & 6 & -14 \\
    22 & 3 & 9 & 12 & -4
    \end{array}\right)
    $$
    
    Найдём базисы пространств $\mathbb{R}^{5}$ и $\mathbb{R}^{4}$, 
    в которых матрица отображения $\varphi$ имеет диагональный вид $D$
    
    Найдём базис ядра:
    $$\mat{    -17 & -15 & 14 & 15 & 18 \\
    10 & 6 & -3 & -3 & -7 \\
    34 & 9 & 1 & 6 & -14 \\
    22 & 3 & 9 & 12 & -4} \implies \text{УСВ: } \mat{1 & 0 & 0 & \frac{18}{53} & -\frac{17}{53} \\
    0 & 1 & 0 & -\frac{37}{53} & -\frac{21}{53} \\
    0 & 0 & 1 & \frac{39}{53} & \frac{25}{53} \\
    0 & 0 & 0 & 0 & 0}$$
    $$\mat{x_1\\x_2\\x_3\\x_4\\x_5} = 
    \mat{-\frac{18}{53}x_4+\frac{17}{53}x_5\\
    \frac{37}{53}x_4+\frac{21}{53}x_5
    \\-\frac{39}{53}x_4-\frac{25}{53}x_5
    \\x_4\\x_5} = \mat{-\frac{18}{53}\\
    \frac{37}{53}
    \\-\frac{39}{53}
    \\1\\0}x_4 + \mat{\frac{17}{53}\\
    \frac{21}{53}
    \\-\frac{25}{53}
    \\0\\1}x_5$$
    Координаты базиса ядра:
    $$e_4' = \mat{-18\\37\\-39\\53\\0}, e_5' = \mat{17\\21\\-25\\0\\53}$$
    Дополним до $\RR^4$:
    $$\mat{1\\0\\0\\0}, \mat{0\\1\\0\\0}, \mat{0\\0\\1\\0}$$
    Следовательно, координаты образа:
    $$f_1' = \mat{-17\\10\\34\\22}, f_2' = \mat{-15\\6\\9\\3}, f_3'= \mat{14\\-3\\1\\9}$$
    Дополним их до базиса $\RR^4$:
    $$f_4' = \mat{0\\0\\0\\1}$$
    Пусть:
    $$e_1' = \mat{1\\0\\0\\0\\0}, e_2' = \mat{0\\1\\0\\0\\0}, 
    e_3' = \mat{0\\0\\1\\0\\0}$$
    Тогда:
    $$\phi(e_1') = f_1', \quad \phi(e_2') = f_2', \quad \phi(e_3') = f_3'$$
    Получаем два новых базиса:
    $$\mat{e_1'\\ e_2'\\ e_3'\\ e_4'\\ e_5'} =
     \mat{e_1\\ e_2\\ e_3\\ -18e_1+37e_2-39e_3+53e_4 \\ 17e_1+21e_2-25e_3+53e_5}$$
    $$\mat{f_1'\\ f_2'\\ f_3'\\ f_4'} = 
    \mat{-17f_1+10f_2+34f_3+22f_4\\-15f_1+6f_2+9f_3+3f_4 
    \\ 14f_1 -3 f_2 + f_3 + 9f_4 \\ f_4}$$
    \[
     D = \begin{pmatrix}
     1 & 0 & 0 & 0 & 0 \\
     0 & 1 & 0 & 0 & 0 \\
     0 & 0 & 1 & 0 & 0 \\
     0 & 0 & 0 & 0 & 0
     \end{pmatrix}
     \]
    Соответствующие матрицы перехода:
    $$
    C_{1}=\begin{pmatrix}
    -17 & -15 & 14 & 0 \\
    10 & 6 & -3 & 0 \\
    34 & 9 & 1 & 0 \\
    22 & 3 & 9 & 1
    \end{pmatrix}, \quad C_{2}=\begin{pmatrix}
        1 & 0 & 0 & -18 & 17 \\
        0 & 1 & 0 & 37 & 21 \\
        0 & 0 & 1 & -39 & -25 \\
        0 & 0 & 0 & 53 & 0 \\
        0 & 0 & 0 & 0 & 53
        \end{pmatrix}
    $$\\
    
    \item[\textbf{№5}]Пусть $ V = \mathbb{R}[x]_{\leqslant 2} $
     - пространство многочленов степени не выше 2 от переменной $ x $ 
     с действительными коэффициентами. 
     Пусть $ \left(\varepsilon_{1}, \varepsilon_{2}, \varepsilon_{3}\right) $ 
     - базис пространства $ V^{*} $, двойственный к базису
    $$
    p_1 = -1 + 3 x^{2}, \quad p_2 = 6 - 19 x + x^{2}, \quad p_3 = -3 + 11 x - 3 x^{2}
    $$
    пространства $ V $, а $ \left(f_{1}, f_{2}, f_{3}\right) $
    - базис пространства $ V $, для которого двойственным является базис 
    $ \left(\rho_{1}, \rho_{2}, \rho_{3}\right) $ пространства $ V^{*} $, где
    $$
    \rho_{1}(f) = f(1), \quad \rho_{2}(f) = f^{\prime}(-1), \quad \rho_{3}(f) = 
    \frac{3}{2} \int_{0}^{2} f(x) d x.
    $$
    Рассмотрим линейную функцию $ \alpha \in V^{*} $, имеющую в базисе 
    $ \left(\varepsilon_{1}, \varepsilon_{2}, \varepsilon_{3}\right) $ 
    координаты $ (-1, 1, 2) $, и многочлен $ h \in V $, имеющий в базисе 
    $ \left(f_{1}, f_{2}, f_{3}\right) $ координаты $ (-1, 1, 2) $.
    Найдём значение $ \alpha(h) $.

    Для начала найдём базис $f_1, f_2, f_3$. 
    Пусть \( f_j = a_j + b_j x + c_j x^2 \). Условия двойственности:
    \[
    \rho_i(f_j) = \delta_{ij}.
    \]
    
    Для \( f_1 \):
    \[ \rho_1(f_1) = f_1(1) = a_1 + b_1 + c_1 = 1 \]
    \[ \rho_2(f_1) = f_1'(-1) = b_1 + 2c_1(-1) = b_1 - 2c_1 = 0 \]
    \[ \rho_3(f_1) = \frac{3}{2} \int_{0}^{2} (a_1 + b_1 x + c_1 x^2) 
    dx = \frac{3}{2} \left(2a_1 + 2b_1 + \frac{8}{3}c_1 \right) = 
    3a_1 + 3b_1 + 4c_1 = 0 \]
    Получаем:
    \[
    \begin{cases}
    a_1 + b_1 + c_1 = 1 \\
    b_1 - 2c_1 = 0 \\
    3a_1 + 3b_1 + 4c_1 = 0
    \end{cases}
    \implies f_1 = 10 - 6x - 3x^2 \]
    Аналогично для \( f_2 \) и \( f_3 \):
    \[ f_2 = -1 + x \]
    \[ f_3 = -3 + 2x + x^2 \]
    Теперь выразим \( h \) через базис $f$.

    Координаты \( h \) в базисе \( (f_1, f_2, f_3) \): \( (-1, 1, 2) \).
    \[
    h = -f_1 + f_2 + 2f_3 = - (10 - 6x - 3x^2) + (-1 + x) + 2(-3 + 2x + x^2)=
    \]\[= -17 + 11x + 5x^2
    \]

    Выразим \( \varepsilon_i \) через двойственный базис.

    Стандартный двойственный базис \( (e^1, e^2, e^3) \):

    \[ e^1(a + bx + cx^2) = a \]
    \[ e^2(a + bx + cx^2) = b \]
    \[ e^3(a + bx + cx^2) = c \]

    Матрица перехода от стандартного базиса к базису \( p_1, p_2, p_3 \):
    \[
    P = \begin{pmatrix}
    -1 & 6 & -3 \\
    0 & -19 & 11 \\
    3 & 1 & -3
    \end{pmatrix}
    \]
    Обратная матрица \( P^{-1} \) позволяет выразить \( \varepsilon_i \) 
    через \( e^1, e^2, e^3 \)
    \[
    P^{-1} = \frac{1}{19} \begin{pmatrix}
    46 & 33 & 57 \\
    15 & 12 & 19 \\
    9 & 11 & 19
    \end{pmatrix}
    \]
    Тогда:
    \[
    \varepsilon_1 = -\frac{46}{19}e^1 - \frac{15}{19}e^2 - \frac{9}{19}e^3
    \]
    \[
    \varepsilon_2 = -\frac{33}{19}e^1 - \frac{12}{19}e^2 - \frac{11}{19}e^3
    \]
    \[
    \varepsilon_3 = -3e^1 - e^2 - e^3.
    \]
    Выразим \( \alpha \) через стандартный базис.

    Координаты \( \alpha \): \( (-1, 1, 2) \) в базисе 
    \( (\varepsilon_1, \varepsilon_2, \varepsilon_3) \):
    \[
    \alpha = -\varepsilon_1 + \varepsilon_2 + 2\varepsilon_3.
    \]
    Подставляем выражения \( \varepsilon_i \):
    \[
    \alpha = -\left(-\frac{46}{19}e^1 - \frac{15}{19}e^2 -
     \frac{9}{19}e^3\right) + \left(-\frac{33}{19}e^1 -
      \frac{12}{19}e^2 - \frac{11}{19}e^3\right) + 2\left(-3e^1 - e^2 - e^3\right)
    \]
    \[
    \alpha = -\frac{101}{19}e^1 - \frac{35}{19}e^2 - \frac{40}{19}e^3
    \]
    Вычислим \( \alpha(h) \)

    Многочлен \( h = -17 + 11x + 5x^2 \), коэффициенты: \( a = -17 \), 
    \( b = 11 \), \( c = 5 \)
    \[
    \alpha(h) = -\frac{101}{19} \cdot (-17) + \left(-\frac{35}{19}\right)
     \cdot 11 + \left(-\frac{40}{19}\right) \cdot 5
    \]
    Вычисляем:
    \[
    \alpha(h) = \frac{1717}{19} - \frac{385}{19} - \frac{200}{19} =
     \frac{1717 - 385 - 200}{19} = \frac{1132}{19}
    \]
    \textbf{Ответ: } $\frac{1132}{19}$
        
\end{enumerate}
\end{document}