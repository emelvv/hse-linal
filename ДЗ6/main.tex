\documentclass[a4paper]{article}
\usepackage{setspace}
\usepackage[T2A]{fontenc} %
\usepackage[utf8]{inputenc} % подключение русского языка
\usepackage[russian]{babel} %
\usepackage[14pt]{extsizes}
\usepackage{mathtools}
\usepackage{graphicx}
\usepackage{fancyhdr}
\usepackage{amssymb}
\usepackage{amsmath, amsfonts, amssymb, amsthm, mathtools}
\usepackage{tikz}

\usetikzlibrary{positioning}
\setstretch{1.3}

\newcommand{\mat}[1]{\begin{pmatrix} #1 \end{pmatrix}}
\renewcommand{\det}[1]{\begin{vmatrix} #1 \end{vmatrix}}
\renewcommand{\f}[2]{\frac{#1}{#2}}
\newcommand{\dspace}{\space\space}
\newcommand{\s}[2]{\sum\limits_{#1}^{#2}}
\newcommand{\sq}[1]{\left[ {#1} \right]}
\newcommand{\gath}[1]{\left[ \begin{array}{@{}l@{}} #1 \end{array} \right.}
\newcommand{\case}[1]{\begin{cases} #1 \end{cases}}
\newcommand{\ts}{\text{\space}}
\newcommand{\lm}[1]{\lim\limits_{#1}}

\newcommand{\lr}{\Leftrightarrow}
\renewcommand{\r}{\Rightarrow}
\newcommand{\rr}{\rightarrow}
\renewcommand{\geq}{\geqslant}
\renewcommand{\leq}{\leqslant}
\newcommand{\RR}{\mathbb{R}}
\newcommand{\CC}{\mathbb{C}}
\newcommand{\QQ}{\mathbb{Q}}
\newcommand{\ZZ}{\mathbb{Z}}
\newcommand{\VV}{\mathbb{V}}
\newcommand{\NN}{\mathbb{N}}

\DeclarePairedDelimiter\abs{\lvert}{\rvert} %
\makeatletter                               % \abs{}
\let\oldabs\abs                             %
\def\abs{\@ifstar{\oldabs}{\oldabs*}}       %

\begin{document}

\section*{Домашняя работа №6 (Линейная алгебра)}
{\large Емельянов Владимир, ПМИ гр №247}\\\\
\begin{enumerate}
    \item[\textbf{1.}]
    \begin{enumerate}
        \item[1.1.] $\det{5 & 4 \\ 7 & 3} = 5*3 - 7*4 = 15 - 28 = -13$
        \item[1.2.] $\det{2 & 1 &  3 \\ 5 & 3 & 2 \\ 1 & 4 & 3} = 18 + 2 + 60 - 9 - 15-16 = 80 - 40 = 40$
        \item[1.3.] $\det{n+1 & n \\ n & n-1} = n^2-1-n^2 = -1$
    \end{enumerate}

    \item[\textbf{2.}] Для $a_{62}a_{i5}a_{33}a_{k4}a_{46}a_{21}$ мы используем перестановку:
    $$\sigma = \mat{i & 2 & 3 & 4 & k & 6 \\ 5 & 1 & 3 & 6 & 4 & 2} \r i,k \in \{1, 5\}$$
    Пусть $i=1, \ts k = 5$, тогда число инверсий:
    $$4+ 0 + 1 + 2 + 1 = 8 \r sgn(\sigma) = 1$$
    Пусть $k=1, \ts i = 5$, тогда число инверсий:
    $$3+ 0 + 1 + 2 + 1 = 7 \r sgn(\sigma) = -1$$
    \textbf{Ответ: } $k = 1, \ts i = 5$
    
    \item[\textbf{3.}]$A \in \text{Mat}_n$
    $$\mat{A^{(1)} & A^{(2)} & \dots & A^{(n)}}$$
    $$A^{(1)} \leftrightarrow A^{(2)}$$
    $$\mat{A^{(2)} & A^{(1)} & \dots & A^{(n)}}$$
    $$A^{(1)} \leftrightarrow A^{(3)}$$
    $$\mat{A^{(2)} & A^{(3)} & A^{(1)} & \dots & A^{(n)}}$$
    $$\dots$$
    $$A^{(1)} \leftrightarrow A^{(n)}$$
    $$\mat{A^{(2)} & A^{(3)} & A^{(4)} & \dots & A^{(n)} & A^{(1)}}$$
    После этого определитель стал $(-1)^{n-1}detA$\\
    Теперь:
    $$A^{(2)} \leftrightarrow A^{(3)}$$
    $$\dots$$
    $$A^{(2)} \leftrightarrow A^{(n-1)}$$
    После этого определитель стал $(-1)^{n-1}(-1)^{n-2}detA$\\
    Повторим так для каждого столбца и получим:
    $$\mat{A^{(n)} & \dots & A^{(2)} &A^{(1)}}$$
    При этом определитель стал:
    $$(-1)^{n-1}(-1)^{n-2}\dots(-1)^1detA =$$
    $$= (-1)^{1 + 2 + 3 + \dots + (n-1)}detA = (-1)^{\f{1+(n-1)}{2}\cdot(n-1)}detA = $$
    $$=(-1)^{\f{n(n-1)}{2}}detA$$
    \textbf{Ответ: } он умножится на $(-1)^{\f{n(n-1)}{2}}$

    \item[\textbf{4.}]$A \in \text{Mat}_n$
    $$A = \mat{A_{(1)} \\ A_{(2)} \\ \dots \\ A_{(n)}} \r \det{A_{(1)} - A_{(2)}\\ A_{(2)}- A_{(3)} \\ \dots \\ A_{(n)} - A_{(1)}} = \det{A_{(1)} - A_{(2)}\\ A_{(2)}- A_{(3)} \\ \dots \\ A_{(n)}} -\det{A_{(1)} - A_{(2)}\\ A_{(2)}- A_{(3)} \\ \dots \\ A_{(1)}} =$$
    $$=detA - \det{A_{(1)} - A_{(2)}\\ A_{(2)}- A_{(3)} \\ \dots \\A_{(n-1)}- A_{(n)} \\ A_{(1)}} = detA - (-1)^{n-1} \det{A_{(1)} \\ A_{(1)} - A_{(2)}\\ A_{(2)}- A_{(3)} \\ \dots \\ A_{(n-1)} - A_{(n)}} = $$
    $$=detA - (-1)^{n-1} (-1)^{n-1}\det{A_{(1)} \\ A_{(2)} - A_{(1)}\\ A_{(3)}- A_{(2)} \\ \dots \\ A_{(n)} - A_{(n-1)}} = detA - (-1)^{n-1} (-1)^{n-1}detA =$$
    $$=detA - (-1)^{2n-2}detA = (1- (-1)^{2n-2})detA = (1 +(-1)^{2n-1})detA$$
    \textbf{Ответ: } он умножается на $1 +(-1)^{2n-1}$

    \item[\textbf{5.}]
    \begin{enumerate}
        \item[5.1.]$detA=\det{0 & 0 & 4 \\ -2 & -11 & 7 \\ 10 & 80 & -20}$
        $$A[0], A[2] = A[2], A[0] \r \det{10 & 80 & -20 \\ -2 & -11 & 7 \\ 0 & 0 & 4} = -detA$$
        $$A[0] = A[0]/10 \r \det{1 & 8 & -2 \\ -2 & -11 & 7 \\ 0 & 0 & 4}= -10detA$$
        $$A[1] = A[1] + 2*A[0] \r \det{1 & 8 & -2 \\ 0 & 5 & 3 \\ 0 & 0 & 4}=-10detA = 20 \r$$
        $$\r detA = -200$$
        \textbf{Ответ: }$200$\\

        \item[5.2.]$detA = \det{1 & 2 & 3 \\ 4 & 5 & 6 \\ 7 & 8 & 9}$
        \[\begin{cases}A[1] = A[1]-4*A[0]\\ A[2] = A[2]-7*A[0]\end{cases} \r \det{1 & 2 & 3 \\ 0 & -3 & -6 \\ 0 & -6 & -12} = detA\]
        $$A[2] = A[2]-2*A[1] \r \det{1 & 2 & 3 \\ 0 & -3 & -6 \\ 0 & 0 & 0} = detA = 0$$
        \textbf{Ответ: }$0$\\

        \item[5.3]$\det{0 & 1 & 3 & 5 \\ 1& 0 & 1 & 3 \\ 3 & 1 & 0 & 1 \\ 5 & 3 & 1 & 0}$
        $$\case{A[2] = A[2] - 3*A[1]\\
        A[3] = A[3] - 5*A[1]} \r \det{0 & 1 & 3 & 5 \\ 1 & 0 & 1 & 3 \\ 0 & 1 & -3 & -8 \\ 0 & 3 & -4 & -15} = detA$$
        $$A[0] = A[0] - A[2] \r \det{0 & 0 & 6 & 13 \\ 1 & 0 & 1 & 3 \\ 0 & 1 & -3 & -8 \\ 0 & 3 & -4 & -15} = detA$$
        $$A[3] = A[3] - 3*A[2] \r \det{0 & 0 & 6 & 13 \\ 1 & 0 & 1 & 3 \\ 0 & 1 & -3 & -8 \\ 0 & 0 & 5 & 9} = detA$$
        $$A[3] = A[3]/5 \r\det{0 & 0 & 6 & 13 \\ 1 & 0 & 1 & 3 \\ 0 & 1 & -3 & -8 \\ 0 & 0 & 1 & \frac{9}{5}} = \f{1}{5}detA$$
        $$A[0] = A[0] - 6*A[3] \r \det{0 & 0 & 0 & \frac{11}{5} \\ 1 & 0 & 1 & 3 \\ 0 & 1 & -3 & -8 \\ 0 & 0 & 1 & \frac{9}{5}} = \f{1}{5}detA$$
        $$\case{A[0], A[1] = A[1], A[0]\\
        A[1], A[2]= A[2], A[1]\\
        A[2], A[3] = A[3], A[2]} \r \det{1 & 0 & 1 & 3 \\ 0 & 1 & -3 & -8 \\0 & 0 & 1 & \frac{9}{5} \\ 0 & 0 & 0 & \frac{11}{5}} = -\f{1}{5}detA = \f{11}{5} \r$$
        $$\r detA = -11$$
        \textbf{Ответ: } $-11$
    \end{enumerate}
    
    \item[\textbf{6.}]$\det{1 & 2 & 175 & -38\\ 3 & 4 & 137 & -91\\ 0 & 0 & 5 & -1 \\ 0 & 0 & 3 & -2}$
    $$A[1] = A[1]-3*A[0] \r \det{1 & 2 & 175 & -38 \\ 0 & -2 & -388 & 23 \\ 0 & 0 & 5 & -1 \\ 0 & 0 & 3 & -2} = detA$$
    $$A[2] = A[2]/5 \r \det{1 & 2 & 175 & -38 \\ 0 & -2 & -388 & 23 \\ 0 & 0 & 1 & -\frac{1}{5} \\ 0 & 0 & 3 & -2} =detA/5$$
    $$A[3] = A[3]-A[2]*3 \r \det{1 & 2 & 175 & -38 \\ 0 & -2 & -388 & 23 \\ 0 & 0 & 1 & -\frac{1}{5} \\ 0 & 0 & 0 & -\frac{7}{5}}=detA/5 = \f{14}{5} \r $$
    $$\r detA = 14$$
    \textbf{Ответ:} $14$

    \item[\textbf{7.}]$\det{a_1 + b_1 & a_1 + b_2 & \ldots & a_1 + b_n\\
	a_2 + b_1 & a_2 + b_2 & \ldots & a_2 + b_n\\
	\ldots & \ldots & \ldots & \ldots\\
	a_n + b_1 & a_n + b_2 & \ldots & a_n + b_n}$
    $$A = \mat{a_1 & a_1 & \ldots & a_1\\
	a_2 & a_2  & \ldots & a_2\\
	\ldots & \ldots & \ldots & \ldots\\
	a_n  & a_n& \ldots & a_n}\ts
    B = \mat{b_1 & b_1 & \ldots & b_1\\
	b_2 & b_2  & \ldots & b_2\\
	\ldots & \ldots & \ldots & \ldots\\
	b_n  & b_n& \ldots & b_n}$$ 
    $$\det{a_1 + b_1 & a_1 + b_2 & \ldots & a_1 + b_n\\
	a_2 + b_1 & a_2 + b_2 & \ldots & a_2 + b_n\\
	\ldots & \ldots & \ldots & \ldots\\
	a_n + b_1 & a_n + b_2 & \ldots & a_n + b_n} = \det{A_{(1)} + B_{(1)} \\ A_{(2)} + B_{(2)}\\ \dots \\ A_{(n)} + B_{(n)}} = \det{A_{(1)} \\ A_{(2)} + B_{(2)}\\ \dots \\ A_{(n)} + B_{(n)}} + \det{B_{(1)} \\ A_{(2)} + B_{(2)}\\ \dots \\ A_{(n)} + B_{(n)}} = $$
    $$= \det{A_{(1)} \\ A_{(2)}\\  A_{(3)} + B_{(3)} \\ \dots \\ A_{(n)} + B_{(n)}} + \det{A_{(1)} \\ B_{(2)}\\ A_{(3)} + B_{(3)} \\ \dots \\ A_{(n)} + B_{(n)}} + \det{B_{(1)} \\ A_{(2)}\\ A_{(3)} + B_{(3)}\\ \dots \\ A_{(n)} + B_{(n)}} + \det{B_{(1)} \\ B_{(2)}\\ A_{(3)} + B_{(3)}\\ \dots \\ A_{(n)} + B_{(n)}}$$
    В итоге получиться сумма определителей всех возможных перестановок строк из A и из B, при этом каждая матрица где будет хотя бы 2 строки из A или хотя бы две строки из B уйдут, т.к. определитель будет равен нулю, потому что все строки в A и в B одинаковые. Следовательно, так как $n\geq 3$, то в каждой такой матрице будет хотя бы две одинаковые строки, значит вся сумма равна нулю\\
    \textbf{Ответ:} $0$
\end{enumerate}
\end{document}