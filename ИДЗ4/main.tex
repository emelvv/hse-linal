\documentclass[a4paper]{article}
\usepackage{setspace}
\usepackage[T2A]{fontenc} %
\usepackage[utf8]{inputenc} % подключение русского языка
\usepackage[russian]{babel} %
\usepackage[12pt]{extsizes}
\usepackage{mathtools}
\usepackage{graphicx}
\usepackage{fancyhdr}
\usepackage{amssymb}
\usepackage{amsmath, amsfonts, amssymb, amsthm, mathtools}
\usepackage{tikz}

\usetikzlibrary{positioning}
\setstretch{1.3}

\newcommand{\mat}[1]{\begin{pmatrix} #1 \end{pmatrix}}
\renewcommand{\det}[1]{\begin{vmatrix} #1 \end{vmatrix}}
\renewcommand{\f}[2]{\frac{#1}{#2}}
\newcommand{\dspace}{\space\space}
\newcommand{\s}[2]{\sum\limits_{#1}^{#2}}
\newcommand{\mul}[2]{\prod_{#1}^{#2}}
\newcommand{\sq}[1]{\left[ {#1} \right]}
\newcommand{\gath}[1]{\left[ \begin{array}{@{}l@{}} #1 \end{array} \right.}
\newcommand{\case}[1]{\begin{cases} #1 \end{cases}}
\newcommand{\ts}{\text{\space}}
\newcommand{\lm}[1]{\underset{#1}{\lim}}
\newcommand{\suplm}[1]{\underset{#1}{\overline{\lim}}}
\newcommand{\inflm}[1]{\underset{#1}{\underline{\lim}}}

\renewcommand{\phi}{\varphi}
\newcommand{\lr}{\Leftrightarrow}
\renewcommand{\r}{\Rightarrow}
\newcommand{\rr}{\rightarrow}
\renewcommand{\geq}{\geqslant}
\renewcommand{\leq}{\leqslant}
\newcommand{\RR}{\mathbb{R}}
\newcommand{\CC}{\mathbb{C}}
\newcommand{\QQ}{\mathbb{Q}}
\newcommand{\ZZ}{\mathbb{Z}}
\newcommand{\VV}{\mathbb{V}}
\newcommand{\NN}{\mathbb{N}}
\newcommand{\OO}{\underline{O}}
\newcommand{\oo}{\overline{o}}


\DeclarePairedDelimiter\abs{\lvert}{\rvert} %
\makeatletter                               % \abs{}
\let\oldabs\abs                             %
\def\abs{\@ifstar{\oldabs}{\oldabs*}}       %

\begin{document}

\section*{ИДЗ №4, 11 вариант (Линейная алгебра)}
 {\large Емельянов Владимир, ПМИ гр №247}\\\\
\begin{enumerate}
    \item[\textbf{№1}]$A = \left(\begin{array}{cc}
        17 + 19i & 8 + 8i \\
        -40 - 40i & -19 - 17i
        \end{array}\right) \in M_{2 \times 2}(\CC)$\\\\
    Чтобы найти значения $ x \in \mathbb{C} $, при которых матрица $ A - x E $ необратима, необходимо найти такие $ x $, при которых определитель матрицы равен нулю. Здесь $ E $ — единичная матрица $ 2 \times 2 $.

    Матрица $ A - x E $ будет выглядеть следующим образом:

    $$
    A - x E = \left(\begin{array}{cc}
    (17 + 19i) - x & 8 + 8i \\
    -40 - 40i & (-19 - 17i) - x
    \end{array}\right)
    $$
    
    Теперь вычислим определитель этой матрицы:

    $$
    det(A - x E) = det\left(\begin{array}{cc}
    (17 + 19i) - x & 8 + 8i \\
    -40 - 40i & (-19 - 17i) - x
    \end{array}\right)
    $$
    $$
    det(A - x E) = \left(17 + 19i - x\right)\left(-19 - 17i - x\right) - (8 + 8i)(-40 - 40i)
    $$
    Теперь упростим каждую часть:\\
    \begin{enumerate}
        \item[1)]Вычислим $ \left(17 + 19i - x\right)\left(-19 - 17i - x\right) $:\\
        Раскроем скобки:
        $$
        (17 + 19i)(-19 - 17i) - (17 + 19i)x - (-19 - 17i)x + x^2
        $$
        Сначала вычислим $(17 + 19i)(-19 - 17i)$:
        $$
        (17 + 19i)(-19 - 17i) = 17 \cdot (-19) + 17 \cdot (-17i) + 19i \cdot (-19) + 19i \cdot (-17i)
        $$
        $$
        = -323 - 289i - 361i - 323i^2= - 289i - 361i = 0 - 650i = -650i
        $$
        Теперь подставим:
        $$
        -650i - (17 + 19i)x + (19 + 17i)x + x^2
        $$
        Упростим:
        $$
        ab = -650i + x^2 + (-17 + 19 + 17i - 19i)x
        $$
        $$
        = -650i + x^2 + (2 - 2i)x
        $$

        \item[2)]Вычислим $(8 + 8i)(-40 - 40i)$
        Раскроем скобки:
        $$
        (8 + 8i)(-40 - 40i) = 8 \cdot (-40) + 8 \cdot (-40i) + 8i \cdot (-40) + 8i \cdot (-40i)=
        $$
        $$
        = -320 - 320i - 320i - 320i^2 = -320 - 320i - 320i + 320 = -640i
        $$
    \end{enumerate}
    Подставим вычисленные выражения:
    $$det(A - x E) = -650i + x^2 + (2 - 2i)x  +640i = x^2 + (2 - 2i)x  -10i = 0$$
    $$D = (2 - 2i)^2 - 4 \cdot 1 \cdot (-10i)$$
    Вычислим $(2 - 2i)^2 $:
    $$
    (2 - 2i)^2 = 4 - 8i + 4i^2 = 4 - 8i - 4 = -8i.
    $$
    Подставим:
    $$
    D = -8i + 40i = 32i.
    $$
    $$
    \sqrt{D} = \sqrt{32i}.
    $$
    Чтобы найти $\sqrt{32i}$, представим $32i$ в тригонометрической форме. Модуль равен $32$, а аргумент равен $\frac{\pi}{2}$:
    $$
    \sqrt{32i} = 4\sqrt{2} \cdot \left(\cos\frac{\pi}{4} + i\sin\frac{\pi}{4}\right) = 4\sqrt{2} \cdot \left(\frac{\sqrt{2}}{2} + i\frac{\sqrt{2}}{2}\right) = 4 + 4i.
    $$
    Следовательно:
    $$
    x = \frac{-(2 - 2i) \pm (4 + 4i)}{2}=\frac{-2 + 2i \pm (4 + 4i)}{2}
    $$
    $$
    \gath{
        x_1 = \frac{-2 + 2i + 4 + 4i}{2} = \frac{2 + 6i}{2} = 1 + 3i\\
        x_2 = \frac{-2 + 2i - 4 - 4i}{2} = \frac{-6 - 2i}{2} = -3 - i
    }
    $$
    \textbf{Следовательно, матрица $ A - x E $ необратима при:}
    $$
    \gath{
        x_1 = 1 + 3i\\
        x_2 = -3 - i
    }
    $$
    \item[\textbf{№2}]$\sqrt[4]{-\frac{81}{2} - \frac{81\sqrt{3}}{2}i}$\\
    Сначала найдем модуль и аргумент комплексного числа $ z = -\frac{81}{2} - \frac{81 \sqrt{3}}{2} i $.
    $$
    |z| = \sqrt{\left(-\frac{81}{2}\right)^2 + \left(-\frac{81 \sqrt{3}}{2}\right)^2}
    $$
    
    Вычислим каждую часть:

    $$
    \left(-\frac{81}{2}\right)^2 = \frac{6561}{4}
    $$
    
    $$
    \left(-\frac{81 \sqrt{3}}{2}\right)^2 = \frac{6561 \cdot 3}{4} 
    $$

    Сложим:
    $$
    |z| = \sqrt{\frac{6561}{4} + \frac{6561 \cdot 3}{4}} = \sqrt{\frac{6561\cdot 4}{4}} = \sqrt{6561} = 81
    $$

    Найдем аргумент $ \varphi $

    Аргумент $ \varphi $ в данном случае вычисляется по формуле:

    $$
    \varphi =\pi - \arctg\left(\frac{y}{x}\right)
    $$
    так как $z$ лежит в третьей четверти
    $$
    \varphi = \arctg\left(\frac{-\frac{81 \sqrt{3}}{2}}{-\frac{81}{2}}\right)-\pi = \arctg(\sqrt{3})-\pi = \frac{\pi}{3}-\pi = -\f{2\pi}{3}
    $$
    Следовательно $z$ равно:
    $$z = 81(\cos(-\f{2\pi}{3} + i\sin(-\f{2\pi}{3})))$$
    
    Теперь мы можем найти четвертые корни $ w_k $:

    $$
    w_k = \sqrt[4]{|z|}\left(\cos \frac{\varphi + 2\pi k}{4} + i \sin \frac{\varphi + 2\pi k}{4}\right)
    $$
    
    где $ |z| = 81 $ и $ \sqrt[4]{|z|} = \sqrt[4]{81} = 3 $.

    Теперь подставим значения для $ k = 0, 1, 2, 3 $:
    \begin{enumerate}
        \item[1)] $k = 0$\\
        $$
        w_0 = 3 \left( \cos \frac{-\frac{2\pi}{3}}{4} + i \sin \frac{-\frac{2\pi}{3}}{4} \right) = 3 \left( \cos \left(-\frac{\pi}{6}\right) + i \sin \left(-\frac{\pi}{6}\right) \right)
        $$
        $$
        = 3 \left( \frac{\sqrt{3}}{2} - i \frac{1}{2} \right) = \frac{3\sqrt{3}}{2} - \frac{3i}{2}
        $$
        \item[2)]$k=1$\\
        $$
        w_1 = 3 \left( \cos \left(\frac{-\frac{2\pi}{3} + 2\pi}{4}\right) + i \sin \left(\frac{-\frac{2\pi}{3} + 2\pi}{4}\right) \right) =
        $$
        $$
        = 3 \left( \cos \left(\frac{\pi}{3}\right) + i \sin \left(\frac{\pi}{3}\right) \right) = 3 \left( \frac{\sqrt{3}}{2} + i \frac{1}{2} \right) = \frac{3}{2} + \frac{3\sqrt{3}}{2}i
        $$
        \item[3)]$k=2$\\
        $$
        w_2 = 3 \left( \cos \left(\frac{-\frac{2\pi}{3} + 4\pi}{4}\right) + i \sin \left(\frac{-\frac{2\pi}{3} + 4\pi}{4}\right) \right) = 3 \left( \cos \left(\frac{10\pi}{12}\right) + i \sin \left(\frac{10\pi}{12}\right) \right)
        $$
        $$
        = 3 \left( \cos \left(\frac{5\pi}{6}\right) + i \sin \left(\frac{5\pi}{6}\right) \right) = 3 \left( -\frac{\sqrt{3}}{2} + i \frac{1}{2} \right) = -\frac{3\sqrt{3}}{2} + \frac{3i}{2}
        $$
        \item[4)]$k=3$\\
        $$
        w_3 = 3 \left( \cos \left(\frac{-\frac{2\pi}{3} + 6\pi}{4}\right) + i \sin \left(\frac{-\frac{2\pi}{3} + 6\pi}{4}\right) \right) = 
        $$
        $$
        = 3 \left( \cos \left(\frac{4\pi}{3}\right) + i \sin \left(\frac{4\pi}{3}\right) \right) = 3 \left( -\frac{1}{2} - i \frac{\sqrt{3}}{2} \right) = -\frac{3}{2} - \frac{3\sqrt{3}}{2} i
        $$
        
    \end{enumerate}
    \textbf{Ответ: }
    $\gath{
        w_0 = \frac{3\sqrt{3}}{2} - \frac{3i}{2}\\
        w_1 = \frac{3}{2} + \frac{3\sqrt{3}}{2}i\\
        w_2 = -\frac{3\sqrt{3}}{2} + \frac{3i}{2}\\
        w_3 = -\frac{3}{2} - \frac{3\sqrt{3}}{2} i
    }$\\

    \item[\textbf{№3}]$$
        v_{1} = \begin{pmatrix}
        5 \\
        -5 \\
        -3 \\
        -1 \\
        2
        \end{pmatrix}, \quad 
        v_{2} = \begin{pmatrix}
        -45 \\
        46 \\
        26 \\
        9 \\
        -21
        \end{pmatrix}, \quad 
        v_{3} = \begin{pmatrix}
        5 \\
        -4 \\
        -4 \\
        a \\
        3
        \end{pmatrix}
        $$
    Чтобы доказать, что векторы $ v_1, v_2, v_3 $ линейно независимы при всех значениях параметра $ a $, мы можем рассмотреть их линейную комбинацию и показать, что единственным решением является тривиальное решение.
    
    Рассмотрим линейную комбинацию векторов:

    $$
    c_1 v_1 + c_2 v_2 + c_3 v_3 = 0
    $$

    где $ c_1, c_2, c_3 $ — скаляры. Подставим векторы:
    
    $$
    c_1 \begin{pmatrix}
    5 \\
    -5 \\
    -3 \\
    -1 \\
    2
    \end{pmatrix} + c_2 \begin{pmatrix}
    -45 \\
    46 \\
    26 \\
    9 \\
    -21
    \end{pmatrix} + c_3 \begin{pmatrix}
    5 \\
    -4 \\
    -4 \\
    a \\
    3
    \end{pmatrix} = \begin{pmatrix}
    0 \\
    0 \\
    0 \\
    0 \\
    0
    \end{pmatrix}
    $$
Это приводит к системе уравнений:
    $$
    \case{
    5c_1 - 45c_2 + 5c_3 = 0 \\
    -5c_1 + 46c_2 - 4c_3 = 0 \\
    -3c_1 + 26c_2 - 4c_3 = 0 \\
    -c_1 + 9c_2 + ac_3 = 0 \\
    2c_1 - 21c_2 + 3c_3 = 0 \\
    }
    $$
    Перепишем в виде матрицы СЛУ:
    $$
    A = \mat{5 & -45 & 5 &|& 0\\ -5 & 46 & -4 &|&0\\ -3 & 26 & -4&|& 0\\ -1 & 9 & a&|& 0\\ 2 & -21 & 3&|&0}
    \implies \begin{pmatrix}
        1 & -9 & 1 & | & 0 \\
        0 & 1 & 1 & | & 0 \\
        0 & -1 & -1 & | & 0 \\
        0 & 0 & a + 1 & | & 0 \\
        0 & -3 & 1 & | & 0
        \end{pmatrix}
    \implies \begin{pmatrix}
        1 & 0 & 10 & | & 0 \\
        0 & 1 & 1 & | & 0 \\
        0 & 0 & 0 & | & 0 \\
        0 & 0 & a + 1 & | & 0 \\
        0 & 0 & 4 & | & 0
        \end{pmatrix}
        \implies$$
    $$\implies \text{УСВ: } \begin{pmatrix}
        1 & 0 & 0 & | & 0 \\
        0 & 1 & 0 & | & 0 \\
        0 & 0 & 0 & | & 0 \\
        0 & 0 & 0 & | & 0 \\
        0 & 0 & 1 & | & 0
        \end{pmatrix}
        $$
    Следовательно:
    $$
    \case{
        c_1 = 0\\
        c_2 = 0\\
        c_3 = 0
    }\text{ - единственное решение}$$

    Таким образом, векторы $ v_1, v_2, v_3 $ линейно независимы для всех значений $ a $.

    Запишем векторы в матричном виде:
    $$A = \mat{5 & -45 & 5\\ -5 & 46 & -4 \\ -3 & 26 & -4\\ -1 & 9 & a\\ 2 & -21 & 3}$$
    Приведём матрицу $A$ к СВ преобразованиями столбцов:
    $$A = \mat{5 & -45 & 5\\ -5 & 46 & -4 \\ -3 & 26 & -4\\ -1 & 9 & a\\ 2 & -21 & 3} \implies \text{ СВ:} \begin{pmatrix}
        1 & 0 & 0 \\
        0 & 1 & 0 \\
        -\frac{8}{5} & -1 & 0 \\
        0 & 0 & 1 \\
        \frac{-9 - 13a}{5(1 + a)} & -3 & \frac{4}{1 + a}
        \end{pmatrix}$$
    После таких преобразований линейная независимость векторов не пропала.
    Следовательно, чтобы дополнить $v_1, v_2, v_3$ до базиса, нужно добавить векторы:
    $$e_3 = \mat{0\\0\\1\\0\\0}, e_5 = \mat{0\\0\\0\\0\\1}$$
    \textbf{Ответ: } $v1, v2, v3, e3, e5$ - базис
    
    \item[\textbf{№4}]
    Запишем векторы $v1, v2, v3, v4$ в матричном виде:
    $$
   A = \begin{pmatrix}
   30 & 4 & -5 & 0 \\
   16 & 7 & -2 & 1 \\
   25 & 7 & -5 & 2 \\
   32 & 8 & -4 & 0 \\
   18 & 5 & -1 & -1
   \end{pmatrix}
   $$
   \begin{enumerate}
    \item[(а)]
    Приведём матрицу к УСВ преобразованиями строк:
    $$A = \begin{pmatrix}
     30 & 4 & -5 & 0 \\
     16 & 7 & -2 & 1 \\
     25 & 7 & -5 & 2 \\
     32 & 8 & -4 & 0 \\
     18 & 5 & -1 & -1
    \end{pmatrix} \implies \text{ УСВ: } \begin{pmatrix}
        1 & 0 & 0 & -\frac{1}{5} \\
        0 & 1 & 0 & \frac{1}{3} \\
        0 & 0 & 1 & -\frac{14}{15} \\
        0 & 0 & 0 & 0 \\
        0 & 0 & 0 & 0
        \end{pmatrix}$$
    Так как такие преобразования не меняют линейную зависимость векторов, то мы можем утверждать, что вектор $v_4$ выражается через остальные, а набор $v_1, v_2, v_3$ - линейно независим.

    Следовательно, так как $v_1, v_2, v_3$ - ЛНЗ, и: 
    $$v_4 \in \langle v_1, v_2, v_3 \rangle \r \langle v_1, v_2, v_3 \rangle = \langle v_1, v_2, v_3, v_4 \rangle$$
    то $v_1, v_2, v_3$ базис подпространства $U$.

    \item[(б)]Пусть $E$ - матрица, составленная из векторов $v_1, v_2, v_3$\\
    Чтобы проверить принадлежность $u_1$ к $U$ решим уравнение ${Ex= u_1}$:
    $$(E|u_1) = \begin{pmatrix}
        30 & 4 & -5 &|& 8 \\
        16 & 7 & -2  &|& 6\\
        25 & 7 & -5 &|& 11 \\
        32 & 8 & -4  &|& 8\\
        18 & 5 & -1&|& 2
       \end{pmatrix} \implies \text{ УСВ: } \begin{pmatrix}
        1 & 0 & 0 &|& -\frac{1}{5} \\
        0 & 1 & 0 & |&\frac{2}{3} \\
        0 & 0 & 1 & |&-\frac{34}{15} \\
        0 & 0 & 0 & |&0 \\
        0 & 0 & 0 &|& 0
        \end{pmatrix}
        $$
    Следовательно, у уравнения $Ax= u_1$ есть единственное решение, а значит $u_1$ можно представить в виде:
    $$u_1 = -\frac{1}{5}v_1 + \frac{2}{3}v_2 -\frac{34}{15}v_3 $$
   
    Теперь проверим принадлежность $u_2$, аналогично решим уравнение $Ex = u_2$:
    $$(E|u_2)= \begin{pmatrix}
        30 & 4 & -5 &|& 9 \\
        16 & 7 & -2  &|& 6\\
        25 & 7 & -5 &|& 10 \\
        32 & 8 & -4  &|& 9\\
        18 & 5 & -1&|& 4
       \end{pmatrix} \implies \text{ УСВ: } \begin{pmatrix}
        1 & 0 & 0 &|& 0 \\
        0 & 1 & 0 &|& 0 \\
        0 & 0 & 1 &|& 0 \\
        0 & 0 & 0 &|& 1 \\
        0 & 0 & 0 &|& 0
        \end{pmatrix}
        \r \varnothing$$
    Следовательно, у уравнения $Ex = u_2$ нет решений, а значит $u_2$ не выражается через базис $v_1, v_2, v_3$, а значит $u_2 \notin U$\\

    \end{enumerate}

    \item[\textbf{№5}]Запишем СЛУ в виде матрицы, и приведём её к улучшенному ступенчатому виду: 
    $$A =
    \begin{pmatrix}
    13 & 12 & 0 & -12 & -6 & | & 0\\
    17 & 13 & -13 & -17 & -11& | & 0 \\
    11 & 9 & -5 & -9 & -7 & | & 0\\
    17 & 13 & -12 & -14 & -12& | & 0
    \end{pmatrix}
    \implies \text{ УСВ: }\begin{pmatrix}
        1 & 0 & 0 & 12 & -6 &| & 0 \\
        0 & 1 & 0 & -14 & 6 &| & 0 \\
        0 & 0 & 1 & 3 & -1 &| & 0 \\
        0 & 0 & 0 & 0 & 0 & | &0
        \end{pmatrix}$$
    Следовательно, получаем:
    $$\case{
        x_3 = -3x_4+x_5\\
        x_2 = 14x_4-6x_5\\
        x_1 = -12x_4+6x_5
    } \quad x_4, x_5 \in \RR$$
    Общий вид решений:
    $$x = \mat{x_1\\x_2\\x_3\\x_4\\x_5} = \mat{-12x_4+6x_5\\14x_4-6x_5\\-3x_4+x_5\\x_4\\x_5}\quad x_4, x_5 \in \RR$$
    При этом:
    $$x = \mat{-12x_4+6x_5\\14x_4-6x_5\\-3x_4+x_5\\x_4\\x_5} = v_1x_4 + v_2x_5, \quad v_1 = \mat{-12\\14\\-3\\1\\0}, v_2 = \mat{6\\-6\\1\\0\\1}$$
    Следовательно набор векторов $v_1, v_2$ является ФСР для данной системы, а соответственно и базисом множества её решений. Его размерность равна двум.\\
    \textbf{Ответ: }$v_1, v_2$ - базис, $\dim{U} = 2$

\end{enumerate}
\end{document}