\documentclass[a4paper]{article}
\usepackage{setspace}
\usepackage[T2A]{fontenc} %
\usepackage[utf8]{inputenc} % подключение русского языка
\usepackage[russian]{babel} %
\usepackage[14pt]{extsizes}
\usepackage{mathtools}
\usepackage{graphicx}
\usepackage{fancyhdr}
\usepackage{amssymb}
\usepackage{amsmath, amsfonts, amssymb, amsthm, mathtools}
\usepackage{tikz}

\usetikzlibrary{positioning}
\setstretch{1.3}

\newcommand{\mat}[1]{\begin{pmatrix} #1 \end{pmatrix}}
\renewcommand{\det}[1]{\begin{vmatrix} #1 \end{vmatrix}}
\renewcommand{\f}[2]{\frac{#1}{#2}}
\newcommand{\dspace}{\space\space}
\newcommand{\s}[2]{\sum\limits_{#1}^{#2}}
\newcommand{\sq}[1]{\left[ {#1} \right]}
\newcommand{\gath}[1]{\left[ \begin{array}{@{}l@{}} #1 \end{array} \right.}
\newcommand{\case}[1]{\begin{cases} #1 \end{cases}}
\newcommand{\ts}{\text{\space}}
\newcommand{\lm}[1]{\lim\limits_{#1}}

\newcommand{\lr}{\Leftrightarrow}
\renewcommand{\r}{\Rightarrow}
\newcommand{\rr}{\rightarrow}
\renewcommand{\geq}{\geqslant}
\renewcommand{\leq}{\leqslant}
\newcommand{\RR}{\mathbb{R}}
\newcommand{\CC}{\mathbb{C}}
\newcommand{\QQ}{\mathbb{Q}}
\newcommand{\ZZ}{\mathbb{Z}}
\newcommand{\VV}{\mathbb{V}}
\newcommand{\NN}{\mathbb{N}}

\DeclarePairedDelimiter\abs{\lvert}{\rvert} %
\makeatletter                               % \abs{}
\let\oldabs\abs                             %
\def\abs{\@ifstar{\oldabs}{\oldabs*}}       %

\begin{document}

\section*{ИДЗ №2 (Линейная алгебра)}
{\large Емельянов Владимир, ПМИ гр №247}\\\\
\begin{enumerate}
    \item[\textbf{1.}]$\case{ax - 4y = -4\\
    6x+by-9z = 1 \\
    -5x-z=1}$\\
    $$\text{Запишем в виде матрицы СЛУ: }\mat{a & -4 & 0 & | & -4 \\ 6 & b & -9 & | & 1 \\ -5 & 0 & -1 & | & 1}$$
    $$A[0] = A[0]/a \r \mat{1 & \f{-4}{a} & 0 & | & \f{-4}{a} \\ 6 & b & -9 & | & 1 \\ -5 & 0 & -1 & | & 1}$$
    $$A[1] = A[1] - 6*A[0] \r \mat{1 & \f{-4}{a} & 0 & | & \f{-4}{a} \\ 0 & b + \f{24}{a} & -9 & | & \f{a + 24}{a} \\ -5 & 0 & -1 & | & 1}$$
    $$A[2] = A[2] + 5*A[0] \r \mat{1 & \f{-4}{a} & 0 & | & \f{-4}{a} \\ 0 & b + \f{24}{a} & -9 & | & \f{a + 24}{a} \\ 0 & -\f{20}{a} & -1 & | & \f{a - 20}{a}}$$
    $$A[1] = A[1]/(b+24/a) \r \mat{1 & \f{-4}{a} & 0 & | & \f{-4}{a} \\ 0 & 1 & -\f{9a}{a b + 24} & | & \f{a + 24}{a b + 24} \\ 0 & -\f{20}{a} & -1 & | & \f{a - 20}{a}}$$
    $$A[2] = A[2]-A[1]*(-20/a) \r \mat{1 & \f{-4}{a} & 0 & | & \f{-4}{a} \\ 0 & 1 & -\f{9a}{a b + 24} & | & \f{a + 24}{a b + 24} \\ 0 & 0 & \f{-a b - 204}{a b + 24} & | & \f{a b - 20 b + 44}{a b + 24}}$$
    $$A[2] = A[2]/((-a*b - 204)/(a*b + 24)) \r \mat{1 & \f{-4}{a} & 0 & | & \f{-4}{a} \\ 0 & 1 & -\f{9a}{a b + 24} & | & \f{a + 24}{a b + 24} \\ 0 & 0 & 1 & | & \f{-a b + 20 b - 44}{a b + 204}}$$
    $$A[0] = A[0]-A[1]*(-4/a) \r \mat{1 & 0 & -\f{36}{a b + 24} & | & \f{4(1 - b)}{a b + 24} \\ 0 & 1 & -\f{9a}{a b + 24} & | & \f{a + 24}{a b + 24} \\ 0 & 0 & 1 & | & \f{-a b + 20 b - 44}{a b + 204}}$$
    $$A[0] = A[0]-A[2]*(-36/(a*b + 24)) \r \mat{1 & 0 & 0 & | & \f{4(-b - 8)}{a b + 204} \\ 0 & 1 & -\f{9a}{a b + 24} & | & \f{a + 24}{a b + 24} \\ 0 & 0 & 1 & | & \f{-a b + 20 b - 44}{a b + 204}}$$
    $$A[1] = A[1]-A[2]*(-9*a/(a*b + 24)) \r \mat{1 & 0 & 0 & | & \f{4(-b - 8)}{a b + 204} \\ 0 & 1 & 0 & | & \f{4(51 - 2a)}{a b + 204} \\ 0 & 0 & 1 & | & \f{-a b + 20 b - 44}{a b + 204}}$$
    Система имеет 0 корней при:
    $$ab+204 = 0 \r ab = -204$$
    $x = y$ при:
    $$\f{4(-b - 8)}{a b + 204} = \f{4(51 - 2a)}{a b + 204} \r a = \f{b+59}{2}$$
    $x = z$ при:
    $$\f{4(-b - 8)}{a b + 204} = \f{-a b + 20 b - 44}{a b + 204} \r a = 24 - \f{12}{b}$$
    $y = z$ при:
    $$\f{4(51 - 2a)}{a b + 204} = \f{-a b + 20 b - 44}{a b + 204} \r a = \f{20b - 248}{b-8}$$
    $x = y = z$ при:
    $$\f{b+59}{2} = 24 - \f{12}{b} \r \gath{b=-8 \r a =\f{51}{2}\\ b=-3 \r a = 28} $$
    \textbf{Ответ:}\begin{itemize}
        \item 
        0 решений при: $ab+204 = 0$
        \item 
        1 решение при: $\gath{b=-8,\ts a =\f{51}{2}\\ b=-3,\ts a = 28}$
        \item 
        2 решения при: $\gath{a = \f{b+59}{2} \\ a = 24 - \f{12}{b} \\ a = \f{20b - 248}{b-8}}$\\\\
    \end{itemize}

    \item[\textbf{2.}]$A = \mat{1 & 0 & -6 \\ 2 & -4 & 3 \\ 0 & 0 & 0}, \ts X = \mat{a & 0 & b \\ c & d & e \\ 0 & 0 & f}$
    $$AX = \mat{a & 0 & b - 6f \\ 2a - 4c & -4d & 2b - 4e + 3f \\ 0 & 0 & 0}, \; XA = \mat{a & 0 & -6a \\ c + 2d & -4d & -6c + 3d \\ 0 & 0 & 0}$$
    $$\case{
        a = a\\
        b-6f = -6a\r b = 6f-6a\\
        2a-4c = c + 2d \r a = \f{5}{2}c + d\\
        -4d = -4d\\
        2b-4e+3f = -6c+3d \r e = \f{3}{2}c-\f{3}{4}d+\f{1}{2}b+\f{3}{4}f\\
    }\r$$
    $$\r \case{
        b = 6f-6(\f{5}{2}c + d)=6f-15c-6d \\
        a = \f{5}{2}c + d\\
        e = \f{3}{2}c-\f{3}{4}d+\f{1}{2}(6f-15c-6d)+\f{3}{4}f = -6c-\f{15}{4}d+\f{15}{4}f
    }$$
    \textbf{Ответ:} $\case{
        b = 6f-15c-6d\\
        a = \f{5}{2}c + d\\
        e = -6c-\f{15}{4}d+\f{15}{4}f
    } f, c, d \in \RR$

    \item[\textbf{3.}]$A = \mat{26 & 8 & -5 & 1 \\ -15 & -2 &3 & 0 \\ -5 & -1 & 1 & 0}, \; B = \mat{44 & -44 & 32 \\ -33 & 32 & -14 \\ -10 & 10 & -5}$
    {\small
    $$AX = B \r \mat{26 & 8 & -5 & 1 & | & 44 & -44 & 32 \\ -15 & -2 &3 & 0 & | & -33 & 32 & -14 \\ -5 & -1 & 1 & 0 & | & -10 & 10 & -5}$$
    $$A[0] = A[0]/26 \r \mat{1 & \frac{4}{13} & -\frac{5}{26} & \frac{1}{26} & | & \frac{22}{13} & -\frac{22}{13} & \frac{16}{13} \\ -15 & -2 & 3 & 0 & | & -33 & 32 & -14 \\ -5 & -1 & 1 & 0 & | & -10 & 10 & -5}$$
    $$\case{A[1] = A[1]+15*A[0]\\A[2] = A[2]+5*A[0]} \r \mat{1 & \frac{4}{13} & -\frac{5}{26} & \frac{1}{26} & \frac{22}{13} & -\frac{22}{13} & \frac{16}{13} \\ 0 & \frac{34}{13} & \frac{3}{26} & \frac{15}{26} & -\frac{99}{13} & \frac{86}{13} & \frac{58}{13} \\ 0 & \frac{7}{13} & \frac{1}{26} & \frac{5}{26} & -\frac{20}{13} & \frac{20}{13} & \frac{15}{13}}$$
    $$A[1] = A[1]*13/34 \r \mat{1 & \frac{4}{13} & -\frac{5}{26} & \frac{1}{26} & \frac{22}{13} & -\frac{22}{13} & \frac{16}{13} \\ 0 & 1 & \frac{3}{68} & \frac{15}{68} & -\frac{99}{34} & \frac{43}{17} & \frac{29}{17} \\ 0 & \frac{7}{13} & \frac{1}{26} & \frac{5}{26} & -\frac{20}{13} & \frac{20}{13} & \frac{15}{13}}$$
    $$A[2] = A[2]-A[1]*7/13 \r \mat{1 & \frac{4}{13} & -\frac{5}{26} & \frac{1}{26} & \frac{22}{13} & -\frac{22}{13} & \frac{16}{13} \\ 0 & 1 & \frac{3}{68} & \frac{15}{68} & -\frac{99}{34} & \frac{43}{17} & \frac{29}{17} \\ 0 & 0 & \frac{1}{68} & \frac{5}{68} & \frac{1}{34} & \frac{3}{17} & \frac{4}{17}}$$
    $$A[2] = A[2]*68 \r \mat{1 & \frac{4}{13} & -\frac{5}{26} & \frac{1}{26} & \frac{22}{13} & -\frac{22}{13} & \frac{16}{13} \\ 0 & 1 & \frac{3}{68} & \frac{15}{68} & -\frac{99}{34} & \frac{43}{17} & \frac{29}{17} \\ 0 & 0 & 1 & 5 & 2 & 12 & 16}$$
    $$A[0] = A[0]-A[1]*4/13 \r \mat{1 & 0 & -\frac{7}{34} & -\frac{1}{34} & \frac{44}{17} & -\frac{42}{17} & \frac{12}{17} \\ 0 & 1 & \frac{3}{68} & \frac{15}{68} & -\frac{99}{34} & \frac{43}{17} & \frac{29}{17} \\ 0 & 0 & 1 & 5 & 2 & 12 & 16}$$
    $$A[0] = A[0]-A[2]*-7/34 \r \mat{1 & 0 & 0 & 1 & 3 & 0 & 4 \\ 0 & 1 & \frac{3}{68} & \frac{15}{68} & -\frac{99}{34} & \frac{43}{17} & \frac{29}{17} \\ 0 & 0 & 1 & 5 & 2 & 12 & 16}$$
    $$A[1] = A[1]-A[2]*3/68 \r \mat{1 & 0 & 0 & 1 & | & 3 & 0 & 4 \\ 0 & 1 & 0 & 0 &| & -3 & 2 & 1 \\ 0 & 0 & 1 & 5 & | & 2 & 12 & 16}$$
    $$\case{
        x_1 = 3- x_4\\
        x_2 =-3\\
        x_3 = 2 - 5x_4\\
    }\case{
        y_1 = -y_4\\
        y_2 = 2\\
        y_3 = 12 - 5y_4\\ 
    }\case{
        z_1 = 4- z_4\\
        z_2 =1\\
        z_3 =16 - 5z_4\\
    }$$
    \textbf{Ответ: } $\mat{3- x_4 & -y_4& 4- z_4 \\ -3 & 2 & 1 \\ 2 - 5x_4 & 12 - 5y_4 & 16 - 5z_4 \\  x_4 & y_4 & z_4}$
    }\\\\

    \item[\textbf{4.}]$A = \mat{4 & 2 & 1 & -4 & -7 \\ 1 & 1 & 1 & -3 &-2 \\ -2 & -1 & 1 & -1 & 5 \\ 3 & 2 & -1 & 0 & -8}$
    $$(A|E) \to (PA|P) = (A'|P)$$
    {\small
        $$(A|E) = \mat{4 & 2 & 1 & -4 & -7 & | & 1 & 0 & 0 & 0 \\ 1 & 1 & 1 & -3 &-2 & | & 0 & 1 & 0 & 0\\ -2 & -1 & 1 & -1 & 5& | & 0 & 0 & 1 & 0 \\ 3 & 2 & -1 & 0 & -8 & | & 0 & 0 & 0 & 1}$$
        $$A[0] = A[0]/4 \r \mat{1 & \frac{1}{2} & \frac{1}{4} & -1 & -\frac{7}{4} & | & \frac{1}{4} & 0 & 0 & 0 \\ 1 & 1 & 1 & -3 & -2 & | & 0 & 1 & 0 & 0 \\ -2 & -1 & 1 & -1 & 5 & | & 0 & 0 & 1 & 0 \\ 3 & 2 & -1 & 0 & -8 & | & 0 & 0 & 0 & 1}$$    
        $$A[1] = A[1]-A[0] \r \mat{1 & \frac{1}{2} & \frac{1}{4} & -1 & -\frac{7}{4} & | & \frac{1}{4} & 0 & 0 & 0 \\ 0 & \frac{1}{2} & \frac{3}{4} & -2 & -\frac{1}{4} & | & -\frac{1}{4} & 1 & 0 & 0 \\ -2 & -1 & 1 & -1 & 5 & | & 0 & 0 & 1 & 0 \\ 3 & 2 & -1 & 0 & -8 & | & 0 & 0 & 0 & 1}$$
        $$A[2] = A[2]+2*A[0] \r \mat{1 & \frac{1}{2} & \frac{1}{4} & -1 & -\frac{7}{4} & | & \frac{1}{4} & 0 & 0 & 0 \\ 0 & \frac{1}{2} & \frac{3}{4} & -2 & -\frac{1}{4} & | & -\frac{1}{4} & 1 & 0 & 0 \\ 0 & 0 & \frac{3}{2} & -3 & \frac{3}{2} & | & \frac{1}{2} & 0 & 1 & 0 \\ 3 & 2 & -1 & 0 & -8 & | & 0 & 0 & 0 & 1}$$
        $$A[3] = A[3]-3*A[0] \r \mat{1 & \frac{1}{2} & \frac{1}{4} & -1 & -\frac{7}{4} & | & \frac{1}{4} & 0 & 0 & 0 \\ 0 & \frac{1}{2} & \frac{3}{4} & -2 & -\frac{1}{4} & | & -\frac{1}{4} & 1 & 0 & 0 \\ 0 & 0 & \frac{3}{2} & -3 & \frac{3}{2} & | & \frac{1}{2} & 0 & 1 & 0 \\ 0 & \frac{1}{2} & -\frac{7}{4} & 3 & -\frac{11}{4} & | & -\frac{3}{4} & 0 & 0 & 1}$$
        $$A[3] = A[3] - A[1] \r \mat{1 & \frac{1}{2} & \frac{1}{4} & -1 & -\frac{7}{4} & | & \frac{1}{4} & 0 & 0 & 0 \\ 0 & \frac{1}{2} & \frac{3}{4} & -2 & -\frac{1}{4} & | & -\frac{1}{4} & 1 & 0 & 0 \\ 0 & 0 & \frac{3}{2} & -3 & \frac{3}{2} & | & \frac{1}{2} & 0 & 1 & 0 \\ 0 & 0 & -\frac{5}{2} & 5 & -\frac{5}{2} & | & -\frac{1}{2} & -1 & 0 & 1}$$
        $$A[1] = A[1]*2 \r \mat{1 & \frac{1}{2} & \frac{1}{4} & -1 & -\frac{7}{4} & | & \frac{1}{4} & 0 & 0 & 0 \\ 0 & 1 & \frac{3}{2} & -4 & -\frac{1}{2} & | & -\frac{1}{2} & 2 & 0 & 0 \\ 0 & 0 & \frac{3}{2} & -3 & \frac{3}{2} & | & \frac{1}{2} & 0 & 1 & 0 \\ 0 & 0 & -\frac{5}{2} & 5 & -\frac{5}{2} & | & -\frac{1}{2} & -1 & 0 & 1}$$
        $$A[2] = A[2]*2/3 \r \mat{1 & \frac{1}{2} & \frac{1}{4} & -1 & -\frac{7}{4} & | & \frac{1}{4} & 0 & 0 & 0 \\ 0 & 1 & \frac{3}{2} & -4 & -\frac{1}{2} & | & -\frac{1}{2} & 2 & 0 & 0 \\ 0 & 0 & 1 & -2 & 1 & | & \frac{1}{3} & 0 & \frac{2}{3} & 0 \\ 0 & 0 & -\frac{5}{2} & 5 & -\frac{5}{2} & | & -\frac{1}{2} & -1 & 0 & 1}$$
        $$A[3] = A[3]+A[2]*5/2 \r \mat{1 & \frac{1}{2} & \frac{1}{4} & -1 & -\frac{7}{4} & | & \frac{1}{4} & 0 & 0 & 0 \\ 0 & 1 & \frac{3}{2} & -4 & -\frac{1}{2} & | & -\frac{1}{2} & 2 & 0 & 0 \\ 0 & 0 & 1 & -2 & 1 & | & \frac{1}{3} & 0 & \frac{2}{3} & 0 \\ 0 & 0 & 0 & 0 & 0 & | & \frac{1}{3} & -1 & \frac{5}{3} & 1}$$
        $$0 = 1 \ts \varnothing$$
    }
    \textbf{Ответ: } такой матрицы $P$ не существует \\\\

    \item[\textbf{5.}]{\small $$A = \mat{1 & 0 & 0 \\ 0 & 1 &0 \\ 0 & 0 & 1 \\ -7 & 0 & -1},
    B = \mat{1 & 0 & -3 & 2 \\ 0 & 1 & 4 & -5 \\ 3 & 2 & -1 & -4},
    C = \mat{1 & -2 & -2 & -1 \\ -2 & 5 & 4 & 1 \\ 2 & -6 & -3 & -1 \\ -2 & 3 & 5 & 3},$$
    $$D = \mat{10 & 9 & 4 & 0 \\ 94 & 29 & 0 & 16 \\ -84 & -20 & 4 & -16 \\ 64 & 2 & -12 & 16}$$}
    $$AB = \mat{1 & 0 & 0 \\ 0 & 1 &0 \\ 0 & 0 & 1 \\ -7 & 0 & -1} \cdot \mat{1 & 0 & -3 & 2 \\ 0 & 1 & 4 & -5 \\ 3 & 2 & -1 & -4} = 
    \mat{1 & 0 & -3 & 2 \\ 0 & 1 & 4 & -5 \\ 3 & 2 & -1 & -4 \\ -10 & -2 & 22 & -10}$$
    $$ABC^{-1} = \mat{1 & 0 & -3 & 2 \\ 0 & 1 & 4 & -5 \\ 3 & 2 & -1 & -4 \\ -10 & -2 & 22 & -10} \cdot \mat{1 & -2 & 2 & -2 \\ -2 & 5 & -6 & 3 \\ -2 & 4 & -3 & 5 \\ -1 & 1 & -1 & 3} = $$
    $$= \mat{5 & -12 & 9 & -11 \\ -5 & 16 & -13 & 8 \\ 5 & -4 & 1 & -17 \\ -40 & 88 & -64 & 94}$$
    $$ABC^{-1}x = 0 \r \mat{5 & -12 & 9 & -11 &| &0\\ -5 & 16 & -13 & 8&| &0 \\ 5 & -4 & 1 & -17 &| &0\\ -40 & 88 & -64 & 94 &| &0} = F$$
    $$\case{F[0] = F[0]/5\\
    F[1] = F[1]+F[0]*5\\
    F[2] = F[2]-F[0]*5\\
    F[3] = F[3] + F[0]*40} \r \mat{1 & -\frac{12}{5} & \frac{9}{5} & -\frac{11}{5} & | & 0 \\ 0 & 4 & -4 & -3 & | & 0 \\ 0 & 8 & -8 & -6 & | & 0 \\ 0 & -8 & 8 & 6 & | & 0}$$
    $$\case{F[2] = F[2]-2*F[1]\\
    F[3] = F[3]+2*F[1]\\
    F[1] = F[1]/4} \r \mat{1 & -\frac{12}{5} & \frac{9}{5} & -\frac{11}{5} & | & 0 \\ 0 & 1 & -1 & -\frac{3}{4} & | & 0 \\ 0 & 0 & 0 & 0 & | & 0 \\ 0 & 0 & 0 & 0 & | & 0}$$
    $$F[0] = F[0] +F[1]*12/5 \r \mat{1 & 0 & -\frac{3}{5} & -4 & | & 0 \\ 0 & 1 & -1 & -\frac{3}{4} & | & 0 \\ 0 & 0 & 0 & 0 & | & 0 \\ 0 & 0 & 0 & 0 & | & 0}\r$$
    $$\r \case{x_1 = \f{3}{5}x_3 + 4x_4 \\ x_2 = x_3+\f{3}{4}x_4} \r x = \mat{\f{3}{5}x_3 + 4x_4 \\ x_3+\f{3}{4}x_4 \\ x_3 \\ x_4}, \; x_3, x_4 \in \RR$$
    $$Dy = 0 \r \mat{10 & 9 & 4 & 0 & | & 0\\ 94 & 29 & 0 & 16 & | & 0\\ -84 & -20 & 4 & -16 & | & 0\\ 64 & 2 & -12 & 16& | & 0} = G$$
    $$\case{G[0]  = G[0]/10\\
    G[1] = G[1]-G[0]*94\\
    G[2] = G[2]+G[0]*84\\
    G[3] = G[3]-G[0]*64} \r \mat{1 & \frac{9}{10} & \frac{2}{5} & 0 & | & 0 \\ 0 & -\frac{278}{5} & -\frac{188}{5} & 16 & | & 0 \\ 0 & \frac{278}{5} & \frac{188}{5} & -16 & | & 0 \\ 0 & -\frac{278}{5} & -\frac{188}{5} & 16 & | & 0}$$
    $$\case{
    G[1] = G[1]*-5/278\\
    G[2] = G[2] - G[1]*278/5\\
    G[3] = G[3] + G[1]*278/5\\
    }\r \mat{1 & \frac{9}{10} & \frac{2}{5} & 0 & | & 0 \\ 0 & 1 & \frac{94}{139} & -\frac{40}{139} & | & 0 \\ 0 & 0 & 0 & 0 & | & 0 \\ 0 & 0 & 0 & 0 & | & 0}$$

    $$G[0] = G[0]-G[1]*9/10 \r \mat{1 & 0 & -\frac{29}{139} & \frac{36}{139} & | & 0 \\ 0 & 1 & \frac{94}{139} & -\frac{40}{139} & | & 0 \\ 0 & 0 & 0 & 0 & | & 0 \\ 0 & 0 & 0 & 0 & | & 0}\r$$
    $$\r \case{y_1 = \frac{29}{139}y_3 -\frac{36}{139}y_4 \\ y_2 = \frac{40}{139}y_4-\frac{94}{139}y_3} \r y = \mat{\frac{29}{139}y_3 -\frac{36}{139}y_4 \\ \frac{40}{139}y_4-\frac{94}{139}y_3 \\ y_3 \\ y_4}, \; y_3, y_4 \in \RR$$
    Найдём, когда $x=y$:
    $$\mat{\f{3}{5}x_3 + 4x_4 \\ x_3+\f{3}{4}x_4 \\ x_3 \\ x_4} = \mat{\frac{29}{139}y_3 -\frac{36}{139}y_4 \\ \frac{40}{139}y_4-\frac{94}{139}y_3 \\ y_3 \\ y_4}$$
    $$\case{
        \f{3}{5}x_3 + 4x_4 = \frac{29}{139}y_3 -\frac{36}{139}y_4 \r \f{3}{5}x_3 + 4x_4 = \frac{29}{139}x_3 -\frac{36}{139}x_4\\ 
        x_3+\f{3}{4}x_4 = \frac{40}{139}y_4-\frac{94}{139}y_3 \r x_3+\f{3}{4}x_4 = \frac{40}{139}x_4-\frac{94}{139}x_3\\
        x_3 = y_3\\
        x_4 = y_4
    } \r$$
    $$\r \case{
        x_3 = -\f{185}{17}x_4\\
        x_3 = -\f{257}{932}x_4
    }\r x_4 = 0 \r x_3 = 0 \r x_2 =0 \r x_1 = 0$$
    \textbf{Ответ: } уравнения не имеют одинакового множества решений, есть одно общее решение, это - {\small$\mat{0\\0\\0\\0}$}
\end{enumerate}
\end{document}