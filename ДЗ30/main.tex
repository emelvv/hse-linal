\documentclass[a4paper]{article}
\usepackage{setspace}
\usepackage[utf8]{inputenc}
\usepackage[russian]{babel}
\usepackage[12pt]{extsizes}
\usepackage{mathtools}
\usepackage{graphicx}
\usepackage{fancyhdr}
\usepackage{amssymb}
\usepackage{amsmath, amsfonts, amssymb, amsthm, mathtools}
\usepackage{tikz}

\usetikzlibrary{positioning}
\setstretch{1.2}

\newcommand{\mat}[1]{\begin{pmatrix} #1 \end{pmatrix}}
\newcommand{\vmat}[1]{\begin{vmatrix} #1 \end{vmatrix}}
\renewcommand{\f}[2]{\frac{#1}{#2}}
\newcommand{\dspace}{\space\space}
\newcommand{\s}[2]{\sum\limits_{#1}^{#2}}
\newcommand{\mul}[2]{\prod_{#1}^{#2}}
\newcommand{\sq}[1]{\left[ {#1} \right]}
\newcommand{\gath}[1]{\left[ \begin{array}{@{}l@{}} #1 \end{array} \right.}
\newcommand{\case}[1]{\begin{cases} #1 \end{cases}}
\newcommand{\ts}{\text{\space}}
\newcommand{\lm}[1]{\underset{#1}{\lim}}
\newcommand{\suplm}[1]{\underset{#1}{\overline{\lim}}}
\newcommand{\inflm}[1]{\underset{#1}{\underline{\lim}}}
\newcommand{\Ker}[1]{\operatorname{Ker}}

\renewcommand{\phi}{\varphi}
\newcommand{\lr}{\Leftrightarrow}
\renewcommand{\l}{\left(}
\renewcommand{\r}{\right)}
\newcommand{\rr}{\rightarrow}
\renewcommand{\geq}{\geqslant}
\renewcommand{\leq}{\leqslant}
\newcommand{\RR}{\mathbb{R}}
\newcommand{\CC}{\mathbb{C}}
\newcommand{\QQ}{\mathbb{Q}}
\newcommand{\ZZ}{\mathbb{Z}}
\newcommand{\VV}{\mathbb{V}}
\newcommand{\NN}{\mathbb{N}}
\newcommand{\OO}{\underline{O}}
\newcommand{\oo}{\overline{o}}
\renewcommand{\Ker}{\operatorname{Ker}}
\renewcommand{\Im}{\operatorname{Im}}
\newcommand{\vol}{\text{vol}}
\newcommand{\Vol}{\text{Vol}}

\DeclarePairedDelimiter\abs{\lvert}{\rvert} %
\makeatletter                               % \abs{}
\let\oldabs\abs                             %
\def\abs{\@ifstar{\oldabs}{\oldabs*}}       %

\begin{document}

\section*{\scalebox{0.9}{Домашнее задание на 13.06 (Линейная алгебра)}}
{\large Емельянов Владимир, ПМИ гр №247}\\\\
\begin{enumerate}
  \item[\textbf{№1}]Найдём жордановы нормалльные формы:
  \begin{enumerate}
    \item[1.1]Дано    
    $$
    A=\begin{pmatrix}2&2\\-2&-2\end{pmatrix}
    $$
    Характеристический многочлен
    $$
    \det(A-\lambda E)=\begin{vmatrix}2-\lambda&2\\-2&-2-\lambda\end{vmatrix}
    =(2-\lambda)(-2-\lambda)-(-4)=\lambda^2
    $$
    Значит единственный собственный корень $\lambda=0$ кратности 2. 
    Проверка показывает $A^2=0$, но $A\neq0$, значит в Жордановой форме 
    один блок размера 2 при $\lambda=0$:
    $$
    J_A=\begin{pmatrix}0&1\\0&0\end{pmatrix}
    $$

    \item[1.2]Дано
    $$
    A=\begin{pmatrix}
    2 & -1 & 2\\
    5 & -3 & 3\\
    -1 & 0 & -2
    \end{pmatrix}
    $$
    характеристический многочлен
    $$
    \chi_A(\lambda)=\det(A-\lambda E)=\lambda^3+3\lambda^2+3\lambda+1=(\lambda+1)^3
    $$
    то есть единственное собственное значение $\lambda=-1$ алгебраической кратности 3. 
    При этом
    $$
    \dim\ker(A+E)=1
    $$
    следовательно геометрическая кратность равна 1 и вся кратность 3 собирается
     в один жорданов блок размера 3. Итоговая Жорданова форма
    $$
    J_A=P^{-1}AP
    =\begin{pmatrix}
    -1 & 1 & 0\\
    0  & -1 & 1\\
    0  & 0  & -1
    \end{pmatrix}
    $$

    \item[1.3]Дано
    $$
    A=\begin{pmatrix}
    0&1&0\\
    -4&4&0\\
    -2&1&2
    \end{pmatrix}
    $$
    Её характеристический многочлен 
    $$\chi_A(\lambda)=(2-\lambda)^3$$
    т.е. единственный собственный корень $\lambda=2$ кратности 3.

    Геометрическая кратность этого корня равна 2, поэтому разложение на жордановы 
    блоки будет одно блочное звено размером 2 и одно размером 1.
    Иными словами, в Жордановой форме
    $$
    J_A=\mathrm{diag}\bigl(J_2(2),\;2\bigr)
    \;=\;
    \begin{pmatrix}
    2&1&0\\
    0&2&0\\
    0&0&2
    \end{pmatrix}
    $$
    где
    $$
    J_2(2)=\begin{pmatrix}2&1\\0&2\end{pmatrix}
    $$

    \item[1.4]Дано
    $$
    A=\begin{pmatrix}
    3 & -1 & 0 & 0\\
    1 & 1  & 0 & 0\\
    3 & 0  & 5 & -3\\
    4 & -1 & 3 & -1
    \end{pmatrix}
    $$
    $$
    \chi_A(\lambda)=\det(A-\lambda E)
    =\lambda^4-8\lambda^3+24\lambda^2-32\lambda+16
    =(\lambda-2)^4
    $$
    Здесь единственное собственное значение $\lambda=2$ 
    алгебраической кратности 4, и оказывается
    $$
    \dim\ker(A-2E)=2
    $$
    Значит у нас два жордановых блока, каждый размера 2. Жорданова форма
    $$
    J_A=P^{-1}AP
    =\begin{pmatrix}
    2 & 1 & 0 & 0\\
    0 & 2 & 0 & 0\\
    0 & 0 & 2 & 1\\
    0 & 0 & 0 & 2
    \end{pmatrix}
    =\mathrm{diag}\bigl(J_2(2),\,J_2(2)\bigr)
    $$
    $\;$

  \end{enumerate}

  \item[\textbf{№2}]
  Рассмотрим квадратную матрицу \( A \) над полем комплексных чисел. 
  Нужно доказать, что \( A \) подобна своей транспонированной \( A^T \),
  то есть существует обратимая матрица \( P \), такая что 
  \[ A = P^{-1} A^T P \]

  любая квадратная матрица над полем комплексных чисел подобна 
  своей жордановой форме. То есть существует обратимая матрица \( S \), 
  такая что 
  \[ A = S^{-1} J S \]
  где \( J \) — жорданова форма \( A \).

  Рассмотрим транспонированную матрицу \( A^T \).
  
  Имеем:
  \[
  A^T = (S^{-1} J S)^T = S^T J^T (S^{-1})^T
  \]
  Таким образом, \( A^T \) подобна \( J^T \).

  Каждая жорданова клетка \( J_m(\lambda) \)
   подобна своей транспонированной \( J_m(\lambda)^T \). Действительно,
    матрица перестановки \( P_m \), задаваемая как:
  \[
  (P_m)_{ij} = \begin{cases}
  1 & \text{если } i + j = m + 1 \\
  0 & \text{иначе}
  \end{cases}
  \]
  удовлетворяет условию \( J_m(\lambda) = P_m^{-1} J_m(\lambda)^T P_m \). Легко проверить, что \( P_m = P_m^{-1} \).

  Поскольку \( J \) и \( J^T \) состоят из попарно подобных блоков, то 
  и вся матрица \( J \) подобна \( J^T \): \( J = Q^{-1} J^T Q \),
   где \( Q \) — блочно-диагональная матрица, составленная из матриц 
   \( P_{m_i} \) для соответствующих жордановых клеток.

  Следовательно, матрица \( A \) подобна своей транспонированной \( A^T \)
  
  \item[\textbf{№3}]\begin{enumerate}
    \item[3.1]Оператор $A$ с матрицей
    $$
    A=\begin{pmatrix}
    2 & -1 & 2\\
    5 & -3 & 3\\
    -1& 0  & -2
    \end{pmatrix}
    $$
    Мы уже увидели, что у $A$ единственный собственный корень $\lambda=-1$ 
    алгебраической кратности 3, а
    $\dim\ker(A+E)=1$
    Значит Жорданова форма один блок размера 3:
    $$
    J_A \;=\; J_3(-1)\;=\;
    \begin{pmatrix}
    -1 & 1 & 0\\
    0  & -1 & 1\\
    0  & 0  & -1
    \end{pmatrix}
    $$
    Найдём теперь жорданов базис
    $$
    (A+E)v_1=0,\quad (A+E)v_2=v_1,\quad (A+E)v_3=v_2
    $$
    Решая, получаем, например
    $$
    v_1=\begin{pmatrix}-1\\-1\\1\end{pmatrix},\quad
    v_2=\begin{pmatrix}-1\\-2\\0\end{pmatrix},\quad
    v_3=\begin{pmatrix}0\\1\\0\end{pmatrix},
    $$
    Тогда в базисе $\{v_3,v_2,v_1\}$ матрица оператора принимает форму $J_3(-1)$.\\

    Матрица \( A^T \) подобна матрице \( A \): \( A^T = P A P^{-1} \). Но если \( A^T \) подобна \( A \), то и \( A \) подобна \( A^T \) (подобие — отношение симметричное: если \( X = R Y R^{-1} \), то \( Y = R^{-1} X R \)). Следовательно, существует матрица \( R = P^{-1} \) такая, что:
        \[
        A = R^{-1} A^T R
        \]

    \item[3.2]Оператор $B$ с матрицей
    $$
    B=\begin{pmatrix}
    1 & 4  & -12\\
    -1& -3 & 7\\
    0 & 0  & 1
    \end{pmatrix}
    $$
    Характеристический многочлен
    $$
    \chi_B(\lambda)=-(\lambda-1)\,(\lambda+1)^2
    $$
    так что собственные значения $\lambda=1$ (кратность 1) и $\lambda=-1$ (кратность 2).
    Проверка показывает
    $$
    \dim\ker(B-E)=1,\qquad \dim\ker(B+E)=1
    $$
    то есть на $\lambda=1$ блок простого собственного вектора, а на $\lambda=-1$ один блок размера 2.
    
    Итоговая Жорданова форма
    $$
    J_B \;=\;\operatorname{diag}\bigl(J_2(-1),\, [1]\bigr)
    \;=\;
    \begin{pmatrix}
    -1&1&0\\
    0&-1&0\\
    0&0&1
    \end{pmatrix}
    $$
    Построим две жордановы цепочки:
    \begin{enumerate}
    \item[1)] Для $\lambda=-1$
      $$
      (B+E)\,w_1=0,\qquad (B+E)\,w_2=w_1
      $$
      Решение даёт, например,
      $$
      w_1=\begin{pmatrix}-2\\1\\0\end{pmatrix},\quad
      w_2=\begin{pmatrix}-1\\0\\0\end{pmatrix},
      $$

    \item[2)] Для $\lambda=1$
    
      Просто собственный вектор
      $$
      (B-E)\,u=0
      \;\Longrightarrow\;
      u=\begin{pmatrix}-5\\3\\1\end{pmatrix}
      $$
    \end{enumerate}
    
    
      В результате в базисе $\{\,w_2,\;w_1,\;u\}$ матрица $B$ становится

    $$
    J_B = \begin{pmatrix}
    -1 & 1 & 0\\
    0  & -1& 0\\
    0  & 0 & 1
    \end{pmatrix}
    $$

  \end{enumerate}$\;$

  \item[\textbf{№4}]Дано
  $$A = \begin{pmatrix}
  0 & 2 & -2 & 1 & 1 \\
  0 & 3 & -3 & 1 & 1 \\
  0 & 3 & -3 & 1 & 1 \\
  0 & 0 & 0 & 1 & 1 \\
  0 & 0 & 0 & -1 & -1
  \end{pmatrix}$$

  Вычисляем определитель матрицы $ A - \lambda E $:
  $$
  \det(A - \lambda E) = \begin{vmatrix}
  -\lambda & 2 & -2 & 1 & 1 \\
  0 & 3 - \lambda & -3 & 1 & 1 \\
  0 & 3 & -3 - \lambda & 1 & 1 \\
  0 & 0 & 0 & 1 - \lambda & 1 \\
  0 & 0 & 0 & -1 & -1 - \lambda
  \end{vmatrix}
  $$
  Разлагаем по первой строке и находим, что характеристический многочлен равен:
  $$
  \lambda^3 (3 - \lambda)(3 + \lambda)
  $$
  Собственные значения: $\lambda = 0$ (кратность 3), $\lambda = 3$ (кратность 1), $\lambda = -3$ (кратность 1).
  \begin{itemize}
    \item Для $\lambda = 0$:
    
      Ранг матрицы $ A $ равен 3, поэтому геометрическая кратность $\lambda = 0$
       равна $5 - 3 = 2$. Это означает, что для $\lambda = 0$ 
       возможны два жордановских блока. Учитывая структуру матрицы, 
       предполагаем один блок размера $2 \times 2$ и один блок размера $1 \times 1$.
      
    \item Для $\lambda = 3$:
    
    Геометрическая кратность равна 1 (ранг матрицы $A - 3E$ равен 4), 
    поэтому жордановский блок размера $1 \times 1$.
      
    \item Для $\lambda = -3$:
    
    Геометрическая кратность равна 1 (ранг матрицы $A + 3E$ равен 4),
     поэтому жордановский блок размера $1 \times 1$.
  \end{itemize}

  Собираем жордановские блоки для каждого собственного значения:

  \begin{itemize}
    \item Блок для $\lambda = 0$: 
      $$
      J_0 = \begin{pmatrix}
      0 & 1 & 0 & 0 & 0 \\
      0 & 0 & 0 & 0 & 0 \\
      0 & 0 & 0 & 0 & 0 \\
      0 & 0 & 0 & 0 & 0 \\
      0 & 0 & 0 & 0 & 0
      \end{pmatrix} \quad \text{(размер $2 \times 2$)} \quad \text{и} \quad 
      J_0' = \begin{pmatrix}
      0
      \end{pmatrix} \quad \text{(размер $1 \times 1$)}
      $$

    \item Блок для $\lambda = 3$: 
      $$
      J_3 = \begin{pmatrix}
      3
      \end{pmatrix}
      $$

    \item Блок для $\lambda = -3$: 
      $$
      J_{-3} = \begin{pmatrix}
      -3
      \end{pmatrix}
      $$
  \end{itemize}

  Жорданова нормальная форма матрицы $A$:
  $$
  J = \begin{pmatrix}
  0 & 1 & 0 & 0 & 0 \\
  0 & 0 & 0 & 0 & 0 \\
  0 & 0 & 0 & 0 & 0 \\
  0 & 0 & 0 & 3 & 0 \\
  0 & 0 & 0 & 0 & -3
  \end{pmatrix}
  $$



\end{enumerate}
\end{document}