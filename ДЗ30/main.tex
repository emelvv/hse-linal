\documentclass[a4paper]{article}
\usepackage{setspace}
\usepackage[utf8]{inputenc}
\usepackage[russian]{babel}
\usepackage[12pt]{extsizes}
\usepackage{mathtools}
\usepackage{graphicx}
\usepackage{fancyhdr}
\usepackage{amssymb}
\usepackage{amsmath, amsfonts, amssymb, amsthm, mathtools}
\usepackage{tikz}

\usetikzlibrary{positioning}
\setstretch{1.3}

\newcommand{\mat}[1]{\begin{pmatrix} #1 \end{pmatrix}}
\newcommand{\vmat}[1]{\begin{vmatrix} #1 \end{vmatrix}}
\renewcommand{\f}[2]{\frac{#1}{#2}}
\newcommand{\dspace}{\space\space}
\newcommand{\s}[2]{\sum\limits_{#1}^{#2}}
\newcommand{\mul}[2]{\prod_{#1}^{#2}}
\newcommand{\sq}[1]{\left[ {#1} \right]}
\newcommand{\gath}[1]{\left[ \begin{array}{@{}l@{}} #1 \end{array} \right.}
\newcommand{\case}[1]{\begin{cases} #1 \end{cases}}
\newcommand{\ts}{\text{\space}}
\newcommand{\lm}[1]{\underset{#1}{\lim}}
\newcommand{\suplm}[1]{\underset{#1}{\overline{\lim}}}
\newcommand{\inflm}[1]{\underset{#1}{\underline{\lim}}}
\newcommand{\Ker}[1]{\operatorname{Ker}}

\renewcommand{\phi}{\varphi}
\newcommand{\lr}{\Leftrightarrow}
\renewcommand{\l}{\left(}
\renewcommand{\r}{\right)}
\newcommand{\rr}{\rightarrow}
\renewcommand{\geq}{\geqslant}
\renewcommand{\leq}{\leqslant}
\newcommand{\RR}{\mathbb{R}}
\newcommand{\CC}{\mathbb{C}}
\newcommand{\QQ}{\mathbb{Q}}
\newcommand{\ZZ}{\mathbb{Z}}
\newcommand{\VV}{\mathbb{V}}
\newcommand{\NN}{\mathbb{N}}
\newcommand{\OO}{\underline{O}}
\newcommand{\oo}{\overline{o}}
\renewcommand{\Ker}{\operatorname{Ker}}
\renewcommand{\Im}{\operatorname{Im}}
\newcommand{\vol}{\text{vol}}
\newcommand{\Vol}{\text{Vol}}

\DeclarePairedDelimiter\abs{\lvert}{\rvert} %
\makeatletter                               % \abs{}
\let\oldabs\abs                             %
\def\abs{\@ifstar{\oldabs}{\oldabs*}}       %

\begin{document}

\section*{Домашнее задание на 13.06 (Линейная алгебра)}
{\large Емельянов Владимир, ПМИ гр №247}\\\\
\begin{enumerate}
  \item[\textbf{№1}]Найдём жордановы нормалльные формы:
  \begin{enumerate}
    \item[1.1]Дано    
    $$
    A=\begin{pmatrix}2&2\\-2&-2\end{pmatrix}
    $$
    Характеристический многочлен
    $$
    \det(A-\lambda I)=\begin{vmatrix}2-\lambda&2\\-2&-2-\lambda\end{vmatrix}
    =(2-\lambda)(-2-\lambda)-(-4)=\lambda^2
    $$
    Значит единственный собственный корень $\lambda=0$ кратности 2. 
    Проверка показывает $A^2=0$, но $A\neq0$, значит в Жордановой форме 
    один блок размера 2 при $\lambda=0$:
    $$
    J_A=\begin{pmatrix}0&1\\0&0\end{pmatrix}
    $$

    \item[1.2]Дано
    $$
    A=\begin{pmatrix}
    2 & -1 & 2\\
    5 & -3 & 3\\
    -1 & 0 & -2
    \end{pmatrix}
    $$
    характеристический многочлен
    $$
    \chi_A(\lambda)=\det(A-\lambda I)=\lambda^3+3\lambda^2+3\lambda+1=(\lambda+1)^3
    $$
    то есть единственное собственное значение $\lambda=-1$ алгебраической кратности 3. 
    При этом
    $$
    \dim\ker(A+I)=1
    $$
    следовательно геометрическая кратность равна 1 и вся кратность 3 собирается
     в один жорданов блок размера 3. Итоговая Жорданова форма
    $$
    J_A=P^{-1}AP
    =\begin{pmatrix}
    -1 & 1 & 0\\
    0  & -1 & 1\\
    0  & 0  & -1
    \end{pmatrix}
    $$

    \item[1.3]Дано
    $$
    A=\begin{pmatrix}
    0&1&0\\
    -4&4&0\\
    -2&1&2
    \end{pmatrix}
    $$
    Её характеристический многочлен 
    $$\chi_A(\lambda)=(2-\lambda)^3$$
    т.е. единственный собственный корень $\lambda=2$ кратности 3.

    Геометрическая кратность этого корня равна 2, поэтому разложение на жордановы 
    блоки будет одно блочное звено размером 2 и одно размером 1.
    Иными словами, в Жордановой форме
    $$
    J_A=\mathrm{diag}\bigl(J_2(2),\;2\bigr)
    \;=\;
    \begin{pmatrix}
    2&1&0\\
    0&2&0\\
    0&0&2
    \end{pmatrix}
    $$
    где
    $$
    J_2(2)=\begin{pmatrix}2&1\\0&2\end{pmatrix}
    $$

    \item[1.4]Дано
    $$
    A=\begin{pmatrix}
    3 & -1 & 0 & 0\\
    1 & 1  & 0 & 0\\
    3 & 0  & 5 & -3\\
    4 & -1 & 3 & -1
    \end{pmatrix}
    $$
    $$
    \chi_A(\lambda)=\det(A-\lambda I)
    =\lambda^4-8\lambda^3+24\lambda^2-32\lambda+16
    =(\lambda-2)^4
    $$
    Здесь единственное собственное значение $\lambda=2$ 
    алгебраической кратности 4, и оказывается
    $$
    \dim\ker(A-2I)=2
    $$
    Значит у нас два жордановых блока, каждый размера 2. Жорданова форма
    $$
    J_A=P^{-1}AP
    =\begin{pmatrix}
    2 & 1 & 0 & 0\\
    0 & 2 & 0 & 0\\
    0 & 0 & 2 & 1\\
    0 & 0 & 0 & 2
    \end{pmatrix}
    =\mathrm{diag}\bigl(J_2(2),\,J_2(2)\bigr)
    $$

  \end{enumerate}

  \item[\textbf{№2}]
\end{enumerate}
\end{document}