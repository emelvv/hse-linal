\documentclass[a4paper]{article}
\usepackage{setspace}
\usepackage[T2A]{fontenc} %
\usepackage[utf8]{inputenc} % подключение русского языка
\usepackage[russian]{babel} %
\usepackage[12pt]{extsizes}
\usepackage{mathtools}
\usepackage{graphicx}
\usepackage{fancyhdr}
\usepackage{amssymb}
\usepackage{amsmath, amsfonts, amssymb, amsthm, mathtools}
\usepackage{tikz}

\usetikzlibrary{positioning}
\setstretch{1.3}

\newcommand{\mat}[1]{\begin{pmatrix} #1 \end{pmatrix}}
\newcommand{\vmat}[1]{\begin{vmatrix} #1 \end{vmatrix}}
\renewcommand{\f}[2]{\frac{#1}{#2}}
\newcommand{\dspace}{\space\space}
\newcommand{\s}[2]{\sum\limits_{#1}^{#2}}
\newcommand{\mul}[2]{\prod_{#1}^{#2}}
\newcommand{\sq}[1]{\left[ {#1} \right]}
\newcommand{\gath}[1]{\left[ \begin{array}{@{}l@{}} #1 \end{array} \right.}
\newcommand{\case}[1]{\begin{cases} #1 \end{cases}}
\newcommand{\ts}{\text{\space}}
\newcommand{\lm}[1]{\underset{#1}{\lim}}
\newcommand{\suplm}[1]{\underset{#1}{\overline{\lim}}}
\newcommand{\inflm}[1]{\underset{#1}{\underline{\lim}}}
\newcommand{\Ker}[1]{\operatorname{Ker}}

\renewcommand{\phi}{\varphi}
\newcommand{\lr}{\Leftrightarrow}
\renewcommand{\r}{\Rightarrow}
\newcommand{\rr}{\rightarrow}
\renewcommand{\geq}{\geqslant}
\renewcommand{\leq}{\leqslant}
\newcommand{\RR}{\mathbb{R}}
\newcommand{\CC}{\mathbb{C}}
\newcommand{\QQ}{\mathbb{Q}}
\newcommand{\ZZ}{\mathbb{Z}}
\newcommand{\VV}{\mathbb{V}}
\newcommand{\NN}{\mathbb{N}}
\newcommand{\OO}{\underline{O}}
\newcommand{\oo}{\overline{o}}
\renewcommand{\Ker}{\operatorname{Ker}}
\renewcommand{\Im}{\operatorname{Im}}


\DeclarePairedDelimiter\abs{\lvert}{\rvert} %
\makeatletter                               % \abs{}
\let\oldabs\abs                             %
\def\abs{\@ifstar{\oldabs}{\oldabs*}}       %

\begin{document}

\section*{Домашнее задание на 12.02 (Линейная алгебра)}
 {\large Емельянов Владимир, ПМИ гр №247}\\\\
\begin{enumerate}
    \item[\textbf{№1}]
    \begin{enumerate}
        \item[1)]
        Предположим, что существует линейная комбинация:
        \[
        \alpha^1 e_1 + \alpha^2 e_2 + \ldots + \alpha^k e_k = 0_V,
        \]
        где \( \alpha^i \) — скаляры. Применим к обеим частям функционал \( \varepsilon^i \):
        \[
        \varepsilon^i(\alpha^1 e_1 + \ldots + \alpha^k e_k) = \varepsilon^i(0_V) = 0.
        \]
        В силу линейности \( \varepsilon^i \):
        \[
        \alpha^1 \varepsilon^i(e_1) + \ldots + \alpha^k \varepsilon^i(e_k) = 0.
        \]
        Из условия \( \varepsilon^i(e_j) = \delta^i_j \) следует, что все слагаемые, кроме \( \alpha^i \cdot 1 \), равны нулю. Таким образом:
        \[
        \alpha^i = 0 \quad \text{для всех } i = 1, \ldots, k.
        \]
        Следовательно, векторы \( e_1, \ldots, e_k \) линейно независимы.

        \item[2)]
        Предположим, что существует линейная комбинация:
        \[
        \beta_1 \varepsilon^1 + \beta_2 \varepsilon^2 + \ldots + \beta_k \varepsilon^k = 0_{V^*},
        \]
        где \( \beta_i \) — скаляры. Подействуем обеими частями на вектор \( e_j \):
        \[
        (\beta_1 \varepsilon^1 + \ldots + \beta_k \varepsilon^k)(e_j) = 0_{V^*}(e_j) = 0.
        \]
        Раскрывая левую часть:
        \[
        \beta_1 \varepsilon^1(e_j) + \ldots + \beta_k \varepsilon^k(e_j) = 0.
        \]
        Из условия \( \varepsilon^i(e_j) = \delta^i_j \) следует, что все слагаемые, кроме \( \beta_j \cdot 1 \), равны нулю. Таким образом:
        \[
        \beta_j = 0 \quad \text{для всех } j = 1, \ldots, k.
        \]
        Следовательно, функционалы \( \varepsilon^1, \ldots, \varepsilon^k \) линейно независимы.\\
    \end{enumerate}

    \item[\textbf{№2}]Базис \( \{e_0, e_1, \ldots, e_n\} \) в \( V \) двойствен к базису \( \{\beta^0, \beta^1, \ldots, \beta^n\} \) в \( V^* \), если  
    \[
    \beta^i(e_j) = \delta^i_j = 
    \begin{cases} 
    1, & i = j, \\
    0, & i \neq j.
    \end{cases}
    \]
    Пусть:
    \[
    e_j(x) = \frac{x^j}{j!} \quad \text{для } j = 0, 1, \ldots, n.
    \]
    Проверим, подходит ли этот базис $V$ под условие:
    
    \begin{enumerate}
        \item[1)]
        При \( i < j \):  
         
        \( i \)-я производная \( e_j(x) \) равна 
        \[ \frac{x^{j-i}}{(j-i)!} \]
        Подставляя \( x = 0 \), получаем \( 0 \).  
        Следовательно, \( \beta^i(e_j) = 0 \). 
        \item[2)] При \( i > j \):  
         
        \( i \)-я производная \( e_j(x) \) равна 
        \[ 0 \]
        Подставляя \( x = 0 \), получаем \( 0 \).  
        Следовательно, \( \beta^i(e_j) = 0 \). 
        
        \item[3)]При \( i = j \):  
         
        \( i \)-я производная \( e_j(x) \) равна 
        \[ 1 \]
        Подставляя \( x = 0 \), получаем \( 1 \).  
        Следовательно, \( \beta^i(e_j) = 1 \). 

    \end{enumerate}
    Таким образом, условие \( \beta^i(e_j) = \delta^i_j \)
     выполняется для всех \( i, j \), что подтверждает, что базис
      \( \left\{\frac{x^j}{j!}\right\}_{j=0}^n \) двойствен к \( \{\beta^i\}_{i=0}^n \).\\


    \item[\textbf{№3}]Для определения билинейных форм на пространстве матриц 
    \( M_n(\mathbb{R}) \) необходимо проверить линейность функций по каждому аргументу.
    \begin{enumerate}
        \item[3.1.] Линейность по первому аргументу:  
        \[\beta(A_1 + A_2, B) = \operatorname{tr}((A_1 + A_2) B^T) = 
        \operatorname{tr}(A_1 B^T) + \operatorname{tr}(A_2 B^T) =
         \beta(A_1, B) + \beta(A_2, B)\]  
        \[\beta(\alpha A, B) = \operatorname{tr}(\alpha A B^T) =
         \alpha \operatorname{tr}(A B^T) = \alpha \beta(A, B)\]  
      Линейность по второму аргументу:  
        \[\beta(A, B_1 + B_2) = \operatorname{tr}(A (B_1 + B_2)^T) 
        = \operatorname{tr}(A B_1^T + A B_2^T) = \beta(A, B_1) + \beta(A, B_2)\]
        \[\beta(A, \alpha B) = \operatorname{tr}(A (\alpha B)^T) = 
        \alpha \operatorname{tr}(A B^T) = \alpha \beta(A, B)\]
        Функция билинейна.  

        \item[3.2.]\( \beta(A, B) = \operatorname{tr}(AB - BA) \) 
        \[\operatorname{tr}(AB - BA) = \operatorname{tr}(AB) - \operatorname{tr}(BA)\]
        Поскольку \[\operatorname{tr}(AB) = \operatorname{tr}(BA)\]
        получаем \[\operatorname{tr}(AB - BA) = 0\]
        Нулевая функция тривиально билинейна

        \item[3.3.]\( \beta(A, B) = \operatorname{tr}(A + B) \) 
        \[\beta(A, B) = \operatorname{tr}(A) + \operatorname{tr}(B)\]  
        Проверка линейности:  
        \[\beta(A + C, B) = \operatorname{tr}(A + C) + \operatorname{tr}(B) 
        \neq \beta(A, B) + \beta(C, B)\] 
        \[\beta(\alpha A, B) = \alpha \operatorname{tr}(A) + \operatorname{tr}(B) 
        \neq \alpha \beta(A, B)\] 
        Функция не билинейна.  

        \item[3.4.]\( \beta(A, B) = \det(AB) \) 
        \[\det(AB) = \det(A) \det(B)\]
        Проверка линейности:  
        \[\det(AB + CB) \neq \det(AB) + \det(CB)\]  
        \[\det(\alpha A) \det(B) = \alpha^n \det(A) \det(B) 
        \neq \alpha \det(A) \det(B)\]
        Функция не билинейна.  
    \end{enumerate}

    \item[\textbf{№4}]Чтобы проверить, является ли функция 
    \( \beta(f, g) = f(2) g'(1) \) 
    билинейной формой на пространстве 
    \( V = \mathbb{R}[x]_{\leqslant 3} \), 
    необходимо убедиться в линейности по каждому аргументу. 

    1. По первому аргументу:
    Для любых \( f, h \in V \) и \( a \in \mathbb{R} \):
    \[ \beta(f + h, g) = (f + h)(2) g'(1) = f(2) g'(1) + h(2) g'(1) = 
    \beta(f, g) + \beta(h, g) \]
    \[ \beta(af, g) = (af)(2) g'(1) = a f(2) g'(1) = a \beta(f, g) \]

    2. По второму аргументу: Для любых \( g, h \in V \) и \( a \in \mathbb{R} \):
   \[ \beta(f, g + h) = f(2) (g + h)'(1) = f(2) g'(1) + f(2) h'(1) = 
   \beta(f, g) + \beta(f, h) \]
   \[ \beta(f, ag) = f(2) (ag)'(1) = f(2) a g'(1) = a \beta(f, g) \]

    
    Таким образом, \( \beta \) является билинейной формой.

    Элементы матрицы \( \beta(e_i, e_j) \) вычисляются как \( e_i(2) \cdot e_j'(1) \).

    Значения \( e_i(2) \):
    \[e_1(2) = 1, \quad e_2(2) = 2, \quad e_3(2) = 4, \quad e_4(2) = 8 \]

    Значения \( e_j'(1) \):
    \[ e_1'(1) = 0,\quad e_2'(1) = 1, \quad e_3'(1) = 2, \quad e_4'(1) = 3 \]

    Матрица билинейной формы:
    \[
    \begin{pmatrix}
    0 & 1 & 2 & 3 \\
    0 & 2 & 4 & 6 \\
    0 & 4 & 8 & 12 \\
    0 & 8 & 16 & 24
    \end{pmatrix}
    \]
    Билинейная форма \( \beta \) не является симметричной.\\

    \item[\textbf{№5}]Чтобы найти матрицу билинейной формы \( \beta \) в новом базисе,
    используем формулу преобразования матрицы билинейной формы при смене базиса:
     \[ B' = C^T B C \] где \( C \) — матрица перехода, 
     столбцы которой являются координатами новых базисных векторов в старом базисе.

    Новые базисные векторы заданы формулами:
    \[
    \begin{aligned}
    e_{1}' &= e_{1} - e_{2} \\
    e_{2}' &= e_{1} + e_{3} \\
    e_{3}' &= e_{1} + e_{2} + e_{3}
    \end{aligned}
    \]
    Соответствующая матрица перехода \( C \):
    \[
    C = \begin{pmatrix}
    1 & 1 & 1 \\
    -1 & 0 & 1 \\
    0 & 1 & 1
    \end{pmatrix}
    \]
    Транспонированная матрица \( C \):
    \[
    C^T = \begin{pmatrix}
    1 & -1 & 0 \\
    1 & 0 & 1 \\
    1 & 1 & 1
    \end{pmatrix}
    \]
    Умножим матрицы \( C^T \) и \( B \):
    \[
    C^T \cdot B = \begin{pmatrix}
    -3 & -3 & -3 \\
    8 & 10 & 12 \\
    12 & 15 & 18
    \end{pmatrix}
    \]
    Умножим матрицу на \( C \):
    \[
    (C^T \cdot B) \cdot C = \begin{pmatrix}
    0 & -6 & -9 \\
    -2 & 20 & 30 \\
    -3 & 30 & 45
    \end{pmatrix}
    \]
    Итоговая матрица билинейной формы \( \beta \) в новом базисе:
    \[
    \begin{pmatrix}
    0 & -6 & -9 \\
    -2 & 20 & 30 \\
    -3 & 30 & 45
    \end{pmatrix}
    \]
    \textbf{Ответ: } $\begin{pmatrix}
        0 & -6 & -9 \\
        -2 & 20 & 30 \\
        -3 & 30 & 45
        \end{pmatrix}$

    \item[\textbf{№6}]Чтобы найти значение билинейной формы 
    \( \beta(x, y) \), используем формулу для билинейной формы в координатном виде:
    \[
    \beta(x, y) = x^T B y,
    \]
    где \( x \) и \( y \) — координатные столбцы векторов 
    \( x \) и \( y \), а \( B \) — матрица билинейной формы.
    \[
    B = \begin{pmatrix}
    1 & -1 & 1 \\
    -2 & -1 & 3 \\
    0 & 4 & 5
    \end{pmatrix}, \quad
    x = \begin{pmatrix}
    1 \\
    0 \\
    3
    \end{pmatrix}, \quad
    y = \begin{pmatrix}
    -1 \\
    2 \\
    -4
    \end{pmatrix}
    \]
    Вычислим \( B y \):
    \[
    B y = \begin{pmatrix}
    1 & -1 & 1 \\
    -2 & -1 & 3 \\
    0 & 4 & 5
    \end{pmatrix}
    \begin{pmatrix}
    -1 \\
    2 \\
    -4
    \end{pmatrix}=\begin{pmatrix}
        -7 \\
        -12 \\
        -12
        \end{pmatrix}
    \]
    Вычислим \( x^T B y \):
    \small{\[
    x^T B y = \begin{pmatrix}
    1 & 0 & 3
    \end{pmatrix}
    \begin{pmatrix}
    -7 \\
    -12 \\
    -12
    \end{pmatrix}=-43
    \]}
    \textbf{Ответ: } $-43$
        
\end{enumerate}
\end{document}