\documentclass[a4paper]{article}
\usepackage{setspace}
\usepackage[T2A]{fontenc} %
\usepackage[utf8]{inputenc} % подключение русского языка
\usepackage[russian]{babel} %
\usepackage[12pt]{extsizes}
\usepackage{mathtools}
\usepackage{graphicx}
\usepackage{fancyhdr}
\usepackage{amssymb}
\usepackage{amsmath, amsfonts, amssymb, amsthm, mathtools}
\usepackage{tikz}

\usetikzlibrary{positioning}
\setstretch{1.3}

\newcommand{\mat}[1]{\begin{pmatrix} #1 \end{pmatrix}}
\renewcommand{\det}[1]{\begin{vmatrix} #1 \end{vmatrix}}
\renewcommand{\f}[2]{\frac{#1}{#2}}
\newcommand{\dspace}{\space\space}
\newcommand{\s}[2]{\sum\limits_{#1}^{#2}}
\newcommand{\mul}[2]{\prod_{#1}^{#2}}
\newcommand{\sq}[1]{\left[ {#1} \right]}
\newcommand{\gath}[1]{\left[ \begin{array}{@{}l@{}} #1 \end{array} \right.}
\newcommand{\case}[1]{\begin{cases} #1 \end{cases}}
\newcommand{\ts}{\text{\space}}
\newcommand{\lm}[1]{\underset{#1}{\lim}}
\newcommand{\suplm}[1]{\underset{#1}{\overline{\lim}}}
\newcommand{\inflm}[1]{\underset{#1}{\underline{\lim}}}

\renewcommand{\phi}{\varphi}
\newcommand{\lr}{\Leftrightarrow}
\renewcommand{\r}{\Rightarrow}
\newcommand{\rr}{\rightarrow}
\renewcommand{\geq}{\geqslant}
\renewcommand{\leq}{\leqslant}
\newcommand{\RR}{\mathbb{R}}
\newcommand{\CC}{\mathbb{C}}
\newcommand{\QQ}{\mathbb{Q}}
\newcommand{\ZZ}{\mathbb{Z}}
\newcommand{\VV}{\mathbb{V}}
\newcommand{\NN}{\mathbb{N}}
\newcommand{\OO}{\underline{O}}
\newcommand{\oo}{\overline{o}}


\DeclarePairedDelimiter\abs{\lvert}{\rvert} %
\makeatletter                               % \abs{}
\let\oldabs\abs                             %
\def\abs{\@ifstar{\oldabs}{\oldabs*}}       %

\begin{document}

\section*{Домашнее задание на 15.01 (Линейная алгебра)}
 {\large Емельянов Владимир, ПМИ гр №247}\\\\
\begin{enumerate}
    \item[\textbf{№1}]Для доказательства того, что векторы $ b_1, b_2, b_3 $ образуют базис векторного пространства $ V $, необходимо показать, что они линейно независимы.
    
    
Запишем векторы $ b_1, b_2, b_3 $:
    $$\case{
    b_1 = a_1 + 2a_2 + 5a_3\\
    b_2 = 3a_1 - a_2 + a_3\\
    b_3 = 2a_1 + 0a_2 - 2a_3}
    $$
    
    Теперь представим их в виде матрицы:

    $$
    B = 
    \begin{pmatrix}
    1 & 3 & 2 \\
    2 & -1 & 0 \\
    5 & 1 & -2
    \end{pmatrix}
    $$
        
    Найдем определитель этой матрицы:
    \[
    det(B) = 
    \begin{vmatrix}
    1 & 3 & 2 \\
    2 & -1 & 0 \\
    5 & 1 & -2
    \end{vmatrix}
    =28 \neq 0\]

    Следовательно, $b_1, b_2, b_3$ - ЛНЗ. Следовательно, $b_1, b_2, b_3$ - базис V т.к. линейная оболочка не менялась.
    
    Матрица перехода \( C \) состоит из координат векторов \( b_1, b_2, b_3 \) в базисе \( a_1, a_2, a_3 \), что уже представлено матрицей \( B \). Следовательно:
    \[
    C = B.
    \]
    Теперь найдем обратную матрицу \( C^{-1} \), которая позволяет переходить от базиса \( a_1, a_2, a_3 \) к базису \( b_1, b_2, b_3 \).

    
    Обратная матрица \( C^{-1} \) (переход от базиса \( a_1, a_2, a_3 \) к базису \( b_1, b_2, b_3 \)) равна:
    $$(C | E) \to (E|C^{-1})$$
    $$ \begin{pmatrix}
        1 & 2 & 5 & | & 1 & 0 & 0 \\
        3 & -1 & 1& | & 0 & 1 & 0\\
        2 & 0 & -2& | & 0 & 0 & 1
        \end{pmatrix} \to \begin{pmatrix}
             1 & 0 & 0 & | &\frac{1}{14} & \frac{1}{7} & \frac{1}{4} \\
             0 & 1 & 0 & | &\frac{2}{7} & -\frac{3}{7} & \frac{1}{2} \\
             0 & 0 & 1 & | &\frac{1}{14} & \frac{1}{7} & -\frac{1}{4}
            \end{pmatrix}$$

    
    Теперь вычислим координаты вектора \( a_1 + 4a_2 - 3a_3 \) в новом базисе, используя формулу:
    \[
    x' = C^{-1} \cdot x,
    \]
    где \( x = \begin{pmatrix} 1 \\ 4 \\ -3 \end{pmatrix} \) — координаты вектора в старом базисе.

    Координаты вектора \( a_1 + 4a_2 - 3a_3 \) в базисе \( b_1, b_2, b_3 \) равны:
    $$x' = \begin{pmatrix}
        \frac{1}{14} & \frac{1}{7} & \frac{1}{4} \\
        \frac{2}{7} & -\frac{3}{7} & \frac{1}{2} \\
       \frac{1}{14} & \frac{1}{7} & -\frac{1}{4}
       \end{pmatrix}\cdot \begin{pmatrix} 1 \\ 4 \\ -3 \end{pmatrix} = 
        \begin{pmatrix} 
        -\frac{3}{28} \\ 
        -\frac{41}{14} \\ 
        \frac{39}{28} 
        \end{pmatrix}
        $$

    \textbf{Ответ: } $\begin{pmatrix} 
        -\frac{3}{28} \\ 
        -\frac{41}{14} \\ 
        \frac{39}{28} 
        \end{pmatrix}$
    
    \item[\textbf{№2}]Так как в матрице $C$ в i-ом столбце записаны координаты $e_i'$ в первом базисе, то при записи второго базиса в обратном порядке, в матрице $C$ строки будут тоже в обратном порядке. При записи первого базиса в обратном порядке, столбцы будут в обратном порядке
    
    \item[\textbf{№3}]Чтобы доказать, что они являются базисами, достачно доказать их ЛНЗ. Для этого посчитаем определитель матрицы векторов каждого базиса:
    $$
    \det{
    -4 & 3 & 4 \\
    -3 & -3 & 3 \\
    2 & -1 & -5
    } = -63 \neq 0 
    \quad \quad
    \det{-19 & 4 & 29 \\
    -51 & 3 & 6 \\
    37 & -2 & -34} = -4599 \neq 0$$
    Следовательно, базисы $e$ и $e'$ - ЛНЗ, а значит они базисы в $\RR^3$ т.к. $$\dim(\RR^3) = \dim(e) = \dim(e')$$

    Найдём матрицу перехода от $e$ к $e'$, для этого найдём координаты $e'$ в $e$:
    $$
    \mat{-4\\-3\\2}x_1 + \mat{3\\-3\\-1}x_2 + \mat{4 \\ 3\\-5}x_3 = \mat{-19 \\-51\\37} \implies \mat{x_1 \\ x_2\\x_3} = \mat{2 \\ 7\\ -8}
    $$
    $$
    \mat{-4\\-3\\2}x_1 + \mat{3\\-3\\-1}x_2 + \mat{4 \\ 3\\-5}x_3 = \mat{4 \\ 3\\ -2} \implies \mat{x_1 \\ x_2\\x_3} = \mat{-1 \\ 0\\ 0}
    $$
    $$
    \mat{-4\\-3\\2}x_1 + \mat{3\\-3\\-1}x_2 + \mat{4 \\ 3\\-5}x_3 = \mat{29 \\ 6\\ -34} \implies \mat{x_1 \\ x_2\\x_3} = \mat{2 \\ 3\\ 7}
    $$
    \textbf{Ответ:} $\begin{pmatrix}2 & -1 & 2 \\ 7 & 0 & 3 \\ -8 & 0 & 7\end{pmatrix}$\\

    \item[\textbf{№4}]
    Подпространство $ U $ задается векторами:
    $$
    \mathbf{u_1} = \begin{pmatrix} 1 \\ 2 \\ 1 \\ 0 \end{pmatrix}, \quad \mathbf{u_2} = \begin{pmatrix} -1 \\ 1 \\ 1 \\ 1 \end{pmatrix}
    $$

    Подпространство $ W $ задается векторами:
    $$
    \mathbf{w_1} = \begin{pmatrix} 2 \\ -1 \\ 0 \\ 1 \end{pmatrix}, \quad \mathbf{w_2} = \begin{pmatrix} 1 \\ -1 \\ 3 \\ 7 \end{pmatrix}
    $$
    \begin{enumerate}
        \item[\textbf{4.1}]
        
        Следовательно, подпространство $U+W$ задается векторами:
        $$u_1, u_2, w_1, w_2$$
        То есть:
        $$U+W = \langle u_1, u_2, w_1, w_2 \rangle$$
        Чтобы найти базис, нужно проверить их ЛНЗ:
        $$\begin{pmatrix} 1 & -1 & 2 & 1\\ 2 & 1 & -1 & -1\\ 1 & 1 & 0 & 3 \\ 0 & 1 & 1 & 7\end{pmatrix} \to \text{УСВ: }\begin{pmatrix} 1 & 0 & 0 & -1 \\ 0 & 1 & 0 & 4 \\ 0 & 0 & 1 & 3 \\ 0 & 0 & 0 & 0 \end{pmatrix}$$
        Следовательно, веткторы:
        $$u_1, u_2, w_1 - \text{ЛНЗ}$$
        А значит $u_1, u_2, w_1$ - базис в $U+W$, значит:
        $$\dim{(U+W)} = 3$$
        А $\dim(U \cap W)$ можно найти по формуле:
        $$\dim(U \cap W) = dim(U)+dim(W)-dim(U+W)$$
        Так $u_1, u_2$ и $w_1, w_2$ - ЛНЗ, то:
        $$\dim U = 2, \quad \dim W = 2$$
        Значит 
        $$\dim(U \cap W) = 2+2 -3 = 1$$

        \item[\textbf{4.2}]
        Зададим $U$ через ОСЛУ:
        $$U:
        \gath{
            u_1:\\
            u_2:
        } \; \mat{1 & 2 & 1 & 0 &|&0\\
        -1 & 1 & 1 & 1&|&0} \implies \text{УСВ: }\begin{pmatrix}
            1 & 0 & -\frac{1}{3} & -\frac{2}{3} & |&0 \\
            0 & 1 & \frac{2}{3} & \frac{1}{3} & |&0
            \end{pmatrix}
        $$
        Следовательно,
        $$\case{
            x_1 = \frac{1}{3}x_3+\frac{2}{3}x_4\\
            x_2 = -\frac{2}{3}x_3 - \frac{1}{3}x_4
        }\implies \begin{pmatrix}
            \frac{1}{3} \\
            -\frac{2}{3} \\
            1 \\
            0
            \end{pmatrix}, \begin{pmatrix}
                \frac{2}{3} \\
                -\frac{1}{3} \\
                0 \\
                1
                \end{pmatrix} \text{ - ФСР}$$
        Следовательно, матрица ОСЛУ для $U$:
        $$\mat{\frac{1}{3} &-\frac{2}{3} &1 &0 & | & 0\\
        \frac{2}{3} &-\frac{1}{3} &0 &1& | & 0}$$

        Зададим $W$ через ОСЛУ:
        $$W:
        \gath{
            w_1:\\
            w_2:
        } \; \mat{2 & -1 & 0 & 1 &|&0\\
        1 & -1 & 3 & 7&|&0} \implies \text{УСВ: }\begin{pmatrix}
            1 & 0 & -3 & -6 & |&0 \\
            0 & 1 & -6 & -13 & |&0
            \end{pmatrix}
        $$
        Следовательно,
        $$
        \case{x_1 = 3x_3+6x_4\\
        x_2 = 6x_3+13x_4}
        \implies \begin{pmatrix}
            3 \\
            6 \\
            1 \\
            0
            \end{pmatrix}, \begin{pmatrix}
                6 \\
                13 \\
                0 \\
                1
                \end{pmatrix} \text{- ФСР}$$
        Следовательно, матрица ОСЛУ для $U$:
        $$\mat{3 &6 &1 &0&|&0\\
        6 &13 &0 &1&|&0}$$

        Тогда $U \cap W$ задаётся как:
        $$U\cap W:\mat{\frac{1}{3} &-\frac{2}{3} &1 &0 & | & 0\\
        \frac{2}{3} &-\frac{1}{3} &0 &1& | & 0\\
        3 &6 &1 &0&|&0\\
        6 &13 &0 &1&|&0} \implies \text{УСВ: }\begin{pmatrix}
            1 & 0 & 0 & \frac{5}{4} &| & 0 \\
            0 & 1 & 0 & -\frac{1}{2} & | &0 \\
            0 & 0 & 1 & -\frac{3}{4} & | &0 \\
            0 & 0 & 0 & 0 &| & 0
            \end{pmatrix}$$
        $$
        \case{
            x_1 = -\frac{5}{4}x_4\\
            x_2 = \frac{1}{2}x_4\\
            x_3 = \frac{3}{4}x_4\\
        } \implies \left\{\mat{-\frac{5}{4}\\\frac{1}{2}\\\frac{3}{4}\\1}\right\} \text{ - ФСР} 
        $$
        Следовательно,
        $$\dim{(U \cap W)} = 1$$

        \item[\textbf{4.3}]Запшием матрицу:
        $$(u_1, u_2, -w_1, -w_2) = \begin{pmatrix} 1&-1&-2&-1 \\ 2&1&1&1 \\ 1&1&0&-3 \\ 0&1&-1&-7 \end{pmatrix} \implies \text{ УСВ: } \begin{pmatrix} 
            1 & 0 & 0 & 1 \\ 
            0 & 1 & 0 & -4 \\ 
            0 & 0 & 1 & 3 \\ 
            0 & 0 & 0 & 0 
            \end{pmatrix}$$
        Следовательно,
        $$\case{\lambda_1 = -\mu_2\\
        \lambda_2 = 4\mu_2\\
        \mu_1 = -3\mu_2}\implies \text{ФСР: } \mat{-1\\4\\-3\\1}$$
        Значит:
        $$-u_1+4u_2 = \mat{-1\\-2\\-1\\0}+\mat{-4\\4\\4\\4} = \mat{-5\\2\\3\\4} \text{-базис U} \implies \dim U = 1$$
        
    \end{enumerate}
    
    \item[\textbf{№5}]
        
    
\end{enumerate}
\end{document}