\documentclass[a4paper]{article}
\usepackage{setspace}
\usepackage[T2A]{fontenc} %
\usepackage[utf8]{inputenc} % подключение русского языка
\usepackage[russian]{babel} %
\usepackage[12pt]{extsizes}
\usepackage{mathtools}
\usepackage{graphicx}
\usepackage{fancyhdr}
\usepackage{amssymb}
\usepackage{amsmath, amsfonts, amssymb, amsthm, mathtools}
\usepackage{tikz}

\usetikzlibrary{positioning}
\setstretch{1.3}

\newcommand{\mat}[1]{\begin{pmatrix} #1 \end{pmatrix}}
\newcommand{\vmat}[1]{\begin{vmatrix} #1 \end{vmatrix}}
\renewcommand{\f}[2]{\frac{#1}{#2}}
\newcommand{\dspace}{\space\space}
\newcommand{\s}[2]{\sum\limits_{#1}^{#2}}
\newcommand{\mul}[2]{\prod_{#1}^{#2}}
\newcommand{\sq}[1]{\left[ {#1} \right]}
\newcommand{\gath}[1]{\left[ \begin{array}{@{}l@{}} #1 \end{array} \right.}
\newcommand{\case}[1]{\begin{cases} #1 \end{cases}}
\newcommand{\ts}{\text{\space}}
\newcommand{\lm}[1]{\underset{#1}{\lim}}
\newcommand{\suplm}[1]{\underset{#1}{\overline{\lim}}}
\newcommand{\inflm}[1]{\underset{#1}{\underline{\lim}}}
\newcommand{\Ker}[1]{\operatorname{Ker}}

\renewcommand{\phi}{\varphi}
\newcommand{\lr}{\Leftrightarrow}
\renewcommand{\l}{\left(}
\renewcommand{\r}{\right)}
\newcommand{\rr}{\rightarrow}
\renewcommand{\geq}{\geqslant}
\renewcommand{\leq}{\leqslant}
\newcommand{\RR}{\mathbb{R}}
\newcommand{\CC}{\mathbb{C}}
\newcommand{\QQ}{\mathbb{Q}}
\newcommand{\ZZ}{\mathbb{Z}}
\newcommand{\VV}{\mathbb{V}}
\newcommand{\NN}{\mathbb{N}}
\newcommand{\OO}{\underline{O}}
\newcommand{\oo}{\overline{o}}
\renewcommand{\Ker}{\operatorname{Ker}}
\renewcommand{\Im}{\operatorname{Im}}
\newcommand{\vol}{\text{vol}}
\newcommand{\Vol}{\text{Vol}}

\DeclarePairedDelimiter\abs{\lvert}{\rvert} %
\makeatletter                               % \abs{}
\let\oldabs\abs                             %
\def\abs{\@ifstar{\oldabs}{\oldabs*}}       %

\begin{document}

\section*{ИДЗ №9 (вариант 10) (Линейная алгебра)}
 {\large Емельянов Владимир, ПМИ гр №247}\\\\
\begin{enumerate}
    \item[\textbf{№1}]Пусть $P(-35,20,-15)$ --- точка на искомой прямой. Пусть $Q$ - точка на пересекающей прямой, при которой вектор $PQ$ --- это направляющий вектор искомой прямой.
    $$Q(t) = (-t+26, -4t-6, -5t+20)$$
    Тогда вектор $PQ$:
    $$PQ(t) = (-t+26+35, -4t-6-20, -5t+20+15) = (61-t, -4t-26, 35-5t)$$
    По условию искомая прямая параллельна плоскости 
    $$3x+4y-2z=-3$$ 
    Это означает, что прямая перпендикулярна к нормали этой плоскости, то есть должно выполняться:
    $$(PQ, n) = 0 \quad \lr \quad  3\cdot(61-t) + 4\cdot(-4t-26) -2\cdot(35-5t) = 0 \quad \lr \quad t = 1$$
    Получаем направляющий вектор искомой прямой, поделим его на $30$ для удобства:
    $$\vec{a} = \f{1}{30}PQ= \f{1}{30}(60, -30, 30) = (2, -1, 1)$$
    То есть уравнение искомой прямой:
    $$\f{x+35}{2} = \f{y-20}{-1} = \f{z+15}{1}$$
    \textbf{Ответ: }$\f{x+35}{2} = \f{y-20}{-1} = \f{z+15}{1}$

\end{enumerate}
\end{document}