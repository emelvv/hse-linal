\documentclass[a4paper]{article}
\usepackage{setspace}
\usepackage[T2A]{fontenc} %
\usepackage[utf8]{inputenc} % подключение русского языка
\usepackage[russian]{babel} %
\usepackage[12pt]{extsizes}
\usepackage{mathtools}
\usepackage{graphicx}
\usepackage{fancyhdr}
\usepackage{amssymb}
\usepackage{amsmath, amsfonts, amssymb, amsthm, mathtools}
\usepackage{tikz}

\usetikzlibrary{positioning}
\setstretch{1.3}

\newcommand{\mat}[1]{\begin{pmatrix} #1 \end{pmatrix}}
\newcommand{\vmat}[1]{\begin{vmatrix} #1 \end{vmatrix}}
\renewcommand{\f}[2]{\frac{#1}{#2}}
\newcommand{\dspace}{\space\space}
\newcommand{\s}[2]{\sum\limits_{#1}^{#2}}
\newcommand{\mul}[2]{\prod_{#1}^{#2}}
\newcommand{\sq}[1]{\left[ {#1} \right]}
\newcommand{\gath}[1]{\left[ \begin{array}{@{}l@{}} #1 \end{array} \right.}
\newcommand{\case}[1]{\begin{cases} #1 \end{cases}}
\newcommand{\ts}{\text{\space}}
\newcommand{\lm}[1]{\underset{#1}{\lim}}
\newcommand{\suplm}[1]{\underset{#1}{\overline{\lim}}}
\newcommand{\inflm}[1]{\underset{#1}{\underline{\lim}}}
\newcommand{\Ker}[1]{\operatorname{Ker}}

\renewcommand{\phi}{\varphi}
\newcommand{\lr}{\Leftrightarrow}
\renewcommand{\l}{\left(}
\renewcommand{\r}{\right)}
\newcommand{\rr}{\rightarrow}
\renewcommand{\geq}{\geqslant}
\renewcommand{\leq}{\leqslant}
\newcommand{\RR}{\mathbb{R}}
\newcommand{\CC}{\mathbb{C}}
\newcommand{\QQ}{\mathbb{Q}}
\newcommand{\ZZ}{\mathbb{Z}}
\newcommand{\VV}{\mathbb{V}}
\newcommand{\NN}{\mathbb{N}}
\newcommand{\OO}{\underline{O}}
\newcommand{\oo}{\overline{o}}
\renewcommand{\Ker}{\operatorname{Ker}}
\renewcommand{\Im}{\operatorname{Im}}
\newcommand{\vol}{\text{vol}}
\newcommand{\Vol}{\text{Vol}}

\DeclarePairedDelimiter\abs{\lvert}{\rvert} %
\makeatletter                               % \abs{}
\let\oldabs\abs                             %
\def\abs{\@ifstar{\oldabs}{\oldabs*}}       %

\begin{document}

\section*{ИДЗ №9 (вариант 10) (Линейная алгебра)}
 {\large Емельянов Владимир, ПМИ гр №247}\\\\
\begin{enumerate}
    \item[\textbf{№1}]Пусть $P(-35,20,-15)$ --- точка на искомой прямой. Пусть $Q$ - точка на пересекающей прямой, при которой вектор $PQ$ --- это направляющий вектор искомой прямой.
    $$Q(t) = (-t+26, -4t-6, -5t+20)$$
    Тогда вектор $PQ$:
    $$PQ(t) = (-t+26+35, -4t-6-20, -5t+20+15) = (61-t, -4t-26, 35-5t)$$
    По условию искомая прямая параллельна плоскости 
    $$3x+4y-2z=-3$$ 
    Это означает, что прямая перпендикулярна к нормали этой плоскости, то есть должно выполняться:
    $$(PQ, n) = 0 \quad \lr \quad  3\cdot(61-t) + 4\cdot(-4t-26) -2\cdot(35-5t) = 0 \quad \lr \quad t = 1$$
    Получаем направляющий вектор искомой прямой, поделим его на $30$ для удобства:
    $$\vec{a} = \f{1}{30}PQ= \f{1}{30}(60, -30, 30) = (2, -1, 1)$$
    То есть уравнение искомой прямой:
    $$\f{x+35}{2} = \f{y-20}{-1} = \f{z+15}{1}$$
    \textbf{Ответ: }$\f{x+35}{2} = \f{y-20}{-1} = \f{z+15}{1}$\\


    \item[\textbf{№2}]Направим ось $x$ по лучу $BA$, ось $y$ по лучу $BC$, ось $z$ по лучу $BB'$, тогда по отношениям в условии найдём координаты точек:
    $$\begin{aligned}
        A = (10, 0, 0), \quad E = (0, 0, 6), \quad & D' = (10, 10, 10), \quad F = (0, 0, 5)\\
        AE = (-10, 0, 6)\quad & D'F = (-10, -10, -5) 
    \end{aligned}$$
    Найдём косинус угла между $AE$ и $D'F$:
    $$
    \vec{AE}\cdot\vec{D'F}=(-10)\cdot(-10)+0\cdot(-10)+6\cdot(-5)=100-30=70,
    $$
    $$
    |\vec{AE}|=2\sqrt{34},\qquad
    |\vec{D'F}|=15
    $$
    $$
    \cos\angle(AE,D'F)
    =\frac{\vec{AE}\cdot\vec{D'F}}{|\vec{AE}|\;|\vec{D'F}|}
    =\frac{70}{2\sqrt{34}\,\cdot15}
    =\frac{70}{30\sqrt{34}}
    =\frac{7}{3\sqrt{34}}
    =\frac{7\sqrt{34}}{102}
    $$
    Найдём расстояние между ними:
    $$
    d=\frac{\bigl|\overrightarrow{A D'}\cdot\bigl(\overrightarrow{AE}\times\overrightarrow{D'F}\bigr)\bigr|}
    {\bigl|\overrightarrow{AE}\times\overrightarrow{D'F}\bigr|},
    $$
    где
    $$
    \overrightarrow{AE}=(-10,0,6),\quad
    \overrightarrow{D'F}=(-10,-10,-5),\quad
    \overrightarrow{A D'}=(0,10,10).
    $$
    найдём векторное произведение
    $$
    \overrightarrow{AE}\times\overrightarrow{D'F}
    =\det\begin{pmatrix}
    \mathbf{i}&\mathbf{j}&\mathbf{k}\\
    -10&0&6\\
    -10&-10&-5
    \end{pmatrix}
    =(60,\,-110,\,100),
    $$
    $$
    \Bigl|\overrightarrow{AE}\times\overrightarrow{D'F}\Bigr|
    =\sqrt{60^2+(-110)^2+100^2}
    =\sqrt{25700}
    =10\sqrt{257}.
    $$
    найдём скалярное произведение с вектором $\overrightarrow{A D'}$:
    $$
    \overrightarrow{A D'}\cdot\bigl(\overrightarrow{AE}\times\overrightarrow{D'F}\bigr)
    =(0,10,10)\cdot(60,-110,100)
    =-1100+1000=-100,
    $$ 
    Подставляем в формулу:
    $$
    d=\frac{100}{10\sqrt{257}}
    =\frac{10}{\sqrt{257}}
    $$
    \textbf{Ответ: } $\arccos(\frac{7\sqrt{34}}{102})$ и $\frac{10}{\sqrt{257}}$\\

    \item[\textbf{№3}]
    \begin{enumerate}
    \item[а)]
    Найдём с.з. и с.в. линейного оператора $\phi$ с матрицей:
    $$A=\begin{pmatrix}
        3 & 4 & -8\\
        -2 & -4 & 3\\
        2 & 2 & -5
    \end{pmatrix}$$
    Найдём характеристический многочлен:
    
    $$
    \chi_A(\lambda)=\det(\lambda I - A)
    =\det\begin{pmatrix}
    \lambda-3 & -4 & 8\\
    2 & \lambda+4 & -3\\
    -2 & -2 & \lambda+5
    \end{pmatrix} = \lambda^3+6\lambda^2+11\lambda+6
    $$
    Получаем:
    $$\chi_A(\lambda)=(\lambda+1)(\lambda+2)(\lambda+3)$$
    Значит:
    $$\text{spec}\; \phi = \{-1, -2, -3\}$$
    Найдём базис $V_{-3}$:
    $$
    (A+3E)v=0
    \;\Longrightarrow\;
    v_1=\begin{pmatrix}2\\-1\\1\end{pmatrix}.
    $$
    Найдём базис $V_{-2}$
    $$
    (A+2E)v=0
    \;\Longrightarrow\;
    v_2=\begin{pmatrix}2\\-\tfrac12\\1\end{pmatrix}.
    $$
    Найдём базис $V_{-1}$
    $$
    (A+E)v=0
    \;\Longrightarrow\;
    v_3=\begin{pmatrix}3\\-1\\1\end{pmatrix}.
    $$
    Поскольку матрица имеет три различных собственных значения, она диагонализуема. Например, в базисе
    $$
    \{\,v_1,\,v_2,\,v_3\}
    $$
    матрица оператора имеет диагональный вид
    $$
    P^{-1}AP=\operatorname{diag}(-3,\,-2,\,-1).
    $$\\

    \item[б)]Найдём с.з. и с.в. линейного оператора $\phi$ с матрицей:
    $$A=\begin{pmatrix}
        -2 & -1 & 1\\
        -3 & -2 & -2\\
        0 & 2 & -4
        \end{pmatrix}$$

    \end{enumerate}
    Найдём характеристический многочлен:
    $$
    \chi_A(\lambda)=\det(\lambda E - A)
    =\det\begin{pmatrix}
    \lambda+2 & 1 & -1\\
    3 & \lambda+2 & 2\\
    0 & -2 & \lambda+4
    \end{pmatrix}
    =\lambda^3+8\lambda^2+21\lambda+18
    $$
    Значит:
    $$\chi_A(\lambda)=(\lambda+2)(\lambda+3)^2$$
    Следовательно:
    $$\text{spec}\; \phi = \{-1, -3\}$$
    Найдём базис $V_{-3}$:
    $$
    (A+3E)v=0
    \;\Longrightarrow\;
    \begin{cases}
    x_1 - x_2 + x_3 =0,\\
    -3x_1 + x_2 -2x_3 =0,\\
    2x_2 - x_3 =0.
    \end{cases}
    \implies 
    v=\langle \begin{pmatrix}-1\\1\\2\end{pmatrix} \rangle
    $$
    Найдём базис $V_{-1}$:
    $$
    (A+2I)v=0
    \;\Longrightarrow\;
    \begin{cases}
    0\cdot x_1 - x_2 + x_3 =0,\\
    -3x_1 + 0\cdot x_2 -2x_3 =0,\\
    0\cdot x_1 +2x_2 -2x_3 =0.
    \end{cases} \implies 
    v = \langle \begin{pmatrix}-2\\3\\3\end{pmatrix}\rangle
    $$

    Так как для $\lambda = -1$ алгебраическая кратность ($a_{-1} = 2$) не равна геометрической ($g_{-1} = 1$), то линейный оператор $\phi$ \textbf{не диагонализуем}\\
\end{enumerate}
\end{document}