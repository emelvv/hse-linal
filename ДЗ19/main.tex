\documentclass[a4paper]{article}
\usepackage{setspace}
\usepackage[T2A]{fontenc} %
\usepackage[utf8]{inputenc} % подключение русского языка
\usepackage[russian]{babel} %
\usepackage[12pt]{extsizes}
\usepackage{mathtools}
\usepackage{graphicx}
\usepackage{fancyhdr}
\usepackage{amssymb}
\usepackage{amsmath, amsfonts, amssymb, amsthm, mathtools}
\usepackage{tikz}

\usetikzlibrary{positioning}
\setstretch{1.3}

\newcommand{\mat}[1]{\begin{pmatrix} #1 \end{pmatrix}}
\newcommand{\vmat}[1]{\begin{vmatrix} #1 \end{vmatrix}}
\renewcommand{\f}[2]{\frac{#1}{#2}}
\newcommand{\dspace}{\space\space}
\newcommand{\s}[2]{\sum\limits_{#1}^{#2}}
\newcommand{\mul}[2]{\prod_{#1}^{#2}}
\newcommand{\sq}[1]{\left[ {#1} \right]}
\newcommand{\gath}[1]{\left[ \begin{array}{@{}l@{}} #1 \end{array} \right.}
\newcommand{\case}[1]{\begin{cases} #1 \end{cases}}
\newcommand{\ts}{\text{\space}}
\newcommand{\lm}[1]{\underset{#1}{\lim}}
\newcommand{\suplm}[1]{\underset{#1}{\overline{\lim}}}
\newcommand{\inflm}[1]{\underset{#1}{\underline{\lim}}}
\newcommand{\Ker}[1]{\operatorname{Ker}}

\renewcommand{\phi}{\varphi}
\newcommand{\lr}{\Leftrightarrow}
\renewcommand{\r}{\Rightarrow}
\newcommand{\rr}{\rightarrow}
\renewcommand{\geq}{\geqslant}
\renewcommand{\leq}{\leqslant}
\newcommand{\RR}{\mathbb{R}}
\newcommand{\CC}{\mathbb{C}}
\newcommand{\QQ}{\mathbb{Q}}
\newcommand{\ZZ}{\mathbb{Z}}
\newcommand{\VV}{\mathbb{V}}
\newcommand{\NN}{\mathbb{N}}
\newcommand{\OO}{\underline{O}}
\newcommand{\oo}{\overline{o}}
\renewcommand{\Ker}{\operatorname{Ker}}
\renewcommand{\Im}{\operatorname{Im}}


\DeclarePairedDelimiter\abs{\lvert}{\rvert} %
\makeatletter                               % \abs{}
\let\oldabs\abs                             %
\def\abs{\@ifstar{\oldabs}{\oldabs*}}       %

\begin{document}

\section*{Домашнее задание на 24.02 (Линейная алгебра)}
 {\large Емельянов Владимир, ПМИ гр №247}\\\\
\begin{enumerate}
    \item[\textbf{№1}]
    \begin{enumerate}
        \item[1.1]$Q(x_1, x_2, x_3) = 2x_1^2 - 12x_3^2 - 2x_1x_2 + 2x_1x_3 - 6x_2x_3 $

        Матрица квадратичной формы:
        $$\mat{
            2 & -1 & 1 \\
            -1 & 0 & -3\\
            1 & -3 & -12
        }$$
        $$\delta_1 = 2, \quad \delta_2 = 1, \quad \delta_3 = 3+3+12-18 = 0$$
        Нормальный вид:
        $$Q(x_1, x_2, x_3) = x_1' + x_2'$$\\

        \item[1.2]$ Q(x_1, x_2, x_3) = x_1^2 + 9x_2^2 + 4x_3^2 + 6x_1x_2 - 4x_1x_3 + 10x_2x_3 $
        
        Матрица квадратичной формы:
        $$\mat{
            1 & 3 & -2 \\
            3 & 9 & 5\\
            -2& 5 & 4   
        } \implies \mat{
            1 & -2& 3 \\
            -2& 4& 5   \\
            3 & 5& 9 \\
        }$$
        $$\delta_1 = 1, \quad \delta_2 = 8, \quad \delta_3 = 36-60-36-25-36 = -121$$
        Нормальный вид:
        $$Q(x_1, x_2, x_3) = x_1' + x_2' -x_3'$$\\
    \end{enumerate}

    \item[\textbf{№2}]$Q(x) = x_1^2 + x_2^2 + 5x_3^2 + 2\lambda x_1x_2 - 2x_1x_3 + 4x_2x_3$
    
    Матрица квадратичной формы:
    $$\mat{
        1 & \lambda & -1 \\
        \lambda & 1 & 2\\
        -1 & 2 & 5
    }$$
    $$\delta_1 = 1, \quad \delta_2 = 1-\lambda^2, \quad \delta_3 = 5 -4\lambda-1-5\lambda^2-4 = -4\lambda - 5\lambda^2$$
    Положительна определена при:
    $$\case{
        1-\lambda^2 > 0 \lr \lambda \in (-1; 1)\\
        -4\lambda-5\lambda^2 > 0 \lr \lambda(4+5\lambda) < 0 \lr \lambda \in (-\f{4}{5};0)
    }$$
    \textbf{Ответ: } $\lambda \in (-\f{4}{5};0)$

    \item[\textbf{№3}]$Q(x) = x_1^2 + 4x_2^2 + x_3^2 + 2\lambda x_1x_2 + 10x_1x_3 + 6x_2x_3$
    
    Матрица квадратичной формы:
    $$\mat{
        1 & \lambda & 5 \\
        \lambda & 4 & 3\\
        5 & 3 & 1
    }$$
    $$\delta_1 = 1, \quad \delta_2 = 4-\lambda^2, \quad \delta_3 = 4+30\lambda-100-\lambda^2-9 = -105+30\lambda -\lambda^2$$
    \begin{enumerate}
        \item[1)] $\lambda \in (-2, 2) \implies \delta_2 > 0$
        $$f(x) = -\lambda^2 + 30 \lambda-105 = 0 \implies \lambda = -15 \pm 5\sqrt{5}$$
        $$f(x) < 0 : \lambda \in (-\infty; -15-2\sqrt{30})\cap (-15+2\sqrt{30};+\infty)$$
        $$f(x) > 0 : \lambda \in (-15-2\sqrt{30}; -15+2\sqrt{30})$$
        Т.к. $-2 > -15+2\sqrt{30}$, то:
        $$\delta_3 < 0 $$
        Нормальный вид:
        $$Q(x) = x_1' + x_2' - x_3'$$
        \item[2)] $\lambda \in (-\infty;-2)\cup(2;+\infty) \implies \delta_2 < 0$
        
        При $\lambda < -15-2\sqrt{30}$:
        $$\delta_3 < 0$$
        $$Q(x) = x_1' - x_2' + x_3'$$
        При $\lambda > -15-2\sqrt{30}$ и $\lambda < -15+2\sqrt{30}$ :
        $$\delta_3 > 0$$
        $$Q(x) = x_1' - x_2' - x_3'$$
        При $\lambda \in (-15+2\sqrt{30};-2) \cup (2;+\infty)$:
        $$\delta_3 < 0$$
        $$Q(x) = x_1' - x_2' + x_3'$$
        При $\lambda \in \{-15-2\sqrt{30}, -15+2\sqrt{30}\}$:
        $$\delta_3 = 0$$
        $$Q(x) = x_1' - x_2'$$

        \item[3)]$\lambda = -2 $:
        
        Матрица квадратичной формы:
        $$\mat{
            1 & -2 & 5 \\
            -2 & 4 & 3\\
            5 & 3 & 1
        } \implies \mat{
            1 & 5& -2  \\
            5  & 1& 3 \\
            -2& 3 & 4 
        }$$
        $$\delta_1 = 1, \quad \delta_2 = -24, \quad \delta_3 = 4-60-4-100-9 = -169$$
        Нормальный вид:
        $$Q(x) = x_1' -x_2'+x_3'$$

        \item[3)]$\lambda = 2 $:
        
        Матрица квадратичной формы:
        $$\mat{
            1 & 2 & 5 \\
            2 & 4 & 3\\
            5 & 3 & 1
        } \implies \mat{
            1 & 5& 2  \\
            5  & 1& 3 \\
            2& 3 & 4 
        }$$
        $$\delta_1 = 1, \quad \delta_2 = -24, \quad \delta_3 = 4+60-4-100-9 = -49$$
        Нормальный вид:
        $$Q(x) = x_1' -x_2'+x_3'$$
    \end{enumerate}

    Мы рассмотрели все случаи.

    \item[\textbf{№4}]Например $Q(x) = x_1^2 + 4x_2^2 + x_3^2 + 4 x_1x_2 + 10x_1x_3 + 6x_2x_3$:
    $$\mat{
        1 & 2 & 5 \\
        2 & 4 & 3\\
        5 & 3 & 1
    }$$
    $$\delta_1 = 1, \quad \delta_2 = 0, \quad \delta_3 = 4+60-100-9-4 = -49$$
    Но при этом по задаче выше:
    $$Q(x) = x_1' -x_2'+x_3'$$
    Следовательно, $Q$ неопределённая.

    \item[\textbf{№5}]Пусть $Q$ не неопределённая, тогда рассмотрим случаи:
    \begin{enumerate}
        \item[1)] $\forall v \in V: Q(v) > 0$:
        
        Тогда у нас не существует ни одного изотропного вектора.

        Следовательно, не существует базиса, состоящего из изотропных векторов.


        \item[3)] $\forall v \in V: Q(v) < 0$:
        
        Аналогично не существует ни одного изотропного вектора.

        Следовательно, не существует базиса, состоящего из изотропных векторов.

    \end{enumerate}

\end{enumerate}
\end{document}