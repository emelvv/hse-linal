\documentclass[a4paper]{article}
\usepackage{setspace}
\usepackage[T2A]{fontenc} %
\usepackage[utf8]{inputenc} % подключение русского языка
\usepackage[russian]{babel} %
\usepackage[12pt]{extsizes}
\usepackage{mathtools}
\usepackage{graphicx}
\usepackage{fancyhdr}
\usepackage{amssymb}
\usepackage{amsmath, amsfonts, amssymb, amsthm, mathtools}
\usepackage{tikz}

\usetikzlibrary{positioning}
\setstretch{1.3}

\newcommand{\mat}[1]{\begin{pmatrix} #1 \end{pmatrix}}
\newcommand{\vmat}[1]{\begin{vmatrix} #1 \end{vmatrix}}
\renewcommand{\f}[2]{\frac{#1}{#2}}
\newcommand{\dspace}{\space\space}
\newcommand{\s}[2]{\sum\limits_{#1}^{#2}}
\newcommand{\mul}[2]{\prod_{#1}^{#2}}
\newcommand{\sq}[1]{\left[ {#1} \right]}
\newcommand{\gath}[1]{\left[ \begin{array}{@{}l@{}} #1 \end{array} \right.}
\newcommand{\case}[1]{\begin{cases} #1 \end{cases}}
\newcommand{\ts}{\text{\space}}
\newcommand{\lm}[1]{\underset{#1}{\lim}}
\newcommand{\suplm}[1]{\underset{#1}{\overline{\lim}}}
\newcommand{\inflm}[1]{\underset{#1}{\underline{\lim}}}
\newcommand{\Ker}[1]{\operatorname{Ker}}

\renewcommand{\phi}{\varphi}
\newcommand{\lr}{\Leftrightarrow}
\renewcommand{\l}{\left(}
\renewcommand{\r}{\right)}
\newcommand{\rr}{\rightarrow}
\renewcommand{\geq}{\geqslant}
\renewcommand{\leq}{\leqslant}
\newcommand{\RR}{\mathbb{R}}
\newcommand{\CC}{\mathbb{C}}
\newcommand{\QQ}{\mathbb{Q}}
\newcommand{\ZZ}{\mathbb{Z}}
\newcommand{\VV}{\mathbb{V}}
\newcommand{\NN}{\mathbb{N}}
\newcommand{\OO}{\underline{O}}
\newcommand{\oo}{\overline{o}}
\renewcommand{\Ker}{\operatorname{Ker}}
\renewcommand{\Im}{\operatorname{Im}}
\newcommand{\vol}{\text{vol}}
\newcommand{\Vol}{\text{Vol}}

\DeclarePairedDelimiter\abs{\lvert}{\rvert} %
\makeatletter                               % \abs{}
\let\oldabs\abs                             %
\def\abs{\@ifstar{\oldabs}{\oldabs*}}       %

\begin{document}

\section*{Домашнее задание на 24.04 (Линейная алгебра)}
 {\large Емельянов Владимир, ПМИ гр №247}\\\\
\begin{enumerate}
    \item[\textbf{№1}]\begin{enumerate}
        \item[1.1]Для любого линейного оператора $\phi$ должно быть $\phi(0)=0$.  
        Но  
        \[
          \phi(0,0,0) = (0+2,\;0+5,\;0) = (2,5,0) \neq (0,0,0).
        \]  
        \textbf{Ответ: } нет\\

        \item[1.2]Для любых \(f,g\) из \(\RR[x]_{\le n}\):
        $$
          \phi\bigl(f+g\bigr) = (f+g)(x+1) - (f+g)(x)=$$
          $$
            = \bigl[f(x+1)-f(x)\bigr] + \bigl[g(x+1)-g(x)\bigr]
            = \phi(f) + \phi(g)
        $$
        Также, для любого числа \(c\in\RR\) и любого \(f\):
        \[
          \phi\bigl(c\,f\bigr) = (c\,f)(x+1) - (c\,f)(x)
                           = c\bigl[f(x+1)-f(x)\bigr]
                           = c\,\phi(f).
        \]
        и:
        \[\;\phi(0)=0(x+1)-0(x)=0\]
        Следовательно $\phi$ является линейным оператором.
        \textbf{Ответ: } да\\
    \end{enumerate}
    \item[\textbf{№2}]Рассмотрим:
    $$ \varphi(e_1) = 3e_1 - 2e_3 + e_4$$
    $$ \varphi(e_2) = 4e_2$$
    $$(\varphi - 5 \, id)(e_3) = 0 \Rightarrow \varphi(e_3) = 5e_3$$
    $$(\varphi - 7 \, id)(e_4) = e_3 \Rightarrow \varphi(e_4) = 7e_4 + e_3$$

   То есть:
   \[
   [\varphi] =
   \begin{pmatrix}
   3 & 0 & 0 & 0 \\
   0 & 4 & 0 & 0 \\
   -2 & 0 & 5 & 1 \\
   1 & 0 & 0 & 7 \\
   \end{pmatrix}
   \]

   \item[\textbf{№3}]
   \begin{enumerate}
    \item[3.1]\[
    A = 
    \begin{pmatrix}
    2 & -1 & 2 \\
    5 & -3 & 3 \\
    -1 & 0 & -2
    \end{pmatrix}
    \]
    Решим:
    \[
    \det(A - \lambda E) = 0
    \]
    Найдём определитель матрицы:
    \[
    \det \begin{pmatrix}
    2 - \lambda & -1 & 2 \\
    5 & -3 - \lambda & 3 \\
    -1 & 0 & -2 - \lambda
    \end{pmatrix} = 0 \implies \lambda = -1
    \]
    Подставляем \( \lambda = -1 \) в уравнение:
    \[
    (A + E)v = 0
    \]
    Или:
    \[
    \begin{pmatrix}
    3 & -1 & 2 \\
    5 & -2 & 3 \\
    -1 & 0 & -1
    \end{pmatrix}
    \cdot
    \begin{pmatrix}
    x \\
    y \\
    z
    \end{pmatrix}
    = 0
    \implies v = \begin{pmatrix} -1 \\ -1 \\ 1 \end{pmatrix} \]
    \textbf{Ответ: } с/з --- $-1$, с/в --- $v$\\

    \item[3.2]
    Дана матрица:
    \[
    A = \begin{pmatrix}
    1 & 1 \\
    -1 & 1
    \end{pmatrix}
    \]
    Ищем корни уравнения:
    \[
    \det(A - \lambda E) = 0
    \]
    Определитель:
    \[
    (1 - \lambda)^2 + 1 = 0
    \]
    Раскроем:
    \[
    (1 - \lambda)^2 = \lambda^2 - 2\lambda + 1
    \Rightarrow
    \lambda^2 - 2\lambda + 1 + 1 = \lambda^2 - 2\lambda + 2 = 0
    \]
    Решаем уравнение:
    \[
    \lambda^2 - 2\lambda + 2 = 0
    \]
    Дискриминант:
    \[
    D = (-2)^2 - 4 \cdot 1 \cdot 2 = 4 - 8 = -4
    \]
    Комплексные корни:
    \[
    \lambda = \frac{2 \pm \sqrt{-4}}{2} = \frac{2 \pm 2i}{2} = 1 \pm i
    \]
    
    Рассмотрим \( \lambda = 1 + i \), решаем:
    \[
    (A - \lambda E)v = 0
    \Rightarrow
    \left(
    \begin{pmatrix}
    1 & 1 \\
    -1 & 1
    \end{pmatrix}
    -
    (1 + i)
    \begin{pmatrix}
    1 & 0 \\
    0 & 1
    \end{pmatrix}
    \right)
    \cdot
    \begin{pmatrix}
    x \\
    y
    \end{pmatrix}
    = 0
    \]
    Система уравнений:
    \[
    \begin{cases}
    -ix + y = 0 \Rightarrow y = ix \\
    - x - i y = 0
    \end{cases}
    \]
    Значит, собственный вектор:
    \[
    v = \begin{pmatrix} 1 \\ i \end{pmatrix}
    \]
    Аналогично для \( \lambda = 1 - i \):
    \[
    v = \begin{pmatrix} 1 \\ -i \end{pmatrix}
    \]
    \textbf{Ответ: } Над $\RR$: нету, над $\CC$:
    \[
    \lambda_1 = 1 + i,\quad v_1 = \begin{pmatrix} 1 \\ i \end{pmatrix}\quad
    \lambda_2 = 1 - i,\quad v_2 = \begin{pmatrix} 1 \\ -i \end{pmatrix}
    \]

    \item[3.3]Заметим, что \(M\) блочная:
    \[
    M=\begin{pmatrix}
    A & 0\\
    B & C
    \end{pmatrix},
    \quad
    A=\begin{pmatrix}3&-1\\1&1\end{pmatrix},
    \quad
    C=\begin{pmatrix}5&-3\\3&-1\end{pmatrix}
    \]
    Тогда
    \[
    \chi_M(\lambda)=\det(A-\lambda E)\,\det(C-\lambda E)
    \]
    Вычислим каждое:
    \[
    \det\!\begin{pmatrix}3-\lambda&-1\\1&1-\lambda\end{pmatrix}
    =(3-\lambda)(1-\lambda)+1
    =\lambda^2-4\lambda+4=(\lambda-2)^2,
    \]
    \[
    \det\!\begin{pmatrix}5-\lambda&-3\\3&-1-\lambda\end{pmatrix}
    =(5-\lambda)(-1-\lambda)+9
    =\lambda^2-4\lambda+4=(\lambda-2)^2
    \]
    Итого  
    \[
    \chi_M(\lambda)=(\lambda-2)^4,
    \]  
    то есть единственный собственный корень \(\lambda=2\) алгебраической кратности 4.

    Решим \((M-2E)v=0\).  
    \[
    M-2E = 
    \begin{pmatrix}
    1 & -1 & 0 & 0\\
    1 & -1 & 0 & 0\\
    3 &  0 & 3 & -3\\
    4 & -1 & 3 & -3
    \end{pmatrix},
    \quad
    v=(x_1,x_2,x_3,x_4)^T
    \]
    множество решений:
    $$\langle (1, 1, 0, 1), (-1, -1, 1, 0)\rangle$$

    \textbf{Ответ: } с/з --- $2$, с/в --- $V_2 = \langle e_1 + e_2 + e_4, -e_1-e_2+e_3 \rangle$\\


   \end{enumerate}

    \textbf{№4} Пусть \(\varphi^2\) имеет собственное значение \(\lambda^2\), существует ненулевой \(v\in V\) такой, что
    \[
    \varphi^2(v) = \lambda^2\,v.
    \]
    Рассмотрим два случая.
    \begin{enumerate}
      \item[1)]\(\varphi(v)\) и \(v\) ЛЗ, то есть \(\varphi(v)=\mu v\).  
   
      Тогда из \(\varphi^2(v)=\mu^2v=\lambda^2v\) следует \(\mu^2=\lambda^2\), значит \(\mu=\lambda\) или \(\mu=-\lambda\).  
      В первом случае \(\varphi(v)=\lambda v\), во втором — \(\varphi(v)=-\lambda v\).  
      В обоих случаях найден ненулевой собственный вектор оператора \(\varphi\) с собственным значением \(\lambda\) или \(-\lambda\).
    
      \item[2)] \(v\) и \(\varphi(v)\) ЛНЗ. 
      
      Построим в \(V\) вектор
      \[
      u = \varphi(v) - \lambda\,v.
      \]
      Заметим, во‑первых, что \(u\neq0\), иначе \(\varphi(v)=\lambda v\) и мы оказались в случае 1.  
      Во‑вторых, вычислим
      \[
      \varphi(u)
      = \varphi\bigl(\varphi(v) - \lambda v\bigr)
      = \varphi^2(v)-\lambda\,\varphi(v)
      = \lambda^2v - \lambda\,\varphi(v)
      = -\lambda\bigl(\varphi(v)-\lambda v\bigr)
      = -\lambda\,u
      \]
      Итак, \(u\) — ненулевой собственный вектор \(\varphi\) с собственным значением \(-\lambda\).
    \end{enumerate}
    Следовательно, оператор \(\varphi\) имеет собственный вектор с собственным значением либо \(\lambda\), либо \(-\lambda\).\\

    \textbf{№5}\begin{enumerate}
      \item[5.1]Матрица:
      \[
      A = \begin{pmatrix}
      1 & 2\\
      -1 & 0
      \end{pmatrix}
      \]
      Характеристический многочлен:
      \[
      \chi_A(x) = x^2 - x + 2
      \]
      $$D < 0 \implies \text{ нет с/з}$$

      \textbf{Ответ: } не диаг.\\

      \item[5.2]Матрица:
      \[
      A =
      \begin{pmatrix}
      2 & 2\\
      -2 & -2
      \end{pmatrix}
      \]
    \end{enumerate}

    Характеристический многочлен:
    \[
    \chi_A(x) = x^2
    \]
    Собственные значения:
    \[
    \lambda = 0\quad\text{(кратность 2)}
    \]
    Собственные векторы:
    \[v = (-1, 1)^T\]
    Не диагонализуема, так как геометрическая кратность меньше алгебраической.
    
    \textbf{Ответ: } не диаг.\\

    \item[5.3]Матрица:
    \[
    A = 
    \begin{pmatrix}
    1 & 1 & 1 & 1\\
    1 & 1 & -1 & -1\\
    1 & -1 & 1 & -1\\
    1 & -1 & -1 & 1
    \end{pmatrix}
    \]
    Характеристический многочлен:
    \[
    \chi_A(x) = x^4 - 4x^3 + 16x - 16
    \]
    Собственные значения:
    \[
    \lambda_1 = -2,\quad \lambda_2 = 2
    \]
    Собственные векторы (базис):
    \[
    v_1 = (-1,1,1,1)^T,\;
    v_2 = (1,0,0,1)^T,\;
    v_3 = (1,0,1,0)^T,\;
    v_4 = (1,1,0,0)^T
    \]
    Собственные векторы линейно независимы, значит образуют базис $V$, следовательно, матрица диагонализуема  
    
    Диагональная форма:
    \[
    A' = \mathrm{diag}(-2,\;2,\;2,\;2)
    \]

    Матрица перехода \(C\):
    \[
    C = \begin{pmatrix}
    -1 & 1 & 1 & 1\\
    1  & 0 & 0 & 1\\
    1  & 0 & 1 & 0\\
    1  & 1 & 0 & 0
    \end{pmatrix}
    \quad\text{и}\quad
    A' = C^{-1}AC
    \]

    \textbf{Ответ: } диаг.\\

    \item[5.4]Матрица:
    \[
    A =
    \begin{pmatrix}
    -1 & 3 & -1\\
    -3 & 5 & -1\\
    -3 & 3 & 1
    \end{pmatrix}
    \]
    Характеристический многочлен:
    \[
    \chi_A(x) = -x^3 + 5x^2 - 8x + 4
    \]
    Собственные значения:
    \[
    \lambda_1 = 1,\quad \lambda_2 = 2
    \]
    Собственные векторы:
    \[
    v_1 = (-1,0,3)^T,\;
    v_2 = (1,1,0)^T,\;
    v_3 = (1,1,1)^T
    \]
    Они ЛНЗ, следовательно, матрица диагонализуема
    \[
    A' = \mathrm{diag}(2,\;2,\;1)
    \]
    Матрица перехода \(C\):
    \[
    C = \begin{pmatrix}
    -1 & 1 & 1\\
    0 & 1 & 1\\
    3 & 0 & 1
    \end{pmatrix},
    \quad A' = C^{-1}AC
    \]

    \textbf{Ответ: } диаг.\\
    
  \end{enumerate}
\end{document}