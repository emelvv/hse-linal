\documentclass[a4paper]{article}
\usepackage{setspace}
\usepackage[T2A]{fontenc} %
\usepackage[utf8]{inputenc} % подключение русского языка
\usepackage[russian]{babel} %
\usepackage[12pt]{extsizes}
\usepackage{mathtools}
\usepackage{graphicx}
\usepackage{fancyhdr}
\usepackage{amssymb}
\usepackage{amsmath, amsfonts, amssymb, amsthm, mathtools}
\usepackage{tikz}

\usetikzlibrary{positioning}
\setstretch{1.3}

\newcommand{\mat}[1]{\begin{pmatrix} #1 \end{pmatrix}}
\renewcommand{\det}[1]{\begin{vmatrix} #1 \end{vmatrix}}
\renewcommand{\f}[2]{\frac{#1}{#2}}
\newcommand{\dspace}{\space\space}
\newcommand{\s}[2]{\sum\limits_{#1}^{#2}}
\newcommand{\mul}[2]{\prod_{#1}^{#2}}
\newcommand{\sq}[1]{\left[ {#1} \right]}
\newcommand{\gath}[1]{\left[ \begin{array}{@{}l@{}} #1 \end{array} \right.}
\newcommand{\case}[1]{\begin{cases} #1 \end{cases}}
\newcommand{\ts}{\text{\space}}
\newcommand{\lm}[1]{\underset{#1}{\lim}}
\newcommand{\suplm}[1]{\underset{#1}{\overline{\lim}}}
\newcommand{\inflm}[1]{\underset{#1}{\underline{\lim}}}

\renewcommand{\phi}{\varphi}
\newcommand{\lr}{\Leftrightarrow}
\renewcommand{\r}{\Rightarrow}
\newcommand{\rr}{\rightarrow}
\renewcommand{\geq}{\geqslant}
\renewcommand{\leq}{\leqslant}
\newcommand{\RR}{\mathbb{R}}
\newcommand{\CC}{\mathbb{C}}
\newcommand{\QQ}{\mathbb{Q}}
\newcommand{\ZZ}{\mathbb{Z}}
\newcommand{\VV}{\mathbb{V}}
\newcommand{\NN}{\mathbb{N}}
\newcommand{\OO}{\underline{O}}
\newcommand{\oo}{\overline{o}}


\DeclarePairedDelimiter\abs{\lvert}{\rvert} %
\makeatletter                               % \abs{}
\let\oldabs\abs                             %
\def\abs{\@ifstar{\oldabs}{\oldabs*}}       %

\begin{document}

\section*{Домашнее задание на 09.12 (Линейная алгебра)}
 {\large Емельянов Владимир, ПМИ гр №247}\\\\
\begin{enumerate}
    \item[\textbf{№1}] Для нахождения фундаментальной системы решений (ФСР) данной однородной системы линейных уравнений, начнем с записи системы в матричном виде:
    
    $$Ax = 0 \implies
    \begin{pmatrix}
    1 & -2 & -1 & 2 \\
    -1 & 2 & 2 & -7
    \end{pmatrix}
    \begin{pmatrix}
    x_1 \\
    x_2 \\
    x_3 \\
    x_4
    \end{pmatrix}
    =
    \begin{pmatrix}
    0 \\
    0
    \end{pmatrix}
    $$
    
    Теперь мы можем привести матрицу $A$ к улучшенному ступенчатому виду:
    $$A = \begin{pmatrix}
    1 & -2 & -1 & 2 \\
    -1 & 2 & 2 & -7
    \end{pmatrix} \implies \begin{pmatrix}
        1 & -2 & -1 & 2 \\
        0 & 0 & 1 & -5
        \end{pmatrix} \implies $$

    $$\implies \begin{pmatrix}
        1 & -2 & 0 & -3 \\
        0 & 0 & 1 & -5
        \end{pmatrix} \implies \case{x_1 = 3x_4+2x_2 \\ x_3 = 5x_4} \implies$$
    
    $$\implies x = \mat{3x_4+2x_2 \\ x_2 \\ 5x_4 \\ x_4}, \quad x_2, x_4 \in \RR$$
    При этом:
    $$x = \mat{3x_4+2x_2 \\ x_2 \\ 5x_4 \\ x_4} = \mat{3 \\ 0 \\ 5 \\ 1}x_4+\mat{2\\1\\0\\0}x_2 \text{ - ФСР}$$
    \textbf{Ответ: } $\left\{\mat{3 \\ 0 \\ 5 \\ 1}, \mat{2\\1\\0\\0} \right\}$\\

    \item[\textbf{№2}]Чтобы найти базис в пространстве $ U = \langle u_{1}, u_{2}, u_{3}, u_{4} \rangle $, нам нужно определить линейную независимость векторов $ u_{1}, u_{2}, u_{3}, u_{4} $ и, возможно, отобрать из них базис.
    
    Составим матрицу из векторов
    $$A = \mat{u_1 \; u_2 \; u_3 \; u_4}= \begin{pmatrix}
        1 & 2 & 1 & 0 \\
        0 & 1 & 1 & 1 \\
        0 & 1 & 1 & 1 \\
        -1 & 0 & 1 & 3 \\
        0 & 1 & 1 & 4
        \end{pmatrix}
        $$
    Приведем матрицу к ступенчатому виду с помощью элементарных преобразований столбцов:
    $$A= \begin{pmatrix}
        1 & 2 & 1 & 0 \\
        0 & 1 & 1 & 1 \\
        0 & 1 & 1 & 1 \\
        -1 & 0 & 1 & 3 \\
        0 & 1 & 1 & 4
        \end{pmatrix}
        \implies \begin{pmatrix}
            1 & 0 & 1 & 0 \\
            0 & 1 & 1 & 1 \\
            0 & 1 & 1 & 1 \\
            -1 & 2 & 1 & 3 \\
            0 & 1 & 1 & 4
            \end{pmatrix}
            \implies \begin{pmatrix}
                1 & 0 & 0 & 0 \\
                0 & 1 & 1 & 1 \\
                0 & 1 & 1 & 1 \\
                -1 & 2 & 2 & 3 \\
                0 & 1 & 1 & 4
                \end{pmatrix} \implies$$
    $$\implies \begin{pmatrix}
        1 & 0 & 0 & 0 \\
        0 & 1 & 0 & 1 \\
        0 & 1 & 0 & 1 \\
        -1 & 2 & 0 & 3 \\
        0 & 1 & 0 & 4
        \end{pmatrix} \implies \begin{pmatrix}
            1 & 0 & 0 & 0 \\
            0 & 1 & 1 & 0 \\
            0 & 1 & 1 & 0 \\
            -1 & 2 & 3 & 0 \\
            0 & 1 & 4 & 0
            \end{pmatrix} \implies \begin{pmatrix}
                1 & 0 & 0 & 0 \\
                0 & 1 & 0 & 0 \\
                0 & 1 & 0 & 0 \\
                -1 & 2 & 1 & 0 \\
                0 & 1 & 3 & 0
                \end{pmatrix} = \mat{e_1 \; e_2 \; e_3 \; 0}$$
    После преобразований линейная оболочка векторов не поменялась следовательно:
    $$\langle e_1, e_2, e_3 \rangle = U$$
    При этом они находяться в ступенчатом виде, следовательно, они линейно независимы. Поэтому $\{e_1, e_2, e_3\}$ - базис пространства $U$
    \\\textbf{Ответ: } $\left\{\mat{1 \\ 0 \\ 0 \\ -1\\ 0}, \mat{0 \\ 1\\ 1\\ 2\\3}, \mat{0 \\ 0 \\0 \\ 1 \\ 3} \right\}$\\

    \item[\textbf{№3}]Запишем векторы в виде матрицы:
    $$A = \begin{pmatrix}
        1 & 1 & 0 & 3 & 5 \\
        0 & 1 & -1 & 3 & 2 \\
        -1 & 1 & -2 & 3 & -1
        \end{pmatrix}$$
    
    Теперь применим элементарные преобразования строк для приведения матрицы к ступенчатому виду.
    $$A = \begin{pmatrix}
        1 & 1 & 0 & 3 & 5 \\
        0 & 1 & -1 & 3 & 2 \\
        -1 & 1 & -2 & 3 & -1
        \end{pmatrix} \implies \begin{pmatrix}
            1 & 1 & 0 & 3 & 5 \\
            0 & 1 & -1 & 3 & 2 \\
            0 & 2 & -2 & 6 & 4
            \end{pmatrix}\implies \begin{pmatrix}
                1 & 1 & 0 & 3 & 5 \\
                0 & 1 & -1 & 3 & 2 \\
                0 & 0 & 0 & 0 & 0
                \end{pmatrix}$$
    
    Теперь мы видим, что векторы $ u_1 $ и $ u_2 $ являются линейно независимыми, а векторы $ u_3, u_4, u_5 $ можно выразить через них.
    
    Выразим $u_3$:
    $$u_3 = u_1-u_2$$
    Выразим $u_4$:
    $$u_4 = 3u_2$$
    Выразим $u_5$:
    $$u_5 = 3u_1 + 2u_2$$
    \textbf{Ответ: }$u_1, u_2$ - базис. $\case{u_3 = u_1-u_2\\u_4 = 3u_2\\u_5 = 3u_1 + 2u_2}$\\

    \item[\textbf{№4}]Запишем векторы в матричном виде:
    $$A = \begin{pmatrix}
        1 & 1 & 2 \\
        8 & -7 & -4 \\
        12 & 0 & 8 \\
        7 & 1 & 3
        \end{pmatrix}$$
    Приведём её к ступенчатому виду, преобразованиями строк:
    $$A = \begin{pmatrix}
        1 & 1 & 2 \\
        8 & -7 & -4 \\
        12 & 0 & 8 \\
        7 & 1 & 3
        \end{pmatrix} \implies \begin{pmatrix}
            1 & 1 & 2 \\
            0 & -15 & -20 \\
            0 & -12 & -16 \\
            0 & -6 & -11
            \end{pmatrix} \implies 
                \begin{pmatrix}
                1 & 1 & 2 \\
                0 & -3 & -4 \\
                0 & -3 & -4 \\
                0 & -6 & -11
                \end{pmatrix} \implies A = \begin{pmatrix}
                    1 & 1 & 2 \\
                    0 & 1 & \frac{4}{3} \\
                    0 & 0 & 1 \\
                    0 & 0 & 0
                    \end{pmatrix}$$
    $$\implies \begin{pmatrix}
        1 & 0 & 0 \\
        0 & 1 & 0 \\
        0 & 0 & 1 \\
        0 & 0 & 0
        \end{pmatrix}$$
    
    Следовательно, $v_1, v_2, v_3$ - линейно независимы. Теперь дополним их до базиса $\RR^4$.
    Опять запишем её в матричном виде и приведём к ступенчатому виду преобразованиями столбцов:
    $$A = \begin{pmatrix}
        1 & 1 & 2 \\
        8 & -7 & -4 \\
        12 & 0 & 8 \\
        7 & 1 & 3
        \end{pmatrix} \r \begin{pmatrix}
            1 & 0 & 0 \\
            8 & -15 & -20 \\
            12 & -12 & -16 \\
            7 & -6 & -11
            \end{pmatrix} \r \begin{pmatrix}
                1 & 0 & 0 \\
                8 & -5 & 0 \\
                12 & -4 & 0 \\
                7 & -2 & -3
                \end{pmatrix}$$
    Следовательно, векторы $v_1, v_2, v_3$ нужно дополнить до базиса вектором:
    $$e_3 = \mat{0\\0\\1\\0}$$
    \textbf{Ответ: }$v_1, v_2, v_3, e_3$
\end{enumerate}
\end{document}