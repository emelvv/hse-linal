\documentclass[a4paper]{article}
\usepackage{setspace}
\usepackage[T2A]{fontenc} %
\usepackage[utf8]{inputenc} % подключение русского языка
\usepackage[russian]{babel} %
\usepackage[12pt]{extsizes}
\usepackage{mathtools}
\usepackage{graphicx}
\usepackage{fancyhdr}
\usepackage{amssymb}
\usepackage{amsmath, amsfonts, amssymb, amsthm, mathtools}
\usepackage{tikz}

\usetikzlibrary{positioning}
\setstretch{1.3}

\newcommand{\mat}[1]{\begin{pmatrix} #1 \end{pmatrix}}
\renewcommand{\det}[1]{\begin{vmatrix} #1 \end{vmatrix}}
\renewcommand{\f}[2]{\frac{#1}{#2}}
\newcommand{\dspace}{\space\space}
\newcommand{\s}[2]{\sum\limits_{#1}^{#2}}
\newcommand{\mul}[2]{\prod_{#1}^{#2}}
\newcommand{\sq}[1]{\left[ {#1} \right]}
\newcommand{\gath}[1]{\left[ \begin{array}{@{}l@{}} #1 \end{array} \right.}
\newcommand{\case}[1]{\begin{cases} #1 \end{cases}}
\newcommand{\ts}{\text{\space}}
\newcommand{\lm}[1]{\underset{#1}{\lim}}
\newcommand{\suplm}[1]{\underset{#1}{\overline{\lim}}}
\newcommand{\inflm}[1]{\underset{#1}{\underline{\lim}}}

\renewcommand{\phi}{\varphi}
\newcommand{\lr}{\Leftrightarrow}
\renewcommand{\r}{\Rightarrow}
\newcommand{\rr}{\rightarrow}
\renewcommand{\geq}{\geqslant}
\renewcommand{\leq}{\leqslant}
\newcommand{\RR}{\mathbb{R}}
\newcommand{\CC}{\mathbb{C}}
\newcommand{\QQ}{\mathbb{Q}}
\newcommand{\ZZ}{\mathbb{Z}}
\newcommand{\VV}{\mathbb{V}}
\newcommand{\NN}{\mathbb{N}}

\DeclarePairedDelimiter\abs{\lvert}{\rvert} %
\makeatletter                               % \abs{}
\let\oldabs\abs                             %
\def\abs{\@ifstar{\oldabs}{\oldabs*}}       %

\begin{document}

\section*{Домашнее задание на 20.11 (Линейная алгебра)}
 {\large Емельянов Владимир, ПМИ гр №247}\\\\
\begin{enumerate}
    \item[\textbf{1.}]\begin{enumerate}
        \item[1.1] Доказать: $0 \cdot x = \overline{0} \;\; \forall x\in V$
        $$0x = (0-0)x= 0x-0x = \overline{0}$$

        \item[1.2] Доказать: $(-\alpha)\cdot x = - ( \alpha \cdot x) \;\; \forall \alpha \in \RR \;\; \forall x\in V$
        $$\alpha-\alpha = 0 \r 1\cdot \alpha-\alpha = 0 \r  -\alpha = -1\cdot \alpha$$
        $$(-\alpha)\cdot x =(-1 \cdot \alpha)\cdot x = -1 \cdot (\alpha\cdot x) = -(\alpha\cdot x)$$
    \end{enumerate}

    \item[\textbf{2.}]
    \begin{enumerate}
        \item[2.2] $U = \{(x,2y,x+y)^T \in \RR^3 \mid x,y \in \RR\}$\\
        Проверим условия подпространства:
        \begin{enumerate}
            \item[(1)]Нулевой вектор $\overline{0} \in U$:
            $$x, y = 0: (0, 0, 0) \in U$$
            \item[(2)]Замкнутость относительно сложения:
            $$(x_1, 2y_1, x_1+y_1)^T + (x_2, 2y_2, x_2+y_2)^T =$$
            $$= (x_2 + x_1, 2y_2 + 2y_1, x_2+y_2+x_1+y_1)^T \in U$$
            \item[(3)]Замкнутость относительно умножения на скаляр:
            $$(x, 2y, x+y)^T \in U, \; c \in \RR$$
            $$c(x, 2y, x+y)^T = (xc, 2yc, xc+yc)^T \in U$$
        \end{enumerate}
        \textbf{Следовательно, $U$ подпространство $\RR^\NN$}\\

        $U = \{(x, y, x, y, \dots)^T \in \RR^3 \mid x,y \in \RR\}$\\
        Проверим условия подпространства:
        \begin{enumerate}
            \item[(1)]Нулевой вектор $\overline{0} \in U$:
            $$x, y = 0: (0, 0, 0, 0, \dots) \in U$$
            \item[(2)]Замкнутость относительно сложения:
            $$(x_1, y_1, x_1, y_1, \dots)^T + (x_2, y_2, x_2, y_2, \dots)^T =$$
            $$= (x_1 + x_2, y_1 + y_2, \dots)^T \in U$$
            \item[(3)]Замкнутость относительно умножения на скаляр:
            $$(x, y, x, y, \dots)^T \in U, \; c \in \RR$$
            $$c(x, y, x, y, \dots)^T = (cx, cy, cx, cy, \dots)^T \in U$$
        \end{enumerate}
        \textbf{Следовательно, $U$ подпространство $\RR^\NN$}\\

        \item[2.2]$U = \{(x_1, \ldots, x_n)^T  \in \RR^n \mid x_1 \ne 0, \ldots, x_n \ne 0\}$\\
        Проверим условия подпространства:
        \begin{enumerate}
            \item[(1)]Нулевой вектор $\overline{0} \in U$
            $$(0, \ldots, 0)^T \notin U$$
        \end{enumerate}
        Первое условие не выполняется.\\
        \textbf{Следовательно, $U$ не подпространство $\RR^\NN$}\\
        
        $U = \{(x_1, \ldots, x_n)^T  \in \RR^n \mid x_1 \ge x_2 \ge \ldots \ge x_n\}$\\
        Проверим условия подпространства:
        \begin{enumerate}
            \item[(1)]Нулевой вектор $\overline{0} \in U$:
            $$x_1, \ldots, x_n = 0: (0, 0, 0, 0, \dots) \in U$$
            \item[(2)]Замкнутость относительно сложения:
            $$(x_1, \ldots, x_n)^T + (y_1, \ldots, y_n)^T =$$
            $$= (x_1+y_1, \ldots, x_n+y_n)^T \in? \; U$$
            Сложение может нарушить порядок убывания.
        \end{enumerate}
        Второе условие не выполняется.\\
        \textbf{Следовательно, $U$ не подпространство $\RR^\NN$}\\

        \item[2.3]$U = \{(x_1, \ldots, x_n)^T  \in \RR^n \mid x_1, \ldots, x_n \in \ZZ\}$;\\
        Проверим условия подпространства:
        \begin{enumerate}
            \item[(1)]Нулевой вектор $\overline{0} \in U$:
            $$x_1, \ldots, x_n = 0: (0, 0, 0, 0, \dots) \in U$$
            \item[(2)]Замкнутость относительно сложения:
            $$(x_1, \ldots, x_n)^T + (y_1, \ldots, y_n)^T =$$
            $$= (x_1+y_1, \ldots, x_n+y_n)^T \in \; U$$
            \item[(3)]Замкнутость относительно умножения на скаляр:
            $$(x_1, \ldots, x_n)^T \in U, \; c \in \RR$$
            $$c(x_1, \ldots, x_n)^T = (cx_1, \ldots, cx_n)^T \in? \; U$$
            При $c \in \RR \backslash \QQ$ числа перестанут быть целыми.
        \end{enumerate}
        Третье условие не выполняется.\\
        \textbf{Следовательно, $U$ не подпространство $\RR^\NN$}\\

        $U = \{(x, x^3, x^5, \ldots, x^{2n-1})^T \mid x \in \RR\}$;\\
        Проверим условия подпространства:
        \begin{enumerate}
            \item[(1)]Нулевой вектор $\overline{0} \in U$:
            $$x = 0: (0, 0, 0, 0, \dots) \in U$$
            \item[(2)]Замкнутость относительно сложения:
            $$(x_1, x_1^3, x_1^5, \ldots, x_1^{2n-1})^T + (x_2, x_2^3, x_2^5, \ldots, x_1^{2n-1})^T =$$
            $$= (x_1+x_2, x_1^3+x_2^3, x_1^5+x_2^5, \ldots, x_1^{2n-1}+x_2^{2n-1})^T \in? \; U$$
            При $x_1 = 1$ и $x_2 = 1$:
            $$(1, \ldots, 1)^T+(1, \ldots, 1)^T = (2, \ldots, 2)^T \notin U$$
        \end{enumerate}
        Второе условие не выполняется.\\
        \textbf{Следовательно, $U$ не подпространство $\RR^\NN$}\\

        $U = \{(x, x)^T \mid x \in \RR\} \cup \{(x, -x)^T \mid x \in \RR\}$\\
        Проверим условия подпространства:
        \begin{enumerate}
            \item[(1)]Нулевой вектор $\overline{0} \in U$:
            $$x= 0: (0, 0)^T \in U$$
            \item[(2)]Замкнутость относительно сложения:
            $$(x, x)^T + (y, y)^T = (x+y, x+y)^T \in U$$
            $$(x, x)^T + (y, -y)^T = (x+y, x-y)^T \in? \; U$$
            При $x = 1$ и $y= 1$:
            $$(1, 1)^T + (1, -1)^T = (2, 0) \notin U$$
            
        \end{enumerate}
        Второе условие не выполняется.\\
        \textbf{Следовательно, $U$ не подпространство $\RR^\NN$}\\
    \end{enumerate}

    \item[\textbf{3}]
    \begin{enumerate}
        \item[3.1]Множество симметрических матриц; множество кососимметрических матриц
        \begin{enumerate}
            \item[1)]Симметрические матрицы. Проверим условия подпространства:
            \begin{enumerate}
                \item[(1)]Нулевая матрица симметрична.
                \item[(2)]Сумма двух симметричных матриц \( A \) и \( B \) также симметрична, так как $$ (A + B)^T = A^T + B^T = A + B $$
                \item[(3)]Умножение симметричной матрицы \( A \) на скаляр \( \alpha \) сохраняет симметричность: $$(\alpha A)^T = \alpha A^T = \alpha A$$ 
            \end{enumerate}
            \textbf{Следовательно, множество симметрических матриц является подпространством $V$}\\

            \item[2)]Кососимметрические матрицы. Проверим условия подпространства:
            \begin{enumerate}
                \item[(1)]Нулевая матрица кососимметрична.
                \item[(2)]Сумма двух кососимметричных матриц \( A \) и \( B \) также кососимметрична, так как $$ (A + B)^T = A^T + B^T = -A - B = -(A + B) $$
                \item[(3)]Умножение кососимметричной матрицы \( A \) на скаляр \( \alpha \) сохраняет кососимметричность:  $$(\alpha A)^T = \alpha A^T = \alpha (-A) = -(\alpha A)$$ 
            \end{enumerate}
            \textbf{Следовательно, множество кососимметрических матриц является подпространством $V$}\\\\
        \end{enumerate}

        \item[3.2]Множество матриц со следом 0. Проверим условия подпространства:
        \begin{enumerate}
            \item[(1)] Нулевая матрица имеет след 0
            \item[(2)] Если \( A \) и \( B \) имеют след 0, то
            $$\operatorname{Tr}(A + B) = \operatorname{Tr}(A) + \operatorname{Tr}(B) = 0 + 0 = 0$$
            \item[(3)]Если \( \alpha \in F \) и \( \operatorname{Tr}(A) = 0 \), то 
            $$\operatorname{Tr}(\alpha A) = \alpha \operatorname{Tr}(A) = \alpha \cdot 0 = 0$$
        \end{enumerate}
        \textbf{Следовательно, множество матриц со следом 0 является подпространством $V$}\\\\

        \item[3.3]Множество невырожденных матриц; множество вырожденных матриц
        \begin{enumerate}
            \item[1)]Невырожденные матрицы. Проверим условия подпространства:
            \begin{enumerate}
                \item[(1)]Нулевая матрица вырожденная (\( det(0) = 0 \)), следовательно, множество невырожденных матриц не содержит нулевую матрицу и не является подпространством.
            \end{enumerate}
            Первое условие не выполняется.\\
            \textbf{Следовательно, множество невырожденных матриц не является подпространством $V$}\\\\
            
            \item[1)]Невырожденные матрицы. Проверим условия подпространства:
            \begin{enumerate}
                \item[(1)]Нулевая матрица вырожденная.
                \item[(2)]Сумма двух вырожденных матриц может не быть вырожденной (например, \( A \) и \( -A \) вырожденные, но \( A + (-A) = 0 \) - невырожденная).
            \end{enumerate}
            Второе условие не выполняется.\\
            \textbf{Следовательно, множество вырожденных матриц не является подпространством $V$}\\\\
        \end{enumerate}
    \end{enumerate}

    \item[\textbf{4.}]
    \begin{enumerate}
        \item[4.1]Функции с фиксированным значением в точке \( 5 \)
        \begin{enumerate}
            \item[1)]Множество функций, значение которых в точке \( 5 \) равно \( 0 \). Проверим условия подпространства:
            \begin{enumerate}
                \item[(1)]Нулевая функция \( f(x) = 0 \) принадлежит \( U \), потому что \( f(5) = 0 \)
                \item[(2)]Если \( f(5) = 0 \) и \( g(5) = 0 \), то $$ (f + g)(5) = f(5) + g(5) = 0 + 0 = 0 $$
                \item[(3)]Если \( f(5) = 0 \) и \( \alpha \in \mathbb{R} \), то 
                $$(\alpha f)(5) = \alpha f(5) = \alpha \cdot 0 = 0 $$
            \end{enumerate}
            \textbf{Следовательно, множество функций, значение которых в точке \( 5 \) равно \( 0 \),  является подпространством $V$}\\\\

            \item[2)]Множество функций, значение которых в точке \( 5 \) равно \( 1 \). Проверим условия подпространства:
            \begin{enumerate}
                \item[(1)]Нулевая функция \( f(x) = 0 \) не принадлежит \( U \), так как \( f(5) \neq 1 \).
            \end{enumerate}
            Первое условие не выполняется.\\
            \textbf{Следовательно, множество функций, значение которых в точке \( 5 \) равно \( 1 \), не является подпространством $V$}\\\\
        \end{enumerate}

        \item[4.2]Возрастающие и неубывающие функции
        \begin{enumerate}
            \item[1)]Множество строго возрастающих функций. Проверим условия подпространства:
            \begin{enumerate}
                \item[(1)]Нулевая функция \( f(x) = 0 \) не строго возрастает (\( f(x_1) = f(x_2) \)).
            \end{enumerate}
            Первое условие не выполняется.\\
            \textbf{Следовательно, множество строго возрастающих функций, не является подпространством $V$}\\\\

            \item[2)]Множество неубывающих функций. Проверим условия подпространства:
            \begin{enumerate}
                \item[(1)]Нулевая функция \( f(x) = 0 \) не убывает.
                \item[(2)]Если \( f \) и \( g \) не убывают, то \( (f + g)(x_1) \leq (f + g)(x_2) \) сохраняется, так как:
                $$ f(x_1) + g(x_1) \leq f(x_2) + g(x_2)$$
                \item[(3)]Если \( f \) не убывает и \( \alpha \geq 0 \), то \( (\alpha f)(x_1) \leq (\alpha f)(x_2) \). Если \( \alpha < 0 \), порядок меняется, и функция может стать убывающей.
            \end{enumerate}
            Третье условие не выполняется.\\
            \textbf{Следовательно, множество неубывающих функций, не является подпространством $V$}\\\\
        \end{enumerate}

        \item[4.3]Нестрого монотонные функции\\
        $U = \{f \mid f \text{ не убывает}\} \cup \{f \mid f \text{ не возрастает}\}.$
        Проверим условия подпространства:
        \begin{enumerate}
            \item[(1)]Нулевая функция \( f(x) = 0 \) принадлежит \( U \), так как она одновременно не убывает и не возрастает.
            \item[(2)]Если \( f \) не убывает и \( g \) не возрастает, то \( f + g \) может не быть ни неубывающей, ни невозрастающей. Например, если \( f(x) = \ln{x} \) (не убывает), \( g(x) = -2^x \) (не возрастает), то:
            $$(f + g)(x) = ln(x)-2^x \text{ (немонотонная)}$$
        \end{enumerate}
        Второе условие не выполняется.\\
        Третье условие не выполняется.\\
        \textbf{Следовательно, множество нестрого монотонных функции, не является подпространством $V$}\\\\
    \end{enumerate}

    \item[\textbf{5.}] Чтобы определить, при каких значениях параметра \( a \) вектор \((0, 6, -1, a)^T\) является линейной комбинацией векторов \((1, 3, -2, 1)^T\), \((2, 0, -3, 4)^T\) и \((3, 3, -5, 5)^T\), нужно решить систему уравнений:
    \[
    (0, 6, -1, a)^T = x_1(1, 3, -2, 1)^T + x_2(2, 0, -3, 4)^T + x_3(3, 3, -5, 5)^T
    \]
    \[
    \case{
    0 = x_1 + 2x_2 + 3x_3, \\
    6 = 3x_1 + 0x_2 + 3x_3, \\
    -1 = -2x_1 - 3x_2 - 5x_3, \\
    a = x_1 + 4x_2 + 5x_3.
    }
    \]
    Запишем в виде СЛУ:
    \[A = 
    \mat{
    1 & 2 & 3 & | & 0 \\
    3 & 0 & 3 & | & 6 \\
    -2 & -3 & -5 & | & -1 \\
    1 & 4 & 5 & | & a
    }
    \]
    $$\case{
        A_{(2)} - 3A_{(1)}\\
        A_{(3)} + 2A_{(1)}\\
        A_{(4)} - A_{(1)}
    }\r \mat{
        1 & 2 & 3 & | & 0 \\
        0 & -6 & -6 & | & 6 \\
        0 & 1 & 1 & | & -1 \\
        0 & 2 & 2 & | & a
        } = \mat{
            1 & 2 & 3 & | & 0 \\
            0 & 1 & 1 & | & -1 \\
            0 & 2 & 2 & | & a
            }$$
    $$A_{(3)} - 2A_{(2)} \r \mat{
        1 & 2 & 3 & | & 0 \\
        0 & 1 & 1 & | & -1 \\
        0 & 0 & 0 & | & a+2
        }$$
    При $a \neq -2$: нет решений \\
    При $a = -2$: есть решения 

    \((0, 6, -1, a)^T\) является линейной комбинацией векторов \((1, 3, -2, 1)^T\), \((2, 0, -3, 4)^T\) и \((3, 3, -5, 5)^T\) только, когда существует решение этой системы.
    То есть при $0 =a+2 \r a = -2$

    \textbf{Ответ:} $a = -2$
\end{enumerate}
\end{document}