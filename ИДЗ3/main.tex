\documentclass[a4paper]{article}
\usepackage{setspace}
\usepackage[T2A]{fontenc} %
\usepackage[utf8]{inputenc} % подключение русского языка
\usepackage[russian]{babel} %
\usepackage[12pt]{extsizes}
\usepackage{mathtools}
\usepackage{graphicx}
\usepackage{fancyhdr}
\usepackage{amssymb}
\usepackage{amsmath, amsfonts, amssymb, amsthm, mathtools}
\usepackage{tikz}

\usetikzlibrary{positioning}
\setstretch{1.3}

\newcommand{\mat}[1]{\begin{pmatrix} #1 \end{pmatrix}}
\renewcommand{\det}[1]{\begin{vmatrix} #1 \end{vmatrix}}
\renewcommand{\f}[2]{\frac{#1}{#2}}
\newcommand{\dspace}{\space\space}
\newcommand{\s}[2]{\sum\limits_{#1}^{#2}}
\newcommand{\mul}[2]{\prod_{#1}^{#2}}
\newcommand{\sq}[1]{\left[ {#1} \right]}
\newcommand{\gath}[1]{\left[ \begin{array}{@{}l@{}} #1 \end{array} \right.}
\newcommand{\case}[1]{\begin{cases} #1 \end{cases}}
\newcommand{\ts}{\text{\space}}
\newcommand{\lm}[1]{\underset{#1}{\lim}}
\newcommand{\suplm}[1]{\underset{#1}{\overline{\lim}}}
\newcommand{\inflm}[1]{\underset{#1}{\underline{\lim}}}

\newcommand{\lr}{\Leftrightarrow}
\renewcommand{\r}{\Rightarrow}
\newcommand{\rr}{\rightarrow}
\renewcommand{\geq}{\geqslant}
\renewcommand{\leq}{\leqslant}
\newcommand{\RR}{\mathbb{R}}
\newcommand{\CC}{\mathbb{C}}
\newcommand{\QQ}{\mathbb{Q}}
\newcommand{\ZZ}{\mathbb{Z}}
\newcommand{\VV}{\mathbb{V}}
\newcommand{\NN}{\mathbb{N}}

\DeclarePairedDelimiter\abs{\lvert}{\rvert} %
\makeatletter                               % \abs{}
\let\oldabs\abs                             %
\def\abs{\@ifstar{\oldabs}{\oldabs*}}       %

\begin{document}

\section*{ИДЗ №3 (Линейная алгебра)}
 {\large Емельянов Владимир, ПМИ гр №247}\\\\
\begin{enumerate}
    \item[\textbf{1.}]
    $A = \mat{2 & -2 & 1 & -1 \\ 1 & -4 & -3 & 3 \\ -1 & 0 & -1 & 2 \\ -1 & 1 & 0 & 1}$\\
    Запишем в виде $(A|E)$ и приведём к $(E|A^{-1})$
    $$(A|E) = \mat{2 & -2 & 1 & -1 & | & 1 & 0 & 0 & 0\\ 1 & -4 & -3 & 3 & | & 0 & 1 & 0 & 0\\ -1 & 0 & -1 & 2 & | & 0 & 0 & 1 & 0 \\ -1 & 1 & 0 & 1 & | & 0 & 0 & 0 & 1} $$
    $$A[0] = A[0]/2 \to \mat{1 & -1 & \frac{1}{2} & -\frac{1}{2} & | & \frac{1}{2} & 0 & 0 & 0 \\ 1 & -4 & -3 & 3 & | & 0 & 1 & 0 & 0 \\ -1 & 0 & -1 & 2 & | & 0 & 0 & 1 & 0 \\ -1 & 1 & 0 & 1 & | & 0 & 0 & 0 & 1}$$
    $$\case{A[1] = A[1]-A[0]\\
    A[2] = A[2]+A[0]\\
    A[3] = A[3]+A[0]\\} \to \mat{1 & -1 & \frac{1}{2} & -\frac{1}{2} & | & \frac{1}{2} & 0 & 0 & 0 \\ 0 & -3 & -\frac{7}{2} & \frac{7}{2} & | & -\frac{1}{2} & 1 & 0 & 0 \\ 0 & -1 & -\frac{1}{2} & \frac{3}{2} & | & \frac{1}{2} & 0 & 1 & 0 \\ 0 & 0 & \frac{1}{2} & \frac{1}{2} & | & \frac{1}{2} & 0 & 0 & 1}$$
    $$\case{
        A[1], A[2] = A[2], A[1] \\
        A[1] = A[1]*-1} \to \mat{1 & -1 & \frac{1}{2} & -\frac{1}{2} & | & \frac{1}{2} & 0 & 0 & 0 \\ 0 & 1 & \frac{1}{2} & -\frac{3}{2} & | & -\frac{1}{2} & 0 & -1 & 0 \\ 0 & -3 & -\frac{7}{2} & \frac{7}{2} & | & -\frac{1}{2} & 1 & 0 & 0 \\ 0 & 0 & \frac{1}{2} & \frac{1}{2} & | & \frac{1}{2} & 0 & 0 & 1}$$

    $$A[2] = A[2]+3*A[1] \to \mat{1 & -1 & \frac{1}{2} & -\frac{1}{2} & | & \frac{1}{2} & 0 & 0 & 0 \\ 0 & 1 & \frac{1}{2} & -\frac{3}{2} & | & -\frac{1}{2} & 0 & -1 & 0 \\ 0 & 0 & -2 & -1 & | & -2 & 1 & -3 & 0 \\ 0 & 0 & \frac{1}{2} & \frac{1}{2} & | & \frac{1}{2} & 0 & 0 & 1}$$
    $$A[2] = A[2]/(-2) \to \mat{1 & -1 & \frac{1}{2} & -\frac{1}{2} & | & \frac{1}{2} & 0 & 0 & 0 \\ 0 & 1 & \frac{1}{2} & -\frac{3}{2} & | & -\frac{1}{2} & 0 & -1 & 0 \\ 0 & 0 & 1 & \frac{1}{2} & | & 1 & -\frac{1}{2} & \frac{3}{2} & 0 \\ 0 & 0 & \frac{1}{2} & \frac{1}{2} & | & \frac{1}{2} & 0 & 0 & 1}$$
    $$A[3] = A[3]-A[2]/2 \to \mat{1 & -1 & \frac{1}{2} & -\frac{1}{2} & | & \frac{1}{2} & 0 & 0 & 0 \\ 0 & 1 & \frac{1}{2} & -\frac{3}{2} & | & -\frac{1}{2} & 0 & -1 & 0 \\ 0 & 0 & 1 & \frac{1}{2} & | & 1 & -\frac{1}{2} & \frac{3}{2} & 0 \\ 0 & 0 & 0 & \frac{1}{4} & | & 0 & \frac{1}{4} & -\frac{3}{4} & 1}$$
    $$A[3] = A[3]*4 \to \mat{1 & -1 & \frac{1}{2} & -\frac{1}{2} & | & \frac{1}{2} & 0 & 0 & 0 \\ 0 & 1 & \frac{1}{2} & -\frac{3}{2} & | & -\frac{1}{2} & 0 & -1 & 0 \\ 0 & 0 & 1 & \frac{1}{2} & | & 1 & -\frac{1}{2} & \frac{3}{2} & 0 \\ 0 & 0 & 0 & 1 & | & 0 & 1 & -3 & 4}$$
    $$A[0] = A[0]+A[1] \to \mat{1 & 0 & 1 & -2 & | & 0 & 0 & -1 & 0 \\ 0 & 1 & \frac{1}{2} & -\frac{3}{2} & | & -\frac{1}{2} & 0 & -1 & 0 \\ 0 & 0 & 1 & \frac{1}{2} & | & 1 & -\frac{1}{2} & \frac{3}{2} & 0 \\ 0 & 0 & 0 & 1 & | & 0 & 1 & -3 & 4}$$
    $$A[0] = A[0]-A[2] \to \mat{1 & 0 & 0 & -\frac{5}{2} & | & -1 & \frac{1}{2} & -\frac{5}{2} & 0 \\ 0 & 1 & \frac{1}{2} & -\frac{3}{2} & | & -\frac{1}{2} & 0 & -1 & 0 \\ 0 & 0 & 1 & \frac{1}{2} & | & 1 & -\frac{1}{2} & \frac{3}{2} & 0 \\ 0 & 0 & 0 & 1 & | & 0 & 1 & -3 & 4}$$
    $$A[0] = A[0]+A[3]*5/2 \to \mat{1 & 0 & 0 & 0 & | & -1 & 3 & -10 & 10 \\ 0 & 1 & \frac{1}{2} & -\frac{3}{2} & | & -\frac{1}{2} & 0 & -1 & 0 \\ 0 & 0 & 1 & \frac{1}{2} & | & 1 & -\frac{1}{2} & \frac{3}{2} & 0 \\ 0 & 0 & 0 & 1 & | & 0 & 1 & -3 & 4}$$
    $$A[1] = A[1]-A[2]/2 \to \mat{1 & 0 & 0 & 0 & | & -1 & 3 & -10 & 10 \\ 0 & 1 & 0 & -\frac{7}{4} & | & -1 & \frac{1}{4} & -\frac{7}{4} & 0 \\ 0 & 0 & 1 & \frac{1}{2} & | & 1 & -\frac{1}{2} & \frac{3}{2} & 0 \\ 0 & 0 & 0 & 1 & | & 0 & 1 & -3 & 4}$$
    $$A[1] = A[1]+A[3]*7/4 \to \mat{1 & 0 & 0 & 0 & | & -1 & 3 & -10 & 10 \\ 0 & 1 & 0 & 0 & | & -1 & 2 & -7 & 7 \\ 0 & 0 & 1 & \frac{1}{2} & | & 1 & -\frac{1}{2} & \frac{3}{2} & 0 \\ 0 & 0 & 0 & 1 & | & 0 & 1 & -3 & 4}$$
    $$A[2] = A[2]-A[3]/2 \to \mat{1 & 0 & 0 & 0 & | & -1 & 3 & -10 & 10 \\ 0 & 1 & 0 & 0 & | & -1 & 2 & -7 & 7 \\ 0 & 0 & 1 & 0 & | & 1 & -1 & 3 & -2 \\ 0 & 0 & 0 & 1 & | & 0 & 1 & -3 & 4} \r$$
    $$\r A^{-1} = \mat{-1 & 3 & -10 & 10 \\ -1 & 2 & -7 & 7 \\ 1 & -1 & 3 & -2 \\ 0 & 1 & -3 & 4}$$
    \textbf{Ответ: }$\mat{-1 & 3 & -10 & 10 \\ -1 & 2 & -7 & 7 \\ 1 & -1 & 3 & -2 \\ 0 & 1 & -3 & 4}$\\

    \item[\textbf{2.}]
    $$\left(\mat{1 & 2 & 3 & 4 & 5 & 6 & 7 & 8 \\ 3 & 1 & 6 & 7 & 8 & 4 & 2 & 5}^{11}\cdot \mat{1 & 2 & 3 & 4 & 5 & 6 & 7 & 8 \\ 7 & 8 & 5 & 6 & 1 & 3 & 2 & 4}^{-1}\right)^{176} \cdot X = $$
    $$ = \mat{1 & 2 & 3 & 4 & 5 & 6 & 7 & 8 \\ 3 & 2 & 4 & 7 & 8 & 1 & 6 & 5}$$
    \begin{enumerate}
        \item[]
        $\mat{1 & 2 & 3 & 4 & 5 & 6 & 7 & 8 \\ 3 & 1 & 6 & 7 & 8 & 4 & 2 & 5}=(1 \; 3 \; 6\; 4\; 7\; 2)(5 \; 8) \r \mat{1 & 2 & 3 & 4 & 5 & 6 & 7 & 8 \\ 3 & 1 & 6 & 7 & 8 & 4 & 2 & 5}^{11} = $
        $=\mat{1 & 2 & 3 & 4 & 5 & 6 & 7 & 8 \\ 3 & 1 & 6 & 7 & 8 & 4 & 2 & 5}^{-1} = (2\;7\;4\;6\;3)(8\;5)$\\
        \item[]
        $\mat{1 & 2 & 3 & 4 & 5 & 6 & 7 & 8 \\ 7 & 8 & 5 & 6 & 1 & 3 & 2 & 4} = (1 \; 7\; 2\; 8\; 4\; 6\; 3\; 5) \r$ \\
        $\r \mat{1 & 2 & 3 & 4 & 5 & 6 & 7 & 8 \\ 7 & 8 & 5 & 6 & 1 & 3 & 2 & 4}^{-1} = (5\;3\;6\;4\;8\;2\;7\;1)$
    
    \end{enumerate}
    $\mat{1 & 2 & 3 & 4 & 5 & 6 & 7 & 8 \\ 3 & 1 & 6 & 7 & 8 & 4 & 2 & 5}^{11}\cdot \mat{1 & 2 & 3 & 4 & 5 & 6 & 7 & 8 \\ 7 & 8 & 5 & 6 & 1 & 3 & 2 & 4}^{-1} =\\ $
    $= \mat{1 & 2 & 3 & 4 & 5 & 6 & 7 & 8 \\
    2 & 7 & 1 & 6 & 8 & 3 & 4 & 5} \cdot \mat{1 & 2 & 3 & 4 & 5 & 6 & 7 & 8 \\
    5 & 7 & 6 & 8 & 3 & 4 & 1 & 2} = \mat{1 & 2 & 3 & 4 & 5 & 6 & 7 & 8 \\
    8 & 4 & 3 & 5 & 1 & 6 & 2 & 7}$\\
    $\mat{1 & 2 & 3 & 4 & 5 & 6 & 7 & 8 \\
    8 & 4 & 3 & 5 & 1 & 6 & 2 & 7} = (1\; 8 \;7 \;2 \;4 \;5)$\\
    $\mat{1 & 2 & 3 & 4 & 5 & 6 & 7 & 8 \\
    8 & 4 & 3 & 5 & 1 & 6 & 2 & 7}^{176} =\mat{1 & 2 & 3 & 4 & 5 & 6 & 7 & 8 \\
    8 & 4 & 3 & 5 & 1 & 6 & 2 & 7}^{2} = (1 \; 7 \; 4)(8 \; 2 \; 5) $\\
    $((1 \; 7 \; 4)(8 \; 2 \; 5))X = \mat{1 & 2 & 3 & 4 & 5 & 6 & 7 & 8 \\ 3 & 2 & 4 & 7 & 8 & 1 & 6 & 5}$\\
    $((1 \; 7 \; 4)(8 \; 2 \; 5))^{-1} = (4\;7\;1)(5\;2\;8) = \mat{1 & 2 & 3 & 4 & 5 & 6 & 7 & 8 \\
    4 & 8 & 3 & 7 & 2 & 6 & 1 & 5}$\\
    $X = \mat{1 & 2 & 3 & 4 & 5 & 6 & 7 & 8 \\
    4 & 8 & 3 & 7 & 2 & 6 & 1 & 5} \cdot \mat{1 & 2 & 3 & 4 & 5 & 6 & 7 & 8 \\ 3 & 2 & 4 & 7 & 8 & 1 & 6 & 5}  =\\= \mat{1 & 2 & 3 & 4 & 5 & 6 & 7 & 8 \\ 3 & 8 & 7 & 1 & 5 & 4 & 6 & 2}$\\
    \textbf{Ответ: }$\mat{1 & 2 & 3 & 4 & 5 & 6 & 7 & 8 \\ 3 & 8 & 7 & 1 & 5 & 4 & 6 & 2}$

    \item[\textbf{3.}]
    $\sigma = \mat{1 & 2 & \dots & 97 & 98 & \dots & 412 & 413 & \dots & 560 \\ 464 & 465 & \dots & 560 & 149 & \dots & 463 & 1 & \dots & 148}$\\\\
    Число инверсий: $463 + 463 + \dots + 463 + 148 + 148 + \dots + 148 + 0 + 0 +\dots + 0 = $\\
    $= 463\cdot 97 + 148\cdot 315 = 91531 \r sgn(\sigma) = (-1)^{91531} = -1$\\
    \textbf{Ответ: } $-1$

    \item[\textbf{4.}]
    $\det{0 & 4 & 0 & 0 & x & 0 \\
    x & x & 2 & x & 0 & 7 \\ 
    0 & 3 & 0 & 4 & 1 & 4 \\
    0 &9 & 0 & 7 &2 & 4 \\
    0 & 3 & 3 & 7 & 1 & 7 \\
    7 & x & 2 & 8 & x & 4} = -4\det{x & 2 & x & 0 & 7 \\ 0 & 0 & 4 & 1 & 4 \\ 0 & 0 & 7 & 2 & 4 \\ 0 & 3 & 7 & 1 & 7 \\ 7 & 2 & 8 & x & 4} + x \det{x & x & 2 & x & 7 \\ 0 & 3 & 0 & 4 & 4 \\ 0 & 9 & 0 & 7 & 4 \\ 0 & 3 & 3 & 7 & 7 \\ 7 & x & 2 & 8 & 4} =$\\
    \begin{enumerate}
        \item[1)]$-4x\det{0 & 4 & 1 & 4 \\ 0 & 7 & 2 & 4 \\ 3 & 7 & 1 & 7 \\ 2 & 8 & x & 4} -28\det{2 & x & 0 & 7 \\ 0 & 4 & 1 & 4 \\ 0 & 7 & 2 & 4 \\ 3 & 7 & 1 & 7} = $\\
        $= -4x \left(3 \cdot  \begin{vmatrix} 4 & 1 & 4 \\ 7 & 2 & 4 \\ 8 & x & 4 \end{vmatrix} - 2 \cdot \begin{vmatrix} 4 & 1 & 4 \\ 7 & 2 & 4 \\ 7 & 1 & 7 \end{vmatrix} \right) -28 \left( 2 \cdot\begin{vmatrix} 4 & 1 & 4 \\ 7 & 2 & 4 \\ 7 & 1 & 7 \end{vmatrix}- 3 \cdot \begin{vmatrix} x & 0 & 7 \\ 4 & 1 & 4 \\ 7 & 2 & 4 \end{vmatrix} \right) = $\\
        $= -28(-18 - 3(7 - 4x)) - 4x(18 + 3(-28 + 12x)) =\\ = -144x^2-72x+1092$

        \item[2)]$x^2 \cdot  \begin{vmatrix} 3 & 0 & 4 & 4 \\ 9 & 0 & 7 & 4 \\ 3 & 3 & 7 & 7 \\ x & 2 & 8 & 4 \end{vmatrix} +7x \cdot \begin{vmatrix} x & 2 & x & 7 \\ 3 & 0 & 4 & 4 \\ 9 & 0 & 7 & 4 \\ 3 & 3 & 7 & 7 \end{vmatrix} = $\\
        $= x^2(-3 \cdot \begin{vmatrix}
            3 & 4 & 4 \\
            9 & 7 & 4 \\
            x & 8 & 4
            \end{vmatrix} + 2 \cdot \begin{vmatrix}
            3 & 4 & 4 \\
            9 & 7 & 4 \\
            3 & 7 & 7
            \end{vmatrix})+ \\ + 7 x  (-2 \cdot \begin{vmatrix}
            3 & 4 & 4 \\
            9 & 7 & 4 \\
            3 & 7 & 7
            \end{vmatrix} + 3 \cdot \begin{vmatrix}
            x & x & 7 \\
            3 & 4 & 4 \\
            9 & 7 & 4
            \end{vmatrix}) =\\
            = x^2 \cdot (54 - 3 \cdot (132 - 12x)) + 7x \cdot (-54 + 3 \cdot (-105 + 12x)) = \\
            = -2583x - 90x^2 + 36x^3$
    \end{enumerate}
    $-144x^2-72x+1092 -2583x - 90x^2 + 36x^3 =\\= 36x^3-234x^2-2655x+1092$\\
    \textbf{Ответ: } $36x^3-234x^2-2655x+1092$\\

    \item[\textbf{5.}]$\det{2 & x & 3 & 9 & 7 & 1 & 10 \\
    x & 1 & 7 & 7 & 1 & 4 & x \\
    3 & 7 & 4 & 2 & 7 & x & 3 \\
    9 & 7 & 2 & 4 & x & 9 & 4 \\
    7 & 1 & 7 & x & 7 & 6 & 8 \\
    1 & 4 & x & 9 & 6 & 2 & 2 \\
    10 & x & 3 & 4 & 8 & 2 & 9 \\} =\det{2 & x & 3 & 9 & 7 & 1 & 10 \\ x & 1 & 7 & 7 & 1 & 4 & x \\ 3 & 7 & 4 & 2 & 7 & x & 3 \\ 9 & 7 & 2 & 4 & x & 9 & 4 \\ 7 & 1 & 7 & x & 7 & 6 & 8 \\ 1 & 4 & x & 9 & 6 & 2 & 2 \\ 8 & 0 & 0 & -5 & 1 & 1 & -1} = 
    \det{2 & x & 3 & 9 & 7 & 1 & 8 \\ x & 1 & 7 & 7 & 1 & 4 & 0 \\ 3 & 7 & 4 & 2 & 7 & x & 0 \\ 9 & 7 & 2 & 4 & x & 9 & -5 \\ 7 & 1 & 7 & x & 7 & 6 & 1 \\ 1 & 4 & x & 9 & 6 & 2 & 1 \\ 8 & 0 & 0 & -5 & 1 & 1 & -9}$\\\\
    Так как в 6ти строках есть $x$, то $x^5$ встречается только в $\binom{6}{5} = 6$ перестановках, то есть достаточно рассмотреть 6 случаев когда каждый из $x$ не в перестановке:
    \begin{enumerate}
        \item[]\underbar{$x$ из 1го столбца не в перестановке}: \\
        Единственный возможный многочлен степени 5: 
        $$sgn(\sigma_1)a_{12}a_{27}a_{36}a_{45}a_{54}a_{63}a_{71} = 0$$
        \item[]\underbar{$x$ из 2го столбца не в перестановке}: \\
        Единственный возможный многочлен степени 5: 
        $$sgn(\sigma_2)a_{17}a_{21}a_{36}a_{45}a_{54}a_{63}a_{72} = 0$$
        \item[]\underbar{$x$ из 3го столбца не в перестановке}: \\
        Единственный возможный многочлен степени 5: 
        $$sgn(\sigma_3)a_{12}a_{21}a_{36}a_{45}a_{54}a_{67}a_{73} = 0$$
        \item[]\underbar{$x$ из 4го столбца не в перестановке}: \\
        Единственный возможный многочлен степени 5: 
        $$sgn(\sigma_4)a_{12}a_{21}a_{36}a_{45}a_{57}a_{63}a_{74} = -5x^5$$
        \item[]\underbar{$x$ из 5го столбца не в перестановке}: \\
        Единственный возможный многочлен степени 5: 
        $$sgn(\sigma_5)a_{12}a_{21}a_{36}a_{47}a_{54}a_{63}a_{75} = -5x^5$$ 
        \item[]\underbar{$x$ из 5го столбца не в перестановке}: \\
        Единственный возможный многочлен степени 5: 
        $$sgn(\sigma_6)a_{12}a_{21}a_{37}a_{45}a_{54}a_{63}a_{76} = 0$$ 

    \end{enumerate} 
    Сложим все многочлены степени 5, получившиеся во всех случаях:
    $$0 + 0+ 0 -5x^5 - 5x^5 + 0 = -10x^5$$

    \textbf{Ответ: } $-10$
\end{enumerate}
\end{document}