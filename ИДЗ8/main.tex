\documentclass[a4paper]{article}
\usepackage{setspace}
\usepackage[T2A]{fontenc} %
\usepackage[utf8]{inputenc} % подключение русского языка
\usepackage[russian]{babel} %
\usepackage[12pt]{extsizes}
\usepackage{mathtools}
\usepackage{graphicx}
\usepackage{fancyhdr}
\usepackage{amssymb}
\usepackage{amsmath, amsfonts, amssymb, amsthm, mathtools}
\usepackage{tikz}

\usetikzlibrary{positioning}
\setstretch{1.3}

\newcommand{\mat}[1]{\begin{pmatrix} #1 \end{pmatrix}}
\renewcommand{\f}[2]{\frac{#1}{#2}}
\newcommand{\dspace}{\space\space}
\newcommand{\s}[2]{\sum\limits_{#1}^{#2}}
\newcommand{\mul}[2]{\prod_{#1}^{#2}}
\newcommand{\sq}[1]{\left[ {#1} \right]}
\newcommand{\gath}[1]{\left[ \begin{array}{@{}l@{}} #1 \end{array} \right.}
\newcommand{\case}[1]{\begin{cases} #1 \end{cases}}
\newcommand{\ts}{\text{\space}}
\newcommand{\lm}[1]{\underset{#1}{\lim}}
\newcommand{\suplm}[1]{\underset{#1}{\overline{\lim}}}
\newcommand{\inflm}[1]{\underset{#1}{\underline{\lim}}}
\newcommand{\Ker}[1]{\operatorname{Ker}}

\renewcommand{\phi}{\varphi}
\newcommand{\lr}{\Leftrightarrow}
\renewcommand{\r}{\Rightarrow}
\newcommand{\rr}{\rightarrow}
\renewcommand{\geq}{\geqslant}
\renewcommand{\leq}{\leqslant}
\newcommand{\RR}{\mathbb{R}}
\newcommand{\CC}{\mathbb{C}}
\newcommand{\QQ}{\mathbb{Q}}
\newcommand{\ZZ}{\mathbb{Z}}
\newcommand{\VV}{\mathbb{V}}
\newcommand{\NN}{\mathbb{N}}
\newcommand{\OO}{\underline{O}}
\newcommand{\oo}{\overline{o}}
\renewcommand{\Ker}{\operatorname{Ker}}
\renewcommand{\Im}{\operatorname{Im}}
\newcommand{\Vol}{\operatorname{Vol}}
\newcommand{\vol}{\operatorname{vol}}
\newcommand{\pr}{\operatorname{pr}}
\newcommand{\ort}{\operatorname{ort}}


\DeclarePairedDelimiter\abs{\lvert}{\rvert} %
\makeatletter                               % \abs{}
\let\oldabs\abs                             %
\def\abs{\@ifstar{\oldabs}{\oldabs*}}       %

\begin{document}

\section*{ИДЗ №8 (вариант 12) (Линейная алгебра)}
 {\large Емельянов Владимир, ПМИ гр №247}\\\\
\begin{enumerate}
    \item[\textbf{№1}]Мы знаем, что:
    $$v = \f{1}{10}(3, 9, 3, 1)$$
    Пусть:
    $$v' = (3, 9, 3, 1)$$
    
    Для начала найдём какие-нибудь ЛНЗ решения уравнения:
    $$3a_1 + 9a_2 + 3a_3 + a_4 = 0 \implies \begin{aligned}
        u_1 = (1, 0, 0, -3)^T \quad u_2 = (0, 1, 0, -9)^T\\
        u_3 = (0, 0, 1, -3)^T
    \end{aligned}$$
    Следовательно:
    $$v', u_1, u_2, u_3 \text{ --- базис $\RR^4$}$$
    Теперь применим метод ортогонализации Грамма-Шмидта. 
    Векторы $u_1, u_2, u_3$ уже перпендикулярны $v'$. Найдём векторы $u_1', u_2', u_3'$, такие, что векторы:
    $$u_1', u_2', u_3'$$
    Перпендикулярны между собой ($v'$ будет им перпендикулярен).
    $$u_1' = u_1 = (1, 0, 0, -3)^T$$
    $$u_2' = u_2 - \text{pr}_{u_1'}(u_2) = \left(-\frac{27}{10},\ 1,\ 0,\ -\frac{9}{10}\right)^T$$
    $$u_3' = u_3 - \text{pr}_{\langle u_1', u_2' \rangle}(u_2) =  \left( -\frac{9}{91},\ -\frac{27}{91},\ 1,\ -\frac{3}{91} \right)^T$$
    Теперь отнормируем векторы:
    $$v', u_1', u_2', u_3'$$
    Получаем:
    $$v = \f{1}{10}(3, 9, 3, 1)$$
    $$e_1 = \left( \frac{1}{\sqrt{10}},\ 0,\ 0,\ -\frac{3}{\sqrt{10}} \right)^T$$
    $$e_2 =  \left( \frac{-27}{\sqrt{910}},\ \sqrt{\frac{10}{91}},\ 0,\ \frac{-9}{\sqrt{910}} \right)^T$$
    $$e_3 = \left( \frac{-9}{10\sqrt{91}},\ \frac{-27}{10\sqrt{91}},\ \frac{\sqrt{91}}{10},\ \frac{-3}{10\sqrt{91}} \right)^T$$
    Получаем ортонормированный базис $\RR^4$\\
    \textbf{Ответ: } $(v, e_1, e_2, e_3)$\\


    \item[\textbf{№2}]Для начала решим ОСЛУ:
    $$\mat{ -2 & 3 & -6 & 9 & | & 0 \\ 
    -5 & -2 & -15 & -6 & | & 0} \implies \text{ УСВ: } \mat{ 1 & 0 & 3 & 0 & | & 0 \\ 
    0 & 1 & 0 & 3 & | & 0} $$
    Следовательно, ФСР:
    $$\mat{x_1\\x_2\\x_3\\x_4} = \mat{-3x_3\\-3x_4\\x_3\\x_4} = \mat{-3\\0\\1\\0}x_3 + \mat{0\\-3\\0\\1}x_4$$
    Значит:
    $$U = \langle u_1, u_2 \rangle, \quad \text{где } \quad u_1= (-3, 0, 1, 0)^T, \quad  u_2 = (0, -3, 0, 1)^T$$
    Найдём проекцию $v$ на $U$:
    $$v = (-2, 3, -6, 1)^T$$
    $$\text{pr}_{\langle u_1, u_2\rangle} v = \f{(v, u_1)}{u_1, u_1}u_1 + \f{(v, u_2)}{u_2, u_2}u_2 = 0 + \f{-10}{10}u_2 = -u_2 = (0, 3, 0, 1)^T$$
    Найдём состовляющую:
    $$\text{ort}_{\langle u_1, u_2\rangle} v = v - \text{pr}_{\langle u_1, u_2\rangle} v = (-2, 0, -6, 0)^T$$
    Найдём расстояние от $U$ до $v$
    $$\rho(U, v) = \abs{\text{ort}_{\langle u_1, u_2\rangle} v} = \sqrt{4+36} = 2\sqrt{10}$$
    \textbf{Ответ: }$(0, 3, 0, 1)^T, (-2, 0, -6, 0)^T$ и $2\sqrt{10}$\\


    \item[\textbf{№3}]Найдём ортогональное дополнение $U^\perp$ к $U$, для этого найдём ФСР системы:
    $$\case{
        (u_1, x) = 0\\
        (u_2, x) = 0
    } \implies \case{
        x_1+2x_2 + 2x_3 + x_4 = 0\\
        x_1-3x_2-3x_3 + x_4 = 0
    }\implies$$
    $$ \implies \mat{x_1\\x_2\\x_3\\x_4} = \mat{0\\-1\\1\\0}x_3 + \mat{-1\\0\\0\\1}x_4$$
    Пусть
    $$u_1^\perp = (0, -1, 1, 0), \quad u_2^\perp = (-1, 0, 0, 1)$$
    Тогда:
    $$U^\perp = \langle u_1^\perp, u_2^\perp \rangle$$
    Мы знаем, что:
    $$v = pr_{\langle u_1, u_2 \rangle} v  + ort_{\langle u_1, u_2 \rangle} v = 
     (-10, 0, 0, -10)  + \lambda_1 u_1^\perp + \lambda_2 u_2^\perp $$
    $$v = pr_{\langle w_1, w_2 \rangle} v  + ort_{\langle w_1, w_2 \rangle} v = 
     \mu_1 w_1 + \mu_2 w_2 + (6, 0, -8, -10)$$
    Приравняем:
    $$(-10, 0, 0, -10)  + \lambda_1 u_1^\perp + \lambda_2 u_2^\perp  = \mu_1 w_1 + \mu_2 w_2 + (6, 0, -8, -10)$$
    Распишем:
    $$(-10, 0, 0, -10)  + \lambda_1 (0, -1, 1, 0) + \lambda_2 (-1, 0, 0, 1)  $$
    $$= \mu_1(1, 2, 2, -1) + \mu_2 (1, -1, -3, 3) + (6, 0, -8, -10)$$
    Получаем СЛУ:
    $$\begin{cases}
        -10 - \lambda_2 = \mu_1 + \mu_2 + 6 \\
        -\lambda_1 = 2\mu_1 - \mu_2 \\
        \lambda_1 = 2\mu_1 - 3\mu_2 - 8 \\
        -10 + \lambda_2 = -\mu_1 + 3\mu_2 - 10
        \end{cases} \implies \begin{pmatrix}
            0 & -1 & -1 & -1 & 16 \\
            -1 & 0 & -2 & 1 & 0 \\
            1 & 0 & -2 & 3 & -8 \\
            0 & 1 & 1 & -3 & 0 \\
            \end{pmatrix} \implies$$
    $$\implies\text{УСВ: } 
    \begin{pmatrix}
        1 & 0 & 0 & 0 & 0 \\
        0 & 1 & 0 & 0 & -10 \\
        0 & 0 & 1 & 0 & -2 \\
        0 & 0 & 0 & 1 & -4
        \end{pmatrix}
    $$
    Следовательно,
    
    \[
    \lambda_1 = 0,\quad \lambda_2 = -10,\quad \mu_1 = -2,\quad \mu_2 = -4
    \]
    Получается:
    $$v = -2(1, 2, 2, -1) + -4 (1, -1, -3, 3) + (6, 0, -8, -10) = (0, 0, 0, -20)$$
    \textbf{Ответ: } $(0, 0, 0, -20)$\\


    \item[\textbf{№4}]Пусть:
    $$v_1 = (-4, 5, -1), \quad v_2 = (16, -18, 3), \quad v_3 = (8, -14, 3)$$
    $$w_1 = (-3, 3, -4), \quad w_2 = (-3, -1, -7), \quad w_3 = (6, -9, 10)$$
    - координаты соответствующих многочленов в ортонормированном базисе $1, x, x^2$. Пусть тогда
    $$V = [v_1, v_2, v_3] = \mat{-4 & 16 & 8\\5 &-18 & -14\\-1& 3 & 3}$$
    $$W = [w_1, w_2, w_3] = \mat{-3 & -3 & 6\\ 3 & -1 & -9 \\ -4 & -7 & 10}$$
    % Тогда, т.к. векторы $v_1, v_2, v_3$ ЛНЗ и векторы $w_1, w_2, w_3$ ЛНЗ, то можем
    % найти матрицу перехода $A$ от $v_1, v_2, v_3$ к $w_1, w_2, w_3$:
    % $$VA = W \lr \small{\mat{-4 & 16 & 8 & | & -3 & -3 & 6\\5 & -18 & -14 & | &3 & -1 & -9\\-1 & 3 & 3& | & -4 & -7 & 10}} \implies$$
    % $$\implies \text{УСВ: }
    %     \mat{
    %     1 & 0 & 0 &|& \frac{71}{2} & \frac{155}{2} & -82 \\
    %     0 & 1 & 0 &|& \frac{55}{8} & \frac{119}{8} & -\frac{65}{4} \\
    %     0 & 0 & 1 &|& \frac{29}{8} & \frac{69}{8} & -\frac{31}{4}
    %     } \implies A =         \mat{
    %         \frac{71}{2} & \frac{155}{2} & -82 \\
    %         \frac{55}{8} & \frac{119}{8} & -\frac{65}{4} \\
    %         \frac{29}{8} & \frac{69}{8} & -\frac{31}{4}
    %         }
    % $$
    Теперь мы может найти $\Vol(P(w_1, w_2, w_3))$:
    $$\Vol(P(w_1, w_2, w_3)) = |\det A| \Vol(P(v_1, v_2, v_3)) = 5|\det A|$$
    Найдём для этого $\det A$:
    $$\det V =  \det \mat{-4 & 16 & 8\\5 &-18 & -14\\-1& 3 & 3} = 8$$
    $$\det W = \det \mat{-3 & -3 & 6\\ 3 & -1 & -9 \\ -4 & -7 & 10} = 51$$
    $$\det V \det A = \det W \implies \det A = \f{\det W}{\det V} = \f{51}{8}$$
    Получаем:
    $$\Vol(P(w_1, w_2, w_3)) = 5|\det A| = 5\cdot \f{51}{8} = \f{255}{8}$$
    \textbf{Ответ: }$\f{255}{8}$\\

    \item[\textbf{№5}]Пусть:
    $$a_1 = v_1 - v_0, \quad a_2 = v_2 - v_0, \quad a_3 = v_3 - v_0$$
    \begin{itemize}
        \item \( a_1 = v_1 - v_0 = (1,\ 7,\ -8,\ -4,\ -7) \)
        \item \( a_2 = v_2 - v_0 = (-4,\ -7,\ 0,\ -1,\ -7) \)
        \item \( a_3 = v_3 - v_0 = (-7,\ 1,\ 0,\ -7,\ 4) \)
    \end{itemize}
    Проверим, что они ЛНЗ:
   $$ \begin{pmatrix}
        1 & -4 & -7 \\
        7 & -7 & 1 \\
        -8 & 0 & 0 \\
        -4 & -1 & -7 \\
        -7 & -7 & 4
        \end{pmatrix} \implies \text{УСВ: } \begin{pmatrix}
            1 & 0 & 0 \\
            0 & 1 & 0 \\
            0 & 0 & 1 \\
            0 & 0 & 0 \\
            0 & 0 & 0
            \end{pmatrix}$$
    Следовательно, они образуют базис направляющего подпространства линейного многообразия $L$, то есть $L$ можно записать как:
    $$L = v_0 + \langle a_1, a_2, -a_3 \rangle$$
    Пусть:
    $$b = v-v_0 =  (-8, 1, 0, -10, -2) - (2, 3, 8, -7, -6) = (-10,\ -2,\ -8,\ -3,\ 4)$$
    Обозначим матрицу из векторов \( a_1, a_2, a_3 \) как столбцы:
    \[
    A = \begin{pmatrix} a_1 & a_2 & a_3 \end{pmatrix} = \begin{pmatrix}
        1 & -4 & -7 \\
        7 & -7 & 1 \\
        -8 & 0 & 0 \\
        -4 & -1 & -7 \\
        -7 & -7 & 4
        \end{pmatrix}
    \]
    Найдём проекцию на направляющее подпространство $L$:
    $$\pr_{\langle a_1, a_2, a_3 \rangle} b = A(A^TA)^{-1}A^{T} = \left(
        - \frac{6503}{895},\ 
        \frac{1778}{20585},\ 
        - \frac{192}{179},\ 
        - \frac{147796}{20585},\ 
        \frac{19523}{20585}
        \right)$$
    Следовательно, ближайшая точка $a_0$ к $v$ на трёхмерной плоскости $L$:
    $$a_0 = v_0 + \pr_{\langle a_1, a_2, a_3 \rangle}b = \left(
        - \frac{4713}{895},\ 
        \frac{63533}{20585},\ 
        \frac{1240}{179},\ 
        - \frac{291891}{20585},\ 
        - \frac{103987}{20585}
        \right)$$
    Расстояние от $a_0$ до $v$:
    $$||a_0 - v|| = \sqrt{\frac{1782651}{20585}}$$
    \textbf{Ответ: } $a_0 = \left(
        - \frac{4713}{895},\ 
        \frac{63533}{20585},\ 
        \frac{1240}{179},\ 
        - \frac{291891}{20585},\ 
        - \frac{103987}{20585}
        \right)$
        и
    $ \sqrt{\frac{1782651}{20585}}$

\end{enumerate}
\end{document}