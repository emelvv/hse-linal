\documentclass[a4paper]{article}
\usepackage{setspace}
\usepackage[T2A]{fontenc} %
\usepackage[utf8]{inputenc} % подключение русского языка
\usepackage[russian]{babel} %
\usepackage[12pt]{extsizes}
\usepackage{mathtools}
\usepackage{graphicx}
\usepackage{fancyhdr}
\usepackage{amssymb}
\usepackage{amsmath, amsfonts, amssymb, amsthm, mathtools}
\usepackage{tikz}

\usetikzlibrary{positioning}
\setstretch{1.3}

\newcommand{\mat}[1]{\begin{pmatrix} #1 \end{pmatrix}}
\renewcommand{\det}[1]{\begin{vmatrix} #1 \end{vmatrix}}
\renewcommand{\f}[2]{\frac{#1}{#2}}
\newcommand{\dspace}{\space\space}
\newcommand{\s}[2]{\sum\limits_{#1}^{#2}}
\newcommand{\mul}[2]{\prod_{#1}^{#2}}
\newcommand{\sq}[1]{\left[ {#1} \right]}
\newcommand{\gath}[1]{\left[ \begin{array}{@{}l@{}} #1 \end{array} \right.}
\newcommand{\case}[1]{\begin{cases} #1 \end{cases}}
\newcommand{\ts}{\text{\space}}
\newcommand{\lm}[1]{\underset{#1}{\lim}}
\newcommand{\suplm}[1]{\underset{#1}{\overline{\lim}}}
\newcommand{\inflm}[1]{\underset{#1}{\underline{\lim}}}
\newcommand{\Ker}[1]{\operatorname{Ker}}

\renewcommand{\phi}{\varphi}
\newcommand{\lr}{\Leftrightarrow}
\renewcommand{\r}{\Rightarrow}
\newcommand{\rr}{\rightarrow}
\renewcommand{\geq}{\geqslant}
\renewcommand{\leq}{\leqslant}
\newcommand{\RR}{\mathbb{R}}
\newcommand{\CC}{\mathbb{C}}
\newcommand{\QQ}{\mathbb{Q}}
\newcommand{\ZZ}{\mathbb{Z}}
\newcommand{\VV}{\mathbb{V}}
\newcommand{\NN}{\mathbb{N}}
\newcommand{\OO}{\underline{O}}
\newcommand{\oo}{\overline{o}}
\renewcommand{\Ker}{\operatorname{Ker}}
\renewcommand{\Im}{\operatorname{Im}}


\DeclarePairedDelimiter\abs{\lvert}{\rvert} %
\makeatletter                               % \abs{}
\let\oldabs\abs                             %
\def\abs{\@ifstar{\oldabs}{\oldabs*}}       %

\begin{document}

\section*{ИДЗ №8 (вариант 12) (Линейная алгебра)}
 {\large Емельянов Владимир, ПМИ гр №247}\\\\
\begin{enumerate}
    \item[\textbf{№1}]Мы знаем, что:
    $$v = \f{1}{10}(3, 9, 3, 1)$$
    Пусть:
    $$v' = (3, 9, 3, 1)$$
    
    Для начала найдём какие-нибудь ЛНЗ решения уравнения:
    $$3a_1 + 9a_2 + 3a_3 + a_4 = 0 \implies \begin{aligned}
        u_1 = (1, 0, 0, -3)^T \quad u_2 = (0, 1, 0, -9)^T\\
        u_3 = (0, 0, 1, -3)^T
    \end{aligned}$$
    Следовательно:
    $$v', u_1, u_2, u_3 \text{ --- базис $\RR^4$}$$
    Теперь применим метод ортогонализации Грамма-Шмидта. 
    Векторы $u_1, u_2, u_3$ уже перпендикулярны $v'$. Найдём векторы $u_1', u_2', u_3'$, такие, что векторы:
    $$u_1', u_2', u_3'$$
    Перпендикулярны между собой ($v'$ будет им перпендикулярен).
    $$u_1' = u_1 = (1, 0, 0, -3)^T$$
    $$u_2' = u_2 - \text{pr}_{u_1'}(u_2) = \left(-\frac{27}{10},\ 1,\ 0,\ -\frac{9}{10}\right)^T$$
    $$u_3' = u_3 - \text{pr}_{\langle u_1', u_2' \rangle}(u_2) =  \left( -\frac{9}{91},\ -\frac{27}{91},\ 1,\ -\frac{3}{91} \right)^T$$
    Теперь отнормируем векторы:
    $$v', u_1', u_2', u_3'$$
    Получаем:
    $$v = \f{1}{10}(3, 9, 3, 1)$$
    $$e_1 = \left( \frac{1}{\sqrt{10}},\ 0,\ 0,\ -\frac{3}{\sqrt{10}} \right)^T$$
    $$e_2 =  \left( \frac{-27}{\sqrt{910}},\ \sqrt{\frac{10}{91}},\ 0,\ \frac{-9}{\sqrt{910}} \right)^T$$
    $$e_3 = \left( \frac{-9}{10\sqrt{91}},\ \frac{-27}{10\sqrt{91}},\ \frac{\sqrt{91}}{10},\ \frac{-3}{10\sqrt{91}} \right)^T$$
    Получаем ортонормированный базис $\RR^4$\\
    \textbf{Ответ: } $(v, e_1, e_2, e_3)$\\


    \item[\textbf{№2}]Для начала решим ОСЛУ:
    $$\mat{ -2 & 3 & -6 & 9 & | & 0 \\ 
    -5 & -2 & -15 & -6 & | & 0} \implies \text{ УСВ: } \mat{ 1 & 0 & 3 & 0 & | & 0 \\ 
    0 & 1 & 0 & 3 & | & 0} $$
    Следовательно, ФСР:
    $$\mat{x_1\\x_2\\x_3\\x_4} = \mat{-3x_3\\-3x_4\\x_3\\x_4} = \mat{-3\\0\\1\\0}x_3 + \mat{0\\-3\\0\\1}x_4$$
    Значит:
    $$U = \langle u_1, u_2 \rangle, \quad \text{где } \quad u_1= (-3, 0, 1, 0)^T, \quad  u_2 = (0, -3, 0, 1)^T$$
    Найдём проекцию $v$ на $U$:
    $$v = (-2, 3, -6, 1)^T$$
    $$\text{pr}_{\langle u_1, u_2\rangle} v = \f{(v, u_1)}{u_1, u_1}u_1 + \f{(v, u_2)}{u_2, u_2}u_2 = 0 + \f{-10}{10}u_2 = -u_2 = (0, 3, 0, 1)^T$$
    Найдём состовляющую:
    $$\text{ort}_{\langle u_1, u_2\rangle} v = v - \text{pr}_{\langle u_1, u_2\rangle} v = (-2, 0, -6, 0)^T$$
    Найдём расстояние от $U$ до $v$
    $$\rho(U, v) = \abs{\text{ort}_{\langle u_1, u_2\rangle} v} = \sqrt{4+36} = 2\sqrt{10}$$
    \textbf{Ответ: }$(0, 3, 0, 1)^T, (-2, 0, -6, 0)^T$ и $2\sqrt{10}$
\end{enumerate}
\end{document}