\documentclass[a4paper]{article}
\usepackage{setspace}
\usepackage[utf8]{inputenc}
\usepackage[russian]{babel}
\usepackage[12pt]{extsizes}
\usepackage{mathtools}
\usepackage{graphicx}
\usepackage{fancyhdr}
\usepackage{amssymb}
\usepackage{amsmath, amsfonts, amssymb, amsthm, mathtools}
\usepackage{tikz}

\usetikzlibrary{positioning}
\setstretch{1.3}

\newcommand{\mat}[1]{\begin{pmatrix} #1 \end{pmatrix}}
\newcommand{\vmat}[1]{\begin{vmatrix} #1 \end{vmatrix}}
\renewcommand{\f}[2]{\frac{#1}{#2}}
\newcommand{\dspace}{\space\space}
\newcommand{\s}[2]{\sum\limits_{#1}^{#2}}
\newcommand{\mul}[2]{\prod_{#1}^{#2}}
\newcommand{\sq}[1]{\left[ {#1} \right]}
\newcommand{\gath}[1]{\left[ \begin{array}{@{}l@{}} #1 \end{array} \right.}
\newcommand{\case}[1]{\begin{cases} #1 \end{cases}}
\newcommand{\ts}{\text{\space}}
\newcommand{\lm}[1]{\underset{#1}{\lim}}
\newcommand{\suplm}[1]{\underset{#1}{\overline{\lim}}}
\newcommand{\inflm}[1]{\underset{#1}{\underline{\lim}}}
\newcommand{\Ker}[1]{\operatorname{Ker}}

\renewcommand{\phi}{\varphi}
\newcommand{\lr}{\Leftrightarrow}
\renewcommand{\l}{\left(}
\renewcommand{\r}{\right)}
\newcommand{\rr}{\rightarrow}
\renewcommand{\geq}{\geqslant}
\renewcommand{\leq}{\leqslant}
\newcommand{\RR}{\mathbb{R}}
\newcommand{\CC}{\mathbb{C}}
\newcommand{\QQ}{\mathbb{Q}}
\newcommand{\ZZ}{\mathbb{Z}}
\newcommand{\VV}{\mathbb{V}}
\newcommand{\NN}{\mathbb{N}}
\newcommand{\OO}{\underline{O}}
\newcommand{\oo}{\overline{o}}
\renewcommand{\Ker}{\operatorname{Ker}}
\renewcommand{\Im}{\operatorname{Im}}
\newcommand{\vol}{\text{vol}}
\newcommand{\Vol}{\text{Vol}}

\DeclarePairedDelimiter\abs{\lvert}{\rvert} %
\makeatletter                               % \abs{}
\let\oldabs\abs                             %
\def\abs{\@ifstar{\oldabs}{\oldabs*}}       %

\begin{document}

\section*{Домашнее задание на 29.05 (Линейная алгебра)}
{\large Емельянов Владимир, ПМИ гр №247}\\\\
\begin{enumerate}
  \item[\textbf{№1}]\begin{enumerate}
    \item[1)]Проверим, что $U$ и $V$ ортогональны
    \begin{itemize}
      \item Матрица \( U \):
     \[
     U = \begin{pmatrix}
     -\frac{2}{\sqrt{5}} & \frac{1}{\sqrt{5}} \\
     \frac{1}{\sqrt{5}} & \frac{2}{\sqrt{5}}
     \end{pmatrix}
     \]
     Убедимся, что \( U^T U = E \):
     \[
     U^T U = \begin{pmatrix}
     \left(-\frac{2}{\sqrt{5}}\right)^2 + \left(\frac{1}{\sqrt{5}}\right)^2 & \left(-\frac{2}{\sqrt{5}}\right)\left(\frac{1}{\sqrt{5}}\right) + \left(\frac{1}{\sqrt{5}}\right)\left(\frac{2}{\sqrt{5}}\right) \\
     \left(\frac{1}{\sqrt{5}}\right)\left(-\frac{2}{\sqrt{5}}\right) + \left(\frac{2}{\sqrt{5}}\right)\left(\frac{1}{\sqrt{5}}\right) & \left(\frac{1}{\sqrt{5}}\right)^2 + \left(\frac{2}{\sqrt{5}}\right)^2
     \end{pmatrix}= \]
     \[= \begin{pmatrix}
     1 & 0 \\
     0 & 1
     \end{pmatrix}
     \]

     \item Матрица \( V \):
     \[
     V^T = \begin{pmatrix}
     -\frac{2}{3} & -\frac{2}{3} & \frac{1}{3} \\
     -\frac{1}{3} & \frac{2}{3} & \frac{2}{3} \\
     \frac{2}{3} & -\frac{1}{3} & \frac{2}{3}
     \end{pmatrix}
     \]
     Убедимся, что \( V^T V = E \):
     \[
     V^T V = \begin{pmatrix}
     \left(-\frac{2}{3}\right)^2 + \left(-\frac{2}{3}\right)^2 + \left(\frac{1}{3}\right)^2 & \text{...} \\
     \text{...} & \text{...}
     \end{pmatrix} = E
     \]\\
    \end{itemize}

    \item[2)] Проверим матрицу $\Sigma$:
      \[
      \Sigma = \begin{pmatrix}
      9\sqrt{5} & 0 & 0 \\
      0 & 3\sqrt{5} & 0
      \end{pmatrix}
      \]
      Диагональные элементы \( 9\sqrt{5} \) и \( 3\sqrt{5} \) упорядочены по убыванию.\\

  \item[3)]Представление \( A = u_1\sigma_1v_1^T + u_2\sigma_2v_2^T \):
  \begin{itemize}
    \item Вычислим \( u_1\sigma_1v_1^T \):
    \[
    u_1 = \begin{pmatrix} -\frac{2}{\sqrt{5}} \\ \frac{1}{\sqrt{5}}
      \end{pmatrix}, \quad \sigma_1 = 9\sqrt{5}, \quad v_1^T = 
      \begin{pmatrix} -\frac{2}{3} & -\frac{2}{3} & \frac{1}{3} \end{pmatrix},
    \]
    \[
    u_1\sigma_1v_1^T = \begin{pmatrix} 12 & 12 & -6 \\ -6 & -6 & 3 \end{pmatrix}
    \]

    \item Вычислим \( u_2\sigma_2v_2^T \):
    \[
    u_2 = \begin{pmatrix} \frac{1}{\sqrt{5}} \\ \frac{2}{\sqrt{5}} \end{pmatrix}, \quad \sigma_2 = 
    3\sqrt{5}, \quad v_2^T = \begin{pmatrix} 
      -\frac{1}{3} & \frac{2}{3} & \frac{2}{3} \end{pmatrix}
    \]
    \[
    u_2\sigma_2v_2^T = \begin{pmatrix} -1 & 2 & 2 \\ -2 & 4 & 4 \end{pmatrix}
    \]

    \item Сумма:
    \[
    u_1\sigma_1v_1^T + u_2\sigma_2v_2^T = \begin{pmatrix} 11 & 14 & -4 \\ -8 & -2 & 7 \end{pmatrix} = A
    \]
  \end{itemize}

  \item[4)] Найдём $B$ ранга 1:
  \[
  B = u_1\sigma_1v_1^T = \begin{pmatrix} 12 & 12 & -6 \\ -6 & -6 & 3 \end{pmatrix}
  \]
  Норма Фробениуса разности:
  \[
  \|A - B\| = \sqrt{(3\sqrt{5})^2} = 3\sqrt{5}
  \]
  \end{enumerate}

  \item[\textbf{№2}]Полное сингулярное разложение матрицы \( A \):  
  \[
  A = U \Sigma V^T = \begin{pmatrix}
  -2/\sqrt{5} & 1/\sqrt{5} \\
  1/\sqrt{5} & 2/\sqrt{5}
  \end{pmatrix} \begin{pmatrix}
  9\sqrt{5} & 0 & 0 \\
  0 & 3\sqrt{5} & 0
  \end{pmatrix} \begin{pmatrix}
  -2/3 & -2/3 & 1/3 \\
  -1/3 & 2/3 & 2/3 \\
  2/3 & -1/3 & 2/3
  \end{pmatrix}^T
  \]
  Матрица \( A \) имеет размер \( 2 \times 3 \), поэтому \( \min(m, n) = 2 \).  
  Размеры усечённых матриц:
  \begin{itemize}
    \item \( U' \): \( 2 \times 2 \) (первые 2 столбца \( U \)),  
    \item \( \Sigma' \): \( 2 \times 2 \) (первые 2 сингулярных значения),
    \item \( V' \): \( 2 \times 3 \) (первые 2 строки \( V^T \)).
  \end{itemize}
  Сформируем:
  \begin{itemize}
    \item \( U' = U \):  
    \[
    U' = \begin{pmatrix}
    -2/\sqrt{5} & 1/\sqrt{5} \\
    1/\sqrt{5} & 2/\sqrt{5}
    \end{pmatrix}
    \]
    
    \item \( \Sigma' \):  
    \[
    \Sigma' = \begin{pmatrix}
    9\sqrt{5} & 0 \\
    0 & 3\sqrt{5}
    \end{pmatrix}
    \]

    \item \( (V')^T \):  
    \[
    (V')^T = \begin{pmatrix}
    -2/3 & -2/3 & 1/3 \\
    -1/3 & 2/3 & 2/3
    \end{pmatrix}
    \]
  \end{itemize}
  Проверим:
  Вычислим произведение \( U' \Sigma' (V')^T \):  
  \begin{itemize}
    \item \( U' \Sigma' \):  
    \[
    U' \Sigma' = \begin{pmatrix}
    -2/\sqrt{5} \cdot 9\sqrt{5} & 1/\sqrt{5} \cdot 3\sqrt{5} \\
    1/\sqrt{5} \cdot 9\sqrt{5} & 2/\sqrt{5} \cdot 3\sqrt{5}
    \end{pmatrix} = \begin{pmatrix}
    -18 & 3 \\
    9 & 6
    \end{pmatrix}
    \]
    \item Умножение на \( (V')^T \):  
    \[
    \begin{pmatrix}
    -18 & 3 \\
    9 & 6
    \end{pmatrix} \begin{pmatrix}
    -2/3 & -2/3 & 1/3 \\
    -1/3 & 2/3 & 2/3
    \end{pmatrix} = \begin{pmatrix}
    11 & 14 & -4 \\
    -8 & -2 & 7
    \end{pmatrix} = A
    \]
  \end{itemize}
  \textbf{Ответ: }\[
  A = \underbrace{\begin{pmatrix}
  -2/\sqrt{5} & 1/\sqrt{5} \\
  1/\sqrt{5} & 2/\sqrt{5}
  \end{pmatrix}}_{U'} \underbrace{\begin{pmatrix}
  9\sqrt{5} & 0 \\
  0 & 3\sqrt{5}
  \end{pmatrix}}_{\Sigma'} \underbrace{\begin{pmatrix}
  -2/3 & -2/3 & 1/3 \\
  -1/3 & 2/3 & 2/3
  \end{pmatrix}}_{(V')^T}
  \]  

  \item[\textbf{№3}]Ненулевая матрица-строка имеет ранг 1.
    Её единственное сингулярное значение:  
  \[
  \sigma = \|\mathbf{a}\| = \sqrt{a_1^2 + a_2^2 + \ldots + a_n^2}
  \]
  Матрица \( U' \) размера \( 1 \times 1 \):  
  \[
  U' = \begin{pmatrix} 1 \end{pmatrix}
  \]
  Нормированный вектор направления \( \mathbf{a} \):  
  \[
  \mathbf{v}_1 = \frac{\mathbf{a}^T}{\sigma} =
    \frac{1}{\sigma} \begin{pmatrix} a_1 \\ a_2 \\ \vdots \\ a_n \end{pmatrix}
  \]  
  Матрица \( (V')^T \) размера \( 1 \times n \):  
  \[
  (V')^T = \mathbf{v}_1^T = \frac{1}{\sigma} 
  \begin{pmatrix} a_1 & a_2 & \ldots & a_n \end{pmatrix}
  \]
  Усечённое SVD:
  \[
  \mathbf{a} = U' \Sigma' (V')^T,
  \]  
  \textbf{Ответ:} Усечённое сингулярное разложение:  
  \[
    \underbrace{\begin{pmatrix} 1 \end{pmatrix}}_{U'}
      \underbrace{\begin{pmatrix} \sigma \end{pmatrix}}_{\Sigma'}
      \underbrace{\left( \frac{1}{\sigma} 
      \begin{pmatrix} a_1 & a_2 & \ldots & a_n \end{pmatrix} \right)}_{(V')^T}
  \]

  \item[\textbf{№4}]Найдём усечённые и полные сингулярные разложения
  \begin{enumerate}
  \item[\textbf{4.1.}] Матрица:  
  \[
  A = \begin{pmatrix}
  3 & 2 & 2 \\
  2 & 3 & -2
  \end{pmatrix}
  \]
  Вычисляем \( A^T A \):  
  \[
  A^T A = \begin{pmatrix}
  3 & 2 \\
  2 & 3 \\
  2 & -2
  \end{pmatrix}^T \begin{pmatrix}
  3 & 2 & 2 \\
  2 & 3 & -2
  \end{pmatrix} = \begin{pmatrix}
  17 & 8 \\
  8 & 17
  \end{pmatrix}
  \]
  Находим собственные значения \( A^T A \):  
  \[
  \det(A^T A - \lambda I) = (17 - \lambda)^2 - 64 = 0 \implies \lambda_1 = 25, \, \lambda_2 = 9.
  \]  
  Сингулярные значения:  
  \[
  \sigma_1 = \sqrt{25} = 5, \quad \sigma_2 = \sqrt{9} = 3
  \]
  Правые сингулярные векторы \( V \):  
  \begin{itemize}
    \item 
    Для \( \lambda_1 = 25 \):  
    \[
    (A^T A - 25I)\mathbf{v} = \begin{pmatrix}
    -8 & 8 \\
    8 & -8
    \end{pmatrix}\mathbf{v} = 0 \implies \mathbf{v}_1 = \frac{1}{\sqrt{2}} 
    \begin{pmatrix} 1 \\ 1 \end{pmatrix}
    \]  
    \item 
    Для \( \lambda_2 = 9 \):  
    \[
    (A^T A - 9I)\mathbf{v} = \begin{pmatrix}
    8 & 8 \\
    8 & 8
    \end{pmatrix}\mathbf{v} = 0 \implies \mathbf{v}_2 = 
    \frac{1}{\sqrt{2}} \begin{pmatrix} 1 \\ -1 \end{pmatrix}
    \]  
    \item
    Третий вектор \( \mathbf{v}_3 = 
    \begin{pmatrix} 0 \\ 0 \\ 1 \end{pmatrix} \) (ортогонален первым двум).  
  \end{itemize}
  Матрица \( V \):  
  \[
  V = \begin{pmatrix}
  \frac{1}{\sqrt{2}} & \frac{1}{\sqrt{2}} & 0 \\
  \frac{1}{\sqrt{2}} & -\frac{1}{\sqrt{2}} & 0 \\
  0 & 0 & 1
  \end{pmatrix}
  \]
  Левые сингулярные векторы \( U \):  
  \[
  \mathbf{u}_1 = \frac{1}{\sigma_1} A 
  \mathbf{v}_1 = \frac{1}{5} \begin{pmatrix} \frac{5}{\sqrt{2}} \\
     \frac{5}{\sqrt{2}} \end{pmatrix} = \begin{pmatrix} \frac{1}{\sqrt{2}} \\
       \frac{1}{\sqrt{2}} \end{pmatrix}, \quad \mathbf{u}_2 = \frac{1}{\sigma_2} A 
       \mathbf{v}_2 = \frac{1}{3} \begin{pmatrix} \frac{1}{\sqrt{2}} \\
         -\frac{1}{\sqrt{2}} \end{pmatrix}
  \]  
  Матрица \( U \):  
  \[
  U = \begin{pmatrix}
  \frac{1}{\sqrt{2}} & \frac{1}{\sqrt{2}} \\
  \frac{1}{\sqrt{2}} & -\frac{1}{\sqrt{2}}
  \end{pmatrix}
  \]
  Полное SVD:  
  \[
  A = U \Sigma V^T = \begin{pmatrix}
  \frac{1}{\sqrt{2}} & \frac{1}{\sqrt{2}} \\
  \frac{1}{\sqrt{2}} & -\frac{1}{\sqrt{2}}
  \end{pmatrix} \begin{pmatrix}
  5 & 0 & 0 \\
  0 & 3 & 0
  \end{pmatrix} \begin{pmatrix}
  \frac{1}{\sqrt{2}} & \frac{1}{\sqrt{2}} & 0 \\
  \frac{1}{\sqrt{2}} & -\frac{1}{\sqrt{2}} & 0 \\
  0 & 0 & 1
  \end{pmatrix}
  \]
  Усечённое сингулярное разложение:
  \[
  A = U' \Sigma' (V')^T = \begin{pmatrix}
  \frac{1}{\sqrt{2}} & \frac{1}{\sqrt{2}} \\
  \frac{1}{\sqrt{2}} & -\frac{1}{\sqrt{2}}
  \end{pmatrix} \begin{pmatrix}
  5 & 0 \\
  0 & 3
  \end{pmatrix} \begin{pmatrix}
  \frac{1}{\sqrt{2}} & \frac{1}{\sqrt{2}} & 0 \\
  \frac{1}{\sqrt{2}} & -\frac{1}{\sqrt{2}} & 0
  \end{pmatrix}
  \]

  \item[\textbf{4.2.}]Матрица:  
  \[
  A = \begin{pmatrix}
  0 & 0 \\
  0 & 0 \\
  -3 & 4
  \end{pmatrix}
  \]
  Вычисляем \( A^T A \):  
  \[
  A^T A = \begin{pmatrix}
  0 & 0 & -3 \\
  0 & 0 & 4
  \end{pmatrix} \begin{pmatrix}
  0 & 0 \\
  0 & 0 \\
  -3 & 4
  \end{pmatrix} = \begin{pmatrix}
  9 & -12 \\
  -12 & 16
  \end{pmatrix}
  \]
  Собственные значения:  
  \[
  \det(A^T A - \lambda I) = \lambda^2 - 
  25\lambda = 0 \implies \lambda_1 = 25, \, \lambda_2 = 0
  \]  
  Сингулярные значения:  
  \[
  \sigma_1 = 5, \quad \sigma_2 = 0
  \]
  Правые сингулярные векторы \( V \):  
  \begin{itemize}
    \item 
    Для \( \lambda_1 = 25 \):  
    \[
    (A^T A - 25I)\mathbf{v} = \begin{pmatrix}
    -16 & -12 \\
    -12 & -9
    \end{pmatrix}\mathbf{v} = 0 \implies \mathbf{v}_1 = 
    \frac{1}{5} \begin{pmatrix} -3 \\ 4 \end{pmatrix}
    \]
    
    \item
    Для \( \lambda_2 = 0 \):  
    \[
    \mathbf{v}_2 = \frac{1}{5} \begin{pmatrix} 4 \\ 3 \end{pmatrix}
    \]
  \end{itemize}
  Левые сингулярные векторы \( U \):  
  \begin{itemize}
    \item 
    Для \( \sigma_1 = 5 \):  
    \[
    \mathbf{u}_1 = \frac{1}{5} A \mathbf{v}_1 =
     \begin{pmatrix} 0 \\ 0 \\ 1 \end{pmatrix}
    \]  
    \item
    Остальные векторы \( \mathbf{u}_2, \mathbf{u}_3 \)
     выбираются ортогональными к \( \mathbf{u}_1 \)
  \end{itemize}
  Полное SVD:  
  \[
  U = \begin{pmatrix}
  0 & 0 & 1 \\
  0 & 1 & 0 \\
  1 & 0 & 0
  \end{pmatrix}, \quad \Sigma = \begin{pmatrix}
  5 & 0 \\
  0 & 0 \\
  0 & 0
  \end{pmatrix}, \quad V^T = \begin{pmatrix}
  -\frac{3}{5} & \frac{4}{5} \\
  \frac{4}{5} & \frac{3}{5}
  \end{pmatrix}
  \]
  \[
  A = \begin{pmatrix}
  0 & 0 & 1 \\
  0 & 1 & 0 \\
  1 & 0 & 0
  \end{pmatrix} \begin{pmatrix}
  5 & 0 \\
  0 & 0 \\
  0 & 0
  \end{pmatrix} \begin{pmatrix}
  -\frac{3}{5} & \frac{4}{5} \\
  \frac{4}{5} & \frac{3}{5}
  \end{pmatrix}^T
  \]
  
  Усечённое SVD:
  \[
  A = U' \Sigma' (V')^T = \begin{pmatrix} 0 \\ 0 \\ 1 \end{pmatrix} 
  \begin{pmatrix} 5 \end{pmatrix} \begin{pmatrix} -\frac{3}{5} & 
    \frac{4}{5} \end{pmatrix}
  \]
  \end{enumerate}

  \item[\textbf{№5}]Найдём усечённые SVD и матрицы ранга 1:
  \begin{enumerate}
  
  \item[\textbf{5.1}]Матрица:  
  \[
  A = \begin{pmatrix}
  3 & 2 & 4 \\
  1 & 4 & -2
  \end{pmatrix}
  \]
  Вычисляем \( A^T A \):  
  \[
  A^T A = \begin{pmatrix}
  3 & 1 \\
  2 & 4 \\
  4 & -2
  \end{pmatrix} \begin{pmatrix}
  3 & 2 & 4 \\
  1 & 4 & -2
  \end{pmatrix} = \begin{pmatrix}
  10 & 10 & 10 \\
  10 & 20 & 0 \\
  10 & 0 & 20
  \end{pmatrix}
  \]
  Собственные значения \( A^T A \):  
  \[
  \det(A^T A - \lambda I) = -\lambda^3 + 50\lambda^2 -
   600\lambda = 0 \implies \lambda_1 = 30, \, \lambda_2 = 10, \, \lambda_3 = 0
  \]  
  Сингулярные значения:  
  \[
  \sigma_1 = \sqrt{30}, \quad \sigma_2 = \sqrt{10}
  \]
  Правые сингулярные векторы \( V \):  
  \begin{itemize}
    \item 
    Для \( \lambda_1 = 30 \):  
    \[
    (A^T A - 30I)\mathbf{v} = \begin{pmatrix}
    -20 & 10 & 10 \\
    10 & -10 & 0 \\
    10 & 0 & -10
    \end{pmatrix}\mathbf{v} = 0 \implies \mathbf{v}_1 = 
    \frac{1}{\sqrt{3}} \begin{pmatrix} 1 \\ 1 \\ 1 \end{pmatrix}
    \]  
    \item
    Для \( \lambda_2 = 10 \):  
    \[
    (A^T A - 10I)\mathbf{v} = \begin{pmatrix}
    0 & 10 & 10 \\
    10 & 10 & 0 \\
    10 & 0 & 10
    \end{pmatrix}\mathbf{v} = 0 \implies \mathbf{v}_2 = 
    \frac{1}{\sqrt{2}} \begin{pmatrix} 1 \\ -1 \\ 0 \end{pmatrix}
    \]
  \end{itemize}
  Левые сингулярные векторы \( U \):  
  \[
  \mathbf{u}_1 = \frac{1}{\sigma_1} A \mathbf{v}_1 = 
  \frac{1}{\sqrt{30}} \begin{pmatrix} 6 \\ 3 \end{pmatrix}, 
  \quad \mathbf{u}_2 = \frac{1}{\sigma_2} A \mathbf{v}_2 =
   \frac{1}{\sqrt{10}} \begin{pmatrix} 1 \\ -3 \end{pmatrix}
  \]  
  Матрица \( U' \):  
  \[
  U' = \begin{pmatrix}
  \frac{2}{\sqrt{5}} & \frac{1}{\sqrt{5}} \\
  \frac{1}{\sqrt{5}} & -\frac{2}{\sqrt{5}}
  \end{pmatrix}
  \]
  Усечённое SVD:
  \[
  A = U' \Sigma' (V')^T = \begin{pmatrix}
  \frac{2}{\sqrt{5}} & \frac{1}{\sqrt{5}} \\
  \frac{1}{\sqrt{5}} & -\frac{2}{\sqrt{5}}
  \end{pmatrix} \begin{pmatrix}
  \sqrt{30} & 0 \\
  0 & \sqrt{10}
  \end{pmatrix} \begin{pmatrix}
  \frac{1}{\sqrt{3}} & \frac{1}{\sqrt{3}} & \frac{1}{\sqrt{3}} \\
  \frac{1}{\sqrt{2}} & -\frac{1}{\sqrt{2}} & 0
  \end{pmatrix}
  \]
  Матрица \( B \) ранга 1:
  \[
  B = \sigma_1 \mathbf{u}_1 \mathbf{v}_1^T =
  \sqrt{30} \cdot \frac{2}{\sqrt{5}} \cdot \frac{1}{\sqrt{3}}
   \begin{pmatrix} 1 \\ 0.5 \end{pmatrix} \begin{pmatrix} 1 & 1 & 1 \end{pmatrix}
   = \begin{pmatrix}
  12 & 12 & -6 \\
  -6 & -6 & 3
  \end{pmatrix}
  \]  
  Норма Фробениуса:  
  \[
  \|A - B\| = \sqrt{\sigma_2^2} = \sqrt{10}
  \]

  \item[\textbf{5.2}]
  Вычисляем \( A^T A \):  
  \[
  A^T A = \begin{pmatrix}
  1 & 0 & 0 \\
  0 & 1 & 0 \\
  0 & 0 & 1 \\
  1 & 1 & 1
  \end{pmatrix} \begin{pmatrix}
  1 & 0 & 0 & 1 \\
  0 & 1 & 0 & 1 \\
  0 & 0 & 1 & 1
  \end{pmatrix} = \begin{pmatrix}
  1 & 0 & 0 & 1 \\
  0 & 1 & 0 & 1 \\
  0 & 0 & 1 & 1 \\
  1 & 1 & 1 & 3
  \end{pmatrix}
  \]
  Собственные значения \( A^T A \):  
  \[
  \lambda_1 = 4, \, \lambda_2 = 1, \, \lambda_3 = 1, \, \lambda_4 = 0
  \]  
  Сингулярные значения:  
  \[
  \sigma_1 = 2, \quad \sigma_2 = 1, \quad \sigma_3 = 1
  \]
  Правые сингулярные векторы \( V \):  
  \begin{itemize}
    \item 
    Для \( \lambda_1 = 4 \):  
    \[
    \mathbf{v}_1 = \frac{1}{2} \begin{pmatrix} 1 \\ 1 \\ 1 \\ 1 \end{pmatrix}
    \]  

    \item
    Для \( \lambda_2 = 1 \) и \( \lambda_3 = 1 \): векторы ортогональны 
    \( \mathbf{v}_1 \).
  \end{itemize}

  Левые сингулярные векторы \( U \):  
  \[
  \mathbf{u}_1 = \frac{1}{2} A \mathbf{v}_1 = 
  \begin{pmatrix} 1 \\ 1 \\ 1 \end{pmatrix}, \quad \mathbf{u}_2 =
   \frac{1}{1} A \mathbf{v}_2, \quad \mathbf{u}_3 = \frac{1}{1} A \mathbf{v}_3
  \]
  Усечённое SVD:
  \[
  A = U' \Sigma' (V')^T = \begin{pmatrix}
  1 & 0 & 0 \\
  0 & 1 & 0 \\
  0 & 0 & 1
  \end{pmatrix} \begin{pmatrix}
  2 & 0 & 0 \\
  0 & 1 & 0 \\
  0 & 0 & 1
  \end{pmatrix} \begin{pmatrix}
  \frac{1}{2} & \frac{1}{2} & \frac{1}{2} & \frac{1}{2} \\
  \text{...} & \text{...} & \text{...} & \text{...}
  \end{pmatrix}
  \]
  Матрица \( B \) ранга 1:
  \[
  B = \sigma_1 \mathbf{u}_1 \mathbf{v}_1^T = 2 \cdot
   \begin{pmatrix} 1 \\ 1 \\ 1 \end{pmatrix} \cdot \frac{1}{2}
    \begin{pmatrix} 1 & 1 & 1 & 1 \end{pmatrix} = \begin{pmatrix}
  1 & 1 & 1 & 1 \\
  1 & 1 & 1 & 1 \\
  1 & 1 & 1 & 1
  \end{pmatrix}
  \]  
  Норма Фробениуса:  
  \[
  \|A - B\| = \sqrt{\sigma_2^2 + \sigma_3^2} = \sqrt{1 + 1} = \sqrt{2}
  \]
  \end{enumerate}
\end{enumerate}
\end{document}