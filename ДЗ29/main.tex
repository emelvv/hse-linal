\documentclass[a4paper]{article}
\usepackage{setspace}
\usepackage[utf8]{inputenc}
\usepackage[russian]{babel}
\usepackage[12pt]{extsizes}
\usepackage{mathtools}
\usepackage{graphicx}
\usepackage{fancyhdr}
\usepackage{amssymb}
\usepackage{amsmath, amsfonts, amssymb, amsthm, mathtools}
\usepackage{tikz}

\usetikzlibrary{positioning}
\setstretch{1.3}

\newcommand{\mat}[1]{\begin{pmatrix} #1 \end{pmatrix}}
\newcommand{\vmat}[1]{\begin{vmatrix} #1 \end{vmatrix}}
\renewcommand{\f}[2]{\frac{#1}{#2}}
\newcommand{\dspace}{\space\space}
\newcommand{\s}[2]{\sum\limits_{#1}^{#2}}
\newcommand{\mul}[2]{\prod_{#1}^{#2}}
\newcommand{\sq}[1]{\left[ {#1} \right]}
\newcommand{\gath}[1]{\left[ \begin{array}{@{}l@{}} #1 \end{array} \right.}
\newcommand{\case}[1]{\begin{cases} #1 \end{cases}}
\newcommand{\ts}{\text{\space}}
\newcommand{\lm}[1]{\underset{#1}{\lim}}
\newcommand{\suplm}[1]{\underset{#1}{\overline{\lim}}}
\newcommand{\inflm}[1]{\underset{#1}{\underline{\lim}}}
\newcommand{\Ker}[1]{\operatorname{Ker}}

\renewcommand{\phi}{\varphi}
\newcommand{\lr}{\Leftrightarrow}
\renewcommand{\l}{\left(}
\renewcommand{\r}{\right)}
\newcommand{\rr}{\rightarrow}
\renewcommand{\geq}{\geqslant}
\renewcommand{\leq}{\leqslant}
\newcommand{\RR}{\mathbb{R}}
\newcommand{\CC}{\mathbb{C}}
\newcommand{\QQ}{\mathbb{Q}}
\newcommand{\ZZ}{\mathbb{Z}}
\newcommand{\VV}{\mathbb{V}}
\newcommand{\NN}{\mathbb{N}}
\newcommand{\OO}{\underline{O}}
\newcommand{\oo}{\overline{o}}
\renewcommand{\Ker}{\operatorname{Ker}}
\renewcommand{\Im}{\operatorname{Im}}
\newcommand{\vol}{\text{vol}}
\newcommand{\Vol}{\text{Vol}}

\DeclarePairedDelimiter\abs{\lvert}{\rvert} %
\makeatletter                               % \abs{}
\let\oldabs\abs                             %
\def\abs{\@ifstar{\oldabs}{\oldabs*}}       %

\begin{document}

\section*{Домашнее задание на 05.06 (Линейная алгебра)}
{\large Емельянов Владимир, ПМИ гр №247}\\\\
\begin{enumerate}
  \item[\textbf{№1}]\begin{enumerate}
    \item[1.1]Дано
    \[6x^2 - 5y^2 + 12x - 10y + 31 = 0\]
    Сгруппируем члены с \(x\) и \(y\):  
    \[(6x^2 + 12x) + (-5y^2 - 10y) + 31 = 0\]  
    Вынесем коэффициенты за скобки:  
    \[6(x^2 + 2x) - 5(y^2 + 2y) + 31 = 0\]  
    Дополним до полных квадратов:  
    \[6(x^2 + 2x + 1 - 1) - 5(y^2 + 2y + 1 - 1) + 31 = 0\]  
    \[6[(x + 1)^2 - 1] - 5[(y + 1)^2 - 1] + 31 = 0\]  
    Раскроем скобки:  
    \[6(x + 1)^2 - 6 - 5(y + 1)^2 + 5 + 31 = 0\]  
    \[6(x + 1)^2 - 5(y + 1)^2 + 30 = 0\]  
    Перенесем константу вправо:  
    \[6(x + 1)^2 - 5(y + 1)^2 = -30\]  
    Разделим обе части на \(-30\):  
    \[\frac{6(x + 1)^2}{-30} - \frac{5(y + 1)^2}{-30} = 1\]  
    Упростим:  
    \[-\frac{(x + 1)^2}{5} + \frac{(y + 1)^2}{6} = 1\]  
    \[\frac{(y + 1)^2}{6} - \frac{(x + 1)^2}{5} = 1\]  
    Каноническое уравнение:  
    \[\frac{(y')^2}{6} - \frac{(x')^2}{5} = 1\]  
    где \(x' = x + 1\), \(y' = y + 1\).  
    Выражение старых координат через новые:  
    \[x = x' - 1, \quad y = y' - 1\]  
    Тип кривой: гипербола (так как уравнение имеет вид \(\frac{y^2}{a^2} - \frac{x^2}{b^2} = 1\)).


    \item[1.2]
  \end{enumerate}
\end{enumerate}
\end{document}