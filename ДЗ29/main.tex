\documentclass[a4paper]{article}
\usepackage{setspace}
\usepackage[utf8]{inputenc}
\usepackage[russian]{babel}
\usepackage[12pt]{extsizes}
\usepackage{mathtools}
\usepackage{graphicx}
\usepackage{fancyhdr}
\usepackage{amssymb}
\usepackage{amsmath, amsfonts, amssymb, amsthm, mathtools}
\usepackage{tikz}

\usetikzlibrary{positioning}
\setstretch{1.3}

\newcommand{\mat}[1]{\begin{pmatrix} #1 \end{pmatrix}}
\newcommand{\vmat}[1]{\begin{vmatrix} #1 \end{vmatrix}}
\renewcommand{\f}[2]{\frac{#1}{#2}}
\newcommand{\dspace}{\space\space}
\newcommand{\s}[2]{\sum\limits_{#1}^{#2}}
\newcommand{\mul}[2]{\prod_{#1}^{#2}}
\newcommand{\sq}[1]{\left[ {#1} \right]}
\newcommand{\gath}[1]{\left[ \begin{array}{@{}l@{}} #1 \end{array} \right.}
\newcommand{\case}[1]{\begin{cases} #1 \end{cases}}
\newcommand{\ts}{\text{\space}}
\newcommand{\lm}[1]{\underset{#1}{\lim}}
\newcommand{\suplm}[1]{\underset{#1}{\overline{\lim}}}
\newcommand{\inflm}[1]{\underset{#1}{\underline{\lim}}}
\newcommand{\Ker}[1]{\operatorname{Ker}}

\renewcommand{\phi}{\varphi}
\newcommand{\lr}{\Leftrightarrow}
\renewcommand{\l}{\left(}
\renewcommand{\r}{\right)}
\newcommand{\rr}{\rightarrow}
\renewcommand{\geq}{\geqslant}
\renewcommand{\leq}{\leqslant}
\newcommand{\RR}{\mathbb{R}}
\newcommand{\CC}{\mathbb{C}}
\newcommand{\QQ}{\mathbb{Q}}
\newcommand{\ZZ}{\mathbb{Z}}
\newcommand{\VV}{\mathbb{V}}
\newcommand{\NN}{\mathbb{N}}
\newcommand{\OO}{\underline{O}}
\newcommand{\oo}{\overline{o}}
\renewcommand{\Ker}{\operatorname{Ker}}
\renewcommand{\Im}{\operatorname{Im}}
\newcommand{\vol}{\text{vol}}
\newcommand{\Vol}{\text{Vol}}

\DeclarePairedDelimiter\abs{\lvert}{\rvert} %
\makeatletter                               % \abs{}
\let\oldabs\abs                             %
\def\abs{\@ifstar{\oldabs}{\oldabs*}}       %

\begin{document}

\section*{Домашнее задание на 05.06 (Линейная алгебра)}
{\large Емельянов Владимир, ПМИ гр №247}\\\\
\begin{enumerate}
  \item[\textbf{№1}]\begin{enumerate}
    \item[1.1]Дано
    \[6x^2 - 5y^2 + 12x - 10y + 31 = 0\]
    Сгруппируем члены с \(x\) и \(y\):  
    \[(6x^2 + 12x) + (-5y^2 - 10y) + 31 = 0\]  
    Вынесем коэффициенты за скобки:  
    \[6(x^2 + 2x) - 5(y^2 + 2y) + 31 = 0\]  
    Дополним до полных квадратов:  
    \[6(x^2 + 2x + 1 - 1) - 5(y^2 + 2y + 1 - 1) + 31 = 0\]  
    \[6[(x + 1)^2 - 1] - 5[(y + 1)^2 - 1] + 31 = 0\]  
    Раскроем скобки:  
    \[6(x + 1)^2 - 6 - 5(y + 1)^2 + 5 + 31 = 0\]  
    \[6(x + 1)^2 - 5(y + 1)^2 + 30 = 0\]  
    Перенесем константу вправо:  
    \[6(x + 1)^2 - 5(y + 1)^2 = -30\]  
    Разделим обе части на \(-30\):  
    \[\frac{6(x + 1)^2}{-30} - \frac{5(y + 1)^2}{-30} = 1\]  
    Упростим:  
    \[-\frac{(x + 1)^2}{5} + \frac{(y + 1)^2}{6} = 1\]  
    \[\frac{(y + 1)^2}{6} - \frac{(x + 1)^2}{5} = 1\]  
    Каноническое уравнение:  
    \[\frac{(y')^2}{6} - \frac{(x')^2}{5} = 1\]  
    где \(x' = x + 1\), \(y' = y + 1\).  
    Выражение старых координат через новые:  
    \[x = x' - 1, \quad y = y' - 1\]  
    Тип кривой: гипербола (так как уравнение имеет вид \(\frac{y^2}{a^2} - \frac{x^2}{b^2} = 1\)).\\

    \item[1.2]Дано
    \[x^2 - 2xy + y^2 - 10x - 6y + 25 = 0\]
    Квадратичная форма: \[ x^2 - 2xy + y^2 \] 
    Матрица квадратичной формы:  
    \[
    A = \begin{pmatrix}
    1 & -1 \\
    -1 & 1
    \end{pmatrix}
    \]  
    Характеристическое уравнение:  
    \[
    \det(A - \lambda I) = \begin{vmatrix}
    1 - \lambda & -1 \\
    -1 & 1 - \lambda
    \end{vmatrix} = (1 - \lambda)^2 - 1 = \lambda^2 - 2\lambda = 0
    \]  
    Собственные значения: \( \lambda_1 = 0 \), \( \lambda_2 = 2 \).
    \begin{itemize}
    \item Для \( \lambda_1 = 0 \):  
    \[
    (A - 0I)\mathbf{v} = \begin{pmatrix} 1 & -1 \\ -1 & 1 \end{pmatrix}
      \begin{pmatrix} v_1 \\ v_2 \end{pmatrix} = 0 \implies v_1 - v_2 =
      0 \implies v_1 = v_2
    \]  
    Нормированный собственный вектор:
      \( \mathbf{v}_1 = \left( \frac{1}{\sqrt{2}}, \frac{1}{\sqrt{2}} \right)^\top \)  
    
    \item Для \( \lambda_2 = 2 \):  
    \[
    (A - 2I)\mathbf{v} = \begin{pmatrix} -1 & -1 \\ -1 & -1 \end{pmatrix} \begin{pmatrix} v_1 \\ v_2 \end{pmatrix} = 0 \implies -v_1 - v_2 = 0 \implies v_1 = -v_2.
    \]  
    Нормированный собственный вектор: \( \mathbf{v}_2 = \left( -\frac{1}{\sqrt{2}}, \frac{1}{\sqrt{2}} \right)^\top \).  
    \end{itemize}
    Матрица перехода:  
    \[
    P = \begin{pmatrix}
    \frac{1}{\sqrt{2}} & -\frac{1}{\sqrt{2}} \\
    \frac{1}{\sqrt{2}} & \frac{1}{\sqrt{2}}
    \end{pmatrix}
    \]  
    Подставляем:  
    \[
    \begin{pmatrix} x \\ y \end{pmatrix} = P \begin{pmatrix} x' \\ y' \end{pmatrix} \implies x = \frac{x' - y'}{\sqrt{2}}, \quad y = \frac{x' + y'}{\sqrt{2}}.
    \]  
    Сумма квадратичных членов:  
    \[
    x^2 - 2xy + y^2 = \frac{1}{2}(x'^2 - 2x'y' + y'^2) - (x'^2 - y'^2) + 
    \frac{1}{2}(x'^2 + 2x'y' + y'^2) = 2y'^2
    \]  
    Исходное уравнение после подстановки:  
    \[
    2y'^2 + (-8\sqrt{2}  x' + 2\sqrt{2}  y') + 25 = 0 \implies 2y'^2 - 8\sqrt{2}  x' + 2\sqrt{2}  y' + 25 = 0.
    \]  
    Группируем члены с \( y' \):  
    \[
    2y'^2 + 2\sqrt{2}  y' = 8\sqrt{2}  x' - 25.
    \]  
    Выделяем полный квадрат для \( y' \):  
    \[
    2\left( y'^2 + \sqrt{2}  y' \right) = 8\sqrt{2}  x' - 25.
    \]  
    \[
    y'^2 + \sqrt{2}  y' = \left( y' + \frac{\sqrt{2}}{2} \right)^2 - \left( \frac{\sqrt{2}}{2} \right)^2 = \left( y' + \frac{\sqrt{2}}{2} \right)^2 - \frac{1}{2}.
    \]  
    Подставляем:  
    \[
    2\left[ \left( y' + \frac{\sqrt{2}}{2} \right)^2 - \frac{1}{2} \right] = 8\sqrt{2}  x' - 25,
    \]  
    \[
    2\left( y' + \frac{\sqrt{2}}{2} \right)^2 - 1 = 8\sqrt{2}  x' - 25,
    \]  
    \[
    2\left( y' + \frac{\sqrt{2}}{2} \right)^2 - 8\sqrt{2}  x' = -24.
    \]  
    Переносим константу:  
    \[
    2\left( y' + \frac{\sqrt{2}}{2} \right)^2 = 8\sqrt{2}  x' - 24.
    \]  
    \[
    \left( y' + \frac{\sqrt{2}}{2} \right)^2 = 4\sqrt{2} \left( x' - \frac{24}{8\sqrt{2}} \right) = 4\sqrt{2} \left( x' - \frac{3}{\sqrt{2}} \right) = 4\sqrt{2} \left( x' - \frac{3\sqrt{2}}{2} \right).
    \]  
    Из формул поворота:  
    \[
    x' = \frac{x + y}{\sqrt{2}}, \quad y' = -\frac{x - y}{\sqrt{2}}.
    \]  
    Выражаем через \( x'' \), \( y'' \):  
    \[
    x' = x'' + \frac{3\sqrt{2}}{2}, \quad y' = y'' - \frac{\sqrt{2}}{2}.
    \]  
    Подставляем в формулы поворота:  
    \[
    x = \frac{x' - y'}{\sqrt{2}} = \frac{ \left( x'' + \frac{3\sqrt{2}}{2} \right) - \left( y'' - \frac{\sqrt{2}}{2} \right) }{\sqrt{2}} = \frac{x'' - y'' + 2\sqrt{2}}{\sqrt{2}} = \frac{x'' - y''}{\sqrt{2}} + 2,
    \]  
    \[
    y = \frac{x' + y'}{\sqrt{2}} = \frac{ \left( x'' + \frac{3\sqrt{2}}{2} \right) + \left( y'' - \frac{\sqrt{2}}{2} \right) }{\sqrt{2}} = \frac{x'' + y'' + \sqrt{2}}{\sqrt{2}} = \frac{x'' + y''}{\sqrt{2}} + 1.
    \]  
    Таким образом, старые координаты выражаются через новые (\( x'' \), \( y'' \)) как:  
    \[
    x = \frac{x'' - y''}{\sqrt{2}} + 2, \quad y = \frac{x'' + y''}{\sqrt{2}} + 1.
    \]  
    переобозначим новые координаты: \( x' = x'' \), \( y' = y'' \).  

    Каноническое уравнение:  
    \[
    (y')^2 = 4\sqrt{2}  x'.
    \]  
    Тип кривой: парабола (уравнение вида \( y^2 = 4px \) с \( p = \sqrt{2} \)).

    Новая система координат: старые координаты выражаются через новые (\( x', y' \)) как:  
    \[
    x = \frac{x' - y'}{\sqrt{2}} + 2, \quad y = \frac{x' + y'}{\sqrt{2}} + 1.
    \]  
  \end{enumerate}

  \item[\textbf{№2}]\begin{enumerate}
    \item[2.1]Дано 
    \[x^2 + 4y^2 + 9z^2 - 6x + 8y - 36z = 0\]
    Сгруппируем члены с \(x\), \(y\), \(z\):  
    \[(x^2 - 6x) + (4y^2 + 8y) + (9z^2 - 36z) = 0\] 
    Выделяем полные квадраты:  
    \[(x - 3)^2 - 9 + 4(y + 1)^2 - 4 + 9(z - 2)^2 - 36 = 0\]  
    \[(x - 3)^2 + 4(y + 1)^2 + 9(z - 2)^2 = 49\]  
    Разделим на 49:  
    \[\frac{(x - 3)^2}{49} + \frac{4(y + 1)^2}{49} + \frac{9(z - 2)^2}{49} = 1\]  
    Упростим:  
    \[\frac{(x - 3)^2}{7^2} + \frac{(y + 1)^2}{\left(\frac{7}{2}\right)^2} + \frac{(z - 2)^2}{\left(\frac{7}{3}\right)^2} = 1\]  
    Новые координаты:  
    \[x' = x - 3, \quad y' = y + 1, \quad z' = z - 2\]  
    Каноническое уравнение:  
    \[\frac{(x')^2}{49} + \frac{(y')^2}{\frac{49}{4}} + \frac{(z')^2}{\frac{49}{9}} = 1\]  
    Тип поверхности: эллипсоид\\

    \item[2.2]Дано 
    \[2xy + 2x + 2y + 2z - 1 = 0\]
    Квадратичная форма: \(2xy\) 

    Матрица квадратичной формы:  
    \[
    A = \begin{pmatrix}
    0 & 1 & 0 \\
    1 & 0 & 0 \\
    0 & 0 & 0
    \end{pmatrix}
    \]  
    Характеристическое уравнение:  
    \[
    \det(A - \lambda I) = \begin{vmatrix}
    -\lambda & 1 & 0 \\
    1 & -\lambda & 0 \\
    0 & 0 & -\lambda
    \end{vmatrix} = -\lambda(\lambda^2 - 1) = 0
    \]  
    Собственные значения: \(\lambda_1 = 1\), \(\lambda_2 = -1\), \(\lambda_3 = 0\).  
    \begin{itemize}
      \item Для \(\lambda_1 = 1\):  
      \[
      (A - I)\mathbf{v} = \begin{pmatrix}
      -1 & 1 & 0 \\
      1 & -1 & 0 \\
      0 & 0 & -1
      \end{pmatrix} \mathbf{v} = 0 \implies \mathbf{v}_1 = \frac{1}{\sqrt{2}} (1, 1, 0)^\top
      \]  

      \item Для \(\lambda_2 = -1\):  
      \[
      (A + I)\mathbf{v} = \begin{pmatrix}
      1 & 1 & 0 \\
      1 & 1 & 0 \\
      0 & 0 & 1
      \end{pmatrix} \mathbf{v} = 0 \implies \mathbf{v}_2 = \frac{1}{\sqrt{2}} (-1, 1, 0)^\top
      \]  

      \item  Для \(\lambda_3 = 0\):  
      \[
      A\mathbf{v} = 0 \implies \mathbf{v}_3 = (0, 0, 1)^\top
      \]  
    \end{itemize}
    Матрица перехода:  
    \[
    P = \begin{pmatrix}
    \frac{1}{\sqrt{2}} & -\frac{1}{\sqrt{2}} & 0 \\
    \frac{1}{\sqrt{2}} & \frac{1}{\sqrt{2}} & 0 \\
    0 & 0 & 1
    \end{pmatrix}
    \]  
    Формулы поворота:  
    \[
    x = \frac{x' - y'}{\sqrt{2}}, \quad y = \frac{x' + y'}{\sqrt{2}}, \quad z = z'
    \]  
    Подставляем:  
    \[
    2 \left( \frac{x' - y'}{\sqrt{2}} \right) \left( \frac{x' + y'}{\sqrt{2}} \right) + 2 \left( \frac{x' - y'}{\sqrt{2}} \right) + 2 \left( \frac{x' + y'}{\sqrt{2}} \right) + 2z' - 1 = 0
    \]  
    Упрощаем:  
    \[
    2 \cdot \frac{(x')^2 - (y')^2}{2} + \sqrt{2} (x' - y') + \sqrt{2} (x' + y') + 2z' - 1 = 0
    \]  
    \[
    (x')^2 - (y')^2 + 2\sqrt{2} x' + 2z' - 1 = 0
    \]  
    Группируем члены с \(x'\):  
    \[
    (x')^2 + 2\sqrt{2} x' = (x' + \sqrt{2})^2 - 2
    \]  
    Подставляем:  
    \[
    (x' + \sqrt{2})^2 - 2 - (y')^2 + 2z' - 1 = 0
    \]  
    \[
    (x' + \sqrt{2})^2 - (y')^2 + 2z' = 3
    \]  
    Вводим новые переменные:  
    \[
    x'' = x' + \sqrt{2}, \quad y'' = y', \quad z'' = z'
    \]  
    Уравнение:  
    \[
    (x'')^2 - (y'')^2 + 2z'' = 3
    \]  
    Переносим константу:  
    \[
    (x'')^2 - (y'')^2 = -2z'' + 3
    \]  
    Чтобы убрать свободный член, сдвигаем \(z''\):  
    \[
    z''' = z'' - \frac{3}{2}
    \]  
    Тогда:  
    \[
    (x'')^2 - (y'')^2 = -2 \left( z''' + \frac{3}{2} \right) + 3 = -2z'''
    \]  
    Каноническое уравнение:  
    \[
    (x'')^2 - (y'')^2 = -2z'''
    \]  
    Из поворота:  
    \[
    x' = \frac{x + y}{\sqrt{2}}, \quad y' = \frac{-x + y}{\sqrt{2}}, \quad z' = z
    \]  
    Из сдвигов:  
    \[
    x'' = x' + \sqrt{2} = \frac{x + y}{\sqrt{2}} + \sqrt{2}, \quad y'' = y' = \frac{-x + y}{\sqrt{2}}, \quad z''' = z' - \frac{3}{2} = z - \frac{3}{2}
    \]  
    Подставляем:  
    \[
    x'' = \frac{x + y + 2}{\sqrt{2}}, \quad y'' = \frac{-x + y}{\sqrt{2}}, \quad z''' = z - \frac{3}{2}
    \]  
    Выражаем старые координаты через новые (\(x'', y'', z'''\)):  
    \[
    x = \frac{\sqrt{2}}{2} (x'' - y'') - 1, \quad y = \frac{\sqrt{2}}{2} (x'' + y'') - 1, \quad z = z''' + \frac{3}{2}
    \]  
    Каноническое уравнение:  
    \[
    (x'')^2 - (y'')^2 = -2z'''
    \]  
    Тип поверхности: гиперболический параболоид.\\

    \item[2.3]Дано 
    \[x^2 = 12y + z - 39\]
    Линейная часть \(-12y - z\) соответствует направлению вектора \(\mathbf{n} = (0, -12, -1)\). Длина вектора:  
    \[ \|\mathbf{n}\| = \sqrt{0^2 + (-12)^2 + (-1)^2} = \sqrt{145} \]  
    Введем ортогональную замену координат, где ось \(z'\) направлена вдоль \(\mathbf{n}\), а ось \(y'\) — вдоль ортогонального вектора \(\mathbf{m} = (0, 1, -12)\) (так как \(\mathbf{n} \cdot \mathbf{m} = 0 \cdot 0 + (-12) \cdot 1 + (-1) \cdot (-12) = 0\)). Длина \(\mathbf{m}\):  
    \[ \|\mathbf{m}\| = \sqrt{0^2 + 1^2 + (-12)^2} = \sqrt{145} \]  
    Единичные векторы:  
    \[ \mathbf{e}_{y'} = \left(0, \frac{1}{\sqrt{145}}, -\frac{12}{\sqrt{145}}\right),
     \quad \mathbf{e}_{z'} = \left(0, -\frac{12}{\sqrt{145}},
      -\frac{1}{\sqrt{145}}\right) \]  
    Формулы преобразования:  
    \[ x = x', \quad y = \frac{y'}{\sqrt{145}} - \frac{12z'}{\sqrt{145}}, \quad z = -\frac{12y'}{\sqrt{145}} - \frac{z'}{\sqrt{145}} \]  
    Подставим в уравнение:  
    \[ (x')^2 - 12 \left( \frac{y'}{\sqrt{145}} - \frac{12z'}{\sqrt{145}} \right) - \left( -\frac{12y'}{\sqrt{145}} - \frac{z'}{\sqrt{145}} \right) + 39 = 0 \]  
    Упростим линейную часть:  
    \[ -12y - z = \sqrt{145}  z' \]  
    Уравнение принимает вид:  
    \[ (x')^2 + \sqrt{145}  z' + 39 = 0 \]  
    Введем сдвиг: \( z'' = z' + \frac{39}{\sqrt{145}} \), тогда:  
    \[ (x')^2 + \sqrt{145} \left( z'' - \frac{39}{\sqrt{145}} \right) + 39 = 0 \implies (x')^2 + \sqrt{145}  z'' = 0 \]  
    Каноническое уравнение:  
    \[ (x')^2 = -\sqrt{145}  z'' \]  
    Из замены:  
    \[ z' = z'' - \frac{39}{\sqrt{145}} \]  
    Тогда:  
    \[ x = x' \]  
    \[ y = \frac{y'}{\sqrt{145}} - \frac{12}{\sqrt{145}} \left( z'' - \frac{39}{\sqrt{145}} \right) = \frac{y'}{\sqrt{145}} - \frac{12z''}{\sqrt{145}} + \frac{468}{145} \]  
    \[ z = -\frac{12y'}{\sqrt{145}} - \frac{1}{\sqrt{145}} \left( z'' - \frac{39}{\sqrt{145}} \right) = -\frac{12y'}{\sqrt{145}} - \frac{z''}{\sqrt{145}} + \frac{39}{145} \]  
    Каноническое уравнение \((x')^2 = -\sqrt{145}  z''\) описывает параболический 
    цилиндр. \\

    \item[2.4]Дано 
    \[3x^2 + 4xy - 4xz + 4y^2 + 2z^2 - 3y - 6z - 3 = 0 \]
    Квадратичная форма: \( 3x^2 + 4xy - 4xz + 4y^2 + 2z^2 \).  

    Матрица квадратичной формы:  
    \[
    A = \begin{pmatrix}
    3 & 2 & -2 \\
    2 & 4 & 0 \\
    -2 & 0 & 2
    \end{pmatrix}
    \]  
    Характеристическое уравнение:  
    \[
    \det(A - \lambda I) = \begin{vmatrix}
    3-\lambda & 2 & -2 \\
    2 & 4-\lambda & 0 \\
    -2 & 0 & 2-\lambda
    \end{vmatrix} = -\lambda(\lambda - 3)(\lambda - 6) = 0
    \]  
    Собственные значения: \(\lambda_1 = 0\), \(\lambda_2 = 3\), \(\lambda_3 = 6\).  
    \begin{itemize}
      \item Собственный вектор для \(\lambda_1 = 0\):  
      \[
      (A - 0I)\mathbf{v} = \begin{pmatrix}
      3 & 2 & -2 \\
      2 & 4 & 0 \\
      -2 & 0 & 2
      \end{pmatrix} \mathbf{v} = 0 \implies \mathbf{v}_1 = \frac{1}{3} \begin{pmatrix} -2 \\ 1 \\ -2 \end{pmatrix}
      \]  
      
      \item Собственный вектор для \(\lambda_2 = 3\):  
      \[
      (A - 3I)\mathbf{v} = \begin{pmatrix}
      0 & 2 & -2 \\
      2 & 1 & 0 \\
      -2 & 0 & -1
      \end{pmatrix} \mathbf{v} = 0 \implies \mathbf{v}_2 = \frac{1}{3} \begin{pmatrix} 1 \\ -2 \\ -2 \end{pmatrix}
      \]  

      \item Собственный вектор для \(\lambda_3 = 6\):  
      \[
      (A - 6I)\mathbf{v} = \begin{pmatrix}
      -3 & 2 & -2 \\
      2 & -2 & 0 \\
      -2 & 0 & -4
      \end{pmatrix} \mathbf{v} = 0 \implies \mathbf{v}_3 = \frac{1}{3} \begin{pmatrix} -2 \\ -2 \\ 1 \end{pmatrix}
      \]  
    \end{itemize}

    Матрица перехода (ортогональная):  
    \[
    P = \frac{1}{3} \begin{pmatrix}
    -2 & 1 & -2 \\
    1 & -2 & -2 \\
    -2 & -2 & 1
    \end{pmatrix}
    \]  
    Формулы поворота:  
    \[
    x = \frac{1}{3} (-2x' + y' - 2z'), \quad y = \frac{1}{3} (x' - 2y' - 2z'), \quad z = \frac{1}{3} (-2x' - 2y' + z')
    \]  
    После подстановки квадратичная часть становится:  
    \[ 0 \cdot (x')^2 + 3 \cdot (y')^2 + 6 \cdot (z')^2 \]  
    Линейная часть: \(-3y - 6z - 3\). Подставляем:  
    \[
    y = \frac{1}{3} (x' - 2y' - 2z'), \quad z = \frac{1}{3} (-2x' - 2y' + z')
    \]  
    \[
    -3y = - (x' - 2y' - 2z'), \quad -6z = -2 (-2x' - 2y' + z') = 4x' + 4y' - 2z'
    \]  
    \[
    -3y - 6z = 3x' + 6y'
    \]  
    Уравнение:  
    \[ 3(y')^2 + 6(z')^2 + 3x' + 6y' - 3 = 0 \]  
    Делим на 3:  
    \[ (y')^2 + 2(z')^2 + x' + 2y' - 1 = 0 \] 
    Выделяем полные квадраты
    \[ (y')^2 + 2y' = (y' + 1)^2 - 1 \]  
    Подставляем:  
    \[ (y' + 1)^2 - 1 + 2(z')^2 + x' - 1 = 0 \implies (y' + 1)^2 + 2(z')^2 + x' = 2 \]  
    Вводим сдвиг: \( x'' = x' \), \( y'' = y' + 1 \), \( z'' = z' \):  
    \[ (y'')^2 + 2(z'')^2 + x'' = 2 \implies x'' = 2 - (y'')^2 - 2(z'')^2 \]  
    Пусть
    \[ x' = x'', \quad y' = y'' - 1, \quad z' = z'' \]  
    Подставляем в формулы поворота:  
    \[
    x = \frac{1}{3} (-2x'' + (y'' - 1) - 2z'') = \frac{1}{3} (-2x'' + y'' - 1 - 2z'')
    \]  
    \[
    y = \frac{1}{3} (x'' - 2(y'' - 1) - 2z'') = \frac{1}{3} (x'' - 2y'' + 2 - 2z'')
    \]  
    \[
    z = \frac{1}{3} (-2x'' - 2(y'' - 1) + z'') = \frac{1}{3} (-2x'' - 2y'' + 2 + z'')
    \]  
    Каноническое уравнение \( x'' = 2 - (y'')^2 - 2(z'')^2 \) описывает эллиптический параболоид.  \\

  \end{enumerate}


  \item[\textbf{№3}]Для того чтобы уравнение 
  \[3y^2 + 2z^2 - 4a x z + 8x - 6y - 9 = 0\]
  определяло эллиптический параболоид, необходимо, чтобы квадратичная форма, 
  соответствующая уравнению, имела ранг 2 (то есть одно нулевое собственное значение),
  а линейная часть по переменной, соответствующей нулевому собственному значению, 
  была ненулевой. 
  
  Эллиптический параболоид требует, чтобы оставшаяся квадратичная
  форма по двум переменным была положительно определенной.

  Матрица квадратичной формы для уравнения:
  \[
  A = \begin{pmatrix}
  0 & 0 & -2a \\
  0 & 3 & 0 \\
  -2a & 0 & 2
  \end{pmatrix}
  \]
  Характеристическое уравнение:
  \[
  \det(A - \lambda I) = \det \begin{pmatrix}
  -\lambda & 0 & -2a \\
  0 & 3-\lambda & 0 \\
  -2a & 0 & 2-\lambda
  \end{pmatrix} = \]
  \[=(3 - \lambda) \left[ (-\lambda)(2 - \lambda) -
   (-2a)(-2a) \right] = (3 - \lambda)(\lambda^2 - 2\lambda - 4a^2) = 0.
  \]
  Корни:
  \begin{itemize}
    \item \(\lambda_1 = 3\)
    \item \(\lambda_{2,3} = 1 \pm \sqrt{1 + 4a^2}\)
  \end{itemize}

  Эллиптический параболоид требует одного нулевого собственного значения. Проверим возможные случаи:
  \begin{enumerate}
  \item \(\lambda_1 = 3 = 0\) — невозможно.
  \item \(\lambda_2 = 1 + \sqrt{1 + 4a^2} = 0\) — невозможно, так как
   \[1 + \sqrt{1 + 4a^2} \geq 1 > 0\]
  \item \(\lambda_3 = 1 - \sqrt{1 + 4a^2} = 0\):
    \[
    1 - \sqrt{1 + 4a^2} = 0 \implies \sqrt{1 + 4a^2} = 1 \implies \]
    \[
    \implies 1 + 4a^2 = 1 \implies 4a^2 = 0 \implies a = 0
    \]
  \end{enumerate}

  При \(a = 0\) уравнение принимает вид:
  \[
  3y^2 + 2z^2 + 8x - 6y - 9 = 0.
  \]
  Выделим полные квадраты:
  \begin{itemize} 
    \item По \(y\): \[3y^2 - 6y = 3(y^2 - 2y) = 3((y - 1)^2 - 1) = 3(y - 1)^2 - 3\]
    \item По \(z\): 
    
    линейных членов нет, оставляем \(2z^2\).
  \end{itemize}
  Подставляем:
  \[
  3(y - 1)^2 - 3 + 2z^2 + 8x - 9 = 0 \implies 3(y - 1)^2 + 2z^2 + 8x = 12
  \]
  Разрешим относительно \(x\):
  \[
  8x = -3(y - 1)^2 - 2z^2 + 12 \implies x = -\frac{3}{8}(y - 1)^2 -
   \frac{1}{4}z^2 + \frac{3}{2}
  \]
  Это уравнение эллиптического параболоида, так как:
  \begin{itemize}
    \item Квадратичная форма по \(y\) и \(z\): 
    \[-\frac{3}{8}(y - 1)^2 - \frac{1}{4}z^2\] 
    имеет отрицательные коэффициенты, что соответствует параболоиду, открытому в
     направлении отрицательной оси \(x\).
    \item После замены \(x' = -x\) уравнение примет стандартный вид
     эллиптического параболоида.
    \item Линейный член по \(x\) ненулевой.
  \end{itemize}

  \textbf{Ответ: } $a = 0$

\end{enumerate}
\end{document}