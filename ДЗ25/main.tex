\documentclass[a4paper]{article}
\usepackage{setspace}
\usepackage[T2A]{fontenc} %
\usepackage[utf8]{inputenc} % подключение русского языка
\usepackage[russian]{babel} %
\usepackage[12pt]{extsizes}
\usepackage{mathtools}
\usepackage{graphicx}
\usepackage{fancyhdr}
\usepackage{amssymb}
\usepackage{amsmath, amsfonts, amssymb, amsthm, mathtools}
\usepackage{tikz}

\usetikzlibrary{positioning}
\setstretch{1.3}

\newcommand{\mat}[1]{\begin{pmatrix} #1 \end{pmatrix}}
\newcommand{\vmat}[1]{\begin{vmatrix} #1 \end{vmatrix}}
\renewcommand{\f}[2]{\frac{#1}{#2}}
\newcommand{\dspace}{\space\space}
\newcommand{\s}[2]{\sum\limits_{#1}^{#2}}
\newcommand{\mul}[2]{\prod_{#1}^{#2}}
\newcommand{\sq}[1]{\left[ {#1} \right]}
\newcommand{\gath}[1]{\left[ \begin{array}{@{}l@{}} #1 \end{array} \right.}
\newcommand{\case}[1]{\begin{cases} #1 \end{cases}}
\newcommand{\ts}{\text{\space}}
\newcommand{\lm}[1]{\underset{#1}{\lim}}
\newcommand{\suplm}[1]{\underset{#1}{\overline{\lim}}}
\newcommand{\inflm}[1]{\underset{#1}{\underline{\lim}}}
\newcommand{\Ker}[1]{\operatorname{Ker}}

\renewcommand{\phi}{\varphi}
\newcommand{\lr}{\Leftrightarrow}
\renewcommand{\l}{\left(}
\renewcommand{\r}{\right)}
\newcommand{\rr}{\rightarrow}
\renewcommand{\geq}{\geqslant}
\renewcommand{\leq}{\leqslant}
\newcommand{\RR}{\mathbb{R}}
\newcommand{\CC}{\mathbb{C}}
\newcommand{\QQ}{\mathbb{Q}}
\newcommand{\ZZ}{\mathbb{Z}}
\newcommand{\VV}{\mathbb{V}}
\newcommand{\NN}{\mathbb{N}}
\newcommand{\OO}{\underline{O}}
\newcommand{\oo}{\overline{o}}
\renewcommand{\Ker}{\operatorname{Ker}}
\renewcommand{\Im}{\operatorname{Im}}
\newcommand{\vol}{\text{vol}}
\newcommand{\Vol}{\text{Vol}}

\DeclarePairedDelimiter\abs{\lvert}{\rvert} %
\makeatletter                               % \abs{}
\let\oldabs\abs                             %
\def\abs{\@ifstar{\oldabs}{\oldabs*}}       %

\begin{document}

\section*{Домашнее задание на 17.04 (Линейная алгебра)}
 {\large Емельянов Владимир, ПМИ гр №247}\\\\
\begin{enumerate}
    \item[\textbf{№1}]Пусть есть трёхгранный угол $ABCD$ из вершины $A$, т.е. есть три угла: 
    $$\angle ABC, \angle ACD, \angle BAD$$
    Пусть $\vec{b}, \vec{c}, \vec{d}$ - единичные векторы вдоль сторон AB, AC и AD соответственно.
    Тогда направляющие векторы биссектрис углов $\angle ABC, \angle ACD$, это $\vec{b}+ \vec{c}$ и $ \vec{c}+\vec{d}$.
    А направляющий вектор биссектрисы угла, смежного к $\angle BAD$ это $-\vec{b}+\vec{d}$.
    $$\vec{l_1} = \vec{b}+ \vec{c}, \quad \vec{l_2} = \vec{c}+\vec{d}, \quad \vec{l_3} = -\vec{b}+\vec{d}$$
    Сложим первый и третий векторы:
    $$\vec{l_1} + \vec{l_3 } =  \vec{c}+\vec{d} = \vec{l_2}$$
    Следовательно, так как сумма векторов и сами векторы точно лежат в одной плоскости:
     \textbf{$\vec{l_1}, \vec{l_2}, \vec{l_3}$ - компланарны}\\


    \item[\textbf{№2}]Запишем прямую
    $$\case{
        x + y - 3z = -9\\
        3x+y-z=11
    }$$
    В параметрическом виде:
    $$\case{
        x = 10 - t\\
        y = -19 + 4t\\
        z = t
    } \quad t\in \RR$$
    Пусть $M(2, 3, 1)$. Найдём $t$ при котором $MP \perp a$, где $P$ - точка на прямой, а $a$ - направляющий вектор:
    $$MP = P-M = \mat{8-t\\-22+4t\\t-1}, \quad a = \mat{-1\\4\\1}$$
    $$P = \mat{10-t\\-19+4t\\t}, \quad (a, MP) = 0 \lr t = \f{97}{18}$$
    Значит:
    $$P = \begin{pmatrix} \frac{83}{18}\\ \frac{23}{9}\\ \frac{97}{18} \end{pmatrix}, \quad MP = \mat{\frac{47}{18}\\ \frac{-4}{9}\\ \frac{79}{18}}$$
    Значит точка $M'$ симметричная $M$ относительно прямой имеет координаты:
    $$P + MP = \begin{pmatrix} \frac{65}{9} \\ \frac{19}{9} \\ \frac{88}{9} \end{pmatrix}$$   

    \textbf{Ответ: } $(\frac{65}{9}, \frac{19}{9}, \frac{88}{9})^T$\\

    \item[\textbf{№3}]Найдём направляющий вектор прямой $l_1$:
    $$a_1 = [(1 ,3, 2), (0, 1, 1)] = (1, -1, 1)$$
    Найдём направляющий вектор прямой $l_2$:
    $$a_2 = [(3, 0, 1), (1, -1, 0)] = (1, 1, -3)$$
    Точка на $l_2$ имеет вид:
    $$ M = \mat{x\\ y\\ z} = \begin{pmatrix} 0 \\ 2 \\ 1 \end{pmatrix} + t \begin{pmatrix} 1 \\ 1 \\ -3 \end{pmatrix}$$
    $$PM = M - P = \begin{pmatrix} t \\ t + 2 \\ 1 - 3t \end{pmatrix} - \mat{3\\3\\2} = \begin{pmatrix} t - 3 \\ t - 1 \\ -1 - 3t \end{pmatrix}$$
    $PM$ - направляющий вектор прямой $l$. Он должен быть перпендикулярен $a_1$, то есть:
    $$\l \mat{1\\-1\\1},  \begin{pmatrix} t - 3 \\ t - 1 \\ -1 - 3t \end{pmatrix}\r = 0 \lr t=-1 \implies$$
    $$\implies PM = \mat{-4\\-2\\2}$$
    Следовательно,
    $$a = \mat{2\\1\\-1} \text{--- направляющий вектор $l$}$$
    Тогда прямая \( l \) в параметрическом виде:  
    \[
    \begin{cases}
    x = 3 + 2t, \\
    y = 3 + t, \\
    z = 2 - t.
    \end{cases}
    \]
    Возьмём точку \( Q(3, 0, 4) \) на \( l_1 \). Вектор \( PQ \):
    \[
    PQ = (0, -3, 2).
    \]
    Тогда расстояние между прямыми:
    \[
    \rho(l, l_1) = \frac{\left| (\left( [a, a_1] \right) , PQ) \right|}{\left| [\vec{a} , \vec{a_1}] \right|}
    \]
    Векторное произведение направляющих векторов:
    \[
    [\vec{a}, \vec{a_1} ]= (0, -3, -3).
    \]   
    Смешанное произведение:
    \[
    (([\vec{a}, \vec{a_1})], PQ) = ((0, -3, -3), (0, -3, 2) )= 3
    \]
    Расстояние между прямыми:
    \[
    \rho = \frac{|3|}{|[\vec{a}, \vec{a_1}]|} = \frac{3}{3\sqrt{2}} = \frac{\sqrt{2}}{2}
    \]
    \textbf{Ответ: } $\frac{\sqrt{2}}{2}$



\end{enumerate}
\end{document}