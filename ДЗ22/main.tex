\documentclass[a4paper]{article}
\usepackage{setspace}
\usepackage[T2A]{fontenc} %
\usepackage[utf8]{inputenc} % подключение русского языка
\usepackage[russian]{babel} %
\usepackage[12pt]{extsizes}
\usepackage{mathtools}
\usepackage{graphicx}
\usepackage{fancyhdr}
\usepackage{amssymb}
\usepackage{amsmath, amsfonts, amssymb, amsthm, mathtools}
\usepackage{tikz}

\usetikzlibrary{positioning}
\setstretch{1.3}

\newcommand{\mat}[1]{\begin{pmatrix} #1 \end{pmatrix}}
\newcommand{\vmat}[1]{\begin{vmatrix} #1 \end{vmatrix}}
\renewcommand{\f}[2]{\frac{#1}{#2}}
\newcommand{\dspace}{\space\space}
\newcommand{\s}[2]{\sum\limits_{#1}^{#2}}
\newcommand{\mul}[2]{\prod_{#1}^{#2}}
\newcommand{\sq}[1]{\left[ {#1} \right]}
\newcommand{\gath}[1]{\left[ \begin{array}{@{}l@{}} #1 \end{array} \right.}
\newcommand{\case}[1]{\begin{cases} #1 \end{cases}}
\newcommand{\ts}{\text{\space}}
\newcommand{\lm}[1]{\underset{#1}{\lim}}
\newcommand{\suplm}[1]{\underset{#1}{\overline{\lim}}}
\newcommand{\inflm}[1]{\underset{#1}{\underline{\lim}}}
\newcommand{\Ker}[1]{\operatorname{Ker}}

\renewcommand{\phi}{\varphi}
\newcommand{\lr}{\Leftrightarrow}
\renewcommand{\r}{\Rightarrow}
\newcommand{\rr}{\rightarrow}
\renewcommand{\geq}{\geqslant}
\renewcommand{\leq}{\leqslant}
\newcommand{\RR}{\mathbb{R}}
\newcommand{\CC}{\mathbb{C}}
\newcommand{\QQ}{\mathbb{Q}}
\newcommand{\ZZ}{\mathbb{Z}}
\newcommand{\VV}{\mathbb{V}}
\newcommand{\NN}{\mathbb{N}}
\newcommand{\OO}{\underline{O}}
\newcommand{\oo}{\overline{o}}
\renewcommand{\Ker}{\operatorname{Ker}}
\renewcommand{\Im}{\operatorname{Im}}


\DeclarePairedDelimiter\abs{\lvert}{\rvert} %
\makeatletter                               % \abs{}
\let\oldabs\abs                             %
\def\abs{\@ifstar{\oldabs}{\oldabs*}}       %

\begin{document}

\section*{Домашнее задание на 19.03 (Линейная алгебра)}
 {\large Емельянов Владимир, ПМИ гр №247}\\\\
\begin{enumerate}
    \item[\textbf{№1}]Векторы 
    $$(1, 2, 3)^T, (4, 5, 6)^T, (7, 8, 9)^T$$
    Ортогональны пространству $U$.
     Найдём среди них ЛНЗ векторы - это и будет базис $U^\perp$:
     $$\mat{1 & 4 & 7\\2 & 5 & 8 \\ 3 & 6 & 9} \implies \text{УСВ: } 
     \mat{1 & 0 & -1 \\ 0 & 1 & 2 \\ 0 & 0 & 0}$$
    Следовательно:
    $$(1, 2, 3)^T, (4, 5, 6)^T \text{ --- базис } U^\perp$$
    \textbf{Ответ: } $(1, 2, 3)^T, (4, 5, 6)^T$\\

    \item[\textbf{№2}]Пусть:
    $$v = (5, 6, 7, 8)^T$$
    Найдём базис $U^\perp$, это будет:
    $$e = (1, 2, 3, 4)^T$$
    Следовательно $$pr_{U^\perp}=ort_U = \f{(v, e)}{(e, e)}e = 
    \f{7}{3}\left( 1, 2, 3, 4\right)^T$$
    Значит
    $$pr_U = v-ort_U =\left( \f{8}{3}, \f{4}{3}, 0, \f{28}{3}\right)^T$$
    Найдём расстояние от $v$ до $U$:
    $$\rho(v, U) = |ort_U| = \f{7\sqrt{30}}{3}$$

    \textbf{Ответ:} $\f{7}{3}\left( 1, 2, 3, 4\right)^T, \left( \f{8}{3}, \f{4}{3}, 0, \f{28}{3}\right)^T$ и $ \f{7\sqrt{30}}{3}$

    \item[\textbf{№3}]$$A_1 = \begin{pmatrix}
        1/2 & 1/3 \\
        -1/3 & 1/2
        \end{pmatrix}\implies A_1 \cdot A_1^T = \begin{pmatrix}
        \frac{13}{36} & 0 \\
        0 & \frac{13}{36}
        \end{pmatrix}$$

    $$A_2 = \begin{pmatrix}
        -1/\sqrt{3} & 1/\sqrt{3} & -1/\sqrt{3} \\
        -1/\sqrt{3} & -1/\sqrt{3} & 1/\sqrt{3} \\
        -1/\sqrt{3} & 1/\sqrt{3} & -1/\sqrt{3}
        \end{pmatrix}\implies A_2 \cdot A_2^T = \begin{pmatrix}
        1 & -\frac{1}{3} & 1 \\
        -\frac{1}{3} & 1 & -\frac{1}{3} \\
        1 & -\frac{1}{3} & 1
        \end{pmatrix}$$

    $$A_3 = \begin{pmatrix}
        1/2 & 1/2 & 1/2 & 1/2 \\
        1/2 & 1/2 & -1/2 & -1/2 \\
        1/2 & -1/2 & 1/2 & -1/2 \\
        1/2 & -1/2 & -1/2 & 1/2
        \end{pmatrix}\implies A_3 \cdot A_3^T = \begin{pmatrix}
        1 & 0 & 0 & 0 \\
        0 & 1 & 0 & 0 \\
        0 & 0 & 1 & 0 \\
        0 & 0 & 0 & 1
        \end{pmatrix}$$
    \textbf{Ответ:} Только $A_3$

    \item[\textbf{№4}]
    $$A = \mat{1 & 2\\ 0 & 1 \\ 1 & 0}$$
    Пусть 
    $$e_1 = (1, 0, 1)^T, \quad e_2 = (2, 1, 0)^T$$
    Найдём ортонормированный базис:
    $$f_1 = (1, 0, 1)^T$$
    $$f_2 = e_2 - \f{(e_2, f_1)}{(f_1, f_1)}f_1 = (1, 1, -1)^T$$
    $$\f{f_1}{|f_1|} = (1/\sqrt{2}, 0, 1/\sqrt{2}) = q_1, \quad 
    \f{f_2}{|f_2|} = (1/\sqrt{3}, 1/\sqrt{3}, -1/\sqrt{3}) = q_2$$
    Следовательно:
    $$Q = \mat{
        \f{1}{\sqrt{2}} & \f{1}{\sqrt{3}}\\
        0 & \f{1}{\sqrt{3}}\\
        \f{1}{\sqrt{2}} & -\f{1}{\sqrt{3}}
    }$$
    $$R = Q^T\cdot A = \mat{\sqrt{2} & \sqrt{2} \\ 0 & \sqrt{3}}$$
    $$ \mat{
        \f{1}{\sqrt{2}} & \f{1}{\sqrt{3}}\\
        0 & \f{1}{\sqrt{3}}\\
        \f{1}{\sqrt{2}} & -\f{1}{\sqrt{3}}
    } \cdot \mat{\sqrt{2} & \sqrt{2} \\ 0 & \sqrt{3}}$$

    \item[\textbf{№5}]
    $$\case{
        x_1 + 3x_2 = 5\\
        x_2 = 4\\
        x_1+x_2 = 3
    }\implies A = \mat{1 & 3 \\ 0 & 1 \\ 1 & 1}, b = \mat{5 \\ 4\\3}$$
    Найдём $x_0$ такое, что: 
    $$\rho(Ax_0, b) \to \min$$
    $x_0$ - псевдорешение.
    $$\rho(Ax_0, b) = f(x_1, x_2) = \left\| \mat{1 & 3 \\ 0 & 1 \\ 1 & 1}\mat{x_1\\ x_2}-\mat{5\\4\\3} \right\|= 
    \left\|\mat{x_1+3x_2-5\\x_2-4\\x_1+x_2-3}\right\| =$$
    $$=\sqrt{(-4 + x_2)^2 + (-3 + x_1 + x_2)^2 + (-5 + x_1 + 3x_2)^2}$$
    $$g(x_1, x_2) =(-4 + x_2)^2 + (-3 + x_1 + x_2)^2 + (-5 + x_1 + 3x_2)^2=$$
    $$= 50 - 16x_1 + 2x_1^2 - 44x_2 + 8x_1x_2 + 11x_2^2 $$
    $$\case{
        g(x_1, x_2)_{x_1}'=4\,\left(x_{1}+2\,x_{2}-4\right) = 0\\
        g(x_1, x_2)_{x_2}'= 2\,\left(11\,x_{2}+4\,x_{1}-22\right) =0\\
    } \implies \text{Стационарные точки: } (0, 2)$$
    Второй дифференциал:
    $$d^2g_{(x_1, x_2)}(\bar{h}) = \mat{h_1 & h_2}
     \mat{g(x_1, x_2)_{x_1x_1}' & g(x_1, x_2)_{x_1x_2}'\\
     g(x_1, x_2)_{x_2x_1}' & g(x_1, x_2)_{x_2x_2}'} \mat{h_1\\h_2} = $$
    $$=\mat{h_1 & h_2}
    \mat{4 & 8\\
    8 & 22} \mat{h_1\\h_2}$$
    $$\delta_1 = 4, \quad \delta_2 = 88-64=24$$
    Следовательно, $d^2g_{(x_1, x_2)}(\bar{h}) >0 $ по критерию Сильвестра, а значит $(0, 2)$ - точка минимума функции $h$, а значит и точка минимума функции $f$. Поэтому $(0, 2)$ - псевдорешение
    
    \textbf{Ответ: } $(0, 2)$\\


    \item[\textbf{№6}]Пусть матрица $A$ - ортогональная, тогда столбцы и строки
     образуют ортонормированные базисы. Это значит, что:
     $$(A^{(i)}, A^{(j)}) = \case{
        1, \text{ при } i=j\\
        0, \text{ при } i\neq j
     } \quad (A_{(i)}, A_{(j)}) = \case{
        1, \text{ при } i=j\\
        0, \text{ при } i\neq j
     } \quad (*)$$
     Следовательно, в любом столбце и в любой строке ровно одна 1. 
     Т.к. иначе не будет выполняться одно из условий $(*)$.
     
     \textbf{Вывод: } все ортогональные матрицы порядка $n$ получаются перестановкой столбцов и строк единичной матрицы
    
\end{enumerate}
\end{document}